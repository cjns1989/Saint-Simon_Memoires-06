\PassOptionsToPackage{unicode=true}{hyperref} % options for packages loaded elsewhere
\PassOptionsToPackage{hyphens}{url}
%
\documentclass[oneside,8pt,french,]{extbook} % cjns1989 - 27112019 - added the oneside option: so that the text jumps left & right when reading on a tablet/ereader
\usepackage{lmodern}
\usepackage{amssymb,amsmath}
\usepackage{ifxetex,ifluatex}
\usepackage{fixltx2e} % provides \textsubscript
\ifnum 0\ifxetex 1\fi\ifluatex 1\fi=0 % if pdftex
  \usepackage[T1]{fontenc}
  \usepackage[utf8]{inputenc}
  \usepackage{textcomp} % provides euro and other symbols
\else % if luatex or xelatex
  \usepackage{unicode-math}
  \defaultfontfeatures{Ligatures=TeX,Scale=MatchLowercase}
%   \setmainfont[]{EBGaramond-Regular}
    \setmainfont[Numbers={OldStyle,Proportional}]{EBGaramond-Regular}      % cjns1989 - 20191129 - old style numbers 
\fi
% use upquote if available, for straight quotes in verbatim environments
\IfFileExists{upquote.sty}{\usepackage{upquote}}{}
% use microtype if available
\IfFileExists{microtype.sty}{%
\usepackage[]{microtype}
\UseMicrotypeSet[protrusion]{basicmath} % disable protrusion for tt fonts
}{}
\usepackage{hyperref}
\hypersetup{
            pdftitle={SAINT-SIMON},
            pdfauthor={Mémoires VI},
            pdfborder={0 0 0},
            breaklinks=true}
\urlstyle{same}  % don't use monospace font for urls
\usepackage[papersize={4.80 in, 6.40  in},left=.5 in,right=.5 in]{geometry}
\setlength{\emergencystretch}{3em}  % prevent overfull lines
\providecommand{\tightlist}{%
  \setlength{\itemsep}{0pt}\setlength{\parskip}{0pt}}
\setcounter{secnumdepth}{0}

% set default figure placement to htbp
\makeatletter
\def\fps@figure{htbp}
\makeatother

\usepackage{ragged2e}
\usepackage{epigraph}
\renewcommand{\textflush}{flushepinormal}

\usepackage{indentfirst}
\usepackage{relsize}

\usepackage{fancyhdr}
\pagestyle{fancy}
\fancyhf{}
\fancyhead[R]{\thepage}
\renewcommand{\headrulewidth}{0pt}
\usepackage{quoting}
\usepackage{ragged2e}

\newlength\mylen
\settowidth\mylen{...................}

\usepackage{stackengine}
\usepackage{graphicx}
\def\asterism{\par\vspace{1em}{\centering\scalebox{.9}{%
  \stackon[-0.6pt]{\bfseries*~*}{\bfseries*}}\par}\vspace{.8em}\par}

\usepackage{titlesec}
\titleformat{\chapter}[display]
  {\normalfont\bfseries\filcenter}{}{0pt}{\Large}
\titleformat{\section}[display]
  {\normalfont\bfseries\filcenter}{}{0pt}{\Large}
\titleformat{\subsection}[display]
  {\normalfont\bfseries\filcenter}{}{0pt}{\Large}

\setcounter{secnumdepth}{1}
\ifnum 0\ifxetex 1\fi\ifluatex 1\fi=0 % if pdftex
  \usepackage[shorthands=off,main=french]{babel}
\else
  % load polyglossia as late as possible as it *could* call bidi if RTL lang (e.g. Hebrew or Arabic)
%   \usepackage{polyglossia}
%   \setmainlanguage[]{french}
%   \usepackage[french]{babel} % cjns1989 - 1.43 version of polyglossia on this system does not allow disabling the autospacing feature
\fi

\title{SAINT-SIMON}
\author{Mémoires VI}
\date{}

\begin{document}
\maketitle

\hypertarget{chapitre-premier.}{%
\chapter{CHAPITRE PREMIER.}\label{chapitre-premier.}}

1707

~

{\textsc{Mot étrangement marqué échappé à M. le Grand, dans la colère,
au jeu.}} {\textsc{- M. et M\textsuperscript{me} du Maine\,; leur
caractère et leur conduite.}} {\textsc{- Comte de Toulouse\,; son
caractère.}} {\textsc{- Succession femelle aux duchés de Lorraine et de
Bar.}} {\textsc{- État, famille, figure, santé, fortune et caractère de
Vaudémont\,; ses prétentions et ses artifices.}} {\textsc{- Trahison de
Colmenero.}} {\textsc{- Deux cent quatre-vingt mille livres de pension
de France et d'Espagne à M. et à M\textsuperscript{me} de Vaudémont en
arrivant.}} {\textsc{- Soixante mille livres de pension de l'empereur à
la duchesse de Mantoue, qui se retire en Suisse, puis dans un couvent à
Pont-à-Mousson.}} {\textsc{- État de la seigneurie de Commercy.}}
{\textsc{- Vaudémont obstinément refusé de l'ordre du Saint-Esprit.}}
{\textsc{- Cause de ce refus.}} {\textsc{- M\textsuperscript{me} de
Vaudémont à Marly, et comment.}} {\textsc{- Ses prétentions\,; son
embarras\,; son mécontentement\,; son caractère.}} {\textsc{- Sa prompte
éclipse.}} {\textsc{- Artifices et adroites entreprises de Vaudémont
déconcertées\,; sa conduite\,; ses ressources.}} {\textsc{- Raison de
s'être entendu sur ces tentatives.}} {\textsc{- Souplesse de
Vaudémont.}} {\textsc{- Commercy en souveraineté et Vic au prince de
Vaudémont, réversible au duc de Lorraine.}} {\textsc{- M. de Lorraine
donne au prince de Vaudémont la préséance, après ses enfants, au-dessus
de toute la maison de Lorraine.}} {\textsc{- L'un et l'autre demeure
inutile en France à Vaudémont.}} {\textsc{- Vaudémont abandonne enfin
ses chimères, qui demeure brouillé sans retour avec la maison de
Lorraine.}} {\textsc{- Prince Camille mal à son aise en Lorraine.}}
{\textsc{- Scandale de la brillante figure de Vaudémont en France.}}
{\textsc{- Trahisons continuées de Vaudémont et de ses nièces.}}
{\textsc{- Mesures secrètes de M. de Lorraine.}} {\textsc{- Courte
réflexion.}}

~

Telles étaient ces liaisons et leurs puissants appuis lors de l'arrivée
de M. de Vaudemont en France, dont ses nièces ne lui laissèrent rien
ignorer, et dans lesquelles elles l'initièrent le plus tôt qu'elles le
purent. Elles en avaient de grandes avec M. de Vendôme. On a vu ailleurs
que le prince de Conti et fui partageaient la faveur et la cour la plus
particulière de Monseigneur. M\textsuperscript{lle} Choin avait fait
assez d'effort pour rendre entre eux la balance du moins égale. Ses deux
amies, qui pour elle, ou plutôt pour l'intérêt qu'elles y trouvèrent,
avaient abandonné la princesse de Conti en sauvant toujours les
apparences tarit qu'elles le purent, et toujours assez pour éviter
brouillerie, étaient par là même entraînées vers M. de Vendôme.
D'ailleurs le sang de Lorraine, si ce n'est par force, ne fut jamais
pour aimer, encore moins pour s'attacher au sang de Bourbon.

Cela me fait souvenir d'une brutalité qui échappa à M. le Grand, et qui
par cela même montre le fond de l'âme. Il jouait au lansquenet dans le
salon de Marly avec Monseigneur, et il était très gros et très méchant
joueur. Je ne sais par quelle occasion de compliment
M\textsuperscript{me} la grande-duchesse {[}de Toscane\footnote{Voy. la
  note I, à la fin du volume.}{]} y était venue de son couvent, car elle
y était encore, où elle ne devait retourner qu'après avoir soupé avec le
roi. Le hasard fit qu'elle coupait M. le Grand, et qu'elle lui donna un
coupe-gorge. Lui aussitôt donna un coup de poing sur la table, et, se
baissant dessus, s'écria tout haut. «\,La maudite maison, nous
sera-t-elle toujours funeste\,?» La grande-duchesse rougit, sourit et se
tut. Monseigneur et tout ce qui était, hommes et femmes, à la table et
autour l'entendirent clairement. Le grand écuyer se releva le nez de
dessus la table, regarda toute la compagnie toujours bouffant. Personne
ne dit mot, mais à l'oreille après on ne s'en contraignit pas. Je ne
sais si le roi le sut, mais ce qu'il y a de certain, c'est qu'il n'en
fut autre chose, et qu'il n'en fut pas moins bien traité.

M. le prince de Conti de plus ne donnait aux deux soeurs que
M\textsuperscript{me} la Duchesse dont elles étaient bien assurées
d'ailleurs\,; Vendôme leur donnait occasion de gagner M. du Maine, et
pour elles il n'y avait rien de trop. Elles s'étaient donc liées tant
qu'elles avaient pu à Vendôme, et dans cet esprit elles avaient fort
recommandé à leur cher oncle, car c'est ainsi qu'elles l'appelaient et
qu'elles en parlaient toujours, de ne rien oublier pour engager Vendôme,
lorsqu'il alla en Italie, à en revenir assez de ses amis pour qu'ils
pussent compter sur lui. Le cher oncle profita bien de la leçon, et y
réussit tellement qu'à son retour, et toujours depuis, elles n'eurent
rien à désirer là-dessus, et que Vendôme, elles et Vaudemont, M. du
Maine en quart, se lièrent le plus étroitement, mais le dernier, selon
sa coutume, le plus secrètement.

M. du Maine sentait que Monseigneur ne l'aimait point\,; nulle meilleure
voie de l'en rapprocher peu à peu que ses plus confidentes amies\,;
Vendôme n'était pas seul bastant. Le roi avançait en âge, et Monseigneur
vers le trône\,; M. du Maine en tremblait. Avec de l'esprit, je ne dirai
pas comme un ange, mais comme un démon auquel il ressemblait si fort en
malignité, en noirceur, en perversité d'âme, en desservices à tous, en
services à personne, en marches profondes, en orgueil le plus superbe,
en fausseté exquise, en artifices sans nombre, en simulations sans
mesure, et encore en agréments, en l'art d'amuser, de divertir, de
charmer quand il voulait plaire\,; c'était un poltron accompli de coeur
et d'esprit, et à force de l'être, le poltron le plus dangereux, et le
plus propre, pourvu que ce fût par-dessous terre, à se porter aux plus
cruelles extrémités pour parer ce qu'il jugeait avoir à craindre, et se
porter aussi à toutes les souplesses et les bassesses les plus rampantes
auxquelles le diable ne perdait rien.

Il était de plus poussé par une femme de même trempe, dont l'esprit, et
elle en avait aussi infiniment, avait achevé de se gâter et de se
corrompre par la lecture des romans et des pièces de théâtre, dans les
passions desquelles elle s'abandonnait tellement qu'elle a passé des
années à les apprendre par coeur, et à les jouer publiquement elle-même.
Elle avait du courage à l'excès, entreprenante, audacieuse, furieuse, ne
connaissant que la passion présente et y postposant tout, indignée
contre la prudence et les mesures de son mari qu'elle appelait misères
de faiblesse, à qui elle reprochait l'honneur qu'elle lui avait fait de
l'épouser, qu'elle rendit petit et souple devant elle en le traitant
comme un nègre, le ruinant de fond en comble sans qu'il osât proférer
une parole, souffrant tout d'elle dans la frayeur qu'il en avait et dans
la terreur que la tête achevât tout à fait de lui tourner. Quoiqu'il lui
cachât assez de choses, l'ascendant qu'elle avait sur lui était
incroyable, et c'était à coups de bâton qu'elle le poussait en avant.

Nul concert avec le comte de Toulouse\,; c'était un homme fort court,
mais l'honneur, la vertu, la droiture, la vérité, l'équité même, avec un
accueil aussi gracieux qu'un froid naturel, mais glacial, le pouvait
permettre\,; de la valeur et de l'envie de faire, mais par les bonnes
voies, et en qui le sens droit et juste, pour le très ordinaire,
suppléait à l'esprit\,; fort appliqué d'ailleurs à savoir sa marine de
guerre et de commerce et l'entendant très bien. Un homme de ce caractère
n'était pas pour vivre intimement avec son frère et sa belle-soeur. M.
du Maine le voyait aimé et estimé parce qu'il méritait de l'être, il lui
en portait envie. Le comte de Toulouse, sage, silencieux, mesuré, le
sentait, mais n'en faisait aucun semblant. Il ne pouvait souffrir les
folies de sa belle-soeur. Elle le voyait en plein, elle en rageait, elle
ne le pouvait souffrir à son tour, elle éloignait encore les deux frères
l'un de l'autre.

Celui-ci était fort bien avec Monseigneur et M. et M\textsuperscript{me}
la duchesse de Bourgogne qu'il avait toujours fort ménagés et respectés.
Il était timide avec le roi, qui s'amusait beaucoup plus de M. du Maine,
le Benjamin de M\textsuperscript{me} de Maintenon, son ancienne
gouvernante, à qui il sacrifia M\textsuperscript{me} de Montespan, qui
toutes deux ne l'oublièrent jamais. Il avait eu l'art de persuader au
roi qu'avec beaucoup d'esprit, qu'on ne pouvait lui méconnaître, il
était sans aucunes vues, sans nulle ambition, et un idiot de paresse, de
solitude, d'application, et la plus grande dupe du monde en tout genre.
Aussi passait-il sa vie dans le fond de son cabinet, mangeait seul,
fuyait le monde, allait seul à la chasse, et de cette vie sauvage s'en
faisait un vrai mérite auprès, du roi, qu'il voyait tous les jours en
toutes ses heures particulières\,; enfin, suprêmement hypocrite, à la
grand'messe, aux vêpres, au salut toutes les fêtes et dimanches avec
apparat. Il était le coeur, l'âme, l'oracle de M\textsuperscript{me} de
Maintenon, de laquelle il faisait tout ce qu'il voulait, et qui ne
songeait qu'à tout ce qui lui pouvait être le plus agréable et le plus
avantageux, aux dépens de quoi que ce pût être.

Voilà bien de la digression\,; mais on verra dans la suite combien elle
est nécessaire pour l'éclaircissement et le dévoilement de ce qui se
présentera à raconter. Ces personnages remueront bien des choses qui ne
se pourraient entendre sans cette clef. Je l'ai donnée aux approches du
besoin, et lorsque j'en ai trouvé l'occasion. Revenons maintenant à M.
de Vaudemont.

Ce que j'ai expliqué (t. III, p.~195 et suiv.) de ses deux importantes
nièces est si éloigné de l'endroit où nous sommes, que j'ai cru devoir
les remettre ici devant les yeux sans craindre quelque sorte de
répétition, par les choses si importantes où on les va voir figurer. La
même raison me fait négliger la même crainte sur M. de Vaudemont, pour
remettre ici sommairement sous le même coup d'oeil ce qui se trouve
épars en trop de différents endroits. C'est un éclaircissement
nécessaire pour répandre la lumière sur ses prétentions par sa
naissance, et sur les grâces prodigieuses qu'il tira des cours de France
et d'Espagne, qu'il ne dut pas à ce qu'il en avait mérité.

Charles II, ordinairement dit III, duc de Lorraine, si connu pour avoir
eu l'honneur d'épouser, en 1558, la seconde fille d'Henri II et de
Catherine de Médicis, et plus encore par tout ce que cette reine mit en
oeuvre pour le faire succéder à la couronne après ses enfants, au
préjudice d'Henri IV, son autre gendre, et de toute la branche royale de
Bourbon, eut, sans parler des filles, trois fils de ce mariage\,: Henri,
qu'il eut l'honneur de marier, en 1599, à la soeur d'Henri IV, si connu
aussi par tout ce qu'il mit en usage pour faire rompre ce mariage que
les belles lettres du cardinal d'Ossat expliquent si bien, qui la perdit
sans enfants en 1604, qui se remaria en 1606 à une fille du duc Vincent
de Mantoue, d'où est venue à leur postérité la prétention du Montferrat.
Il succéda à son père en 1608 et mourut en 1624, ne laissant que deux
filles\,: Nicole et Claude-Françoise. Le second fut Charles, cardinal,
évêque de Metz et de Strasbourg\,; et le troisième, François, comte de
Vaudemont qui, d'une Salm, eut deux fils\,: Charles et François\,; et
deux filles\,: l'aînée, si connue, sous le nom de princesse de
Phalsbourg, par ses intrigues, et par tous ses étranges mariages\,; et
la cadette, que M. Gaston épousa de la façon que chacun sait, et qui
n'en a laissé que trois filles M\textsuperscript{lle} de
Montpensier\footnote{Il y a dans cette phrase une erreur de généalogie
  qu'on ne peut attribuer qu'à une inadvertance\,; car Saint-Simon
  connaissait parfaitement la famille de M\textsuperscript{lle} de
  Montpensier. Cette princesse n'était pas fille de Marguerite de
  Lorraine dont il est ici question, mais de la première femme de
  Gaston, Marie de Bourbon, duchesse de Montpensier. Gaston eut de son
  second mariage trois filles\,: Marguerite-Louise d'Orléans, mariée à
  Cosme III de Médicis, grand-duc de Toscane\,; Élisabeth d'Orléans, qui
  devint M\textsuperscript{me} de Guise, et Françoise-Madeleine
  d'Orléans, mariée à Charles-Emmanuel, duc de Savoie, et morte peu de
  temps après son mariage.}, M\textsuperscript{me} la grande-duchesse de
Toscane et M\textsuperscript{me} de Guise.

Les duchés de Lorraine et de Bar, très constamment féminins, et déjà une
fois passés dans la maison d'Anjou, au bon roi René par une héritière,
et retournés par une autre héritière d'Anjou dans la maison de Lorraine,
vinrent de droit à Nicole, fille aînée du duc Henri qui, pour les
conserver dans sa maison, la maria trois ans avant sa mort à Charles,
fils aîné de son troisième frère, qui avait lors vingt et un ans, et
Nicole treize, en présence du comte et de la comtesse de Vaudemont, père
et mère de Charles, qui succéda en 1623, trois ans après son mariage, à
son beau-père par le droit de sa femme. C'est celui qui, sous le nom de
Charles IV, est si connu par ses perfidies, dont toute sa vie n'a été
qu'un tissu, et qui lui firent mener une vie si malheureuse avec
beaucoup d'esprit et de valeur, qui lui coûtèrent ses États et ensuite
une longue prison en Espagne. Comme il n'avait point d'enfants dix ans
après son mariage, ils firent celui de François son frère avec
Claude-Françoise, soeur de la duchesse Nicole, pour assurer les deux
duchés dans leur maison. De ce dernier mariage est venu le fameux
Charles, duc de Lorraine et de Bar, beau-frère de l'empereur Léopold,
qui ne vit et ne posséda jamais ses États, qui s'est acquis un si grand
nom à la tête des armées impériales, dont le fils fut rétabli dans ses
États à la paix de Ryswick, lequel, d'une fille de Monsieur, frère de
Louis XIV, a laissé deux fils, dont l'aîné, devenu grand-duc de Toscane,
a cédé pour toujours les duchés de Lorraine et de Bar à la couronne, et
a épousé la fille aînée de Charles VI, dernier empereur et dernier mâle
de la maison d'Autriche.

Charles IV, amoureux de Béatrix de Cusance, veuve du comte de
Cantecroix, et retiré à Bruxelles, servant la maison d'Autriche, la fit
faire par l'empereur princesse de l'empire, se fit annoncer la mort de
la duchesse Nicole, sa femme, en arbora le plus grand deuil, en reçut
tous les compliments à Bruxelles, et en partit subitement pour Besançon,
où un valet déguisé en prêtre le maria dans sa chambre avec
M\textsuperscript{me} de Cantecroix, le 2 avril 1637. La fourbe fut en
peu de jours découverte, la duchesse Nicole n'avait pas seulement été
malade. Son mari eut de M\textsuperscript{me} de Cantecroix une fille en
1639, qui a été M\textsuperscript{me} de Lislebonne, mère de
M\textsuperscript{lle} de Lislebonne et de la princesse d'Espinoy, et
dix ans après un fils qui est le prince de Vaudemont. Il faut remarquer
que Charles IV n'a jamais attaqué la validité de son mariage avec la
duchesse Nicole, et qu'elle n'est morte qu'en 1657, c'est-à-dire plus de
dix-sept ans après la naissance de M. de Vaudemont. Charles IV son père
mourut en 1675 sans enfants légitimes. François, son frère, était mort
dès 1670, Claude-Françoise, sa femme, soeur de Nicole, dès 1648, sans
que François se soit remarié. Ainsi, le célèbre Charles, qui devint dans
la suite beau-frère de l'empereur Léopold et général de ses armées,
succéda de droit à son oncle, Charles IV, sans que ce droit qu'il tenait
de sa mère lui ait été jamais contesté. Charles IV voulut appuyer ses
bâtards de sa propre maison. Il trouva M. de Lislebonne, frère du duc
d'Elboeuf, qui s'attacha à sa fortune, et qui voulut bien épouser sa
bâtarde en 1660, laquelle avait vingt et un ans\,; et neuf ans après, le
même duc d'Elboeuf, qui ne se souciait point de son fils le trembleur du
premier lit, à qui il fit céder son droit d'aînesse au duc d'Elboeuf
d'aujourd'hui, fils de son second lit, donna sa fille du premier lit à
M. de Vaudemont. Elle était soeur de mère de la femme du duc de La
Rochefoucauld, qui a été si bien avec Louis XIV. M. de Vaudemont avait
vingt ans, et sa femme était du même âge.

On a vu ailleurs tout le parti qu'il sut tirer de sa figure, de son
esprit, de sa galanterie, et comme le maréchal de Villeroy, épris de ses
manières et de le voir si à la mode en France, crut du bel air d'être de
ses amis, et se piqua toute sa vie d'en être. Vaudemont ne tarda pas à
s'apercevoir que ses gentillesses ne le mèneraient à rien de solide ici.
Il s'en alla aux Pays-Bas, entra au service des ennemis de la France,
fit sa cour au prince d'Orange et aux ministres de la maison d'Autriche.
Il alla en Espagne, où, appuyé de force patrons qu'il s'était ménagés,
il obtint une grandesse à vie pour se donner un rang et un état de
consistance, puis la Toison d'or pour se décorer. C'était en 1677, au
temps de la plus forte guerre de la France contre la maison d'Autriche.
On a vu en son lieu à quel point il se déchaîna contre elle pour plaire,
et avec tant d'insolence, à Rome, où il alla d'Espagne, que le roi ne
dédaigna pas de se montrer piqué sur le personnel qu'il avait osé
attaquer, et le fit sortir honteusement de Rome par ordre du pape. Il
alla en Allemagne, où il sut se faire un mérite de cette aventure auprès
de l'empereur, qui le protégea toujours depuis et le fit prince de
l'empire, et auprès du prince d'Orange, si personnellement mal avec le
roi. Il sut plaire à ce dernier par ses grâces, par son esprit, par son
adresse, par leur haine commune, au point d'entrer dans sa plus intime
confiance, qu'il accordait à si peu de gens. On en a vu des marques à
l'occasion de la dernière campagne de Louis XIV en Flandre, et de son
brusque retour à Versailles, en 1693. Cette affection du roi Guillaume
le mit à la tête de l'armée de Flandre, où nous l'avons vu échapper si
belle, grâce à M. du Maine, dont le maréchal de Villeroy sut si
habillement faire sa cour au roi. Enfin, la protection du roi Guillaume
et de l'empereur lui valurent de Charles II le gouvernement général du
Milanais.

On a vu avec quelle dangereuse dextérité il s'y comporta, après n'avoir
osé ne pas y faire proclamer Philippe V, et combien sa soumission fut
ici portée, vantée et applaudie. L'aveuglement fut constant sur lui par
son adresse et la puissante cabale qui le portait, et on vient de voir
qu'après la mort de son fils, feld-maréchal des armées impériales, et
servant en Italie, contenu d'ailleurs par Vendôme, dont il redouta les
yeux et le poids auprès du roi, il se rendit plus mesuré et se l'acquit
par ses souplesses.

Enfin, l'Italie perdue, il profita du mérite d'en avoir sauvé et ramené,
par un traité, vingt mille hommes qui étaient restés, après la victoire
de Médavy, de troupes de France et d'Espagne, qui fut mettre le sceau à
la honte et au dommage extrême d'avoir remis l'Italie à l'empereur,
lorsqu'on pouvait s'y soutenir, et empêcher par là l'ennemi d'attaquer
notre frontière et de pénétrer en France.

En y arrivant, il ne tint encore tout de nouveau à notre cour d'ouvrir
les yeux. Colmenero était l'officier général des troupes du roi
d'Espagne, servant en Italie, le plus intimement dans la confidence de
M. de Vaudemont, qui l'avait avancé à tout et mis avec M. de Vendôme sur
le pied d'avoir part à tout. Nos François soupçonnaient fort sa
fidélité, et croyaient avoir des raisons d'être persuadés qu'ils ne s'y
trompaient pas\,; mais avec de tels appuis il fallut se taire. Il avait
rendu Alexandrie, comme on l'a vu en son temps, d'une manière à
augmenter tout à fait ce soupçon. M. de Vaudemont le soutint
hautement\,; et M. de Vendôme, revenu d'Italie, intimement uni avec lui,
et qui était souvent dupe de moins habiles en l'art de tromper, prit
hautement sa défense. Ils ne persuadèrent personne de ceux qui voyaient
les choses de près, mais bien notre cour, accoutumée à les croire à
l'aveugle. La surprise y fut donc grande lorsqu'on y apprit, en même
temps que Vaudemont y arriva, que le prince Eugène, par ordre de
l'archiduc, avait donné le gouvernement du château de Milan à Colmenero,
qui en même temps passa vers lui, et fut conservé chez les Impériaux
dans le même grade qu'il avait dans nos armées. Vaudemont s'en étonna
fort, M. de Vendôme aussi, de Mons où il était alors, et se sentit piqué
de sa méprise\,; mais ce fut tout, il n'entra pas seulement dans la
pensée de trouver mauvais que Vaudemont l'eût tant vanté.

MMMM\hspace{0pt}. de Vendôme et de Vaudemont avaient passé par la même
étamine\,; Vendôme y avait laissé presque tout son nez, Vaudemont les os
des doigts de ses pieds et de ses mains, qui n'étaient plus qu'une chair
informe, sans consistance, qui se rabattait toute l'une sur l'autre\,;
ses mains faisaient peine à regarder. Il en avait eu d'autres suites
très fâcheuses, dont les médecins n'avaient pu venir à bout. Un
empirique le guérit à Bruxelles autant qu'il pouvait l'être et le mit en
état de se tenir à cheval et sur ses pieds. Ce fut son prétexte en
Italie de paraître si peu dans les armées et d'y monter si rarement à
cheval. Du reste, il avait conservé toute sa belle figure à son âge,
fort droit, grande mine et une fort bonne santé. On va voir qu'il sut
tirer parti d'un état dont la source est si honteuse.

M. de Vaudemont et ses nièces étaient fort occupés de sa subsistance et
de son rang. Il avait acquis à Milan des sommes immenses, et dans
quelque splendeur qu'il y eût vécu, il lui en était resté beaucoup,
comme on ne put s'empêcher d'en être convaincu dans la suite. Mais il ne
fallait pas le laisser apercevoir, et pour obtenir gros, et pour ne pas
perdre le mérite d'un homme si grandement établi et qui revient tout nu.
Cela ne leur parut pas le plus difficile, et, en effet, ils furent si
bien servis que, tout en arrivant, le roi donna quatre-vingt-dix mille
livres de pension à M. de Vaudemont, et qu'il écrivit aussi au roi
d'Espagne pour lui recommander ses intérêts. Ils se trouvèrent encore en
meilleures mains auprès de M\textsuperscript{me} des Ursins, qui,
nonobstant l'état fâcheux des finances et des affaires d'Espagne, où
tout manquait, comme on l'a vu, à l'occasion des suites de la bataille
d'Almanza, voulut montrer à M\textsuperscript{me} de Maintenon ce
qu'elle pouvait sur elle, et fit donner, tant à M. qu'à
M\textsuperscript{me} de Vaudemont, cent quatre-vingt-dix mille livres
de pension. Il avait fait sa révérence au roi le 10 mai\,; mais le 15
juin la réponse d'Espagne était arrivée. On aurait pu croire que deux
cent quatre-vingt mille livres de rente auraient dû suffire et les
contenter. Ce ne fut pas tout, et il faut le dire tout de suite, pour ne
pas revenir au pécuniaire.

M. de Vaudemont avait eu une patente de prince de l'empire de l'empereur
Léopold, qui lui avait fait changer son titre de comte de Vaudemont en
celui de prince. On a vu ses liaisons si longtemps intimes à Vienne, et
depuis si peu encore, son fils unique mort en Italie feld-maréchal des
armées impériales, et la seconde personne de celle de Lombardie. Les
mêmes liaisons, il les avait conservées plus à découvert et avec plus de
bienséance avec les deux ducs de Lorraine père et fils. Il avait, en
traitant avec le prince Eugène du retour de nos troupes, demandé une
pension pour le duc de Mantoue, que l'empereur dépouillait totalement,
et une pour M\textsuperscript{me} de Mantoue. Il fut durement refusé de
la première\,; il obtint la seconde, et le prince Eugène convint qu'elle
serait de vingt mille écus. M\textsuperscript{me} de Mantoue partit
aussitôt pour aller attendre à Soleure la permission d'aller en Lorraine
se mettre aux Filles de Sainte-Marie de Pont-à-Mousson, et
M\textsuperscript{me} de Vaudemont, sa soeur de père, l'accompagna dans
ce voyage, sous prétexte d'amitié et de bienséance, mais en effet pour
négocier de plus près auprès de M. de Lorraine ce qu'on avait engagé le
roi de lui demander pour M. de Vaudemont, où par ce peu que dura une
négociation qui coûta tant à M. de Lorraine, et pour rien, on soupçonna
la cour de Vienne d'y être entrée, laquelle pouvait tout sur lui. Quoi
que ce fût, les dames ne séjournèrent pas longtemps à Soleure, passèrent
en Lorraine\,; M\textsuperscript{me} de Mantoue demeura à
Pont-à-Mousson, et M\textsuperscript{me} de Vaudemont s'en vint à Paris
à l'hôtel de Mayenne.

Charles IV, père de M. de Vaudemont, lui avait donné le comté de
Vaudemont, dont son père portait le nom, et qui a été souvent apanage
des puînés des ducs de Lorraine, quoique la terre ne soit pas
considérable. Le même Charles IV avait acquis du cardinal de Retz la
terre de Commercy, qu'il avait eue de sa mère, qui était
Cilly\footnote{La mère de Charles IV était Catherine, comtesse de Salin.},
et il la donna aussi à M. de Vaudemont, lequel y succéda au cardinal de
Retz, qui en avait retenu la jouissance sa vie durant, et qui s'y était
retiré en revenant d'Italie, pour payer ses dettes et y faire pénitence
de sa vie passée dans la solitude. Dans les suites, le duc Léopold de
Lorraine, gendre de Monsieur, acquit Commercy de M. de Vaudemont, et le
laissa jouir du revenu, qui n'est pas considérable. Cette seigneurie
relevait constamment de l'évêché de Metz. Ils l'avaient donnée en fief à
des seigneurs sous le nom de \emph{damoiseaux}\footnote{Ce mot, formé du
  latin \emph{domicellus} (petit ou jeune seigneur), indiquait d'abord
  le fils d'un chevalier. Il servit dans la suite à désigner les
  possesseurs de certains fiefs et spécialement du fief de Commercy.}.
Les comtes de Nassau-Sarrebruck, qui l'ont longtemps possédée, en ont
toujours reconnu les évêques de Metz, et leur en ont rendu leurs
devoirs\,; et les officiers du roi du bailliage de Vitry ayant formé des
prétentions sur la justice de quelques paroisses de cette terre, son
seigneur et le duc Antoine de Lorraine firent lever, en 1540, de la
chambre de Vic, tous les actes qui démontrèrent que tout Commercy
relevait de l'évêché de Metz, et non pas du roi en rien. Le cardinal de
Lenoncourt en reçut tous les devoirs, comme évêque de Metz, en 1551.
Cependant cette seigneurie était peu à peu devenue une espèce de petite
souveraineté. Il s'y forma une manière de chambre de grands jours, où
les procès se jugeaient en dernier ressort. Les Cilly la possédèrent en
cet état\,; mais, en 1680, la chambre royale de Metz reconnut,
nonobstant ces grands jours, et malgré les prétentions du bailliage de
Vitry, duquel quelques paroisses relevaient, que le droit féodal et
direct sur Commercy en entier appartenait à l'évêque de Metz, et lui fut
adjugé. Malgré des empêchements si dirimants, M. de Vaudemont se proposa
de se faire donner par le duc de Lorraine la souveraineté de Commercy, à
lui qui, de plus, avait vendu cette terre à ce prince, qui le laissait
jouir du revenu\,; d'y faire joindre par le même des dépendances
nouvelles, pour en grossir le revenu et en étendre la souveraineté, et
de rendre le roi protecteur de cette affaire\,; et on verra bientôt
qu'il y réussira, et même à davantage.

En attendant, il songeait fort à s'établir un rang distingué. Il avait
celui de grand d'Espagne, mais il n'avait garde de s'en contenter. Comme
prince de l'empire, il n'en pouvait espérer. Celui de ses grands emplois
avait cessé avec eux, et ce groupe de tant de choses accumulées, et qui
éblouissaient les sots, lui parut trop aisé à désosser pour se pouvoir
flatter d'en faire réussir quelque chose de solide. Il avait tenté, au
milieu de sa situation la plus brillante et la plus accréditée en
Italie, d'être fait chevalier de l'ordre\,; il l'avait fait insinuer par
ses amis\,; enfin il l'avait lui-même formellement demandé. Il avait été
refusé à plus d'une reprise, et on ne lui en avait pas caché la raison,
avec force regrets de ne la pouvoir surmonter. Cette raison était un
statut de l'ordre du Saint-Esprit qui en exclut tous les bâtards, sans
aucune autre exception que ceux des rois. Il eut beau insister, piquer
l'orgueil, en représentant que le roi était maître des dispenses, tout
fut inutile. Dès le temps que le roi d'Espagne était en Italie, il y
employa Louville auprès de Torcy et de M. de Beauvilliers, qui me l'a
conté\,; et depuis il y employa encore Tessé, le maréchal de Villeroy et
M. de Vendôme. Tout fut inutile\,; il n'y eut point de crédit ni de
considération qui pût obtenir du roi d'assimiler un bâtard de Lorraine
aux siens en quoi que ce pût être. Mais quoique le refus ne portât que
sur cet intérêt si cher au roi, il ne laissait pas de montrer à
Vaudemont que le roi ne le prendrait jamais que pour ce qu'il était,
c'est-à-dire que pour un bâtard de Lorraine, qui, par la raison qui
vient d'être expliquée, et que Vaudemont et ses nièces avaient trop
d'esprit pour ne pas sentir, se trouverait toujours en obstacle à toutes
ses prétentions. Ce fut apparemment aussi ce qui lui fit imaginer cette
souveraineté de Commercy, et entreprendre encore au delà, comme on le
verra, pour couvrir sa bâtardise de façon que la raison secrète du roi
en pût être détournée.

Mais tout cela n'était pas fait, et, en attendant, il fallait être à la
cour et dans le monde. N'osant donc hasarder de refus, pour demeurer
entier pour quand\footnote{Jusqu'à ce que.} tout son fait de Commercy et
de plus encore serait arrangé, il résolut d'usurper sans avoir l'air de
prétendre ou de laisser douteux, et se servir avec adresse des excès
d'avances qu'il recevait de tout ce qu'il y avait à la cour de plus
grand, de plus distingué, de plus accrédité\,; d'abuser de la sottise du
gros du monde, et de cacher ses entreprises sous l'impotence de sa
personne, pour, ce qu'il aurait ainsi ténébreusement conquis et tourné
adroitement en habitude, le prétendre après en rang qui lui aurait été
acquis.

Il se fit donc porter en chaise à travers les petits salons jusqu'à la
porte du grand, comme très rarement il arrivait aux filles du roi de le
faire, et ne se tenait debout que devant le roi. Il évita d'aller chez
Monseigneur et chez Mgrs ses fils, sous prétexte de ses jambes, sinon,
en arrivant, leur faire la révérence, et de même chez
M\textsuperscript{me} la duchesse de Bourgogne et chez Madame. Chez les
autres, il se mit sur le premier siège qu'il y trouva\,; et il n'y avait
que des tabourets dans ces appartements de Marly, et dans le salon de
même. Il s'y plaçait dans un coin\,; la plus brillante compagnie s'y
rassemblait autour de lui assise et debout, et là il tenait le dé.
Monseigneur en approcha quelquefois\,; Vaudemont, avec adresse,
l'accoutuma à ne se point lever pour lui, et tout aussitôt après il en
usa de même pour M\textsuperscript{me} la duchesse de Bourgogne.

Tous les ministres furent d'abord chez lui\,; il vit seul
M\textsuperscript{me} de Maintenon chez elle, mais cela se réitéra fort
peu, et il n'y vit jamais le roi, dont il n'eut presque point d'audience
dans son cabinet. Rien de si brillant que ce voyagé, et le roi toujours
occupé de lui. Il lui fit donner une calèche à toutes ses chasses. Une
de ses nièces y allait avec lui. Il était assez plaisant de les voir
tous deux suivre celle du roi, qui était seul dans la sienne avec
M\textsuperscript{me} la duchesse de Bourgogne, et figurer ainsi en deux
tête à tête, sans autre calèche que celle du capitaine des gardes, car
Madame montait encore alors à cheval. Ce voyage de Marly, où il était
arrivé et s'était compassé pour cela avec justesse, s'écoula de la sorte
à y faire toute l'attention, à y être l'homme uniquement principal et à
reconnaître son monde.

Il partagea après son temps moins à Versailles qu'à Paris. Versailles
était plus public, moins ramassé, moins pêle-mêle, les milieux plus
difficiles à garder. Il jugea sagement que, son terrain bien sondé, il
fallait disparaître pour réveiller le goût et l'empressement, et ne les
pas user par l'habitude. Au bout d'un mois, il prit congé et s'en alla à
Commercy avec sa soeur, ses nièces et sa femme, qui, sous prétexte de
fatigue et de santé délicate, n'avait vu le jour à Paris que par le trou
d'une bouteille, niais en effet par l'embarras de ses prétentions,
qu'elle ne voulait pas commettre, et savoir, avant de se présenter à la
cour, sur quel pied elle s'y conduirait\,: Vaudemont, en partant,
s'assura, puis s'annonça pour le premier voyage de Marly. C'était une
distinction qu'il lui importait de ne pas négliger. Trois semaines
suffirent à cette course. La santé était bonne quand il le fallait, et
les jambes ne faisaient jamais rien manquer d'utile.
M\textsuperscript{me} de Lislebonne et M\textsuperscript{me} de
Vaudemont demeurèrent à Paris\,; l'oncle et les nièces vinrent à Marly.
Avant son départ, il y avait eu une négociation. M\textsuperscript{me}
de Vaudemont, qui ne savait encore sur quel pied danser, voulait éviter
le cérémonial de Versailles et aller droit à Marly, comme son mari avait
fait. Le roi trouvait cela ridicule, et cela balança. Au retour de M. de
Vaudemont, il insista si bien qu'il en résulta une distinction plus
grande, parce que le roi la trouva moindre que de recevoir de plein
saut, à Marly, une femme qu'il n'avait jamais vue, et qui se tortillait
en prétentions. Vaudemont et ses nièces arrivèrent le samedi à Marly.

Dans le dimanche, M\textsuperscript{me} de Maintenon fit agréer au roi
que, allant elle à Saint-Cyr le mercredi, comme elle y allait de Marly
presque tous les jours, que celui-là même M\textsuperscript{me} de
Vaudemont l'y viendrait voir de Paris\,; que, sans que nome de Vaudemont
lui parlât de Marly, ce serait elle qui lui proposerait de l'y mener. Le
roi y consentit, puis se ravisa, enfin il l'accorda, et ce qui avait été
réglé pour le mercredi ne s'exécuta que le vendredi. Le roi, entrant le
soir chez M\textsuperscript{me} de Maintenon, y trouva
M\textsuperscript{me} de Vaudemont qui arrivait avec elle. L'accueil fut
gracieux, mais court\,; elle ne soupa point, à cause du maigre. Le
lendemain elle fut présentée à M\textsuperscript{me} la duchesse de
Bourgogne, comme elle allait partir pour la messe, et vit un instant
Monseigneur et Mgr le duc de Bourgogne chez eux, puis les princesses
fort uniment, mais fort courtement. Elle fut l'après-dînée, avec le roi
et presque toutes les dames, voir la roulette, où M\textsuperscript{me}
la duchesse de Bourgogne allait, puis à une grande collation dans le
jardin. M\textsuperscript{me} de Vaudemont ne fut pas, à beaucoup près,
si fêtée que son mari. Elle demeura trois jours à Marly, et s'en alla le
mardi à Paris. Elle revint sept ou huit jours après à Marly passer
quelques jours, et se hâta ensuite de regagner Commercy, peu contente de
n'y avoir pu rien usurper en rang et en préférences.

C'était une personne tout occupée de sa grandeur, de ses chimères, de sa
chute du gouvernement du Milanais\,; elle l'était aussi de sa santé,
mais beaucoup moins en effet que comme chausse-pied ou couverture\,;
tout empesée, toute composée, tout embarrassée, un esprit peu naturel,
une dévotion affichée, pleine d'extérieur et de façons\,; en deux mots,
rien d'aimable, rien de sociable, rien de naturel\,; grande, droite, un
air qui voulait imposer, et néanmoins être doux, mais austère et tirant
fort sur l'aigre-doux. Personne ne s'en accommoda, elle ne s'accommoda
de rien ni de personne\,; elle fut ravie d'abréger et de s'en aller, et
personne n'eut envie de la retenir.

Son mari, ployant, insinuant, admirant avec les plus basses flatteries,
paraissant s'accommoder à tout, continua à Marly son manège. Il y avait
dans le salon trois sièges à dos, qui de l'un à l'autre s'y étaient
amassés, et de la même étoffe que les tabourets. Monseigneur, qui avait
fait faire le premier, jouait dessus\,; en son absence,
M\textsuperscript{me} la duchesse de Bourgogne s'y mit, puis sur un
autre qu'on fit faire pour elle pour ses grossesses.
M\textsuperscript{me} la Duchesse hasarda de demander la permission à
Monseigneur d'en faire cacher un semblable dans un coin, et d'y jouer à
l'abri d'un paravent. Vaudemont, qui avisa que les trois n'étaient
presque jamais occupés ensemble, en prit un d'abord les matins, entre le
lever et la messe, où Monseigneur et les deux princesses n'étaient
jamais dans le salon. Il y tint, à son coin ordinaire, ses assises,
l'exquis de la cour autour de lui sur des tabourets\,; et quand il y eut
accoutumé le monde, qui en France trouve tout bon, à condition que ce
soient des entreprises, il se licencia de la garder les soirs pendant le
jeu. Cela dura deux voyages de la sorte, pendant le second desquels il
fit rehausser les pieds de sa chaise, en apparence pour être plus à son
aise, parce qu'il était grand, en effet pour se l'approprier, et
s'établir ainsi la distinction que personne n'avait, et sans se couvrir
d'un paravent comme faisait M\textsuperscript{me} la Duchesse.
Monseigneur venait quelquefois lui parler, sur cette chaise, quelquefois
aussi M\textsuperscript{me} la duchesse de Bourgogne en voltigeant par
le salon\,: il ne se levait point\,; sur la fin il n'en faisait pas même
contenance\,; il les y avait accoutumés.

Après ces voyages, il voulut aller faire sa cour à M\textsuperscript{me}
la duchesse de Bourgogne, comptant que, l'ayant accoutumée à lui parler
assis à Marly, il était temps de prétendre de l'être chez elle. Il eut
la bonté de s'y contenter d'un tabouret, et de n'y prétendre pas plus
que les petits-fils de France. La duchesse du Lude, qui craignait tout
le monde, éblouie du grand pied sur lequel il s'était mis, eut la
faiblesse d'y consentir. Il fallut pourtant le dire à
M\textsuperscript{me} la duchesse de Bourgogne, à qui cela parut fort
sauvage, et qui le dit à Mgr le duc de Bourgogne. Ce prince le trouva
fort mauvais. Voilà la duchesse de Lude dans un étrange embarras.
L'affaire était engagée au lendemain, elle n'y avait fait aucune
difficulté, la voilà désolée. Pour la tirer de presse, Mgr le duc de
Bourgogne consentit au tabouret pour cette fois, mais il voulut être
présent, et ne point s'asseoir lui-même. Cela s'exécuta de la sorte, au
grand soulagement de la duchesse du Lude, mais au grand dépit de
Vaudemont, qui, ayant compté sur cet artifice pour s'établir un rang
très supérieur, se vit réduit à celui de cul-de-jatte, étant assis en
présence de Mgr le duc de Bourgogne debout. Mais, de peur de récidive,
ce prince jugea à propos de conter le fait au roi et de prendre ses
ordres. En lui en rendant compte„ la chaise à dos de Marly, et d'y
parler assis à Monseigneur, et sans se lever, et à M\textsuperscript{me}
la duchesse de Bourgogne, entrèrent dans le récit, et mirent le roi en
colère et en garde. Il lava la tête à la duchesse du Lude, et défendit
que M. de Vaudemont eût un traitement différent de tous les autres
seigneurs chez M\textsuperscript{me} la duchesse de Bourgogne. Il gronda
Bloin de sa facilité sur le siège à dos rehaussé et approprié, puis
s'informa si Vaudemont était effectivement grand d'Espagne. Dès qu'il en
fut certain, et il le fut bientôt, il le fit avertir de ne prétendre
rien au delà de ce rang\,; et qu'il était fort étonné du siège à dos
qu'il avait pris à Marly, et de ce qu'il demeurait assis devant
M\textsuperscript{me} la duchesse de Bourgogne et devant Monseigneur,
encore qu'il eût la bonté de le lui commander.

Vaudemont avala cet amer calice sans faire semblant de rien, et s'en
alla à Commercy. Revenu à Marly, le salon fut surpris de l'y voir en sa
même place, mais sur un tabouret dont les pieds étaient rehaussés, et de
ce qu'il se levait dès que Monseigneur passait, même à sa portée, ou
Mgrs ses fils et M\textsuperscript{me} la duchesse de Bourgogne. Il
affecta même de leur aller parler au jeu, et d'y demeurer debout quelque
temps avant de revenir à son coin sur son tabouret. Il jugea à propos de
ne demander rien, de ployer sur tout, et se nourrit cependant de
l'espérance de revenir avec avantage à ceux qu'il s'était proposés,
quand ce qu'il se ménageait en Lorraine lui aurait pleinement réussi.

Je me suis étendu sur les manèges et les entreprises adroites du prince
de Vaudemont, parce que toute la cour en a été témoin, et souvent
sottement complice, parce qu'elles se sont passées sous mes yeux, qui
les ont attentivement suivies\,; et beaucoup plus encore pour rappeler,
par ce que chacun y a vu, la manière dont les rangs de princes étrangers
se sont établis en France, sans autre titre que de savoir tirer sur le
temps, et tourner en droit ce qu'ils ont d'abord introduit peu à peu
dans les ténèbres avec adresse, et de monter ainsi par échelons. Il faut
achever de suite ceux dont Vaudemont s'échafauda, pour voir le tout
d'une même vue et n'avoir plus à y revenir. Ce récit ne préviendra son
temps que de peu de mois.

Il fallut à Vaudemont tout le reste de cette année pour arriver au but
qu'il s'était proposé, et ce fut au commencement de janvier 1708 qu'il y
parvint. Il coula toute cette année 1707 comme il put sur ses
prétentions. Comme elles n'avaient pas réussi, il laissa entendre qu'il
ne songeait à déplaire à personne, qu'il était grand d'Espagne\,; et il
en prit comme eux le manteau ducal partout à ses armes, qui n'avaient
aucune marque de bâtardise, et coulant avec adresse, sans s'expliquer
s'il se contentait de ce rang, il ajoutait que, comblé des bontés du
roi, il ne cherchait qu'à les mériter, et à s'attirer la bienveillance
et la considération de tout le monde. Il ne fit guère que des
apparitions à Marly depuis la soustraction de sa chaise à dos et ses
autres mécomptes\,; il fit l'impotent plus que jamais, pour éviter
d'aller nulle part, et surtout aux lieux de respect, excepté sur ce
tabouret dans le salon de Marly, et y voir le roi sur ses pieds un peu à
son lever, qui ne le renvoyait jamais s'asseoir, mais qui lui parlait
toujours avec distinction, et le voir passer pour aller et venir de la
messe et de la promenade. Il fit de fréquents voyages à Commercy, sous
prétexte de sa femme et de son établissement en ce pays-là, d'y bâtir,
d'y percer la forêt pour la chasse en calèche, et avoir là-dessus de
quoi entretenir le roi et fournir à la conversation\,; mais, au fond, il
alla souvent à Lunéville, et couvrait cette assiduité de bienséance, qui
en effet n'était que pour ses desseins.

Y étant au commencement de janvier 1708, tout à coup il y fut déclaré
souverain de Commercy par le duc de Lorraine, du consentement du roi, et
de toutes les dépendances de cette seigneurie, sans que l'évêque de
Metz, qui en avait la directe et la suzeraineté, y fût appelé et y
entrât pour rien, réversible, après la mort de M, de Vaudemont et de sa
femme, au duc de Lorraine et aux ducs de Lorraine ses successeurs, en
même et pleine souveraineté. Incontinent après, M. de Vaudemont abdiqua
les chimères de prétention à la souveraineté de la Lorraine, dont
autrefois il avait tenté d'éblouir aux Pays-Bas sur ce beau mariage de
sa mère\,; et le duc de Lorraine, je ne sais, non pas sur quel
fondement, mais sur quelle apparence, le déclara l'aîné, après ses
enfants et leur postérité, de la maison de Lorraine, lui donna le rang
immédiatement après ses enfants et les leurs, et au-dessus du duc
d'Elboeuf et de tous les princes de la maison de Lorraine. Avec cet
avantage et cette souveraineté, M. de Vaudemont, si bien étayé en
France, ne douta plus du succès de tout ce qu'il s'était proposé, et
que, y précédant désormais la maison de Lorraine sans difficulté\,; il
n'en trouverait plus, et par ce droit et par sa souveraineté, à
atteindre au rang le plus grandement distingué. Son affaire faite en
Lorraine, il y précéda le prince Camille, fils de M. le Grand, qui s'y
était établi depuis quelques années avec une grosse pension de M. de
Lorraine\,; et dès qu'il eut ainsi pris possession de ce rang, il
accourut en France pour y en brusquer les fruits avant qu'on eût le
temps de se reconnaître.

Cette double élévation, si peu attendue du gros du monde, fit à la cour
toute l'impression qu'il s'en était proposée, avec un grand bruit, et,
parmi les gens sensés, une grande surprise et beaucoup au delà. En
effet, il n'y a qu'à voir ce qui vient d'être expliqué de la naissance
de M. de Vaudemont d'une part, et de la consistance de la seigneurie de
Commercy de l'autre, pour ne pouvoir, comprendre ni la souveraineté ni
le premier rang dans la maison de Lorraine. Un seul aussi de cette
maison le fit échouer sur l'un et l'autre point.

Le grand écuyer en furie, et accoutumé à tout emporter du roi d'assaut,
alla lui représenter l'injustice que M. de Lorraine leur faisait, lui
dit qu'ils venaient tous de lui en écrire, et ajouta, avec force cris et
force flatteries sur la différence du roi au duc de Lorraine, qu'il
comptait bien que son équité et son autorité ne se soumettraient pas aux
nouvelles lois qu'il plaisait à ce dernier de faire, et qu'il ne se
figurerait jamais que, par complaisance pour M. de Lorraine et pour M.
de Vaudemont, il voulût leur plonger à tous le poignard dans le sein.
Avec cette véhémence, le droit, la raison, la faveur personnelle, M. le
Grand tira parole du roi que ni la souveraineté nouvelle, ni le rang
nouveau que M. de Lorraine venait de donner à M. de Vaudemont, ne
changeraient rien ici au leur ni à son état. M. de Lorraine tint ferme,
dans sa réponse aux princes de sa maison, à ce qu'il avait décidé. Eux
triomphèrent, M. le Grand surtout de ce qu'il avait obtenu du roi, et M.
de Vaudemont fut arrêté tout court dès son arrivée. M. de Lorraine avait
écrit au roi qu'il avait donné à Vaudemont le premier rang dans sa
maison, et la préséance sur tous. Le roi lui répondit qu'il était le
maître de régler chez lui tout ce qui lui plaisait. Il ne lui en dit pas
davantage, mais, en même temps, il fit bien entendre à Vaudemont que, ni
sa nouvelle qualité de souverain, ni sa nouvelle préséance sur la maison
de Lorraine, ne changerait rien à sa cour, où il avait le rang de grand
d'Espagne, comme il l'était, et qu'il était à propos qu'il n'imaginât
pas d'y en avoir d'autre, ni aucune préférence au delà en rien.

On peut juger de la rage, du dépit, de la honte, de la douleur de
l'oncle et des nièces d'une pareille issue de tant d'habiles
excogitations, et de tant de soins, de peines et de menées pour parvenir
à ce qui venait de s'exécuter. Mais l'art surpassa la nature. Ils
comprirent tout d'un coup que le mal était sans remède\,; ils en
avalèrent le calice tout d'un trait, et ils eurent assez de sens rassis
pour comprendre qu'il ne restait plus que la faveur et la considération
première à sauver\,; que paraître piqué, mécontent, prétendant, ce
serait en vain montrer sa faiblesse, avec sûreté, non seulement de ne
pas réussir, mais encore de déplaire et de se livrer à découvert à
beaucoup de choses fâcheuses, dès que les bouches, que leur faveur avait
tenues closes, oseraient s'ouvrir\,; que d'une conduite contraire et
soumise, ils tireraient un gré infini d'un roi qui se plaisait à se
faire obéir sans réplique, et point du tout à être tracassé,
conséquemment une continuation pour le moins du même brillant et de la
même considération.

Pour cette fois ils ne se trompèrent pas. M. de Vaudemont s'ôta enfin
tout à coup toutes chimères de la tête\,; ses jambes en même temps
s'affermirent\,; il vit le roi plus assidûment et plus longuement aux
heures de cour\,; il {[}y{]} alla d'ailleurs un peu davantage. Le roi,
content d'une conduite qui l'affranchissait d'importunités, redoubla
pour lui d'égards et d'attentions, mais de celles qui, sur les
prétentions possibles, ne pouvaient pas être douteuses, et qui les
exclurent toujours\,; et le monde fut étonné de voir presque tout à coup
un cul-de-jatte ingambe, et marchant au moins à peu près comme un autre,
et sans se faire appuyer ni porter. Je vis cela avec plaisir, et ne me
contraignis pas d'en rire.

Mais tout cela ne put apaiser les Lorrains, qui rompirent ouvertement
avec lui, et qui tous, excepté sa soeur, ses nièces et la duchesse
d'Elboeuf, sa belle-mère, c'est-à-dire de sa femme\,; et qui demeura
neutre, cessèrent tous de le voir et ne l'ont jamais revu depuis. Ses
nièces en demeurèrent brouillées avec eux tous, et M. le Grand ne cessa
de jeter feu et flammes.

L'affront qu'il prétendait que son fils avait reçu en Lorraine, par la
préséance de Vaudemont qu'il y avait essuyée, l'outrait d'autant plus
que, brouillé lui-même avec M. de Lorraine, par la hauteur avec laquelle
il avait arrêté ici tout court les prétentions de Vaudemont, et dont il
s'était élevé contre sa préséance sur eux, il lui devenait fort
embarrassant de laisser son fils à la petite cour de M. de Lorraine, et
encore plus amer de lui faire perdre quarante mille livres de rente
qu'il en recevait, en le faisant revenir, et rie voulant pas l'en
dédommager. Après bien des fougues, M\textsuperscript{me} d'Armagnac,
bien moins indifférente que lui à se soulager du prince Camille aux
dépens d'autrui, fit en sorte qu'il demeurât en Lorraine, mais avec le
dégoût d'en disparaître toutes les fois que Vaudemont y venait, et ce
dernier y allait de tous ses voyages de Commercy, ce qui arrivait
plusieurs fois l'année. Néanmoins cela subsista toujours depuis ainsi\,;
et Camille, qui n'était ni aimable ni aimé en Lorraine, y fut sur le
pied gauche plus que jamais le reste de sa vie.

Qui que ce soit de sens et de raisonnant à la cour n'avait pu goûter la
solide et brillante figure que Vaudemont y fit par les grâces
pécuniaires et par les distinctions de considération\,; mais les
Espagnols surtout, et ce qui avait servi dans leurs troupes en Italie,
en étaient indignés. Le duc d'Albe, moins que personne, ne pouvait
comprendre comment ce citoyen de l'univers, affranchi des Hollandais,
confident du roi Guillaume, créature de la maison d'Autriche, serviteur
si attaché et si employé toute sa vie de tous les ennemis personnels du
roi et de la France, et qui les avait peut-être plus utilement servis
depuis que la conservation des grands emplois qu'il leur devait l'avait
fait changer extérieurement de parti, comment, dis-je, ce Protée pouvait
avoir enchanté si complètement le roi et tout ce qui avait le plus
d'accès auprès de lui en tout genre. Ce scandale ne trompait pas le duc
d'Albe, ni ceux qui pensaient comme lui.

Vaudemont, comblé au point qu'on vient de voir, et avec un intérêt si
capital de conserver tout ce qu'il venait d'obtenir et d'entretenir
cette considération éclatante, ne put commencer enfin à devenir fidèle.
Le succès de ses artifices lui donna la confiance de les continuer\,;
tout ce qu'il vit et reçut de notre cour ne put le réconcilier avec
elle, et ne servit qu'à la lui faire mépriser. Il y resserra de plus en
plus ses anciennes et intimes liaisons avec ses ennemis, et logé dans
Paris au temple de la haine contre les Bourbons, avec des Lorraines si
dignes des Guise, lui si digne aussi du trop fameux abbé de
Saint-Nicaise dom Claude de Guise, ils y passaient leur vie en
trahisons. Barrois, depuis le rétablissement du duc de Lorraine, son
envoyé ici, logeait avec eux. C'était un homme d'esprit, de tête et
d'intrigue, qui se fourrait beaucoup, et qui avait l'art de se faire
considérer. Tout ce qu'ils pouvaient découvrir de plus secret sur les
affaires, et soit par la confiance qu'on avait prise en Vaudemont, soit
par l'adresse qu'il avait, lui, ses nièces et Barrois, par diverses
voies, de savoir beaucoup de choses importantes, ils en étaient fort
bien informés\,; ils les mandaient au duc de Lorraine, et ce qui était
trop important pour le confier au papier se disait à Lunéville dans
leurs courts et fréquents voyages, sans toutefois que Barrois bougeât
jamais de Paris ou de la cour, tant pour demeurer au fil des affaires
que pour paraître ne se mêler de rien, et ne donner aucun soupçon par
ses absences. De Lunéville, les courriers portaient cet avis à Vienne.
Le ministre que l'empereur tenait auprès du duc de Lorraine entrait avec
eux dans ce conseil, qu'ils tenaient sur la manière de profiter de leurs
découvertes, et de la conduite à tenir pour y mieux réussir.

Je sus cette dangereuse menée par un ecclésiastique de l'église
d'Osnabrück, domestique de l'évêque frère de M. de Lorraine, et chargé
de ses affaires à Lunéville et à Paris. C'était un homme léger et
imprudent, qui allait, quand il en avait le temps, passer quelques jours
en Beauce, c'est-à-dire un peu au delà d'Étampes, chez un voisin de
Louville, et son ami particulier. Là, il fit connaissance avec
Louville\,; ils se plurent, ils se convinrent l'un à l'autre, et tant et
si bien que cet ecclésiastique lui conta ce que je viens de rapporter.
Il ajouta que M. de Lorraine faisait sous main des amas de blé et de
toutes choses\,; entretenait, sans qu'il y parût, un grand nombre
d'officiers dans son petit État, pour être tout prêt à lever, au premier
ordre, des troupes qui se trouveraient en un instant sur pied, sitôt que
les conjonctures le pourraient permettre. On verra parmi les Pièces,
dans la négociation de M. de Torcy, quelles furent les prétentions de ce
duc de Lorraine, et avec quelle ténacité elles furent soutenues par tous
les alliés, la dissimulation et les artifices de ce prince, jusqu'à ce
qu'il vît jour au succès par la décadence où les malheurs de la guerre à
voient jeté la France, et jusqu'à quel excès et sous quel odieux
prétexte il porta et fit appuyer ses demandes.

Telle est la reconnaissance de la maison de Lorraine, si grandement et
depuis si longtemps établie en France, vivant à ses dépens\,; tels sont
ces louveteaux que le cardinal d'Ossat a dépeints si au naturel dans ses
admirables lettres\,; tel est le peu de profit que nos rois ont tiré de
la prophétie de François l-, en mourant, à Henri II, son fils, que s'il
n'abaissait la maison de Guise, qu'il avait trop élevée, elle le
mettrait en pourpoint et ses enfants en chemise. À quoi a-t-il tenu
qu'elle n'ait été vérifiée à la lettre, et que n'ont-ils pas fait
depuis, tant et toutes les fois qu'ils l'ont pu, sans que nos rois aient
jamais voulu ouvrir les yeux sur leur conduite, leur esprit, leur coeur,
leur voeu le plus exquis (et des rois prodigues envers eux de toutes
sortes de biens, de rangs, de charges, de gouvernements principaux et
d'établissements de toutes les sortes)\,? N'est-ce point là être frappé
du plus prodigieux aveuglement.

\hypertarget{chapitre-ii.}{%
\chapter{CHAPITRE II.}\label{chapitre-ii.}}

1707

~

{\textsc{Procès de M\textsuperscript{me} de Lussan, qui me brouille
publiquement avec M. le Duc et M\textsuperscript{me} la Duchesse.}}
{\textsc{- Fortune, mérite, mort du maréchal d'Estrées.}} {\textsc{-
Vues terribles de Louvois.}} {\textsc{- Mort de la marquise de La
Vallière.}} {\textsc{- Mort de M\textsuperscript{me} de Montespan.}}
{\textsc{- Sa retraite et sa conduite depuis.}} {\textsc{- Son
caractère.}} {\textsc{- Politique des Noailles.}} {\textsc{- Sentiments
sur la mort de M\textsuperscript{me} de Montespan des personnes
intéressées.}} {\textsc{- Caractère et conduite de d'Antin.}} {\textsc{-
Avarice de d'Antin.}} {\textsc{- Il supprime le testament de
M\textsuperscript{me} de Montespan.}}

~

Il m'arriva au printemps de cette année une affaire qui fit un grand
éclat dans l'été. J'en supprimerais ici l'ennui inséparable de ce
détail, si les suites de cette affaire dans le cours de ma vie ne m'y
engageaient pas, nécessairement, par l'influence qu'elles ont eue sur de
plus importantes que les miennes.

Pour entrer dans cette explication, il faut se souvenir que le dernier
connétable de Montmorency avait épousé en secondes noces une Budos,
soeur du marquis de Portes, tué au siège de Privas en 1629, étant
chevalier de l'ordre de 1619, et vice-amiral, près d'être fait maréchal
de France et surintendant des finances. Cette Budos eut le dernier duc
de Montmorency, qui eut la tête coupée en 1632, et M\textsuperscript{me}
la Princesse, mère de M. le Prince le héros, de M. le prince de Conti et
de M\textsuperscript{me} de Longueville. Le marquis de Portes laissa de
la soeur du duc d'Uzès deux filles et point de garçons, lesquelles par
conséquent étaient cousines germaines de M\textsuperscript{me} la
Princesse. Mon père, en premières noces, épousa la cadette des deux,
belle et vertueuse, et ne voulut point de l'aînée pour sa laideur et sa
mauvaise humeur, qui était aussi fort méchante et qui ne le lui pardonna
jamais. De ce premier mariage de mon père, il ne vint (qui ait vécu)
qu'une fille mariée au duc de Brissac, frère de la dernière maréchale de
Villeroy, qui, étant morte sans enfants, me fit son légataire universel.
Sa mère et sa tante ne liquidèrent jamais leurs partages. L'aînée, fort
impérieuse, appuyée de sa mère remariée au frère aîné de mon père, qui
n'a point eu d'enfants, menaçait sans cesse sa soeur d'un testament
bizarre\,; et dans l'espérance de sa succession, parce qu'elle avait
renoncé au mariage, se fit donner en usufruit force choses très
injustement. Cette première duchesse de Saint-Simon mourut jeune\,;
M\textsuperscript{lle} de Portes, fort vieille, grand nombre d'années
après.

Elle fit un testament ridicule, par lequel elle donna beaucoup plus
qu'elle n'avait, et ses terres de Languedoc à M. le prince de Conti,
avec la folle condition que les sceaux, les titres, les bandoulières des
gardes de ces terres, et partout où il y aurait des armoiries, elles
seraient mi-parties en même écu de Bourbon et de Budos.

La succession fut longtemps vacante. J'étais privilégié sur ses biens
pour mes créances\,; je les demandai. Elles étaient si claires qu'aucun
parent ne se présenta pour me les contester, jusqu'à ce que
M\textsuperscript{me} de Lussan s'avisa de prétendre que ce que je
demandais comme faisant partie du legs de ma soeur était un propre en sa
personne, non un acquêt, et pareillement en celle de
M\textsuperscript{lle} de Portes, dont ni l'une ni l'autre n'avaient pu
disposer que d'un quint\footnote{Cinquième.}\,; que les quatre autres
{[}parts{]} appartenaient aux héritiers de M\textsuperscript{lle} de
Portes, morte longtemps après sa soeur et sa nièce\,; et que les
héritiers ayant renoncé à la succession, elle se portait pour héritière.
Jamais il ne nous vint dans l'esprit que cette femme n'eût pas de
qualité pour cela, et nous ne pensâmes qu'à soutenir le droit de la
nature de la rente. Les tribunaux étaient partagés sur la question et la
jugeaient différemment, mais ce que je soutenais était le droit, le plus
communément celui en faveur duquel le plus ordinaire était de prononcer.

Dans ce point de l'affaire, Harlay qui était encore en place de premier
président, et qui n'ignorait pas que cette affaire se poursuivait à la
grand'chambre où il voyait que j'allais la gagner, proposa à cette
occasion une déclaration qui réglât la question, et qui en rendît
partout le jugement uniforme. Il ne put s'empêcher de proposer en même
temps qu'elle ne la décidât en faveur de ce que je soutenais\,; mais
comme il voulait que je perdisse ma cause, il y inséra adroitement une
clause particulière, faite pour moi tout seul et qui rien pouvait
regarder, d'autres, par laquelle, dans l'espèce dont il s'agissait entre
M\textsuperscript{me} de Lussan et moi, mon procès était perdu. Tout
cela se fit si, brusquement et tellement sous la cheminée que je ne pus
être averti à temps\,; tout était fait quand j'en parlai au chancelier
qui, tout mon ami qu'il était, n'y voulut rien entendre, pour n'avoir
pas, à y retoucher et à disputer contre le premier président, plus
profond que lui et avec lequel tout était convenu. Cette déclaration,
avec sa maligne clause, proposée, dressée, enregistrée, ne fut donc
presque que la même chose, après quoi je n'eus plus qu'à m'avouer
vaincu.

La déclaration ne fut pas plutôt publique qu'elle réveilla d'autres
parents à M\textsuperscript{lle} de Portes, qui, n'ayant point renoncé à
sa succession, se portèrent pour héritiers, et dirent juridiquement à
M\textsuperscript{me} de Lussan le \emph{sic vos non vobis} de Virgile.
M\textsuperscript{me} de Lussan en fut outrée et pour l'honneur et pour
le profit. Elle se voyait enlever le fruit de ses travaux, et réduite,
de plus, à prouver une parenté qui emportait nécessairement celle de M.
le Prince, dont elle s'était toujours piquée et prévalue et qu'elle
savait bien n'exister point. C'était donc là un étrange affront.

Son mari était un fort galant homme à M. le Prince père et fils, de tout
temps, qu'une très belle action fit chevalier de l'ordre, que j'ai
racontée ici quelque part, mais alors fort vieux et sourd, qu'on ne
voyait plus et qui laissait tout faire à sa femme.

C'était une grande créature de peu de chose, dont le nom était Raimond,
souple, fine, hardie, audacieuse, entreprenante, et d'une intrigue de
toutes les façons, qui avait tiré tous les meilleurs partis de l'hôtel
de Condé, et qui avait si bien courtisé M\textsuperscript{me} du Maine
qu'elle avait marié sa fille unique au duc d'Albermale, second bâtard du
roi Jacques II, et qui ne bougeait de Sceaux. Elle passait pour riche,
et il se trouva qu'ils n'avaient rien. Elle hasarda sous cette
protection des manières de princesse du sang, dont le duc de Berwick ne
lui avait pas donné l'exemple, et qui aussi ne durèrent pas longtemps.
Elle devint bientôt veuve et sans enfants, et se remaria depuis à
Mahoni, lieutenant général irlandais, qui se signala tant à la surprise
et reprise de Crémone, où j'en ai parlé. Le mariage fut tenu secret pour
conserver son nom et son rang de duchesse\,; et a vécu et est morte il
n'y a pas longtemps dans une grande indigence et dans la plus profonde
obscurité.

Pour en revenir à l'affaire, le bisaïeul de M. de Lussan avait épousé
une Budos en 1558, et MM. de Disimieu, gens de qualité de Dauphine,
étaient fils d'une soeur de la Budos, femme du dernier connétable de
Montmorency, et du marquis de Portes, beau-père de mon père, par
conséquent, comme la première duchesse de Saint-Simon, cousins germains
de la mère de M. le Prince le héros. C'était bien là une parenté réelle
et proche, et non pas celle de Lussan. Ce fut aussi ce cruel soubresaut
qui fit toute l'aigreur de l'affaire. L'aîné de ces deux Disimieu
n'avait laissé qu'une fille, qui fut la comtesse de Verue, mère du comte
de Verue tué à Hochstedt, dont la femme, fille du duc de Luynes, lui fut
enlevée par le duc de Savoie, ainsi que je l'ai rapporté ailleurs, dont
elle a eu M\textsuperscript{me} de Carignan et d'autres enfants. Le
cadet Disimieu avait eu l'abbaye de Saint-Aphrodise de Béziers, sans
avoir jamais pris aucuns ordres. Il fut longtemps en commerce avec la
fille d'un mestre de camp de cavalerie, du nom de Saline, noblement
établi depuis plus de trois cents ans en Dauphiné. Il en eut plusieurs
enfants, l'épousa ensuite en mettant les enfants sous le poêle, et cela
publiquement, en présence des deux parentés, et ont toujours depuis bien
vécu ensemble. Par les lois, ces enfants devinrent légitimes, et jusqu'à
M\textsuperscript{me} de Lussan personne ne s'était avisé de le leur
contester.

L'aîné de ces enfants, muni des pouvoirs et du désistement de
M\textsuperscript{me} de Verue et des siens en sa faveur, fut celui qui
se présenta contre M\textsuperscript{me} de Lussan, et qui, ne
connaissant personne à Paris, s'adressa à nous pour avoir protection
contre les chicanes et le crédit de cette femme. Elle l'attaqua sur sa
naissance, elle se porta à des inscriptions en faux honteuses, et perdit
son procès à la grand'chambre avec infamie. Ce qui l'irrita le plus, fut
que Disimieu lui contesta sa parenté. Il n'y eut détours ni tours de
passe-passe qu'elle ne mît en usage pour éluder et faire perdre terre à
un provincial inconnu et peu pécunieux, et cela seul montrait la corde.
À la fin, pourtant, il fallut prouver. Alors, elle ne put apporter que
des extraits mortuaires, des extraits baptismaux, des contrats de
mariage, par lesquels elle montra bien l'alliance du bisaïeul de son
mari, que j'ai expliquée ci-dessus, mais qui ne prouvaient aucuns
enfants de mariage\,; et comme ce bisaïeul se remaria en secondes noces,
et que les extraits baptismaux et mortuaires des enfants se trouvèrent
exprimant uniquement le nom du père et point celui de la mère, et que
M\textsuperscript{me} de Lussan n'apporta point de contrat de mariage
d'eux, cette affectation fit justement conclure que ces enfants étaient
de la seconde femme et point de la Budos, ce qui faisait tomber tout
droit à rien prétendre aux biens de M\textsuperscript{lle} de Portes et
à toute parenté avec M. le Prince. Outrée de rage, et n'ayant de
ressource qu'à faire perdre terre à Disimieu, elle l'accabla des plus
atroces chicanes, jusqu'à s'inscrire en faux contre l'arrêt qu'il avait
obtenu contre elle à la grand'chambre\,; et, après qu'elle y eut
honteusement succombé, elle se pourvut au conseil en cassation.

Jusque-là tout s'était passé en procès ordinaire. Toute la maison de
Condé avait sollicité publiquement pour M\textsuperscript{me} de Lussan
sur sa périlleuse parole, et moi contre elle, sans que cela eût été plus
loin\,; et c'est pour ce qui va suivre que j'ai été obligé de faire cet
ennuyeux narré. L'affaire s'instruisit au conseil, tandis qu'en même
temps M\textsuperscript{me} de Lussan présenta au parlement une requête
civile, pour n'omettre rien d'étrange, dont elle fut aussitôt déboutée.

Cependant je fus averti de toutes parts que cette femme se déchaînait
contre moi, disait partout que, de dépit d'avoir perdu un procès contre
elle, je lui suscitais le fils d'un moine et d'une servante pour la
tourmenter, et cent autres impertinences que M\textsuperscript{me} la
Princesse et M\textsuperscript{me} la Duchesse voulurent bien croire, ou
en faire le semblant, et répétèrent à demi d'après elle, en sorte que
cela commençait à faire grand bruit. Je ne crus pas devoir m'en tenir
aux démentis avec elle. Je fis donc un mémoire fort court, qui exposait
nettement les faits, la supposition de la parenté, les infâmes chicanes,
et qui, sans ménagement aucun, peignit au naturel cette ardente et
méchante créature. Tout y était si clairement prouvé, qu'il n'y avait
point de réponse possible à y faire.

Avant que de le répandre, je demandai un quart d'heure à M. le Prince.
Je lui expliquai les faits, je lui lus mon mémoire, je lui dis que je ne
pouvais me justifier des mensonges qu'il plaisait à
M\textsuperscript{me} de Lussan de débiter contre moi qu'en prouvant ses
artifices et ses friponneries, et les mettant au net et au jour\,;
j'ajoutai que M. et M\textsuperscript{me} de Lussan ayant l'honneur
d'être à lui et à M\textsuperscript{me} la Princesse, je ne le voulais
pas publier sans lui en demander la permission. M. le Prince glissa sur
M\textsuperscript{me} de Lussan, me répondit qu'il était très fâché
qu'elle se fût attiré une si vive repartie\,; que, si l'affaire était de
nature à pouvoir s'accommoder, il s'y offrirait à moi\,; que, voyant la
chose impossible, j'étais le maître de publier mon mémoire, et qu'il
m'était fort obligé de l'honnêteté que je lui témoignais en cette
occasion. Il m'en fit extrêmement dans toute cette visite, de laquelle
je sortis fort content.

J'allai plusieurs fois chez M. le Duc pour en faire autant à son égard,
et, ne le pouvant rencontrer chez lui ni ailleurs, je priai le duc de
Coislin, son ami particulier, de le lui dire et de lui donner mon
mémoire. Je le portai à Paris à M\textsuperscript{me} la Princesse, qui
me reçut poliment, mais froidement, et qui s'excusa de l'entendre. Je
crus devoir faire la même chose à l'égard de M. le duc du Maine, à cause
de ce que j'ai expliqué du mariage de M\textsuperscript{me} d'Albemarle,
et par cette raison à l'égard de la reine d'Angleterre, qui me reçut le
mieux du monde, et M. du Maine plus poliment encore, s'il se peut, que
n'avait fait M. le Prince. Pour M\textsuperscript{me} la Duchesse, je la
crus trop prévenue pour aller chez elle\,; je lui fis dire que c'était
par ménagement, en lui faisant donner mon mémoire. Content de ces
mesures, je le publiai, j'en donnai à tout le monde, et je l'accompagnai
de tous les propos que M\textsuperscript{me} de Lussan méritait. Je fus
fort appuyé de beaucoup d'amis qui y firent dignement leur devoir. Ainsi
l'éclat fut grand.

M. le Duc poussé par M\textsuperscript{me} la Princesse,
M\textsuperscript{me} la Duchesse, je crois par d'Antin, qui n'avait pu
me pardonner la préférence sur lui de l'ambassade de Rome, quoique je
n'y eusse eu aucune part et qu'elle n'eût point eu d'effet, ne se
laissèrent persuader, ni par mes raisons, ni par mes honnêtetés pour
eux, ni par l'exemple de M. le Prince, qui n'ouvrit jamais la bouche ni
pour ni contre\,; ils éclatèrent en propos. M\textsuperscript{me} la
Duchesse même les voulut entamer par deux fois les soirs dans le cabinet
du roi, et toutes les deux fois elle fut arrêtée tout court par
M\textsuperscript{me} la duchesse d'Orléans qui prit mon parti sans que
je l'eusse fait prévenir. Une autre fois, et au même lieu, elle attaqua
là-dessus M. du Maine, duquel elle n'eut pas lieu d'être contente,
quoique alors en intimité\,; et en effet, lui et M\textsuperscript{me}
du Maine imitèrent le silence de M. le Prince. Cette fougue m'engagea à
prendre des mesures auprès des gens de mes amis à portée de faire
instruire le roi et M\textsuperscript{me} de Maintenon, et Monseigneur
avec qui M\textsuperscript{me} la Duchesse était parfaitement.

L'affaire, en attendant, cheminait au conseil. M\textsuperscript{me} de
Lussan voulut répondre vivement, sinon solidement, à mon mémoire. M. le
Prince, sans que je le susse, le lui défendit, et de plus lui lava
cruellement la tête. Elle se réduisit donc à faire courir quelques
lignes écrites à la main qui, sans entrer dans l'affaire ni dans aucun
fait, exprimaient en termes respectueux, mais artificieux, la surprise
et la douleur de se voir si cruellement déchirée par un homme de mon
mérite et avec si peu de mesure, dans un temps (c'était celui de Pâques)
que j'avais accoutumé de consacrer tous les ans dans la plus sainte
maison de France. Elle voulait dire la Trappe, dont je me cachais fort,
et où je passais d'ordinaire les jours saints, sous prétexte d'aller à
la Ferté pendant la quinzaine de Pâques, qui est un temps fort ordinaire
d'aller à la campagne.

J'eus lieu de soupçonner que M. le Duc n'avait pas dédaigné de
travailler à ce peu de lignes, et que c'était de lui que partait ce
ridicule qu'on essayait de m'y donner. Je pris donc le parti de le
mépriser. Je me contentai de dire qu'une vaine déclamation, qui n'osait
entrer en rien, n'était pas une réponse à un mémoire où la conduite de
M\textsuperscript{me} de Lussan, et beaucoup plus les discours des
personnes dont elle avait surpris la protection, m'avait obligé
d'expliquer des faits fâcheux, et de mettre au net beaucoup de choses
honteuses, à quoi il fallait manquer bien absolument de réponses pour
n'avoir de ressource qu'en de si misérables pauvretés. Néanmoins, je
voulus instruire Mgr le duc de Bourgogne, duquel j'eus une très
favorable audience dans son cabinet, et à qui je lus mon mémoire.
M\textsuperscript{me} la duchesse de Bourgogne la fut aussi, et s'en
expliqua comme je le pouvais désirer.

Enfin le procès, tant et plus allongé, prit fin au conseil. Tous les
juges, sans exception, n'y opinèrent que par des huées et des cris
d'indignation, et, ce qui est rare au conseil, M\textsuperscript{me} de
Lussan y eut la honte des dépens, de l'amende et de tous les plus
injurieux assaisonnements.

Cette femme en attendait l'événement chez M\textsuperscript{me} la
Duchesse. Les filles de Chamillart étaient en ce temps-là la fleur des
pois, et ne bougeaient de chez M\textsuperscript{me} la duchesse de
Bourgogne et de chez M\textsuperscript{me} la Duchesse. Ma belle-soeur
s'y trouva en ce même moment. On vint la demander, c'était son écuyer
qu'elle avait envoyé à la porte du conseil attendre, et qui accourait
lui apprendre le jugement. Elle rentra en sautant et riant, et,
s'adressant à M\textsuperscript{me} la Duchesse, lui dit ce qui venait
d'être décidé, en présence de M\textsuperscript{me} de Lussan et de la
compagnie. M\textsuperscript{me} la Duchesse en fut si piquée, qu'elle
lui répondit qu'elle se passerait bien de marquer tant de joie chez
elle. La duchesse de Lorges répliqua qu'elle était ravie, et, avec une
pirouette, ajouta qu'elle ne la reverrait que quand elle serait de plus
belle humeur, et s'en vint me le conter. M\textsuperscript{me} la
Duchesse la bouda vingt-quatre heures et fut la première à se vouloir
raccommoder.

Ce jugement fit grand bruit, mais il ne put dégoûter
M\textsuperscript{me} de Lussan de ses chicanes. Elle présenta au
parlement une seconde requête civile. Je ne continuerai pas le récit
d'une affaire si criante et si infâme, dont elle ne put jamais venir à
bout. Je ne l'ai rapportée que pour expliquer ce qui me brouilla avec M.
le Duc et M\textsuperscript{me} la Duchesse.

Après ce qui s'était passé, nous ne crûmes pas devoir rien rendre
davantage à l'un ni à l'autre, et nous cessâmes de les voir l'un et
l'autre, même aux occasions marquées. M\textsuperscript{me} la Duchesse,
qui s'en aperçut bientôt, se plaignit modestement. Elle dit qu'elle ne
savait ce qu'elle nous avait fait\,; qu'il était vrai qu'elle avait été
pour M\textsuperscript{me} de Lussan, que cela était libre, qu'elle
n'avait rien dit là-dessus qui pût nous faire peine\,; que d'ailleurs
M\textsuperscript{me} de Lussan était à M\textsuperscript{me} la
Princesse, et qu'elle lui avait des obligations qu'elle n'oublierait
jamais. Je ne sais pas de quelle nature elles pouvaient être, ni si
elles faisaient beaucoup d'honneur à l'une et à l'autre. Ces plaintes se
firent en sorte qu'elles nous revinssent. M\textsuperscript{me} la
Duchesse y ajouta toutes les prévenances possibles à Marly à
M\textsuperscript{me} de Saint-Simon, qui les reçut avec un froid
respectueux, des réponses courtes, sans jamais lui parler la première ni
s'approcher d'elle, sinon à la table du roi, quand elle s'y trouvait
placée auprès d'elle. Elle redoubla ses plaintes à Fontainebleau, sur ce
qu'étant entré chez M\textsuperscript{me} de Blansac, qui était malade,
j'en sortis aussitôt\,; et fit indirectement tout ce qu'elle put pour
raccommoder les choses. Ce n'était pas qu'elle se souciât de nous, mais
ces princesses voudraient dire et faire sur chacun tout ce qui leur
plaît, et leur orgueil est blessé quand on cesse de les voir. Pour M. le
Duc, qui a toujours mené une vie particulière, jusqu'à l'obscurité, et
qu'une férocité naturelle, que son rang appesantissait encore,
renfermait dans un très petit nombre de gens assez étranges pour la
plupart, je n'en reçus ni malhonnêtetés ni agaceries\,; il me salua
seulement lorsqu'il me rencontra depuis d'une façon plus marquée et plus
polie. À l'égard de M. le prince de Conti que je voyais, il ne fallut
aucune précaution avec lui. Il connaissait la pèlerine et ne se
contraignit pas d'en dire son avis. Je le répète, on trouvera dans la
suite qu'il était nécessaire d'expliquer toute cette espèce de démêlé.

Le maréchal d'Estrées mourut au mois de mai, à Paris, à
quatre-vingt-trois ans passés, doyen des maréchaux de France, comme son
père et son fils, singularité sans exemple, et de trois générations de
suite maréchaux de France, et toutes trois doyens, et toutes trois
dignes du bâton, toutes trois aussi chevaliers de l'ordre. Celui-ci
jouissait, depuis près de quatre ans, de la joie de voir son fils
maréchal de France. Il l'avait été fait seul au printemps de 1681, onze
ans après la mort de son père, avec l'applaudissement public, et son
impatience depuis longtemps de l'en voir décoré. Il était estropié d'une
main de sa première campagne, colonel d'infanterie au siège de
Gravelines en 1644. Dès 1655 il fut fait lieutenant général. Il s'était
distingué en beaucoup d'occasions à la tête du régiment de Navarre.

L'ordre du tableau était encore alors heureusement inconnu. On éprouvait
les gens qui montraient de la volonté et des talents\,; on les mettait à
portée de les employer par des commandements plus ou moins
considérables\,; on laissait ceux en qui on voyait les espérances qu'on
en avait conçues trompées, on avançait ceux qui réussissaient\,; et
quoique la faveur, la naissance, les établissements aient toujours eu
quelques droits, la réputation était pesée, le cri de l'armée, l'opinion
des troupes, le sentiment des généraux d'armée étaient écoutés, on ne
passait par-dessus que bien rarement, en bien et en mal.

M. de Louvois, dès lors méditant lé projet de se rendre le maître de la
conduite de la guerre et des fortunes, et de changer pour sa puissance
toute manière de faire l'une et l'autre, songeait aussi à se défaire des
gens qui pointaient, et dont le mérite l'eût embarrassé, comme à la
longue il en vint à bout. Il gémissait sous le poids de M. le Prince, de
M. de Turenne et de leurs élèves\,; il ne voulait plus qu'il s'en pût
faire de nouveaux\,; il en voulait tarir la source, pour que tout,
jusqu'au mérite, vînt de sa main, et que l'ignorance, parvenue de sa
grâce, ne pût se maintenir que par elle.

M. d'Estrées fut un de ceux qui l'embarrassa le plus. Lieutenant général
depuis douze ans par mérite et à force de services et d'actions à
quarante-trois ans, c'était pour arriver bientôt à l'ouverture de la
guerre en 1667. Colbert, son émule, en prit occasion d'exécuter l'utile
projet qu'il avait formé depuis longtemps de rétablir la marine. Il
l'avait dans son département de secrétaire d'État\,; il en avait les
moyens par sa place de contrôleur général des finances, dont avec
Fouquet il avait détruit la surintendance. Louvois n'en avait aucun
d'empêcher ce rétablissement dans un royaume flanqué des deux mers. Il
dégoûta d'Estrées\,; il se brouilla de propos délibéré avec lui\,; il le
réduisit à se jeter à Colbert, qui, ravi de pouvoir faire une si bonne
acquisition pour là marine qu'il s'agissait de créer plutôt que de
rétablir, le proposa au roi pour lui en donner le commandement.

Quoique ce savant métier en soit tout un autre que celui de la guerre
par terre, d'Estrées s'y montra d'abord tout aussi propre. Il fit une
campagne aux îles de l'Amérique qui y répara tout le désordre que les
Anglais y avaient fait. Il en fut fait vice-amiral. Il battit et força
les corsaires d'Alger, de Tunis et de Salé à demander la paix en 1670,
et ne cessa depuis de se distinguer à la mer par de grandes actions.

Quelque soulagé que fût Louvois de s'être défait d'un homme si capable,
il était outré de ses succès\,; il était venu à le haïr après s'être
brouillé avec lui uniquement pour s'en défaire. Sa gloire, unie à celle
de la marine, lui était odieuse\,; c'était pour lui la prospérité de
Colbert, qui effaçait à son égard celle de l'État. Colbert voulait que
la marine eût un maréchal de France, d'Estrées méritait de l'être depuis
longtemps\,; Louvois eut le crédit de l'empêcher de passer avec ceux
qu'on fit à la mort de M. de Turenne en 1675. Estrées et Colbert furent
outrés, mais ils ne se rebutèrent point, l'un de continuer à mériter par
des actions nouvelles, l'autre de représenter ses services, ses actions,
l'importance de ne pas dégoûter la marine dont on tirait tant
d'avantages, et le découragement où la jetait l'exclusion de son
général. Enfin Louvois n'eut pas le crédit de l'arrêter plus longtemps,
et en mars 1681 le roi le fit maréchal de France seul. Quelques années
après, il lui donna le vain titre de vice-roi de l'Amérique sans
fonctions et sans appointements, enfin le gouvernement de Nantes et
cette lieutenance générale de Bretagne que son fils eut à sa mort.

Le maréchal d'Estrées naquit, vécut et mourut pauvre\,; fort honnête
homme, et fort considéré, et toujours dans la plus étroite union avec
ses frères le duc et le cardinal d'Estrées. Il vit aussi son fils grand
d'Espagne, et son autre fils dans les négociations du dehors, mais sans
avoir pu, ni lui ni son frère, vaincre la répugnance que quelque
jeunesse de ce fils avait donnée au roi de le faire évêque.

Peu de jours après la mort du maréchal d'Estrées, mourut la marquise de
La Vallière, veuve du frère de la maîtresse du roi, que sa faveur avait
faite dame du palais de la reine. Son nom était Gié, et fort peu de
chose, ce qui n'était pas surprenant\,; mais une femme de beaucoup
d'esprit, gaie, extrêmement aimable, qui avait de l'intrigue et beaucoup
d'amis, et qui, par là, sut se soutenir à la cour et dans le monde avec
beaucoup de considération après la retraite de sa belle-soeur. Elle
était devenue infirme et dévote, et ne venait presque plus à la cour,
mais toujours, quand elle y paraissait, fort recherchée. Le roi, qui
s'était fort amusé de sa gaieté et de son esprit, la distinguait toutes
les fois qu'il la voyait, et conserva toujours de l'amitié pour elle.

Une autre mort fit bien plus de bruit, quoique d'une personne depuis
longtemps retirée de tout, et qui n'avait conservé aucun resté du crédit
dominant qu'elle avait si longtemps exercé. Ce fut la mort de
M\textsuperscript{me} de Montespan arrivée fort brusquement aux eaux de
Bourbon, à soixante-six ans, le vendredi 27 mai, à trois heures du
matin.

Je ne remonterai pas au delà de mon temps à parler de celui de son
règne. Je dirai seulement, parce que c'est une anecdote assez peu
connue, que ce fut la faute de son mari plus que la sienne\,; elle
l'avertit du soupçon de l'amour du roi pour elle\,; elle ne lui laissa
pas ignorer qu'elle n'en pouvait plus douter. Elle l'assura qu'une fête
que le roi donnait était pour elle\,; elle le pressa, elle le conjura
avec les plus fortes instances de l'emmener dans ses terres de Guyenne,
et de l'y laisser jusqu'à ce que le roi l'eût oubliée et se fût engagé
ailleurs. Rien n'y put déterminer Montespan, qui ne fut pas longtemps
sans s'en repentir, et qui, pour son tourment, vécut toute sa vie et
mourut amoureux d'elle, sans toutefois l'avoir jamais voulu revoir
depuis le premier éclat. Je ne parlerai point, non plus, des divers
degrés que la peur du diable mit à reprises à sa séparation de la cour,
et je parlerai ailleurs de M\textsuperscript{me} de Maintenon qui lui
dut tout, qui prit peu à peu sa place, qui monta plus haut, qui la
nourrit longtemps des plus cruelles couleuvres, et qui enfin la relégua
de la cour. Ce que personne n'osa, ce dont le roi fut bien en peine, M.
du Maine, comme je l'ai dit ailleurs, s'en chargea, M. de Meaux acheva,
elle partit en larmes et en furie, et ne l'a jamais pardonné à M. du
Maine, qui par cet étrange service se dévoua pour toujours le coeur et
la toute-puissance de M\textsuperscript{me} de Maintenon.

La maîtresse, retirée à la communauté de Saint-Joseph, qu'elle avait
bâtie, fut longtemps à s'y accoutumer. Elle promena son loisir et ses
inquiétudes à Bourbon, à Fontevrault, aux terres de d'Antin, et fut des
années sans pouvoir se rendre à elle-même. À la fin Dieu la toucha. Son
péché n'avait jamais été accompagné de l'oubli, elle quittait souvent le
roi pour aller prier Dieu dans un cabinet\,; rien ne lui aurait fait
rompre aucun jeûne ni un jour maigre, elle fit tous les carêmes, et avec
austérité quant aux jeûnes dans tous les temps de son désordre. Des
aumônes, estime des gens de bien, jamais rien qui approchât du doute ni
de l'impiété\,; mais impérieuse, altière, dominante, moqueuse, et tout
ce que la beauté et la toute-puissance qu'elle en tirait entraîne après
soi. Résolue enfin de mettre à profit un temps qui ne lui avait été
donné que malgré elle, elle chercha quelqu'un de sage et d'éclairé et se
mit entre les mains du P. de La Tour, ce général de l'Oratoire si connu
par ses sermons, par ses directions, par ses amis, et par la prudence et
les talents du gouvernement. Depuis ce moment jusqu'à sa mort, sa
conversion ne se démentit point, et sa pénitence augmenta toujours. Il
fallut d'abord renoncer à l'attachement secret qui lui était demeuré
pour la cour, et aux espérances qui, toutes chimériques qu'elles
fussent, l'avaient toujours flattée. Elle se persuadait que la peur du
diable seule avait forcé le roi à la quitter\,; que cette même peur dont
M\textsuperscript{me} de Maintenon s'était habilement servie pour la
faire renvoyer tout à fait, l'avait mise au comble de grandeur où elle
était parvenue\,; que son âge et sa mauvaise santé qu'elle se figurait
l'en pouvaient délivrer\,; qu'alors se trouvant veuf, rien ne
s'opposerait à rallumer un feu autrefois si actif, dont la tendresse et
le désir de la grandeur de leurs enfants communs pouvait aisément
rallumer les étincelles, et qui n'ayant plus de scrupules à combattre,
pouvait la faire succéder à tous les droits de son ennemie.

Ses enfants eux-mêmes s'en flattaient et lui rendaient de grands devoirs
et fort assidus. Elle les aimait avec passion, excepté M. du Maine qui
fut longtemps sans la voir, et qui ne la vit depuis que par bienséance.
C'était peu dire qu'elle eût du crédit sur les trois autres, c'était de
l'autorité, et elle en usait sans contrainte. Elle leur donnait sans
cesse, et par amitié et pour conserver leur attachement, et pour se
réserver ce lien avec le roi qui n'avait avec elle aucune sorte de
commerce, même par leurs enfants. Leur assiduité fut retranchée\,; ils
ne la voyaient plus que rarement et après le lui avoir fait demander.
Elle devint la mère de d'Antin dont elle n'avait été jusqu'alors que la
marâtre, elle s'occupa de l'enrichir.

Le P. de La Tour tira d'elle un terrible acte de pénitence, ce fut de
demander pardon à son mari et de se remettre entre ses mains. Elle lui
écrivit elle-même dans les termes les plus soumis, et lui offrit de
retourner avec lui s'il daignait la recevoir, ou de se rendre en quelque
lieu qu'il voulût lui ordonner. À qui a connu M\textsuperscript{me} de
Montespan, c'était le sacrifice le plus héroïque. Elle en eut le mérite
sans en essuyer l'épreuve\,; M. de Montespan lui fit dire qu'il ne
voulait ni la recevoir, ni lui prescrire rien, ni ouïr parler d'elle de
sa vie. À sa mort, elle en prit le deuil comme une veuve ordinaire, mais
il est vrai que, devant et depuis, elle ne reprit jamais ses livrées ni
ses armes qu'elle avait quittées, et porta toujours les siennes seules
et pleines.

Peu à peu elle en vint à donner presque tout ce qu'elle avait aux
pauvres. Elle travaillait pour eux plusieurs heures par jour à des
ouvrages bas et grossiers, comme des chemises et d'autres besoins
semblables, et y faisait travailler ce qui l'environnait. Sa table,
qu'elle avait aimée avec excès, devint la plus frugale, ses jeûnes fort
multipliés\,; sa prière interrompait sa compagnie et le plus petit jeu
auquel elle s'amusait\,; et à toutes les heures du jour, elle quittait
tout pour aller prier dans son cabinet. Ses macérations étaient
continuelles\,; ses chemises et ses draps étaient de toile jaune la plus
dure et la plus grossière, mais cachés sous des draps et une chemise
ordinaire. Elle portait sans cesse des bracelets, des jarretières et une
ceinture à pointes de fer, qui lui faisaient souvent des plaies\,; et sa
langue, autrefois si à craindre, avait aussi sa pénitence. Elle était,
de plus, tellement tourmentée des affres de la mort, qu'elle payait
plusieurs femmes dont l'emploi unique était de la veiller. Elle couchait
tous ses rideaux ouverts avec beaucoup de bougies dans sa chambre, ses
veilleuses autour d'elle qu'à toutes les fois qu'elle se réveillait elle
voulait trouver causant, joliant\footnote{Vieux mot qui signifie riant,
  \emph{plaisantant, se livrant à la joie}.} ou mangeant, pour se
rassurer contre leur assoupissement.

Parmi tout cela, elle ne put jamais se défaire de l'extérieur de reine
qu'elle avait usurpé dans sa faveur et qui la suivit dans sa retraite.
Il n'y avait personne qui n'y fût si accoutumé de ce temps-là qu'on en
conservât l'habitude sans murmure. Son fauteuil avait le dos joignant le
pied de son lit\,; il n'en fallait point chercher d'autre dans la
chambre, non pas même pour ses enfants naturels, M\textsuperscript{me}
la duchesse d'Orléans pas plus que les autres. Monsieur et la grande
Mademoiselle l'avaient toujours aimée et l'allaient voir assez souvent.
À ceux-là on apportait des fauteuils et à M\textsuperscript{me} la
Princesse\,; mais elle ne songeait pas à se déranger du sien, ni à les
conduire. Madame n'y allait presque jamais, et trouvait cela fort
étrange. On peut juger par là comme elle recevait tout le monde. Il y
avait de petites chaises à dos, lardées de ployants de part et d'autre-,
depuis son fauteuil, vis-à-vis les uns des autres, pour la compagnie qui
venait et pour celle qui logeait chez elle, nièces, pauvres demoiselles,
filles et femmes qu'elle entretenait et qui faisaient les honneurs.

Toute la France y allait. Je ne sais par quelle fantaisie cela s'était
tourné de temps en temps en devoir\,; les femmes de la cour en faisaient
la leur à ses filles\,; d'hommes il y en allait peu sans des raisons
particulières, ou des occasions. Elle parlait à chacun comme une reine
qui tient sa cour et qui honore en adressant la parole. C'était toujours
avec un air de grand respect, qui que ce fût qui entrât chez elle\,; et
de visites elle n'en faisait jamais, non pas même à Monsieur, ni à
Madame, ni à la grande Mademoiselle, ni à l'hôtel de Condé. Elle
envoyait aux occasions aux gens qu'elle voulait favoriser, et point à
tout ce qui la voyait. Un air de grandeur répandu partout chez elle, et
de nombreux équipages toujours en désarroi\,; belle comme le jour
jusqu'au dernier moment de sa vie, sans être malade, et croyant toujours
l'être et aller mourir. Cette inquiétude l'entretenait dans le goût de
voyager\,; et dans ses voyages elle menait toujours sept ou huit
personnes de compagnie. Elle en fut toujours de la meilleure, avec des
grâces qui faisaient passer ses hauteurs et qui leur étaient adaptées.
Il n'était pas possible d'avoir plus d'esprit, de fine politesse, des
expressions singulières, une éloquence, une justesse naturelle qui lui
formait comme un langage particulier, mais qui était délicieux et
qu'elle communiquait si bien par l'habitude, que ses nièces et les
personnes assidues auprès d'elle, ses femmes, celles que, sans l'avoir
été, elle avait élevées chez elle, le prenaient toutes, et qu'on le sent
et on lé reconnaît encore aujourd'hui dans le peu de personnes qui en
restent. C'était le langage naturel de la famille, de son frère et de
ses soeurs. Sa dévotion ou peut-être sa fantaisie était de marier les
gens, surtout les jeunes filles\,; et comme elle avait peu à donner
après toutes ses aumônes, c'était souvent la faim et la soif qu'elle
mariait. Jamais, depuis sa sortie de la cour, elle ne s'abaissa à rien
demander pour soi ni pour autrui. Les ministres, les intendants, les
juges n'entendirent jamais parler d'elle. La dernière fois qu'elle alla
à Bourbon, et sans besoin, comme elle faisait souvent, elle paya deux
ans d'avance toutes les pensions charitables, qu'elle faisait en grand
nombre, presque toutes à de pauvre noblesse, et doubla toutes ses
aumônes. Quoique en pleine santé, et de son aveu, elle disait qu'elle
croyait qu'elle ne reviendrait pas de ce voyage, et que tous ces pauvres
gens auraient, avec ces avances, le temps de chercher leur subsistance
ailleurs. En effet, elle avait toujours la mort présente\,; elle en
parlait comme prochaine dans une fort bonne santé, et avec toutes ses
frayeurs, ses veilleuses et une préparation continuelle, elle n'avait
jamais ni médecin ni même de chirurgien.

Cette conduite concilie avec ses pensées de sa fin les idées éloignées
de pouvoir succéder à M\textsuperscript{me} de Maintenon, quand le roi,
par sa mort, deviendrait libre. Ses enfants s'en flattaient, excepté M.
du Maine, qui n'y aurait pas gagné. La cour intérieure regardait les
événements les plus étranges comme si peu impossibles, qu'on a cru que
cette pensée n'avait pas peu contribué à l'empressement des Noailles
pour le mariage d'une de leurs filles avec le fils aîné de d'Antin. Ils
s'étoient fort accrochés à M\textsuperscript{lle} Choin\,; ils
cultivaient soigneusement M\textsuperscript{me} la Duchesse\,; et pour
ne laisser Monseigneur libre d'eux par aucun côté, ils s'étaient saisis
de M\textsuperscript{me} la princesse de Conti en donnant une de leurs
filles à La Vallière, qui était son cousin germain, et qui pouvait tout
sur elle. Liés comme ils étaient à M\textsuperscript{me} de Maintenon
par le mariage de leur fils avec sa nièce, qui lui tenait lieu de fille,
il semblait que l'alliance de M\textsuperscript{me} de Montespan ne dût
pas leur convenir par la jalousie et la haine extrême que lui portait
M\textsuperscript{me} de Maintenon, et qui se marquait en tout avec une
suite qu'elle n'eut jamais pour aucun autre objet. Une considération si
forte et si délicate ne put les retenir ni les empêcher de profiter de
cette alliance pour faire leur cour à M\textsuperscript{me} de Montespan
comme à quelqu'un dont ils attendaient.

La maréchale de Cœuvres n'avait point d'enfants. Ils prirent l'occasion
de ce voyage de Bourbon pour lui donner leur fille à y mener comme la
sienne, c'est-à-dire allant avec elle, et n'ayant de maison, de table ni
d'équipage que ceux de M\textsuperscript{me} de Montespan. Elle fit sa
cour aux personnes de la compagnie, toutes subalternes qu'elles
fussent\,; et pour M\textsuperscript{me} de Montespan, elle lui rendit
beaucoup plus de respects qu'à M\textsuperscript{me} la duchesse de
Bourgogne, ni à M\textsuperscript{me} de Maintenon. Elle ne fut occupée
que d'elle, de lui plaire, de la gagner, et de gagner toutes celles de
sa maison. M\textsuperscript{me} de Montespan la traitait en reine, s'en
amusait comme d'une poupée, la renvoyait quand elle l'importunait, et
lui parlait extrêmement français. La maréchale avalait tout, et n'en
était que plus flatteuse et plus rampante.

M\textsuperscript{me} de Saint-Simon et M\textsuperscript{me} de Lauzun
étaient à Bourbon lorsque M\textsuperscript{me} de Montespan y arriva.
J'ai remarqué ailleurs qu'elle était cousine issue de germain de ma mère
(petits-enfants du frère et de la soeur)\,; que M\textsuperscript{me} de
Montespan la fit faire dame du palais de la reine lorsqu'on choisit les
premières\,; que mon père refusa\,; et que M\textsuperscript{me} de
Montespan voyait toujours ma mère en tout temps et à toutes heures, et
s'est toujours piquée de la distinguer. Ma mère la voyait donc de temps
en temps à Saint-Joseph, et M\textsuperscript{me} de Saint-Simon
aussi\,; aussi à Bourbon lui fit-elle toutes sortes d'amitiés et de
caresses, on n'oserait dire, de distinctions, avec cet air de grandeur
qui lui était demeuré. La maréchale de Coeuvres en était mortifiée de
jalousie jusqu'à le montrer et l'avouer, et on s'en divertissait. Je
rapporte ces riens pour montrer que l'idée de remplacer
M\textsuperscript{me} de Maintenon, toute chimérique qu'elle fût, était
entrée dans la tête des courtisans les plus intérieurs, et quelle était
la leur du roi et de la cour.

Parmi ces bagatelles, et M\textsuperscript{me} de Montespan dans une
très bonne santé, elle se trouva tout à coup si mal une nuit, que ses
veilleuses envoyèrent éveiller ce qui était chez elle. La maréchale de
Coeuvres accourut des premières, qui, la trouvant prête à suffoquer et
la tête fort embarrassée, lui fit à l'instant donner de l'émétique de
son autorité, mais une dose si forte, que l'opération leur en fit une
telle peur qu'on se résolut à l'arrêter, ce qui peut-être lui coûta la
vie.

Elle profita d'une courte tranquillité pour se confesser et recevoir les
sacrements. Elle fit auparavant entrer tous ses domestiques jusqu'aux
plus bas, fit une confession publique de ses péchés publics, et demanda
pardon du scandale qu'elle avait si longtemps donné, même de ses
humeurs, avec une humilité si sage, si profonde, si pénitente que rien
ne put être plus édifiant. Elle reçut ensuite les derniers sacrements
avec une piété ardente. Les frayeurs de la mort qui, toute sa vie,
l'avaient si continuellement troublée, se dissipèrent subitement et ne
l'inquiétèrent plus. Elle remercia Dieu en présence de tout le monde de
ce qu'il permettait qu'elle mourût dans un lieu où elle était éloignée
des enfants de son péché, et n'en parla durant sa maladie que cette
seule fois. Elle ne s'occupa plus que de l'éternité, quelque espérance
de guérison dont on la voulût flatter, et de l'état d'une pécheresse
dont la crainte était tempérée par une sage confiance en la miséricorde
de Dieu, sans regrets et uniquement attentive à lui rendre son sacrifice
plus agréable, avec une douceur et une paix qui accompagna toutes ses
actions.

D'Antin, à qui on avait envoyé un courrier, arriva comme elle approchait
de sa fin. Elle le regarda et lui dit seulement qu'il la voyait dans un
état bien différent de celui où il l'avait vue à Bellegarde. Dès qu'elle
fut expirée, peu d'heures après l'arrivée de d'Antin, il partit pour
Paris, ayant donné ses ordres, qui furent étranges ou étrangement
exécutés. Ce corps, autrefois si parfait, devint la proie de la
maladresse et de l'ignorance du chirurgien de la femme de Le Gendre,
intendant de Montauban, qui était venu prendre les eaux, et qui mourut
bientôt après, elle-même. Les obsèques furent à la discrétion des
moindres valets, tout le reste de la maison ayant subitement déserté. La
maréchale de Cœuvres se retira sur-le-champ à l'abbaye de Saint-Menou, à
quelques lieues de Bourbon, dont une nièce du P. La Chaise était
abbesse, avec quelques-unes de la compagnie de M\textsuperscript{me} de
Montespan, les autres ailleurs. Le corps demeura longtemps sur la porte
de la maison, tandis que les chanoines de la Sainte-Chapelle et les
prêtres de la paroisse disputaient de leur rang jusqu'à plus que de
l'indécence, Il fut mis en dépôt dans la paroisse comme y eût pu être
celui de la moindre bourgeoise du lieu, et longtemps après porté à
Poitiers dans le tombeau de sa maison à elle, avec une parcimonie
indigne. Elle fut amèrement pleurée de tous les pauvres de la paroisse,
sur qui elle répandait une infinité d'aumônes, et d'autres sans nombre
de toutes les sortes à qui elle en distribuait continuellement.

D'Antin était à Livry, où Monseigneur était allé chasser et coucher une
nuit, lorsqu'il reçut le courrier de Bourbon. En partant pour s'y
rendre, il envoya avertir à Marly les enfants naturels de sa mère. Le
comte de Toulouse l'alla dire au roi, et lui demander la permission
d'aller trouver sa mère. Il la lui accorda, et {[}le comte de
Toulouse{]} partit aussitôt\,; mais il ne fut que jusqu'à Montargis, où
il trouva un courrier qui apportait la nouvelle de sa mort, ce qui fit
aussi rebrousser les médecins et les autres secours qui l'allaient
trouver à Bourbon. Rien n'est pareil à la douleur que
M\textsuperscript{me} la duchesse d'Orléans, M\textsuperscript{me} la
Duchesse et le comte de Toulouse en témoignèrent. Ce dernier l'était
allé cacher de Montargis à Rambouillet. M. du Maine eut peine à contenir
sa joie\,; il se trouvait délivré de tout reste d'embarras. Il n'osa
rester à Marly\,; mais, au bout de deux jours qu'il fut à Sceaux, il
retourna à Marly et y fit mander son frère. Leurs deux soeurs, qui
étaient aussi retirées à Versailles, eurent le même ordre de retour. La
douleur de M\textsuperscript{me} la Duchesse fut étonnante, elle qui
s'était piquée toute sa vie de n'aimer rien, et à qui l'amour même, ou
ce que l'on croyait tel, n'avait jamais pu donner de regrets. Ce qui le
fut davantage, c'est celle de M. le Duc qui fut extrême, lui si peu
accessible à l'amitié, et dont l'orgueil était honteux d'une telle
belle-mère. Cela put confirmer dans l'opinion que j'ai expliquée plus
haut de leurs espérances, auxquelles cette mort mit fin.

M\textsuperscript{me} de Maintenon, délivrée d'une ancienne maîtresse
dont elle avait pris la place, qu'elle avait chassée de la cour, et sur
laquelle elle n'avait pu se défaire de jalousies et d'inquiétudes,
semblait devoir se trouver affranchie. Il en fut autrement\,; les
remords de tout ce qu'elle lui avait dû, et de la façon dont elle l'en
avait payée, l'accablèrent tout à coup à cette nouvelle. Les larmes la
gagnèrent, que faute de meilleur asile, elle fut cacher à sa chaise
percée\,; M\textsuperscript{me} la duchesse de Bourgogne qui l'y
poursuivit en demeura sans parole d'étonnement. Elle ne fut pas moins
surprise de la parfaite insensibilité du roi après un amour si passionné
de tant d'années\,; elle ne put se contenir de le lui témoigner. Il lui
répondit tranquillement que, depuis qu'il l'avait congédiée, il avait
compté ne la revoir jamais, qu'ainsi elle était dès lors morte pour lui.
Il est aisé de juger que la douleur des enfants qu'il en avait ne lui
plut pas. Quoique redouté au dernier point, elle eut son cours, et il
fut long. Toute la cour les fut voir sans leur rien dire, et le
spectacle ne laissa pas d'en être curieux. Un contraste entre eux et la
princesse de Conti ne le fut pas moins, et les humilia beaucoup.
Celle-ci était en deuil de sa tante, M\textsuperscript{me} de La
Vallière, qui venait de mourir. Les enfants du roi et de
M\textsuperscript{me} de Montespan n'osèrent porter aucun deuil d'une
mère non reconnue. Il n'y parut qu'au négligé, au retranchement de toute
parure et de tout divertissement, même du jeu qu'elles s'interdirent
pour longtemps, ainsi que le comte de Toulouse. La vie et la conduite
d'une si fameuse maîtresse depuis sa retraite forcée m'a paru être une
chose assez curieuse pouf s'y étendre, et l'effet de sa mort propre à
caractériser la cour.

D'Antin, délivré des devoirs à rendre à une mère impérieuse, fut plus
sensible à ce soulagement qu'à la cessation de tout ce qu'il tirait
d'elle depuis sa dévotion. Cette raison et celles de ses soeurs bâtardes
et du comte de Toulouse à qui il voulait plaire, et qui aimaient et
rendaient tant à leur mère, l'y rendait plus attentif. La pénitence la
rendait libérale pour lui\,; mais son coeur n'avait jamais pu s'ouvrir
sur le fils qu'elle avait eu de son mari, toute la place en était prise
par ses autres enfants. La contrainte qu'elle se donnait sur ceux-ci
augmentait sa peine à l'égard de l'autre pour qui tout était par effort.
Sa conduite lâchait la bride à l'humeur, et un autre que d'Antin aurait
encore eu le motif de se voir débarrassé d'une mère devenue sa honte et
celle de sa maison. Mais tel n'était pas son caractère\,: né avec
beaucoup d'esprit naturel, il tenait de ce langage charmant de sa mère
et du gascon de son père, mais avec un tour et des grâces naturelles qui
prévenaient toujours. Beau comme le jour étant jeune, il en conserva de
grands restes jusqu'à la fin de sa vie, mais une beauté mâle, et une
physionomie d'esprit. Personne n'avait ni plus d'agréments, de mémoire,
de lumière, de connaissance des hommes et de chacun, d'art et de
ménagements pour savoir les prendre, plaire, s'insinuer, et parler
toutes sortes de langages\,; beaucoup de connaissances et des talents
sans nombre, qui le rendaient propre à tout, avec quelque lecture. Un
corps robuste et qui sans peine fournissait à tout répondait au génie,
et quoique peu à peu devenu fort gros, il ne lui refusait ni veilles ni
fatigues. Brutal par tempérament, doux, poli par jugement, accueillant,
empressé à plaire, jamais il ne lui arrivait de dire mal de personne. Il
sacrifia tout à l'ambition et aux richesses, quoique prodigue, et fut le
plus habile et le plus raffiné courtisan de son temps, comme le plus
incompréhensiblement assidu. Application sans relâche, fatigues
incroyables pour se trouver partout à la fois, assiduité prodigieuse en
tous lieux différents, soins sans nombre, vues en tout, et cent à la
fois, adresses, souplesses, flatteries sans mesure, attention
continuelle et à laquelle rien n'échappait, bassesses infinies, rien ne
lui coûta, rien ne le rebuta vingt ans durant, sans aucun autre succès
que la familiarité qu'usurpait sa gasconne impudence, avec des gens que
tout lui persuadait avec raison qu'il fallait violer quand on était à
portée de le pouvoir. Aussi n'y avait-il pas manqué avec Monseigneur,
dont il était menin et duquel son mariage l'avait fort approché. Il
avait épousé la fille aînée du duc d'Uzès et de la fille unique du duc
de Montausier, dont la conduite obscure et peu régulière ne l'empêcha
jamais de vivre avec elle et avec tous les siens avec une considération
très marquée, et prenant une grande part à eux tous, ainsi qu'à ceux de
la maison de sa mère. Sa table, ses équipages, toute sa dépense était
prodigieuse et la fut dans tous les temps. Son jeu furieux le fit
subsister longtemps\,; il y était prompt, exact en comptes, bon payeur
sans incidents, jouait {[}tous les jeux{]} fort bien, heureux à ceux de
hasard\,; et avec tout cela, fort accusé d'aider la fortune.

Sa servitude fut extrême à l'égard des enfants de sa mère sa patience
infinie aux rebuts. On a vu celui qu'ils essuyèrent pour lui, lorsqu'à
la mort de son père ils demandèrent tous au roi de le faire duc\,; et si
le dénouement qui se verra bientôt n'eût découvert ce qui avait rendu
tant d'années et de ressorts inutiles, on ne pourrait le concevoir. On a
vu comment sa mère lui fit quitter solennellement le jeu en lui assurant
une pension de dix mille écus, combien le roi trouva ridicule l'éclat de
la profession qu'il en fit, et comment peu à peu il le reprit, deux ans
après, tout aussi gros qu'auparavant. Une autre disparate qu'il fit
pendant cette abstinence de jeu lui réussit tout aussi mal. Il se mit
dans la dévotion, dans les jeûnes qu'il ne laissait pas ignorer, et qui
durent coûter à sa gourmandise et à son furieux appétit\,; il affecta
d'aller tous les jours à la messe, et une régularité extérieure. Il
soutint cette tentative près de deux ans. À la fin, la voyant sans
succès, il s'en lassa, et peu à peu, avec le jeu, il reprit son premier
genre de vie. Avec de tels défauts si reconnus, il en eut un plus
malheureux que coupable, puisqu'il ne dépendait pas de lui, dont il
souffrit plus que de pas un. C'était une poltronnerie, mais telle qu'il
est incroyable ce qu'il faut qu'il ait pris sur lui pour avoir servi si
longtemps. Il en a reçu en sa vie force affronts avec une dissimulation
sans exemple. M. le Duc, méchant jusqu'à la barbarie, étant de jour au
bombardement de Bruxelles, le vit venir à la tranchée pour dîner avec
lui. Aussitôt il donna le mot, mit toute la tranchée dans la confidence,
et un peu après s'être mis à table, voilà une vive alarme, une grande
sortie des ennemis et tout l'appareil d'un combat chaud et imminent.
Quand M. le Duc s'en fut assez diverti, il regarda d'Antin\,: «\,
Remettons-nous à table, lui dit-il\,; la sortie n'était que pour toi.\,»
D'Antin s'y remit sans s'en émouvoir, et il n'y parut pas.

Une autre fois, M. le prince de Conti, qui ne l'aimait pas à cause de M.
du Maine et de M. de Vendôme, visitait des postes à je ne sais plus quel
siége, et trouva d'Antin d'ans un assez avancé. Le voilà à faire ses
grands rires qui lui cria\,: «\,Comment, d'Antin, te voilà ici, et tu
n'es pas encore mort\,?» Cela fut avalé avec tranquillité et sans
changer de conduite avec ces deux princes qu'il voyait très
familièrement. La Feuillade, fort envieux et fort avantageux, lui fit
une incartade aussi gratuite que ces deux-là. Il était à Meudon, à deux
pas de Monseigneur, dans la même pièce. Je ne sais sur quoi on vint à
parler de grenadiers, ni ce que dit d'Antin, qui forma une dispute fort
légère, et plutôt matière de conversation. Tout d'un coup\,: «\,C'est
bien à vous, lui dit La Feuillade en élevant le ton, à parler de
grenadiers, et où en auriez-vous vu\,?» D'Antin voulut répondre. «\,Et
moi, interrompit La Feuillade, j'en ai vu souvent en des endroits dont
vous n'auriez osé approcher de bien loin.\,» D'Antin se tut, et la
compagnie resta stupéfaite. Monseigneur, qui l'entendit, n'en fit pas
semblant, et dit après que, s'il avait témoigné l'avoir ouï, il n'avait
plus de parti à prendre que celui de faire jeter La Feuillade par les
fenêtres, pour un si grand manque de respect en sa présence. Cela passa
doux comme lait, et il n'en fut autre chose. En un mot, il était devenu
honteux d'insulter d'Antin.

Il faut convenir que c'était grand dommage qu'il eût un défaut si
infamant, sans lequel on eût peut-être difficilement trouvé un homme
plus propre que lui à commander les armées. Il avait les vues vastes,
justes, exactes, de grandes parties de général, un talent singulier pour
les marches, les détails de troupes, de fourrages, de subsistances, pour
tout ce qui fait le meilleur intendant d'armée, pour la discipline, sans
pédanterie et allant droit au but et au fait, une soif d'être instruit
de tout, qui lui donnait une peine infinie et lui coûtait cher en
espions. Ces qualités le rendaient extrêmement commode à un général
d'armée\,; le maréchal de Villeroy et M. de Vendôme s'en sont très
utilement servis. Il avait toujours un dessinateur ou deux qui prenaient
tant qu'ils pouvaient les plans du pays, des marches, des camps, des
fourrages et de ce qu'ils pouvaient de l'armée des ennemis. Avec tant de
vues, de soins, d'applications différentes à la cour et à la guerre,
toujours à soi, toujours la tête libre et fraîche, despotique sur son
corps et sur son esprit, d'une société charmante, sans tracasserie, sans
embarras, avec de la gaieté et un agrément tout particulier, affable aux
officiers, aimable aux troupes, à qui il était prodigue avec art et avec
goût, naturellement éloquent et parlant à chacun sa propre langue, aisé
en tout, aplanissant tout, fécond en expédients, et capable à fond de
toutes sortes d'affaires, c'était un homme certainement très rare. Cette
raison m'a fait étendre sur lui, et il est bon de faire connaître
d'avance ce courtisan jusqu'ici si délaissé, qui va devenir un
personnage pour le reste de sa vie. Fait et demeuré comme il était, il
n'est pas surprenant qu'il y ait eu autant d'envie de s'accrocher aux
Noailles. Le surprenant est que sa mère y ait non seulement consenti,
mais qu'elle l'ait désiré plus que lui encore, avec sa retraite et sa
dévotion véritable, pour se rapprocher M\textsuperscript{me} de
Maintenon qu'elle avait tant de raisons de haïr et de se la croire
irréconciliable. Elle lui écrivit plusieurs lettres flatteuses à
l'occasion de ce mariage\,; elle n'en recul que des réponses sèches, et
néanmoins fit tout pour le conclure, dans le dessein de lui plaire, tant
sont fortes les chaînes du monde, auquel trop souvent on croit de bonne
foi avoir entièrement renoncé, et que cependant, malgré tout ce qu'on en
a éprouvé, il se trouve qu'on y tient encore.

D'Antin, qui avait bien plus de sens que de valeur et d'honneur, n'avait
jamais ni espéré ni désiré de voir sa mère succéder à
M\textsuperscript{me} de Maintenon. Comme son intérêt là-dessus
n'aveuglait point son esprit, il en avait trop pour n'en pas sentir la
chimère\,; et si, par impossible, la chimère eût réussi, il voyait trop
clair dans sa plus étroite famille pour ignorer que ce ne serait pour
lui qu'un resserrement et un appesantissement de chaînes qui le
rendraient plus esclave des enfants de sa mère, qui tireraient tout le
fruit de ce retour, sans qui il ne pouvait rien espérer d'une femme qui
n'avait jamais eu pour lui d'amitié ni d'estime, et dont le coeur
n'était occupé que des fruits de son péché, quelque violence que la
dévotion lui fît à son égard et au leur. Il comprenait donc qu'avec le
roi de plus dans la balance, et la dissipation que la dévotion
trouverait en ce retour, il ne ferait que ramasser à peine les miettes
qui tomberaient de dessus leur table. Il sentait encore avec justesse,
et ne s'y trompa pas, la cause de l'inutilité de tous ses soins
jusqu'alors\,; que M\textsuperscript{me} de Maintenon était un obstacle
implacable et invincible à toute fortune du fils légitime de son
ancienne dame et maîtresse\,; laquelle n'étant plus, il se flattait
d'arriver enfin, sans que cette ennemie régnante s'y opposât plus, et de
voler enfin de ses propres ailes, sans être obligé à un vil emprunt des
enfants de sa mère, dont il sentait toute la honte, mais dont
jusqu'alors il éprouvait la nécessité. Le deuil épouvantable dont il
affecta de s'envelopper pour leur plaire et pour dissimuler l'aise et le
soulagement qu'il ressentait, ne les put cacher à eux ni au monde. Il ne
voulait pas, d'autre part, avoir le démérite de l'affliction devant
l'insensibilité du roi, ni devant l'ennemie de sa mère. La difficulté
d'ajuster deux choses si peu alliables le trahit\,; et le monde,
follement accoutumé à la vénération de M\textsuperscript{me} de
Montespan, ne pardonna pas à son fils, qui en tirait si gros, de s'être
remis sitôt au jeu, sous prétexte de la partie de Monseigneur, de
laquelle il était. L'indécence des obsèques, et le peu qui fut distribué
à ce nombreux domestique qui perdait tout, fit beaucoup crier contre
lui. Il crut l'apaiser par quelques largesses de gascon à quelques-uns
des plus attachés. Il porta même à M. du Maine un diamant de grand prix,
lui dit qu'il savait qu'il avait toujours aimé ce diamant, et qu'il ne
pouvait ignorer qu'il ne lui eût été destiné. M. du Maine le prit, mais
vingt-quatre heures après le lui renvoya par un ordre supérieur. Tout
cela ne fut rien en comparaison de l'affaire du testament.

On savait que M\textsuperscript{me} de Montespan en avait fait un, il y
avait longtemps\,; elle ne s'en était pas cachée, elle le dit même en
mourant, mais sans ajouter où on le trouverait, parce qu'il était
apparemment dans ses cassettes avec elle\,; ou, comme on n'en doutait
guère, que le P. de La Tour ne l'eût entre les mains. Cependant le
testament ne se trouva point, et le P. de La Tour, qui était alors dans
ses visites des maisons de l'Oratoire, déclara en arrivant qu'il ne
l'avait point, mais sans ajouter qu'il n'en avait point de connaissance.
Cela acheva de persuader qu'il y en avait un, et qu'il était enlevé et
supprimé pour toujours. Le vacarme fut épouvantable, les domestiques
firent de grands cris, et les personnes subalternes attachées à
M\textsuperscript{me} de Montespan qui y perdirent tout jusqu'à cette
ressource. Ses enfants s'indignèrent de tant d'étranges procédés et s'en
expliquèrent durement à d'Antin lui-même. Il ne fit que glisser et
secouer les oreilles sur ce à quoi il s'était bien attendu\,; il avait
été au solide, et il se promettait bien que la colère passerait avec la
douleur et ne lui nuirait pas en choses considérables. La perte commune
réunit pour un temps M\textsuperscript{me} la duchesse d'Orléans et
M\textsuperscript{me} la Duchesse. M\textsuperscript{me} de Saint-Simon
à son retour, ni moi en l'attendant, n'allâmes ni ne fîmes rien dire à
M. le Duc ni à M\textsuperscript{me} la Duchesse. La maréchale de
Cœuvres, qui pendant son voyage avait perdu son beau-père et avait pris
le nom de maréchale d'Estrées, arriva bien dolente d'avoir perdu son
voyage. Elle essaya d'en profiter au moins auprès des filles de
M\textsuperscript{me} de Montespan. Leur douleur dura assez longtemps,
avec elle finit la réunion des deux soeurs, et celle qu'elle avait
produite aussi entre M\textsuperscript{me} la Duchesse et
M\textsuperscript{me} la princesse de Conti, et toutes reprirent à
l'égard les unes des autres leur conduite ordinaire peu à peu, et à
l'égard du monde leur train de vie accoutumé. D'Antin n'en fut pas
quitte sitôt ni si à bon marché qu'il s'en était flatté avec les enfants
de sa mère, mais à la fin tout sécha, passa et disparut. Ainsi va le
cours du monde.

\hypertarget{chapitre-iii.}{%
\chapter{CHAPITRE III.}\label{chapitre-iii.}}

1707

~

{\textsc{Mort de la duchesse de Nemours\,; sa famille.}} {\textsc{-
Branche de Nemours de la maison de Savoie.}} {\textsc{- Caractère de
M\textsuperscript{me} de Nemours.}} {\textsc{- Origine de l'ordre du
Calvaire.}} {\textsc{- Prétendants à Neuchâtel.}} {\textsc{- Droits des
prétendants.}} {\textsc{- Conduite de la France sur Neuchâtel.}}
{\textsc{- Électeur de Brandebourg prétend Neuchâtel, où son ministre
veut précéder le prince de Conti.}} {\textsc{- Neuchâtel adjugé et livré
à l'électeur de Brandebourg.}} {\textsc{- Mort, famille, fortune du
cardinal d'Arquien.}} {\textsc{- Étonnante vérité.}} {\textsc{- Rage de
la reine de Pologne contre la France, et sa cause.}} {\textsc{- Mort de
la duchesse de La Trémoille.}} {\textsc{- Malheur des familles.}}
{\textsc{- Caractère de la maréchale de Créqui.}} {\textsc{- Mort de
Vaillac\,; son extraction\,; ses aventures.}} {\textsc{- Archevêque de
Bourges singulièrement nommé au cardinalat par le roi Stanislas.}}

~

La mort de la duchesse de Nemours, qui suivit celle de
M\textsuperscript{me} de Montespan de fort près, fit encore plus de
bruit dans le monde, mais dans un autre genre. Elle était fille du
premier lit du dernier duc de Longueville qui ait figuré, et de la fille
aînée du comte de Soissons, prince du sang, qui fit et perdit ce procès
fameux contre le prince de Condé, fils de son frère aîné et, père du
héros. L'autre fille du même prince épousa le prince de Carignan, si
connu sous le nom de prince Thomas, dernier fils du célèbre duc de
Savoie, Charles-Emmanuel, vaincu par l'épée de Louis XIII aux barricades
de Suse. M\textsuperscript{me} de Carignan mourut à Paris à
quatre-vingt-six ans, en 1692, mère du fameux muet et du comte de
Soissons mari de la trop célèbre comtesse de Soissons, nièce du cardinal
Mazarin\,; et M\textsuperscript{me} de Carignan et sa soeur aînée,
duchesse de Longueville, étaient soeurs du dernier comte de Soissons,
prince du sang, tué à la bataille de la Marfée, dite de Sedan, qu'il
venait de gagner contre l'armée du roi, où Sa Majesté n'était pas, en
1641, sans avoir été marié, père de ce bâtard obscur reconnu si
longtemps après sa mort, à qui M\textsuperscript{me} de Nemours dont
nous parlons fit de si grands biens, lequel, d'une fille du maréchal de
Luxembourg, laissa une fille devenue unique, infiniment riche, qui
épousa le duc de Luynes, mère du duc de Chevreuse d'aujourd'hui. Ainsi
ce bâtard était cousin germain de M\textsuperscript{me} de Nemours, fils
du frère de sa mère et de la princesse de Carignan. M. de Longueville
devenu veuf, et n'ayant que M\textsuperscript{me} de Nemours non encore
mariée, épousa en secondes noces la soeur de M. le Prince le héros, qui
sous le nom de M\textsuperscript{me} de Longueville a fait tant de bruit
dans le monde, et tant figuré dans la minorité de Louis XIV.
M\textsuperscript{me} de Nemours fut mariée en 1667, qu'elle avait
trente-deux ans, et devint veuve deux ans après, sans enfants, du
dernier de cette branche de Nemours. Elle sortait de Philippe, comte de
Genevois, fils puîné de Philippe II duc de Savoie. Le comte de Genevois
était frère de père de Philibert II, duc de Savoie, et de la mère du roi
François Ier, et de père et de mère de Charles III duc de Savoie. Le
comte de Tende et de Villars si connu, lui et sa courte mais brillante
postérité en France, était leur frère bâtard. François Ier fit le comte
de Genevois duc de Nemours vérifié sans pairie. Le duc de Savoie,
Charles III, son frère, fut grand-père du fameux duc Charles-Emmanuel
dont je viens de parler, et ce Charles-Emmanuel était grand-père d'autre
Charles-Emmanuel, père du premier roi de Sardaigne. On voit ainsi en
quelle distance cette branche de Nemours était tombée du chef de sa
maison.

Ce premier duc de Nemours épousa une Longueville dont la mère était
Bade, de la branche d'Hochberg, héritière par la sienne de Neuchâtel, et
c'est par là que cette espèce de souveraineté, à faute de Longueville
mâles, est tombée à M\textsuperscript{me} de Nemours. De ce premier duc
de Nemours et de cette héritière vint un fils unique Jacques, duc de
Nemours, si connu en son temps par son esprit, ses grâces, ses
galanteries, sa bravoure, qui fit cet enfant à M\textsuperscript{lle} de
La Garnache dont j'ai parlé (t. II, p.~143) à l'occasion des Rohan, et
qui épousa la fameuse Anne d'Este, petite-fille de Louis XII par sa
mère, et veuve du duc de Guise, tué par Poltrot au siège d'Orléans, et
mère des duc et cardinal de Guise, tués à Blois en 1588, du duc de
Mayenne, chef de la Ligue, du cardinal de Guise, et de cette furieuse
duchesse de Montpensier. Ainsi les deux fils de ce second duc de Nemours
étaient frères utérins des Guise que je viens de nommer, fort liés avec
eux, aussi grands ligueurs qu'eux, mais brouillés à la fin avec le duc
de Mayenne qui voulait tout le royaume pour son fils en épousant
l'infante d'Espagne, parce qu'il les convainquit de vouloir livrer au
duc de Savoie leur gouvernement de Lyon, la Provence et le Dauphiné.
L'aîné mourut sans alliance, le cadet épousa la fille aînée et héritière
du duc d'Aumale, le seul des chefs de la Ligue qu'on ne put trouver
moyen de comprendre dans l'amnistie à la paix, et qui, pour l'assassinat
d'Henri III, fut tiré à quatre chevaux en effigie, en Grève, par arrêt
du parlement, et mourut fort vieux, fort gueux et fort délaissé à
Bruxelles.

De ce mariage trois fils, tous trois ducs de Nemours l'un après l'autre.
L'aîné mourut jeune sans alliance\,; le second épousa la fille du duc de
Vendôme, bâtard d'Henri IV, suivit le parti de M. le Prince et fut tué
en duel par le duc de Beaufort, frère de sa femme, qui avait embrassé le
même parti. La jalousie s'était mise entre eux sur tous chapitres, et
c'est ce duel qui commença la fortune du père du maréchal de Villars
dont j'ai parlé (t. Ier, p. 26). Ce duc de Nemours laissa deux filles,
l'aînée fut duchesse de Savoie et mère du premier roi de Sardaigne,
l'autre, reine de Portugal, célèbre pour avoir répudié, détrôné et
confiné son mari, et épousé son beau-frère qui, après sa mort, eut d'une
Neubourg le roi de Portugal d'aujourd'hui. Le troisième frère, nommé à
l'archevêché de Reims sans avoir pris aucuns ordres, quitta ses
bénéfices en 1652, à la mort de son frère, et quatre ou cinq ans après
épousa M\textsuperscript{me} de Nemours dont il s'agit ici, qu'il laissa
veuve sans enfants deux ans après, à laquelle il faut maintenant
revenir. Il faut seulement remarquer auparavant que son père, mort en
1663, avait laissé deux fils de son second mariage avec la soeur de M.
le Prince et de M\,; le prince de Conti. L'aîné, à qui la tête tourna de
bonne heure, qu'on envoya à Rome chez les jésuites, où il prit le petit
collet en 1666, à vingt ans, ayant renoncé à tout en faveur de son
frère, et fut fait prêtre par le pape même en 1669. C'est sur cette
tutelle que M. le Prince père et fils eurent tant de disputés et de
procédés avec M\textsuperscript{me} de Nemours, qui la perdit contre
eux. Le cadet, qui portait le nom de comte de Saint-Paul, fut tué au
passage du Rhin, sans alliance\,; allant être élu roi de Pologne, en
1672. Michel Wiesnowieski le fut en sa place, sur la nouvelle de sa
mort. Son frère, revenu en France, passa le reste de ses jours
honnêtement, enfermé dans l'abbaye de Saint-Georges, près de Rouen, où
il est mort le dernier de cette longue et illustre bâtardise, en 1694.

M\textsuperscript{me} de Nemours, avec une figure fort singulière, une
façon de se mettre en tourière qui ne l'était pas moins, de gros yeux
qui ne voyaient goutte, et un tic qui lui faisait toujours aller une
épaule, avec des cheveux blancs qui lui traînaient partout, avait l'air
du monde le plus imposant. Aussi était-elle altière au dernier point, et
avait infiniment d'esprit avec une langue éloquente et animée, à qui
elle ne refusait rien. Elle avait la moitié de l'hôtel de Soissons, et
M\textsuperscript{me} de Carignan l'autre, avec qui elle avait souvent
des démêlés, quoique soeur de sa mère et princesse du sang. Elle
joignait à la haine maternelle de la branche de Condé celle qu'inspirent
souvent les secondes femmes aux enfants du premier lit. Elle ne
pardonnait point à M\textsuperscript{me} de Longueville les mauvais
traitements qu'elle prétendait en avoir reçus, et moins encore aux deux
princes de Condé de lui avoir emblé la tutelle et le bien de son frère,
et au prince de Conti d'en avoir gagné contre elle la succession et le
testament fait en sa faveur. Ses propos les plus forts, les plus salés
et souvent très plaisants, ne tarissaient point sur ces chapitres, où
elle ne ménageait point du tout la qualité de princes du sang. Elle
n'aimait pas mieux ses héritiers naturels, les Gondi et les Matignon.
Elle vivait pourtant honnêtement avec la duchesse douairière de
Lesdiguières et avec le maréchal et la maréchale de Villeroy, mais pour
les Matignon, elle n'en voulut pas ouïr parler.

Les deux soeurs de son père avaient épousé, l'aînée le fils aîné du
maréchal-duc de Retz, la cadette le fils puîné du maréchal de Matignon.
Cette aînée perdit son mari avant son beau-père, et est devenue célèbre
sous le nom de marquise de Belle-Ile par quantité de bonnes oeuvres,
s'être faite feuillantine, avoir obstinément refusé l'abbaye de
Fontevrault, enfin pour avoir conçu et enfanté le nouvel ordre du
Calvaire, dans lequel elle mourut à Poitiers en 1628. Le duc de Retz,
son fils unique, ne laissa que deux filles. L'aînée épousa Pierre Gondi,
cousin germain de son père, qui, en faveur de ce mariage, eut de
nouvelles lettres de duc et pair de Retz et le rang de leur date. Il
était fils du célèbre père de l'Oratoire qui avait été chevalier de
l'ordre et général des galères, et il était frère du fameux coadjuteur
de Paris ou cardinal de Retz. Il ne laissa qu'une fille, mariée au duc
de Lesdiguières, qui n'eut qu'un fils, gendre du maréchal de Duras, que
nous avons vu mourir fort jeune sans enfants. L'autre fille épousa le
duc de Brissac, dont il n'eut que mon beau-frère, mort sans enfants, et
la maréchale de Villeroy. L'autre tante de M. de Longueville, père de
M\textsuperscript{me} de Nemours, épousa par amour le second fils du
maréchal de Matignon, dont l'aîné n'avait point d'enfants, deux frères
de grand mérite, en grands emplois et tous deux chevaliers de l'ordre.
Cette Longueville fut mère du père du comte et du dernier maréchal de
Matignon, vivants à la mort de M\textsuperscript{me} de Nemours et bien
longtemps depuis, et qui étaient ses héritiers, ainsi que la maréchale
de Villeroy. La marquise de Belle-Ile avait été mariée par sa famille et
en sa présence\,; sa soeur s'était mariée à son gré à leur insu, et
toute la maison de Longueville ne put se résoudre à leur pardonner et à
les voir qu'après un grand nombre d'années, et jamais depuis aucun des
Longueville n'a aimé les Matignon.

M\textsuperscript{me} de Nemours était là-dessus si entière, que,
parlant au roi dans une fenêtre de son cabinet, avec ses yeux qui ne
voyaient guère, elle ne laissa pas d'apercevoir Matignon qui passait
dans la cour. Aussitôt elle se mit à cracher cinq ou six fois tout de
suite, puis dit au roi qu'elle lui en demandait pardon, mais qu'elle ne
pouvait voir un Matignon sans cracher de la sorte. Elle était
extraordinairement riche, et vivait dans une grande splendeur et avec
beaucoup de dignité\,; mais ses procès lui avaient tellement aigri
l'esprit qu'elle ne pouvait pardonner. Elle ne finissait point
là-dessus\,; et quand quelquefois on lui demandait si elle disait le
\emph{Pater}, elle répondait que oui, mais qu'elle passait l'article du
pardon des ennemis sans le dire. On peut juger plue la dévotion ne
l'incommodait pas. Elle faisait elle-même le conte qu'étant entrée dans
un confessionnal sans être suivie dans l'église, sa mine n'avait pas
imposé au confesseur, ni son accoutrement. Elle parla de ses grands
biens, et beaucoup des princes de Condé et de Conti. Le confesseur lui
dit de passer cela. Elle, qui sentait son cas grave, insista pour
l'expliquer, et fit mention de grandes terres et de millions. Le
bonhomme la crut folle et lui dit de se calmer, que c'était des idées
qu'il fallait éloigner, qu'il lui conseillait de n'y plus penser, et
surtout de manger de bons potages, si elle en avait le moyen. La colère
lui prit, et le confesseur à fermer le volet. Elle se leva et prit le
chemin de la porte. Le confesseur, la voyant aller, eut curiosité de ce
qu'elle devenait, et la suivit à la porte. Quand il vit cette bonne
femme qu'il croyait folle reçue par des écuyers, des demoiselles, et ce
grand équipage avec lequel elle marchait toujours, il pensa tomber à la
renverse, puis courut à sa portière lui demander pardon. Elle, à son
tour, se moqua de lui, et gagna pour ce jour de ne point aller à
confesse. Quelques semaines avant sa mort, elle fut si mal qu'on la
pressa de penser à elle. Enfin elle prit sa résolution. Elle envoya son
confesseur avec un de ses gentilshommes à M. le Prince, à M. le prince
de Conti et à MM. de Matignon, leur demander pardon de sa part. Tous
allèrent la voir et en furent bien reçus\,; mais ce fut tout\,: pas un
n'en eut rien. Elle avait quatre-vingt-six ans et acheva de donner ce
qu'elle put aux deux filles de ce bâtard qu'elle avait fait son
héritier, dont l'une mourut jeune, sans être mariée\,; l'autre épousa le
duc de Luynes, comme je l'ai déjà dit.

Cette mort mit promptement bien des gens en campagne. Le duc de Villeroy
et Matignon partirent aussitôt pour Neuchâtel, et M. le prince de Conti
pour Pontarlier, parce que le roi ne voulut pas qu'il se commît, comme
en son premier voyage, au manque de respect qu'il avait éprouvé à
Neuchâtel. De Pontarlier, il était à portée d'y donner ses ordres pour
ses affaires, et d'en savoir des nouvelles à tous moments. Il y envoya
Saintrailles, que M. le Duc lui prêta, et qui était un homme d'esprit
sage et capable, mais qui, pour avoir été gâté par la bonne compagnie et
par ces princes, était devenu très suffisant et passablement
impertinent, d'ailleurs un très simple gentilhomme, et rien moins que
Poton, dont était le fameux Saintrailles, dont les actions ont rendu ce
nom célèbre dans nos histoires. La vieille Mailly, belle-mère de la dame
d'atours de M\textsuperscript{me} la duchesse de Bourgogne, s'était mise
sur les rangs pour la succession à la principauté d'Orange, sur une
alliance tirée par les cheveux de la maison de Châlons, moins dans
l'espérance d'un droit aussi chimérique, que pour faire valoir le
marquis de Nesle, son petit-fils, par des prétentions si hautes. La même
raison la fit se présenter avec aussi peu de fondement pour Neuchâtel.
Elle se flattait qu'avec la protection de M\textsuperscript{me} de
Maintenon, elle en pourrait tirer d'autres partis plus solides.
M\textsuperscript{me} de Maintenon n'y prit pas la moindre part, et on
se moqua à Paris comme en Suisse de ses chimères. Celle de M. le prince
de Conti était fondée sur le testament du dernier duc de Longueville,
mort enfermé, qui l'avait appelé à tous ses biens, après le comte de
Saint-Paul, son frère, et sa postérité. Il avait gagné ce procès contre
M\textsuperscript{me} de Nemours. Restait à voir si une souveraineté se
pouvait donner comme d'autres biens, et si MM. de Neuchâtel défèreraient
à un arrêt du parlement de Paris. Outre qu'ils n'étaient pas soumis à
aucune juridiction du royaume, les héritiers prétendaient que Neuchâtel,
par la qualité souveraine, ou plutôt indépendante de ce petit État, ne
pouvait se donner ni être ôtée aux héritiers du sang, et cela est vrai
en France des duchés. Restait donc à voir à qui il devait appartenir, de
Matignon ou de la duchesse douairière de Lesdiguières, pour laquelle le
duc de Villeroy était allé comme son héritier par sa mère.

Matignon se prétendait préférable par la proximité du sang, parce qu'il
avait un degré sur la duchesse, et celle-ci par l'aînesse. Son droit
contre Matignon ne paraissait pas douteux. Les fiefs de dignités et tous
les grands fiefs ont toujours suivi l'aînesse\,; la loi et la pratique
s'y sont toujours accordées\,; à plus forte raison un fief indépendant,
étendu et considéré comme souverain. Mais de pareils procès ne se
décident guère par les règles, et Matignon avait beau jeu, Chamillart,
comme je l'ai remarqué (t. IV, p.~192), était son ami intime, et il
était devenu ennemi déclaré du maréchal de Villeroy, à l'occasion de la
bataille de Ramillies, comme je l'ai raconté en son lieu. Par cette même
occasion, comme on l'a vu là même, ce maréchal était tombé dans
l'entière disgrâce du roi. Restait le prince de Conti qu'il n'aimait
point, et à qui il n'avait jamais pu pardonner sincèrement son voyage de
Hongrie, et peut-être encore moins son mérite et sa réputation.
Chamillart, dans le fort de sa faveur, n'eut donc pas de peine d'obtenir
du roi de se déclarer neutre. Ce ministre, sûr de ce côté-là à l'égard
d'un prince du sang, ne balança pas à se déclarer ouvertement pour
Matignon. Il le combla d'argent et de tout ce que son crédit lui put
donner. Puysieux, ambassadeur en Suisse, était frère de Sillery, écuyer
depuis longues années du prince de Conti, auquel ils étaient tous
extrêmement attachés. Quelque désir qu'il eût de le servir dans cette
affaire, la neutralité déclarée du roi lui en ôta tous les moyens par
son caractère\,; et l'autorité et la vigilance de Chamillart tous ceux
qui lui pouvaient rester, comme particulier qui s'était fait des amis
dans le pays. La veuve de ce bâtard du dernier comte de Soissons y était
comme les autres, et, fondée par la donation de M\textsuperscript{me} de
Nemours, elle et son mari avaient dès leur mariage pris le nom de prince
et de princesse de Neuchâtel. Lors de l'arrêt du parlement de Paris qui
jugea le testament de M. de Longueville bon au profit du prince de
Conti, et qu'il alla à Neuchâtel en conséquence, et les autres héritiers
pour le lui disputer, il avait essuyé un préjugé fâcheux.
M\textsuperscript{me} de Nemours, qui y était aussi allée, y fut reçue
et reconnue comme souveraine, comme soeur du dernier possesseur, qui
n'avait pu disposer de Neuchâtel comme de ses autres biens. Le prince de
Conti en essuya une récidive confirmative de ce premier préjugé. Ceux de
Neuchâtel s'indignèrent contre la veuve de ce bâtard, contre la donation
de Neuchâtel faite à son mari et à leurs enfants, contre le nom qu'elle
en osait usurper. Ils la chassèrent comme n'ayant aucun droit, et la
firent honteusement sortir de leur ville et de tout leur petit État.
C'était bien déclarer à M. le prince de Conti le peu d'état qu'ils
faisaient d'un droit sur eux, à titre de donation, égale pour
M\textsuperscript{me} de Neuchâtel et pour lui.

Ces fiers bourgeois, pendant ces disputes, voyaient les prétendants
briguer à leurs pieds leurs suffrages, lorsqu'il parut au milieu d'eux
un ministre de l'électeur de Brandebourg, qui commença par oser disputer
le rang au prince de Conti. Cette impudence est remarquable, à ce même
prince de Conti, à qui, volontaire en Hongrie, à lui et à M. son frère,
l'électeur de Bavière, non par un ministre, mais en propre personne et à
la tête de ses troupes, auxiliaires dans l'armée de l'empereur, ne
l'avait pas disputé, avait vécu également et sans façons, et avait
presque toujours marqué attention à passer partout après eux, et à qui
le fameux duc de Lorraine, beau-frère de l'empereur, généralissime de
ses armées et de celles de l'empire, et qui commandait celle-là en chef,
a toujours cédé partout sans milieu et sans balancer\,; et voilà le
premier fruit du changement de cérémonial de nos ducs et de nos généraux
d'armée avec le même électeur de Bavière, par méprise d'abord, puis
suivie, que j'ai racontée en son lieu. D'alléguer que l'électeur de
Brandebourg, qui comme tel passait sans difficulté après l'électeur de
Bavière, était reconnu roi de Prusse partout, excepté en France, en
Espagne et à Rome, de laquelle comme protestant il ne se souciait point,
ç'aurait pu être une raison valable pour sa personne\,; mais pour son
ministre, on n'a jamais vu de nonce, à qui tous les ambassadeurs des
rois, même protestants, et celui de l'empereur, cèdent partout sans
difficulté, disputer rien en lieu tiers à un prince du sang, ni
l'ambassadeur de l'empereur non plus, qui a la préséance partout sur
ceux de tous les rois, dont aucun ne la lui conteste. L'électeur de
Brandebourg tirait sa prétention de la maison de Châlons. Elle était
encore plus éloignée, plus enchevêtrée, s'il était possible, que celle
de M\textsuperscript{me} de Mailly\,; aussi ne s'en avantagea-t-il que
comme d'un prétexte. Je l'ai déjà dit, ces sortes de procès ne se
décident ni par droit ni par justice.

Ses raisons étaient sa religion conforme à celle du pays\,; l'appui des
cantons protestants voisins, alliés, protecteurs de Neuchâtel\,; la
pressante réflexion que, la principauté d'Orange étant tombée, par la
mort du roi Guillaume III, au même prince de Conti, le roi lui en avait
donné récompense et se l'était appropriée, ce que le voisinage de la
France lui donnerait la facilité de faire pour Neuchâtel, s'il tombait à
un de ses sujets, qui, dans d'autres temps et dans un état fort
différent de celui où la maison de Longueville l'avait possédé, ne se
trouverait pas en situation de refuser le roi de l'en accommoder\,;
enfin un traité produit en bonne forme, par lequel, le cas avenant de la
mort de M\textsuperscript{me} de Nemours, l'Angleterre et la Hollande
s'engageaient à se déclarer pour lui, et à l'assister à vives forces
pour lui procurer ce petit État. Ce ministre de Brandebourg était de
concert avec les cantons protestants, qui, sur sa déclaration, prirent
aussitôt l'affirmative, et qui, par l'argent répandu, la conformité de
religion, la puissante de l'électeur, la réflexion de ce qui était
arrivé à Orange, trouvèrent presque tous les suffrages favorables.
Ainsi, à la chaude, ils firent rendre par ceux de Neuchâtel un jugement
provisionnel qui adjugea leur État à l'électeur jusqu'à la paix, en
conséquence duquel son ministre fut mis en possession actuelle\,; et M.
le prince de Conti, qui, depuis la prétention de ce ministre sur le
rang, n'avait pas cru convenable faire des tours de Pontarlier à
Neuchâtel, se vit contraint de revenir plus honteusement que la dernière
fois, et bientôt après fut suivi des deux autres prétendants.
M\textsuperscript{me} de Mailly, qui se donnait toujours pour telle, fit
si bien les hauts cris à la nouvelle de cette intrusion, qu'à la fin la
considération de son alliance avec M\textsuperscript{me} de Maintenon
réveilla nos ministres. Ils l'écoutèrent. Ils trouvèrent après elle
qu'il était de la réputation du roi de ne pas laisser enlever ce morceau
à ses sujets, et qu'il y avait du danger de le laisser entre les mains
d'un aussi puissant prince protestant, en état de faire une place
d'armes en lieu si voisin de la comté de Bourgogne, et dans une
frontière aussi peu couverte. Là-dessus, le roi fit dépêcher un courrier
à Puysieux, avec ordre à lui d'aller à Neuchâtel, et y employer tout,
même jusqu'aux menaces, pour exclure l'électeur, laissant d'ailleurs la
liberté du choix parmi ses sujets à l'égard desquels, pourvu que c'en
fût un, la neutralité demeurait entière. C'était s'en aviser trop tard.
L'affaire en était faite, les cantons engagés sans moyens de se dédire,
et de plus piqués d'honneur par le ministre électoral, sur les menaces
de Puysieux, au mémoire duquel les ministres d'Angleterre et de
Hollande, qui étaient là firent imprimer une réponse fort violente. Le
jugement provisionnel ne reçut aucune atteinte\,; on en eut la honte, on
en témoigna du ressentiment pendant six semaines, après quoi, faute de
mieux pouvoir, on s'apaisa de soi-même. On peut juger quelle espérance
il resta aux prétendants de revenir, à la paix, de ce jugement
provisionnel, et de lutter avec succès contre un prince aussi puissant
et aussi solidement appuyé. Aussi n'en fut-il pas mention depuis, et
Neuchâtel est pleinement et paisiblement demeuré à ce prince, qui fut
même expressément confirmé dans sa possession par la paix de la part de
la France. Le roi, ni Monseigneur, ni par conséquent la cour, ne prirent
point le deuil de M\textsuperscript{me} de Nemours, quoique fille d'une
princesse du sang\,; mais Monseigneur et M\textsuperscript{me} la
duchesse de Bourgogne le prirent à cause de la maison de Savoie.

Le cardinal d'Arquien mourut à Rome presque en même temps que
M\textsuperscript{me} de Montespan et M\textsuperscript{me} de Nemours.
La singularité de sa fortune mérite qu'on s'arrête un moment à lui. Son
nom était La Grange, et son père, qui n'avait point eu d'enfants de la
fille du second maréchal de La Châtre, était frère puîné du maréchal de
Montigny qui lui donna sa charge de capitaine de la porte, quand il eut
celle de premier maître d'hôtel du roi, et lui procura sa lieutenance au
gouvernement de Metz, et les gouvernements de Calais, Gien et Sancerre.
Il conserva cette dernière place contre les efforts de la Ligue, servit
bien et fidèlement, et fut quelque temps lieutenant-colonel du régiment
des gardes. De son premier mariage, il eut un fils, gouverneur de Calais
après M. de Vic, qui épousa une Rochechouart, mais qui ne fit pas grande
figure, non plus que sa postérité qui dure encore. De son troisième
mariage avec une Ancienville il eut deux fils\,: l'aîné s'appela le
marquis d'Espoisses, qui maria sa fille à Guitaut, premier gentilhomme
de la chambre de M. le Prince qui le fit chevalier de l'ordre\,; l'autre
fut le marquis d'Arquien, mort cardinal, dont nous parlons\footnote{Passage
  omis dans les précédentes éditions depuis \emph{Et son père}.}.

Il naquit en 1613, fut homme d'esprit, de bonne compagnie, et fort dans
le monde où il fut fort aidé par le duc de Saint-Aignan et par la
comtesse de Béthune, sa soeur, dame d'atours de la reine Marie-Thérèse,
de la mère desquels, fille du maréchal de Montigny, il était cousin
germain. Il eut le régiment de cavalerie de Monsieur, et fut capitaine
de ses Cent-Suisses. Il avait épousé une La Châtre de la branche de
Brillebaut\footnote{Louise de La Châtre, fille de Claude de La Châtre,
  maréchal de France et de Jeanne Chabot.}, qu'il perdit en 1672, qui
lui laissa un fils et cinq filles dont deux se firent religieuses.
Embarrassé de marier les autres, il se laissa persuader par un
ambassadeur de Pologne, avec qui il avait lié grande amitié, de les
établir en ce pays-là. Il quitta Monsieur pour faire ce voyage avec
l'ambassadeur qui s'en retournait, qui, peu après leur arrivée, fit si
bien, qu'il en fit épouser une à Jacob Radzevil, prince de Zamoski,
palatin de Sandomir. Elle le perdit peu après sans enfants, et demeura
assez riche pour que Jean Sobieski eût envie de l'épouser. Ce mariage se
fit en 1665.

Sobieski, qui avait l'inclination française, était lors grand maréchal
et gouverneur général de Pologne, et le premier homme de la république
par ses victoires et ses grandes actions, qui le portèrent sur le trône
de Pologne par une élection unanime, le 20 mai 1674. La soeur aînée
n'avait point voulu d'établissement étranger. La liaison intime et la
parenté qui était entre son père et la marquise de Béthune, dame
d'atours de la reine, firent, en 1669, son mariage avec le marquis de
Béthune, son fils, en faveur duquel elle eut la survivance de la charge
de sa belle-mère. Sa soeur étant devenue reine, son mari fut aussitôt
envoyé extraordinaire en Pologne, pour complimenter le nouveau roi. Il
revint immédiatement après, fut fait seul extraordinairement chevalier
de l'ordre en décembre 1675, et repartit pour Varsovie avec sa femme,
chargé de porter le collier du Saint-Esprit au roi son beau-frère, qu'il
lui donna à Zolkiew, en novembre suivant, et y demeura ambassadeur
extraordinaire. Sa femme y avait mené son autre soeur, qu'elle maria en
1678 au comte Wicillopolski, grand chancelier de Pologne, avec lequel
elle vint ici pendant son ambassade en 1686, et le perdit deux ans
après. M. et M\textsuperscript{me} de Béthune eurent deux fils et deux
filles. Le roi de Pologne maria l'aînée, en 1690, au prince Radzevil
Kleski, son neveu, grand maréchal de Lituanie, et en secondes noces, au
prince Sapieha, petit maréchal de Lituanie\,; l'autre fille épousa, en
1693, le comte Jablonowski, grand enseigne de Pologne, palatin de
Volhynie, et, l'année suivante, de Russie, frère de la comtesse Bnin
Opalinska, mère du roi Stanislas, père de la reine épouse de Louis XV.

M. de Béthune demeura toujours en Pologne jusqu'en 1691, où il était
extrêmement aimé et considéré, et y acquit beaucoup de réputation. Il en
partit cette année-là pour aller ambassadeur extraordinaire en Suède, et
il y mourut l'année suivante, 1692. C'était un homme d'esprit avec
beaucoup d'agréments, fait pour la société, et fort capable d'affaires.
Il avait conclu et signé avec l'électeur palatin le contrat de mariage
de Monsieur et de Madame. Il avait aussi servi, été gouverneur de
Clèves, et commandé en chef en ce pays-là. Il vivait fort
magnifiquement\,; sa manie était de se mettre entre deux draps à quelque
heure qu'il voulût faire dépêches, et ne se relevait point qu'elles ne
fussent achevées. Ses deux fils refusèrent avec une folle opiniâtreté le
cardinalat à la nomination du roi de Pologne. Ils vinrent dans la suite
mourir de faim en France. L'aîné fut tué sans alliance à la bataille
d'Hochstedt, et l'autre a vécu obscur toute sa vie. Il épousa une soeur
du duc d'Harcourt dont il n'est resté qu'une fille, qui, veuve fort
jeune sans enfants d'un frère du maréchal de Médavy, s'est remariée au
maréchal, de Belle-Ile. Son père s'est remarié à une soeur du duc de
Tresmes\,; se sont séparés fort brouillés, et il est allé vivre à
Lunéville, où le roi Stanislas l'a fait son grand chambellan\footnote{Passage
  supprimé dans les précédentes éditions depuis \emph{Il épousa}.}.
M\textsuperscript{me} de Béthune est morte à Paris en 1728 à
quatre-vingt-neuf ou dix ans. Elle avait un seul frère, qui a passé sa
vie en Pologne où il obtint l'indigénat de la république\,; c'est-à-dire
être naturalisé et rendu capable de toutes charges comme un Polonais. Il
fut capitaine des gardes du roi son beau-frère, colonel de son régiment
de dragons, et staroste\footnote{On appelait \emph{starostes} en Pologne
  les gouverneurs des villes et des châteaux. Leur dignité se nommait
  \emph{starostie}, aussi bien que le pays soumis à leur autorité.}
d'Hiedreseek, Il est mort sans alliance et sans avoir répondu au
personnage qu'il pouvait faire\footnote{Passage supprimé dans les
  précédentes éditions depuis \emph{Elle avait}.}.

Le roi Jean III Sobieski, signalé par ses victoires sans nombre contre
les Turcs et les Tartares avant et depuis son élection, couronna ses
triomphes par le salut de l'Allemagne. Il vint en personne livrer
bataille aux Turcs, qui assiégeaient Vienne et qu'ils étaient sur le
point de prendre. Leur défaite fut complète, et Vienne sauvée avec une
partie de la Hongrie, dont le héros reçut peu de gré. C'était en 1683\,;
son énorme grosseur et la conjoncture des temps l'empêcha depuis de
beaucoup faire parler de lui à la guerre. Il mourut à Varsovie le 17
juin 1696, à soixante-douze ans. Les enfants qu'il a laissés et toute
cette postérité est trop connue pour en faire mention ici. J'en dirai
seulement une vérité très certaine, et en même temps rien moins que
vraisemblable\,; c'est que si l'électeur de Bavière ne s'était pas
trouvé par sa mère cousin issu de germain de M\textsuperscript{me} de
Belle-Ile, il serait demeuré avec ce qu'il avait hérité de son père, et
ne serait parvenu à aucun des degrés de cette prodigieuse grandeur où il
est monté tout à coup. Cette singulière anecdote sera peut-être
expliquée par sa curiosité, quoiqu'elle dépasse de beaucoup le terme que
je me suis proposé.

La reine de Pologne ne fut pas à beaucoup près si française que le roi
son mari. Transportée de se voir une couronne sur la tête, elle eut une
passion ardente de la venir montrer en son pays\,; d'où elle était
partie si petite particulière. La France avait eu tant de part à cette
élection, que ce fut en reconnaissance de l'avoir procurée que le roi de
Pologne donna sa nomination au cardinal de Janson qui y était
ambassadeur de France. Il n'y avait donc nul obstacle à ce voyage qui
fut prétexté des eaux de Bourbon. Tout annoncé, tout préparé, elle fut
avertie que la reine ne lui donnerait point la main, chose qu'il était
étrange qu'elle pût ignorer. Marie Gonzague, mariée à Paris par
procureur, en présence de toute la cour, ne l'avait ni eue ni prétendue,
et plus nouvellement, le roi Casimir, qui a passé les dernières années
de sa singulière vie en France. Les rois ne l'avaient pas anciennement
chez les nôtres, et les électifs n'y ont songé en aucun temps. Le dépit
en fut néanmoins aussi grand que si elle eût reçu un affront. Elle
rompit son voyage, se lia, avec la cour de Vienne et tous les ennemis de
la France, eut grande part à la ligue d'Augsbourg contre elle, et mit
tout son crédit, qui était grand sur le roi son mari, à lui faire
épouser depuis tous les intérêts contraires à la France. Le désir
extrême qu'elle eut de faire son père duc et pair l'en rapprocha depuis,
mais les mécontentements essentiels qu'on avait reçus d'elle l'en firent
constamment refuser. Longtemps après, c'est-à-dire en 1694, elle obtint
pour lui un collier de l'ordre que le roi son gendre lui donna à Zolkiew
par commission du roi, et l'année suivante, 1695, il reçut le chapeau
auquel le roi son gendre l'avait enfin nommé au refus persévérant de ses
deux petits-fils, étant veuf pour la seconde fois dès 1692, et sans
enfants de ce mariage.

Il avait quatre-vingt-deux ans quand il fut cardinal, ne prit jamais
aucuns ordres, et n'eut jamais aucun bénéfice, en sorte qu'il né dit
jamais de bréviaire, et qu'il s'en vantait. Il fut gaillard et eut des
demoiselles fort au-delà de cet âge, ce que la reine sa fille trouvait
fort mauvais. Per sonne n'a ignoré la conduite sordide qu'elle inspira
au roi son mari dans ses dernières années, qui l'empêcha d'être
regretté, et qui fut un obstacle invincible à l'élection de pas un de
ses enfants, nonobstant l'amour des Polonais pour le sang de leurs rois,
et leur coutume de leur donner leur couronne. Tout ce qui se passa après
la mort de ce prince de sa part, et avec l'abbé de Polignac, ambassadeur
de France, se trouvera dans toutes les histoires. Enfin, détestée en
Pologne jusque de ses créatures et de ses propres enfants, elle emporta
ses trésors et se retira à Rome avec son père, et y demeurèrent dans le
même palais. Les mortifications l'y suivirent\,; elle prétendit y être
traitée comme l'avait été la reine Christine de Suède. On lui répondit,
comme autrefois on avait fait en France, qu'il n'y avait point de parité
entre une reine héréditaire et une reine élective, et on en usa avec
elle en conformité de cette différence. Cela contraignit toute sa
manière de vie, et lui donna tant d'embarras et de dépit qu'elle
n'attendait que la mort de son père pour sortir d'un lieu si
désagréable\,; elle arriva le 24 mai, à quatre-vingt-seize ans, par une
très courte maladie, ayant continuellement joui jusqu'alors de la plus
parfaites santé de corps et d'esprit. Sa fille ne tarda guère après à
exécuter ce qu'elle s'était proposé, comme nous le verrons bientôt.

La duchesse de La Trémoille mourut bientôt après n'ayant guère plus de
cinquante ans. C'était une grande, grosse et maîtresse femme, qui, sans
beaucoup d'esprit, sentait fort sa grande dame, et qui tenait de fort
court sa mère et son mari. Elle était plus que très ménagère, venait
fort peu à la cour, et ne voyait presque personne. Elle était fille
unique et très riche du duc de Créqui, qui, en la mariant, avait eu la
survivance de sa charge de premier gentilhomme de la chambre pour son
gendre. M\textsuperscript{me} de La Trémoille avait pensé épouser le duc
d'York, depuis roi d'Angleterre, Jacques II, lorsqu'il s'était retiré en
France après la catastrophe du roi son père. Ce grand mariage manqué, le
duc et le maréchal de Créqui avaient fort envie de marier leurs enfants
ensemble pour conserver ces grands biens dans leur maison, et les âges
étaient faits exprès pour cela\,; mais les frères ne furent pas les
maîtres. Quoique ce fût la fortune du marquis de Créqui que nous avons
vu tué au combat de Luzzara, et que la faveur de son oncle eût pu lui
faire tout espérer dit côté du roi, jamais la maréchale de Créqui n'y
voulut entendre. C'était une créature altière, méchante, qui menait son
mari, tout fier et tout fâcheux qu'il était, et qui n'osait la
contredire. L'éclat dont brillèrent longtemps le duc et la duchesse de
Créqui avait donné une telle jalousie à leur belle-soeur, qu'elle ne les
pouvait souffrir. Elle avait beaucoup d'esprit et poussa tellement la
duchesse de Créqui à bout, qui n'en avait point, qu'avec toute sa
douceur elle ne put s'empêcher de lui rendre haine pour haine, et de
s'opposer autant qu'elle au mariage si sage de leurs enfants. C'est
ainsi que les femmes perdent ou rétablissent les maisons par leur humeur
ou par leur bonne conduite.

Vaillac mourut en ce même temps. C'était un des bons officiers généraux
que le roi eût pour la cavalerie, et lieutenant général qui aurait été
loin, si le vin, la crapule et l'obscurité qui en sont les suites, n'eût
rendu ses talents et ses services inutiles. Il tenait beaucoup de vin,
enivrait sa compagnie et s'enivrait après. Des coquins le marièrent ivre
mort, en garnison, à une gueuse, sans qu'il sût rien de ce qu'il
faisait, sans ban, sans contrat, sans promesse. Quand il eut cuvé son
vin et qu'il fut bien éveillé, il se trouva bien étonné de trouver cette
créature couchée avec lui. Il lui demanda avec surprise qui l'avait mise
là, et ce qu'elle y faisait. La gueuse s'étonne encore plus, dit qu'elle
est sa femme\,; et prend le haut ton. Voilà un homme éperdu, qui se
croit fou, qui ne sait ce qu'on lui veut dire et qui appelle au secours.
La partie était bien liée. Il n'entend que le même langage, et ne voit
que témoins de son mariage du soir précédent. Il maintient qu'ils en ont
menti\,; qu'il n'en a pas le moindre souvenir, et aussi qu'il lui soit
jamais entré dans l'esprit de se déshonorer par un pareil mariage.
Grande rumeur. À la fin ils virent qu'il faudrait se battre ou essuyer
des coups de bâton, et l'aventure prit fin sans qu'il en ait été
question depuis.

On à donné pour véritable, qu'ayant été fort régalé par le magistrat de
Bâle, à titre de grand buveur, et les ayant tous vaincus à boire, il
leur proposa, étant monté à cheval pour, s'en aller, de boire le vin de
l'étrier\,; qu'ils firent apporter des bouteilles, et lui présentèrent
un verre\,; qu'il leur dit que ce n'était pas ainsi qu'il buvait le vin
de l'étrier, et que jetant sa botte, il l'avait fait remplir et l'avait
vidée\,; mais c'est un conte fait à plaisir, qu'on a brodé au point de
dire que ces magistrats l'avaient fait peindre en cette attitude dans
leur hôtel de ville. Son nom était Ricard\,; je ne sais pourquoi ils
aimaient mieux les noms de Gourdon et de Genouillac, qui étaient des
terres. Il venait de père en fils du frère aîné de deux maîtres de
l'artillerie, dont le second, neveu du premier, fut sénéchal d'Armagnac,
gouverneur de Languedoc, grand écuyer de France sous François Ier, et
rendit son nom célèbre sous celui de seigneur d'Acier, dont la fille
héritière porta les biens à Charles de Crussol, vicomte d'Uzès, dont les
ducs d'Uzès écartèlent deux fois leurs armes. Vaillac dont on parle ici
avait un père ami du mien, qui était un des hommes de France le mieux
faits et de la meilleure mine, brave et fort galant homme, que Monsieur
fit faire chevalier de l'ordre en 1661. Il avait toujours été à reculons
dans sa maison. Aussi n'était-ce pas un homme à être en la main du
chevalier de Lorraine. Il était premier écuyer de Monsieur, fut après
capitaine de ses gardes, enfin chevalier d'honneur de Madame, et mourut
dans cette charge en janvier 1681. Je me souviens encore d'avoir été
chez lui au Palais-Royal, avec mon père et ma mère. Je le peindrais
encore, et l'appartement en bas, au fond de la seconde cour, à droite en
entrant. Il laissa d'une Voisins une quantité d'enfants tous mal
établis, et n'en eut point de sa seconde femme, La Vergne-Tressan, qui
vient de mourir, à près de cent ans, veuve du comte de La Motte,
desquels je n'aurai que trop à dire. Le fils aîné de Vaillac ne parut
point. D'une Cambout il laissa un fils marié richement à une héritière
de Saint-Gelais, dont il a des enfants, sans avoir paru plus que son
père.

L'intrigue de la singulière nomination de l'archevêque de Bourges au
cardinalat mérite d'être rapportée. On a vu (t. II, p.~352), en parlant
du duc de Gesvres son père, qu'il avait été camérier d'honneur
d'Innocent XI, et si goûté de ce pape, qu'il n'était pas éloigné de la
pourpre, lorsque l'éclat arrivé entre le roi et Rome, sur les franchises
des ambassadeurs, en fit rappeler tous les François et perdre toute
espérance à l'abbé de Gesvres, qui en fut fait archevêque de Bourges en
arrivant. Le devenir sans avoir été évêque était une chose tout à fait
inusitée, et une compensation de ce que l'obéissance lui avait fait
abandonner. Mais cette compensation n'était rien moins qu'égale dans
l'esprit et les espérances du nouvel archevêque. Son but avait toujours
été le chapeau\,: il avait lié un grand commerce avec Torcy, qu'il avait
fort entretenu par lettres, étant à Rome. À son retour il le cultiva de
plus en plus, et parvint à devenir son ami particulier. Depuis la mort
d'Innocent XI et l'élection d'Ottobon, à qui on se hâta de sacrifier
tout, et dont on ne tira pas la moindre chose, le roi vivait en bonne
intelligence avec Rome, et l'archevêque de Bourges y avait repris ses
anciens errements avec les amis qu'il s'y était faits, sans courir de
risques par sa liaison avec Torcy. Dans cette situation, il avait
imaginé de pousser le roi d'Angleterre de tirer au moins la nomination
d'un chapeau des disgrâces qu'il essuyait pour la religion, et de le
persuader de la lui donner. Le roi le découvrit, et soit qu'il eût des
raisons pour ne vouloir pas pour lors que le roi d'Angleterre
s'embarquât dans cette prétention, soit qu'il fût piqué que l'archevêque
eût lié cette intrigue sans sa participation, il le trouva si mauvais
que la chose fut arrêtée tout court. On le sut, et on ne douta pas d'une
longue disgrâce.

L'archevêque fit quelques tours dans son diocèse, où il n'a jamais guère
été qu'à regret, ni longtemps, ni souvent. Il s'était fort italianisé à
Rome, non pas à la vérité sur l'honneur, mais pour la politique, les
manéges et les démarches sourdes et profondes, quoique avec peu
d'esprit, mais un esprit tout tourné à cela et aux agréments du monde.
Il arriva, quelque temps après cette aventure, que Stanislas reconnu
partout pour roi de Pologne, hors à Rome, en considération de la
conversion du roi Auguste lorsqu'il se fit élire, voulut essayer de s'y
faire reconnaître par sa nomination au cardinalat, et d'en faire une
affaire de couronne et de nation qui forçât le pape. On sait que les
évêques sont en Pologne les premiers sénateurs, qu'ils ne cèdent point
aux cardinaux, qu'ils ne sont point curieux de l'être, et qu'à moins
d'être en même temps cardinal et archevêque de Gnesne, qui est le
primat, à qui tout cède, un cardinal est fort embarrassé en Pologne\,:
c'est ce qui rend cette nomination si aisée à obtenir aux étrangers,
dont nos cardinaux Bonzi et de Janson ont su profiter pour y avoir été
ambassadeurs. Stanislas chercha donc un sujet qui, par lui-même, pût
aplanir les difficultés. Libre d'embarras du côté des Polonais, il
choisit un Français pour avoir l'appui de la France qui ménageait fort
le roi de Suède, et un Français supérieur des missions de Pologne, en
réputation d'un grand savoir et d'une haute piété, afin que son mérite
lui servît encore. Mais il arriva un prodige en ce genre. Le sujet se
trouva en effet si bon et si digne, qu'il refusa la nomination, et si
déterminément, qu'il fallut songer à un autre. Dans l'embarras du
nouveau choix qui répondît à ses vues de faire passer sa nomination,
Stanislas s'en remit au roi pour le gratifier, et s'assurer par là
d'autant plus du succès. Le rare est qu'à son tour le roi se trouva
embarrassé de le faire. Torcy, par qui l'affaire passait, songea à ses
deux amis, Bourges et Polignac, pressait le roi de se déterminer, de
peur que l'affaire ne s'éventât et ne mît des compétiteurs sur les
rangs, et profitant de l'indifférence du roi, lui représenta les
services de l'abbé de Polignac et la considération de l'archevêque de
Bourges à Rome\,; qu'il pouvait se souvenir que, dans la répugnance que
témoigna si longtemps le pape de faire le cardinal de La Trémoille, il
avait de lui-même insisté plusieurs fois qu'on lui demandât l'archevêque
et qu'il le ferait à l'instant.

L'éloignement du roi pour l'abbé de Polignac prévalut sur le
mécontentement de l'affaire de Saint-Germain que je viens de raconter.
Ne s'avisant d'aucun troisième, entre ces deux, il préféra l'archevêque
de Bourges. Il le proposa à Stanislas qui l'accepta, et le pape,
pressenti en conséquence, l'agréa. Dès qu'on eut réponse, non que la
nomination passerait, mais que celui dont il s'agissait était agréable,
on la déclara pour engager l'affaire, et Torcy fut bien aise en même
temps de mettre par là son ami à l'abri des retours. L'étonnement de la
cour fut extrême. On ne pouvait comprendre par quels souterrains un
homme sans nul commerce avec le nord et qui s'était mis mal avec le roi,
il n'y avait pas longtemps, pour s'être ménagé la nomination du roi
Jacques, obtenait celle du roi Stanislas avec le gré et la participation
du roi, et Torcy y acquit beaucoup d'honneur de savoir si lestement
servir ses amis et se donner un cardinal. Cette espérance, néanmoins,
s'en alla en fumée avec le règne de Stanislas. Nous verrons l'archevêque
lutter encore bien des années contre la fortune, et n'obtenir le prix de
tant de désirs, de soins et de veilles, car il ne le perdit jamais de
vue un seul instant, qu'en 1719, après en avoir tant vu passer devant
lui\,: dès 1713, Polignac, à qui il avait été préféré, et par le détour
d'Angleterre qui lui avait rompu aux mains seize ou dix-sept ans avant
que d'arriver, et tant d'autres qui alors ne pouvaient pas seulement y
penser, tel qu'un Bissy qu'il avait si longtemps regardé, pour parler
avec M. de Noyon, comme un évêque du second ordre, promu pourtant quatre
ans devant lui, et tant d'autres comme Dubois, Fleury, qu'il ne
regardait pas.

\hypertarget{chapitre-iv.}{%
\chapter{CHAPITRE IV.}\label{chapitre-iv.}}

1707

~

{\textsc{Campagne de Flandre.}} {\textsc{- Paresse dangereuse de
Vendôme.}} {\textsc{- Belle campagne du Rhin.}} {\textsc{- Pillages et
audace de Villars.}} {\textsc{- Ragotzi proclamé prince de
Transylvanie.}} {\textsc{- L'empereur humilié par le roi de Suède, qui
passe en Russie.}} {\textsc{- Expéditions heureuses à la mer.}}
{\textsc{- Tempête fatale en Hollande.}} {\textsc{- Ravages de la Loire
et leur cause.}} {\textsc{- Expédition du duc de Savoie en Provence et à
Toulon.}} {\textsc{- Conduite de l'évêque de Fréjus avec le duc de
Savoie.}} {\textsc{- Digression curieuse sur ce prélat, devenu cardinal
et maître du royaume.}} {\textsc{- Mesures pour la défense de Toulon et
de la Provence.}} {\textsc{- Retraite de M. de Savoie de Provence.}}

~

Le duc de Marlborough, arrivé à la Haye d'assez bonne heure, en était
reparti pour aller visiter les électeurs de Saxe et de Brandebourg et le
duc d'Hanovre. Pendant ce temps le duc de Vendôme était à Mons qui
prenait du lait. Vers la fin de mai les armées s'assemblèrent et la
campagne se commença. Vendôme, en apparence sous l'électeur de Bavière,
mais en effet à peine sous le roi même, coulait les jours sur sa chaise
percée, au jeu, à table, comme je l'ai représenté (t. V, p.~134 et
suiv.)\,; et comme il s'était rendu incapable désormais de pouvoir faire
autrement, il ne songeait qu'à jouir d'une gloire qu'il n'avait jamais
acquise, et d'honneurs qu'il arrachait, comme que ce pût être, laissant
à l'électeur la permission de jouer le plus gros jeu, et à Puységur tout
le faix de l'armée, dont il n'entendait jamais parler. Ainsi se passa
toute cette campagne, dont il pensa payer la mollesse chèrement.
Paresseux à son ordinaire de décamper et n'en voulant croire personne,
il eut tout à coup l'armée ennemie sur les bras. Puységur le lui avait
prédit sans avoir jamais pu rien gagner sur lui. L'affaire pressa, elle
devenait instante, il alla pour l'avertir, mais ses valets avaient
défense de laisser entrer pour quelque chose que ce fût. Puységur fut à
l'électeur, qui passa la nuit debout, et qui, lassé de l'inutilité de
ses messages dont pas un ne put aborder, alla lui-même forcer les
portes, éveiller Vendôme et lui dire le péril de son retardement.
Vendôme l'écouta en bâillant, et pour toute réponse lui dit que cela
était le mieux du monde, mais qu'il fallait qu'il dormît encore deux
heures, et tout de suite se tourna de l'autre côté.

L'électeur outré sortit et n'osa donner aucun ordre. Cependant les avis
redoublant de toutes parts de l'arrivée imminente des ennemis sur
l'armée, Puységur prit sur soi de faire sonner boute-selle, détendre et
charger, puis avertit le duc de Vendôme, qui persista à ne vouloir rien
croire, mais qui, sachant l'armée prête à marcher, s'habilla enfin et
monta à cheval, comme elle était déjà ébranlée. Il en était temps.
L'arrière-garde fut incontinent harcelée par l'avant-garde des ennemis,
et toute l'armée se fut mal tirée d'une si profonde négligence, si le
bonheur n'eût voulu que cette tête des ennemis se fût perdue la nuit par
la faute de ses guides, et n'eût, de plus, été très malhabilement menée
par ce déserteur de prince d'Auvergne qui la commandait. Quelque temps
après, dans la même campagne, M. de Vendôme pensa être enlevé, disputant
contre toute évidence, et se voulant croire en sûreté partout où il se
trouvait logé à son gré. Marlborough fit contenance de le vouloir
combattre, lui eut la liberté de s'y présenter\,; tout se passa en
propos et en subsistances. Après les tristes succès qui avaient précédé
en Flandre, on n'avait pas dessein de s'y commettre sans nécessité, et
Marlborough content des leurs en Italie, en attendait de si grands
fruits et si promptement, qu'il ne jugea pas à propos de rien risquer en
Flandre, dans des moments où il comptait que le royaume allait être pris
à revers sans aucun moyen de défense. La campagne se passa donc de la
sorte en Flandre. La fin ennuya M. de Vendôme\,; il la voulut hâter, et
il sépara son armée. Celle des ennemis demeura ensemble plus de huit
jours après, et causa par là une grande inquiétude. Mais tout était bon
de M. de Vendôme, tout permis. Il arriva à la cour, et il y fut reçu à
merveilles.

Le maréchal de Villars passa le Rhin de bonne heure. Il eut affaire
cette année au marquis de Bayreuth, qui commanda l'armée de l'empereur
jusque vers la fin de septembre, que le duc d'Hanovre, depuis roi
d'Angleterre, en vint prendre le commandement, et trouva le marquis
parti, qui ne voulut pas l'attendre. Villars fit passer en même temps
que lui Peri par l'île du Marquisat, Vivans par Lauterbourg, et Broglio
plus bas, à Neubourg. Il n'y eut d'opposition nulle part, et cependant
le maréchal marcha aux lignes de Bihel et de Stollhofen. Il n'y trouva
personne. Tout avait fui à son approche. Leurs tentes étaient demeurées
tendues, et ils avaient abandonné presque tout leur bagage et beaucoup
de canon sur les retranchements. Cela se passa le 23 mai, et Beaujeu en
vint apporter la nouvelle. Le roi en fut aise, jusqu'à une sorte
d'engouement. Dans la suite de la campagne Villars se rendit maître du
château d'Heidelberg et de cette capitale de l'électeur palatin, de
Mannheim et de tout le Palatinat. Profitant de la faiblesse des
Impériaux, il se hâta de pénétrer en Allemagne avant qu'on se pût
opposer à lui. Il entra en Franconie, se fit rendre par la ville d'Ulm
d'Argelot, brigadier, et grand nombre d'autres prisonniers retenus là de
la bataille d'Hochstedt, et tira d'ailleurs avec une facilité
merveilleuse autres huit cents prisonniers d'Hochstedt, trente-cinq
pièces de canon, et grande abondance de vivres et de munitions de
guerre. En même temps, il n'oublia pas les contributions. Outre les
sommes immenses qu'il avait tirées du Palatinat et des pays de Bade et
de Würtemberg, il poussa Broglio par la Franconie, Imécourt et La
Vallière par l'autre côté du Danube. Il en eut des trésors par delà
toute espérance. Gorgé ainsi au conspect de toute l'Allemagne et de
toute son armée, il n'espéra pas qu'un si prodigieux brigandage pût
demeurer inconnu. Il paya d'effronterie et manda au roi qu'il avait fait
en sorte que son armée ne lui coûterait rien de toute la campagne, mais
qu'il espérait aussi qu'il ne trouverait pas mauvais qu'elle aidât à le
défaire d'une petite montagne qui lui déplaisait à Villars. Un autre que
lui aurait été déshonoré d'une part, perdu de l'autre. Cela ne fit pas
le plus petit effet contre lui, sinon du public dont il ne se mit guère
en peine. Ses rafles faites, il ne songea plus qu'à se tirer du pays
ennemi et à repasser le Rhin.

Le duc d'Hanovre, en joignant l'armée impériale à la fin de septembre
qui s'était grossie, trouva tous ces pays dans le dernier désespoir. Il
essaya donc d'embarrasser Villars dans son retour pour tâcher à
l'écorner et à lui faire rendre gorge. Vivans, lieutenant général, se
trouva campé près d'Offenbourg avec quinze escadrons, Mercy prit par
derrière les montagnes avec trois mille chevaux, fit plus de trente
lieues en quatre jours, et par un grand brouillard tomba à la pointe du
jour sur Vivans, qui n'en avait eu nul avis. Il monta à cheval,
rassembla à peine huit cents chevaux, mit la petite rivière entre les
ennemis et lui, et fit ferme. Ils ne l'attaquèrent point et se
contentèrent de piller le camp, les chevaux et les bagages, et Vivans,
avec ce qui l'avait pu rejoindre, s'alla mettre sous Kehl. Villars eut à
bricoler pour regagner le Rhin\,; à la fin il y réussit sans
mésaventure. Il le passa tranquillement avec son armée et son immense
butin, et dès qu'il fut en deçà ne songea plus qu'à terminer la campagne
en repos. Ainsi finit une assez belle campagne, si le gain sordide et
prodigieux du général ne l'avait souillée, qui à son retour n'en fut pas
moins bien reçu du roi.

Au commencement de l'été, Ragotzi avait été proclamé prince de
Transylvanie, et avait fait en cette qualité une magnifique entrée dans
la capitale, et bientôt après l'empereur essuya un autre grand dégoût.

L'envoyé de Suède, dans la brillante posture où nous avons vu naguère le
roi son maître en Saxe, demandait avec hauteur la restitution de
quantité d'églises de Silésie que l'empereur avait ôtées aux
protestants, et un grand nombre de Moscovites qui s'y étaient sauvés,
qu'on avait envoyés vers le Rhin pour les dépayser. Des demandes si
nouvelles à la hauteur de la cour de Vienne éprouvèrent force lenteurs.
L'envoyé de Suède parlait avec audace, on chercha à le mortifier\,; on
lui fit des chicanes sur l'audience des archiduchesses, et le comte de
Zabor, grand chambellan de l'empereur, lui refusa le salut dans
l'antichambre de ce prince. L'envoyé se plaignit de l'insulte\,; la
réponse fut que le respect du lieu défendait d'y en rendre à personne.
Le roi de Suède ne tâta point de ce subterfuge\,; il éclata et il
ordonna à son envoyé de partir sans prendre congé, s'il ne recevait la
satisfaction qu'il avait prescrite\,; la cour de Vienne alors craignit
qu'il ne se jetât ouvertement à la France et céda. Tout cela fut long à
terminer, mais à la fin l'envoyé eut l'audience contestée en la manière
qu'il l'avait prétendue, la restitution des Moscovites et des églises de
Silésie accordée, et le comte de Zabor destitué, arrêté et envoyé en
Saxe au roi de Suède, sans stipulation, pour faire de lui tout ce qu'il
lui plairait. Il tint le comte dans une rude prison et le renvoya après
à Vienne, lui faisant fort valoir, et plus encore à l'empereur, de lui
avoir fait grâce de la vie et de la liberté. En arrivant à Vienne, sa
charge, qui n'avait pas été remplie, lui fut rendue\,; mais s'étant
trouvé quelque temps après en même lieu que cet envoyé de Suède, qui
s'appelait le baron de Strahlenheim, c'est-à-dire à Breslau où Zabor
l'alla chercher, Zabor lui demanda raison de ce qu'il avait souffert à
cette occasion, et de ne l'avoir pu avoir du soufflet qu'il avait reçu
de lui. Ils se battirent, mais on a prétendu que sans avoir rien dit, ni
demandé aucune raison, Zabor assassina Strahlenheim, qui était là en
fonction pour les affaires du roi de Suède son maître. Pour la
restitution des Moscovites et celle des églises de Silésie, qui avait si
longtemps traîné, le roi de Suède partit pour la Pologne, et tout de
suite pour sa malheureuse expédition de Moscovie avant qu'elle fût
exécutée, et dès qu'il fut hors de Saxe l'empereur ne le craignit plus,
et les restitutions ne furent jamais faites.

Tout de suite Rabutin rentra en Transylvanie, fit lever aux mécontents
le blocus de Deva, et l'empereur, profitant de ce succès, fit faire à
Ragotzi de nouvelles propositions d'accommodement par les ministres de
Hollande et d'Angleterre\,; mais le nouveau prince de Transylvanie
répondit que les Hongrois avaient déclaré leur trône vacant et qu'il ne
pouvait plus traiter avec l'empereur. Ce prince en ce même temps rendit
ses bonnes grâces au prince de Salm, qui s'était retiré mécontent, et
qui avait été gouverneur du roi des Romains et fait son mariage avec la
princesse d'Hanovre, dont la mère était soeur de M\textsuperscript{me}
la Princesse et de sa défunte femme. Il était très bien avec eux\,; une
intrigue de cour l'avait déposté. L'empereur lui rendit la présidence du
conseil et sa charge de grand maître de la cour du roi des Romains.

Forbin se signala à la mer cette année. Avec des vaisseaux plus faibles
que les quatre Anglais de soixante-dix pièces de canon, qui convoyaient
une flotte de dix-huit vaisseaux chargés de munitions de guerre et de
bouche, qu'il trouva sur les côtes d'Angleterre, comme il sortait de
Dunkerque, il prit deux vaisseaux de guerre qu'il amena à Dunkerque,
ainsi que les dix-huit vaisseaux marchands, après quatre heures de
combat, et mit le feu à un des deux autres vaisseaux de guerre. Trois
mois après il prit, à l'embouchure de la Dwina, dix-sept vaisseaux
marchands Hollandais richement chargés pour la Moscovie. Il en prit ou
coula à fond plus de cinquante pendant cette campagne. Depuis ce calcul,
il prit encore trois gros vaisseaux de guerre anglais qu'il amena à
Brest, coula à fond un autre de cent pièces de canon de cinq qu'ils
étaient à convoyer une flotte marchande en Portugal, sur laquelle il
lâcha nos armateurs, qui y firent bien leurs affaires et celles de M. le
comte de Toulouse. Les Anglais de la Nouvelle-Angleterre et de la
Nouvelle-York ne furent pas plus heureux à l'Acadie\,: ils attaquèrent
notre colonie douze jours durant sans succès, et furent obligés à se
retirer avec beaucoup de perte.

L'année marine finit par une tempête terrible sur les côtes de Hollande,
qui fit périr beaucoup de vaisseaux au Texel, et submergea beaucoup de
pays et de villages. La France eut aussi sa part du fléau des eaux\,: la
Loire se déborda d'une manière jusqu'alors inouïe, rompit les levées,
inonda et ensabla beaucoup de pays, entraîna des villages, noya beaucoup
de monde et une infinité de bétail, et fit pour plus de huit millions de
dommages. C'est une obligation de plus qu'on eut à M. de La Feuillade,
qui du plus au moins s'est perpétuée depuis. La nature plus sage que les
hommes, ou, pour parler plus juste, son auteur, avait posé des rochers
au-dessus de Roanne dans la Loire, qui en empêchaient la navigation
jusqu'à ce lieu, qui est le principal du duché de M. de La Feuillade.
Son père, tenté du profit de cette navigation, les avait voulu faire
sauter. Orléans, Blois, Tours, en un mot tout ce qui est sur le cours de
la Loire, s'y opposa. Ils représentèrent le danger des inondations, ils
furent écoutés\,; et quoique M. de La Feuillade alors fût un favori et
fort bien avec M. Colbert, il fut réglé qu'il ne serait rien innové et
qu'on ne toucherait point à ces rochers. Son fils, par Chamillart son
beau-père, eut plus de crédit. Sans écouter personne, il y fut procédé
par voie de fait\,; on fit sauter les rochers, et on rendit la
navigation libre en faveur de M. de La Feuillade\,; les inondations
qu'ils arrêtaient se sont débordées depuis avec une perte immense pour
le roi et pour les particuliers. La cause en a été bien reconnue après,
mais elle s'est trouvée irréparable.

Le peu d'effort que les ennemis avaient fait en Flandre et en Allemagne
avait une cause qui commença d'être aperçue vers la mi-juillet. Le
prince Eugène, qui avait eu la gloire de nous chasser totalement
d'Italie, y était demeuré, et entra dans le comté de Nice. Sailly,
lieutenant général, qui y commandait quelques troues, se retira en deçà
du Var, qui sépare la Provence de ce comté, et qui se trouva lors
débordé\,; et Parat, maréchal de camp, qui avait commandé l'hiver à
Nice, se retira à Antibes. Le duc de Savoie entra dans Nice n'ayant
encore que six ou sept mille hommes de ses troupes avec lui\,; et la
flotte ennemie, de quarante vaisseaux de guerre, commença à y débarquer
de l'artillerie. Alors le duc de Marlborough ne cacha plus la cause de
son inaction. Il s'expliqua de l'entreprise comme immanquable, et devant
entraîner les plus grandes suites, et qu'il attendrait pour agir
offensivement que l'entreprise sur Toulon eût réussi. Ce projet n'était
pas conçu depuis peu par M. de Savoie, il l'avait formé lors de la
guerre précédente qui fut terminée à Ryswick. Il dit aux principaux de
la flotte qui l'allèrent saluer à Nice qu'il était bien aise de les
voir, mais qu'il y avait quatorze ans qu'il les avait attendus au même
lieu. Il arriva le 18 à Fréjus.

L'évêque, qui nous gouverne aujourd'hui si fort en plein et sans voile
sous le nom de cardinal Fleury, le reçut dans sa maison épiscopale,
comme il ne pouvait s'en empêcher. Il en fut comblé d'honneurs et de
caresses, et {[}le duc de Savoie{]} l'enivra si parfaitement par ses
civilités, que le pauvre homme, également fait pour tromper et pour être
trompé, prit ses habits pontificaux, présenta l'eau bénite et l'encens à
la porte de sa cathédrale à M. de Savoie, et y entonna le \emph{Te Deum}
pour l'occupation de Fréjus. Il y jouit quelques jours des caresses
moqueuses de la reconnaissance de ce prince pour une action tellement
contraire à son devoir et à son serment qu'il n'aurait osé l'exiger. Le
roi en fut dans une telle colère, que Torcy, ami intime du prélat, eut
toutes les peines imaginables de le détourner d'éclater. Fréjus qui le
sut, et qui, après coup, sentit sa faute et quelle peine il aurait à en
revenir auprès du roi, trouva fort mauvais que Torcy ne la lui eût pas
cachée, comme s'il eût été possible qu'une démarche si étrange et si
publique, et dont M. de Savoie s'applaudissait, ne fût pas revenue de
mille endroits\,; et ce que Fréjus pardonna le moins au ministre fut la
franchise avec laquelle il lui en parla, comme s'il eût pu s'en
dispenser, et comme ami et comme tenant la place qu'il occupait.
L'évêque, flatté au dernier point des traitements personnels de M. de
Savoie, le cultiva toujours depuis\,; et ce prince, par qui les choses
les plus apparemment inutiles ne laissaient pas d'être ramassées,
répondit toujours de manière à flatter la sottise d'un évêque frontière,
duquel il pouvait peut-être espérer de tirer quelque parti dans une
autre occasion. Tout cela entre eux se passa toujours fort en secret,
mais dévoua l'évêque au prince. Tout cela, joint à l'éloignement du roi
marqué pour lui et à la peine extrême qu'il avait montrée à le faire
évêque, n'était pas le chemin pour être choisi par lui pour précepteur
de son successeur.

Devenu premier ministre au point d'autorité sans partage avec laquelle
il règne seul et en chef publiquement depuis seize ans, il n'oublia ni
sa rancune contre Torcy, à qui il l'avait si soigneusement cachée depuis
ses premières plaintes, ni son attachement à M. de Savoie. Dès
auparavant, il lui rendait un compte assidu de tout ce qui regardait
l'éducation du roi\,; il me l'a dit à moi-même en s'écriant que c'était
un devoir, que M. de Savoie était son, grand-père, qu'il n'avait de
parents que lui. Premier ministre, il le consulta sur les affaires, il
s'ouvrit de tout avec lui pendant deux ans. Il me le fit entendre
encore, mais sans s'en expliquer aussi nettement qu'il avait fait sur
l'éducation. «\,C'est son grand-père, me dit-il encore\,; le roi est
tout jeune\,; on est en paix\,; M. de Savoie est le plus habile prince
de l'Europe\,; il est mon ami intime\,; il m'a voulu faire précepteur de
son fils, j'ai sa confiance depuis longtemps\,; il ne peut que prendre
grand intérêt au roi. Qui pourrais-je consulter plus utilement et plus
raisonnablement en Europe\,?» À la fin pourtant il s'aperçut que c'était
M. de Savoie qui avait sa confiance, mais qu'il n'avait pas la sienne,
qu'il en abusait et qu'il le trompait cruellement. L'amour-propre fut
longtemps à se convaincre, mais à la fin il le fut, et vit tout d'un
coup d'oeil le précipice qu'il s'était creusé. Il se tut pour ne pas
faire éclater une si lourde duperie, mais il rompit et ne lui pardonna
jamais. Il le lui rendit bien à son emprisonnement par son fils. Jamais
il ne souffrit que le roi fît la moindre démarche, le moindre office
même, pour ce grand-père, pour ce parent unique. Il ne put dissimuler sa
joie de se voir vengé. Ce n'est pas ici le lieu de dire comment il fit
de même le tour de l'Europe, et comment, ni jusqu'à quel point
l'Angleterre très longtemps, l'empereur ensuite, M. de Lorraine, enfin
la Hollande ont utilement pour eux entretenu sa plus aveugle confiance
et cruellement abusé de sa crédulité. J'en rapporterai seulement ici
quelques traits, parce que ces temps dépassent celui où je me suis
proposé de me taire, et qu'ils sont trop curieux pour les omettre,
puisqu'ils peuvent trouver place si naturellement ici.

Il faut se souvenir de la fameuse aventure qui pensa culbuter M. de
Fréjus. Il était toujours présent au travail particulier de M. le Duc,
qu'il avait fait premier ministre à la mort de M. le duc d'Orléans, pour
lui en donner l'écorce et en retenir la réalité pour soi. M. le Duc,
poussé par sa fameuse maîtresse, M\textsuperscript{me} de Prie, voulut
le déposter et travailler seul avec le roi. Il venait de faire son
mariage et pouvait tout sur la reine, qui fit que le roi vint chez elle
un peu avant l'heure de son travail. M. le Duc s'y rendit avec son
portefeuille, tandis que M. de Fréjus attendait dans le cabinet du roi.
Lassé d'y avoir croqué le marmot une heure, il envoya voir chez la reine
ce qui y pouvait retenir le roi si longtemps. Il apprit qu'il y
travaillait seul avec elle dans son cabinet, et M. le Duc, où elle
n'avait pourtant été qu'un peu en tiers. M. de Fréjus, qui connaissait
ce qu'il pouvait sur le roi, s'en alla chez lui, et dès le soir même
s'en alla à Issy, d'où il envoya une lettre au roi qui eut l'effet et
fit le bruit que chacun a su. Robert Walpole gouvernait alors
l'Angleterre comme il la gouverne encore\,; et Horace, son frère, était
ambassadeur ici, qui l'a été si longtemps. Dès le lendemain matin il
alla voir M. de Fréjus à Issy, dans le temps qu'on ignorait encore s'il
était perdu sans retour et chassé, ou si le roi, malgré M. le Duc, le
rappellerait et se servirait de lui à l'ordinaire. M. de Fréjus fut si
touché de la démarche de ce rusé Anglais dans cette crise, qu'il le crut
son ami intime. L'ambassadeur n'y risquait rien et n'avait point à
compter avec M. le Duc si M. de Fréjus demeurait exclu\,; que, s'il
revenait en place, c'était un trait à lui faire valoir et à en tirer
parti. Aussi fit-il, et plusieurs années.

Devenu premier ministre, après avoir renversé M. le Duc et
M\textsuperscript{me} de Prie, auxquels il ne pardonna jamais, non plus
qu'à la reine, la peur qu'ils lui avaient faite, il s'abandonna
entièrement aux Anglais, avec une duperie qui sautait aux yeux de tout
le monde. Je résolus enfin de lui en parler, et on verra en son temps
combien j'en étais à portée, et pourquoi j'en suis demeuré là. Je lui
dis donc un jour ce que je pensais là-dessus, les inconvénients solides
dans lesquels il se laissait entraîner, et beaucoup de choses sur les
affaires qui seraient ici déplacées. Sur les affaires il entra en
matière\,; mais sur sa confiance en Walpole, en son frère et aux Anglais
dominants, il se mit à sourire. «\,Vous ne savez pas tout, me
répondit-il\,; savez-vous bien ce qu'Horace a fait pour moi\,?» et me
fit valoir cette visite comme un trait héroïque d'attachement et
d'amitié, qui levait pour toujours tout scrupule. Puis continuant\,:
«\,Savez-vous, me dit-il, qu'il me montre toutes ses dépêches, que je
lui dicte les siennes, qu'il n'écrit que ce que je veux\,? voilà un
intrinsèque qu'on ignore, et que je veux bien vous confier. Horace est
mon ami intime, il a toute confiance en moi\,; mais je dis, aveugle.
C'est un très habile homme, il me rend compte de tout\,; il n'est qu'un
avec Robert, qui est un des plus habiles hommes de l'Europe, et qui
gouverne tout en Angleterre. Nous nous concertons, nous faisons tout
ensemble et nous laissons dire.\,» Je demeurai stupéfait, moins encore
de la chose que de l'air de complaisance et de repos, et de
conjouissance en lui-même avec laquelle il me le disait. Je ne laissai
pas d'insister, et de lui demander qui l'assurait qu'Horace ne reçût et
n'écrivit pas doubles dépêches, et ne trompât ainsi bien aisément\,?
Autre sourire d'applaudissement en soi\,: «\,Je le connais bien, me
répondit-il, c'est un des plus honnêtes hommes, des plus francs et des
plus incapables de tromper qu'il y ait peut-être au monde.\,» Et de là à
battre la campagne en exemples et en faits dont Horace l'amusait. Le
dénouement de la pièce fut qu'après s'être servis de la France contre
l'Espagne, et contre elle-même, pour leur commerce et pour leur
grandeur, et l'avoir amusée jusqu'au moment de la déclaration de cette
courte guerre de 1733, les Walpole, ses confidents, ses chers amis, qui
n'agissaient que par ses ordres et ses mouvements, se moquèrent de lui
en plein parlement, l'y traitèrent avec cruauté\,; et de point en point
manifestèrent toute la duperie, et l'enchaînement de lourdises où, à
leur profit et à notre grand dommage, ils avaient fait tomber six ans
durant notre premier ministre, qui en conçut une rage difficile à
exprimer\,; mais elle ne le corrigea pas.

Il se jeta à M. de Lorraine, l'ennemi né de la France, et par lui à
l'empereur. Ce prince, esclave de sa grandeur et de sa gravité, ne se
prêtait pas autant que le voulait M. de Lorraine, qui plus près de notre
cour, et par les gens à lui qu'il y avait, la connaissait à revers\,:
Lecheren qui, par mille intrigues de tous les pays, s'était assuré d'un
chapeau du roi Auguste, et l'avait comme perdu par le dérèglement de sa
conduite, le vendit au comte de Zinzendorf pour son fils, qui n'avait
que vingt-trois ou vingt-quatre ans, et qui, appuyé de l'empereur et du
prétexte de la nomination de Pologne, l'attrapa. Lecheren en eut
beaucoup d'argent comptant, l'évêché de Namur, promesse de mieux, et
toute entrée d'affaires auprès de l'empereur, que Zinzendorf gouvernait
alors. Il connaissait notre terrain aussi bien que M. de Lorraine\,; il
fut à son secours, et fit tant auprès de l'empereur, qu'il le persuada
enfin d'écrire de sa main au cardinal de Fleury, de lui faire des
caresses, de l'accabler de louanges et de confiance, de lui témoigner
qu'il se voulait conduire par lui, pour la grande estime qu'il avait
conçue de sa probité et de sa capacité. Le cardinal se sentit transporté
de joie\,; il n'avait peut-être jamais su le manège pareil de
Charles-Quint avec le cardinal Wolsey. Il s'entête de l'empereur et de
M. de Lorraine de plus en plus, à qui il crut devoir toute cette
confiance, fit tout pour ce dernier, et ce fut par lui désormais que le
commerce de lettres passa de lui à l'empereur et de l'empereur à lui, de
leur main et à l'insu de nos ministres et des plus intimes secrétaires
du cardinal, qui ne voyaient que les dos de ces lettres.

J'eus encore la sottise de l'avertir qu'il était trompé. Il me conta
avec ce même air de complaisance et de confiance, ce commerce de
lettres\,: «\,et sans façons, m'ajouta-t-il, je lui écris rondement,
franchement ce que je pense. Il me répond avec une amitié, une
familiarité, une déférence, pour cela, la plus grande du monde\,;» et se
mit à entrer en affaires, mais moins solidement qu'il n'avait fait sur
l'Angleterre, et battit un peu de campagne. Cette courte guerre ne put
lui dessiller les yeux. Il crut avoir fait la paix à son mot par sa
considération personnelle. Il me la conta à Issy, comme je revenais de
la Ferté. «\,Et la Lorraine, lui dis-je, est-ce que vous ne la stipulez
pas\,?» Mon homme s'embarrassa, et me dit que Campredon s'était trop
avancé, et avait signé contre ses ordres. «\,Mais la Lorraine\,?
ajoutai-je. Mais la Lorraine\,! me dit-il, ils n'ont jamais voulu la
céder, Campredon a signé, nous n'avons pas voulu le désavouer, c'était
chose faite.\,» Alors je lui représentai avec force la suite de la
pragmatique\footnote{Les lois constitutives de l'Allemagne portaient le
  nom de \emph{pragmatique ou pragmatique sanction}. Ainsi la bulle d'or
  de 1356 est désignée sous le nom de \emph{pragmatique sanction}, de
  même que l'ordonnance de 1713 relative à l'ordre de succession dans
  les États autrichiens.} qu'il garantis soit, l'étrange danger d'un
empereur duc de Lorraine, qui fortifierait cet État, y entretiendrait
des troupes, couperait l'Alsace et la Franche-Comté, nous obligerait de
faire à neuf une frontière aux Évêchés\footnote{On appelait \emph{les
  Évêchés} ou \emph{les Trois-Évêchés}, dans l'ancienne France, les
  villes et territoires de Toul, Metz et Verdun.} et en Champagne, si
nous voulions éviter de le voir dans Paris quand il voudrait\,; que, si
on se contentait de promesses, il avait l'exemple de Ferdinand le
Catholique avec Louis XII, et de Charles-Quint avec François Ier, avec
l'extrême différence qu'en se départant des prétentions d'Italie, ces
princes demeuraient en repos et en sûreté de ce côté-là, avec les Alpes
et les États de Savoie entre-deux, au lieu que la position de la
Lorraine nous tenait dans un danger imminent et continuel. Ce discours
plus étendu et fort appuyé qu'il écouta, tant que je voulus le pousser,
sans m'interrompre, avec grande attention, le jeta dans une rêverie
profonde qui, après que j'eus achevé, nous tint tous deux assez
longtemps en silence. Il le rompit le premier pour parler d'autre chose.
Un mois après, je sus qu'on nous cédait la Lorraine en plein et pour
toujours\,; j'en fus ravi, et j'avoue que je crus en être cause, mais je
me gardai bien de dire un seul mot qui le pût faire soupçonner.
L'admirable est que, depuis, jamais le cardinal et moi ne nous sommes
parlé de la Lorraine.

On a vu à la mort de l'empereur, duquel jusqu'alors le cardinal fut
toujours pleinement la dupe, tous les traités faits et signés par lui
contre nous, et la même guerre au moment d'éclore, sous laquelle Louis
XIV avait été au moment de succomber. Les bassesses de Zinzendorf à
Soissons, le consentement de l'empereur pour son chapeau, avant la
promotion des couronnes, avaient préparé les voies, dont Lecheren et M.
de Lorraine surent si dangereusement profiter un mois avant là mort de
l'empereur, laquelle fit avorter en même temps que découvrir cette ligue
toute dressée, et à l'instant d'agir. Schmerling qui faisait tout ici
pour l'empereur, tandis que le prince de Lichtenstein y était
ambassadeur de splendeur et de parade, donna dans l'antichambre du
cardinal, et publiquement devant tout le monde, une riche chaîne d'or
avec la médaille de l'empereur de sa part à Barjac, valet de chambre
principal du cardinal, et que tout le monde a connu pour sa familiarité
et son crédit avec lui, et lui fit les remercîments de ce prince, des
soins qu'il prenait de la santé de son maître, et que c'était pour l'en
remercier et l'exhorter à continuer, que l'empereur lui faisait ce
présent. Barjac le reçut, le cardinal fut charmé, et toute la cour en
silence et bien étonnée. Pour conclusion, Vanhoey, ambassadeur de
Hollande, s'était insinué fort avant dans son esprit par ces cajoleries.
Il le goûtait fort, il s'abandonna à lui à cette époque de la mort de
l'empereur. Il crut disposer de la Hollande, et il fut constamment
entretenu dans cette erreur jusqu'au moment que la dernière révolution
de Russie en faveur d'Élisabeth a manifesté la quadruple alliance de
l'Angleterre, de la cour de Vienne, du Danemark et de la Russie, où le
courrier qui en portait les ratifications à Pétersbourg y trouva toute
la face changée, ceux à qui il la portait tombés du trône et
prisonniers, et Élisabeth, jusqu'alors honnêtement prisonnière, portée à
leur place sur ce même trône. En voilà assez, et peut-être trop, pour la
curiosité qui m'a entraîné en cette digression\,; retournons en
Provence.

Tessé y était accouru de Dauphiné, où il avait laissé Médavy. Il avait
rassemblé vingt-neuf bataillons. Saint-Pater commandait dans Toulon, où
il n'avait que deux bataillons, et quatre formés des troupes de la
marine. On y travailla à force, et surtout à un grand retranchement tout
à fait au dehors, à la faveur des précipices, où Goesbriant fut destiné
avec les cinq bataillons qu'avait eus Sailly dans Nice. Il est certain
que tout ce qui se trouva là d'officiers généraux et particuliers,
jusqu'aux soldats, firent des prodiges à avancer ce vaste retranchement
sur les hauteurs de Sainte-Catherine, pour éloigner les attaques à la
ville le plus qu'il se pourrait, et fondèrent toutes leurs espérances
sur sa défense. Toulon ne valait rien, et jusqu'alors on n'y avait rien
fait. Le Languedoc n'était pas paisible, toutes ces provinces ouvertes
sans aucune place. Tessé présidait médiocrement à ces travaux, il
voltigeait de côté et d'autre pour donner ordre à tout\,; il laissait
agir, et se réservait le droit de faire les difficultés qui lui étaient
suggérées. Rien de plus dissemblable à Anne de Montmorency, en cas à peu
près pareil, et sur le même théâtre. Les disputes ralentirent les
ouvrages, et Tessé les décidait peu. La marine, qui y fit merveilles de
la main et de la tête, désarma tous les bâtiments, en enfonça à l'entrée
du port pour le boucher\,; mais, prévoyant qu'il n'était pas possible de
garantir les navires d'être brûlés, on en mit dix-sept sous l'eau, qui,
bien {[}que{]} relevés dans la suite, fut une grande perte.

M. de Savoie avait visité la flotte devant Nice, et demanda l'argent qui
lui était promis. Les Anglais craignirent d'en manquer, et disputèrent
une journée entière au delà du temps fixé pour le départ. À la fin,
voyant ce prince buté à ne bouger de là qu'il ne fût payé, ils lui
comptèrent un million qu'il reçut lui-même. Cette journée de retardement
fut le salut de Toulon, et on peut dire de la France. Elle donna le
temps à vingt et un bataillons d'arriver à Toulon. Ils y entrèrent le
23, le 24 et le 25. Tessé les y vit lui-même, et de là s'en fut à Aix.
Cela fit le nombre de quarante bataillons, dont on mit trente-quatre au
retranchement de Sainte-Catherine. Le chevalier de Sebeville, chef
d'escadre, y périt dans un précipice en voulant monter par un chemin
trop difficile, et ce fut grand dommage sur mer et sur terre. À la
sécurité parfaite sur ces provinces éloignées succédèrent toutes les
offres de voir prendre le royaume à revers. Chamarande eut ordre de ne
laisser qu'une faible garnison dans Suse, et de mener en Provence toutes
les troupes qu'il avait. Cependant M. de Savoie avec le prince Eugène
étaient arrivés à Valette le 26, à une lieue de Toulon, et ils
commencèrent le 30 à attaquer des postes. Le vent contraire empêchait
toujours le débarquement des vivres et de l'artillerie. Cela retardait
les attaques, et mettait la cherté et la désertion dans leur armée. On
tâchait à se mettre en état de profiter du temps par de gros
détachements des armées de Flandre, d'Allemagne et d'Espagne\,; mais aux
plus éloignés, il y avait pour plus de cinquante jours de marche. Tessé
eut encore vingt bataillons qu'il fit camper aux portes de Toulon, et
finalement le 13 août le roi déclara dans son cabinet, après son souper,
que Mgr le duc de Bourgogne allait en Provence pour en chasser le duc de
Savoie, s'il s'opiniâtrait à y demeurer, et que M. le duc de Berry y
accompagnerait M. son frère sans emploi. Monseigneur et ces deux princes
avaient demandé d'y aller. On comptait que tous les détachements des
diverses armées arrivés en Provence formeraient à Mgr le duc de
Bourgogne une armée aussi forte que celle du duc de Savoie, et le duc de
Berwick fut mandé d'Espagne pour la venir commander sous lui.

Le canon des ennemis débarqua à la fin, dont ils battirent le fort
Saint-Louis défendu par quatre-vingts pièces de canon, sur un gros
vaisseau approché tout contre terre. Visconti et le comte de Non
arrivèrent avec de nouvelles troupes de Piémont, et Médavy en amena
aussi du Dauphiné, et se tint à Saint-Maximin avec toute la cavalerie.
Le 15 août le maréchal de Tessé attaqua, à la pointe du jour, les
retranchements que les ennemis avaient vis-à-vis les nôtres de
Sainte-Catherine sur d'autres hauteurs. Le maréchal était à la droite,
Goesbriant au centre, Dillon à la gauche. Ils les emportèrent en trois
quarts d'heure et n'y perdirent que quatre-vingts hommes. Ils leur en
tuèrent quatorze cents, et les princes de Saxe-Gotha et de Würtemberg
seulement blessés. Ils prirent un colonel et soixante officiers et trois
cents soldats, enclouèrent tout leur canon, rasèrent leurs
retranchements, et y demeurèrent quatorze heures sans que les ennemis
fissent contenance de les venir attaquer. Le fort Saint-Louis fut enfin
pris faute d'eau, mais le bombardement fit peu de mal à la ville. Des
galiotes bombardèrent le port pendant vingt-quatre heures, et y
brûlèrent deux vaisseaux de cinquante pièces de canon.

Après ces essais infructueux, l'arrivée de tant de troupes, et les
nouvelles qu'il en accourait tant d'autres de toutes parts, les ennemis
jugèrent leur projet impossible à exécuter. Le retranchement de
Sainte-Catherine ne leur parut pas pouvoir être forcé\,; ils furent
effrayés des travaux qui avaient été faits entre ces retranchements, et
la ville. La maladie, la désertion, la disette même diminuait
considérablement leurs troupes de jour en jour\,; enfin ils se
résolurent à la retraite. Ils l'exécutèrent la nuit du 22 au 23 août,
après avoir rembarqué presque tout leur canon, mais ils laissèrent
beaucoup de bombes. M. de Savoie se retira en grand ordre, mais fort
diligemment. Il fit lui-même l'arrière-garde de tout en repassant le
Var, se mit en bataille derrière et fit rompre tous les ponts, puis
marcha vers Coni. Tessé le suivit mollement, tardivement, avec peu de
troupes, et Médavy de fort loin, parce qu'il était parti d'une grande
distance. Les paysans assommèrent tout ce qu'ils trouvèrent de traîneurs
et de maraudeurs\,: ils étaient enragés de se voir trompés dans leur
espérance. On ne put jamais tirer aucune sorte de secours des peuples de
Provence pour disputer le passage du Var à l'arrivée de M. de Savoie.
Ils refusèrent argent, vivres, milices, et dirent tout haut qu'il ne
leur importait à qui ils fussent, et que M. de Savoie, quoi qu'il fît,
ne pouvait les tourmenter plus qu'ils l'étaient.

Ce prince qui en fut averti répandit partout des placards, par lesquels
il marquait qu'il venait comme ami les délivrer d'esclavage\,; qu'il ne
voulait ni contributions trop fortes ni de vivres même qu'en payant\,;
que c'était à eux à répondre par leur bonne volonté à la sienne, et par
leur courage à secouer le joug. Il tint exactement parole pendant tout
le mois qu'il fut en Provence\,; mais Fréjus pourtant fut bel et bien
pillé, malgré tous les bons traitements faits à l'évêque, à qui tout ce
qu'il avait à la ville ou à la campagne fut soigneusement conservé\,: il
fallait bien le payer de son \emph{Te Deum}. En retournant, et même du
moment qu'ils commencèrent à rembarquer, le besoin d'attirer les peuples
cessant, la politique et le sage traitement cessa aussi. Il y eut force
pillage, qui, joint à la retraite qui ôtait toute espérance de changer
de maître, mit les paysans au désespoir aux trousses de cette armée,
dont ils tuèrent tout ce qu'ils en purent attraper. Tessé occupa Nice de
nouveau, où il laissa Montgeorges pour y commander\,; il alla de là
donner ordre à Villefranche. On craignit pour cette place et pour
Monaco\,; mais les ennemis ne songèrent à l'une ni à l'autre.

\hypertarget{chapitre-v.}{%
\chapter{CHAPITRE V.}\label{chapitre-v.}}

1707

~

{\textsc{Scandaleux éclat entre Chamillart et Pontchartrain à l'occasion
de la nouvelle de la retraite du duc de Savoie.}} {\textsc{- Le fils de
Tessé fait maréchal de camp.}} {\textsc{- Folie de Tessé et de
Pontchartrain.}} {\textsc{- M. de Savoie prend Suse.}} {\textsc{- Tessé
de retour.}} {\textsc{- Naissance du prince des Asturies.}} {\textsc{-
Perte du royaume de Naples.}} {\textsc{- Belle action de Villena,
vice-roi indignement traité par les Impériaux.}} {\textsc{- Conspiration
découverte à Genève.}} {\textsc{- Bains à Forges inutiles au moins.}}
{\textsc{- Service de la communion du roi ôtée aux ducs avec les princes
du sang.}} {\textsc{- Colère du roi sur M\textsuperscript{me} de
Torcy.}} {\textsc{- Femmes de la plus haute robe ne mangent point avec
les filles de France, et les servent.}} {\textsc{- Princesses du sang
très rarement au grand couvert, et sans conséquence.}}

~

L'importante nouvelle d'une délivrance si désirée arriva le matin, à
Marly, du vendredi 26 août, par un courrier de Langeron, qui commandait
là la marine, à Pontchartrain, qui aussitôt fut la porter au roi et le
combla et toute la cour de joie. Ce courrier avait été dépêché à l'insu
de Tessé qui envoya son fils, lequel ne partit que huit heures après le
courrier de Langeron, et arriva à l'Étang où Chamillart était, qui
l'amena à Marly dans le cabinet du roi, comme il était près de sortir de
son souper, bien honteux tous deux d'avoir été prévenus. Le courrier ne
sut du tout rien de ce qu'il conta au roi et ensuite à tout le monde, et
se fit fort moquer de lui. Il n'en fut pas moins fait maréchal de
camp\,; il n'y avait pas un mois qu'il était brigadier. Chamillart,
piqué à l'excès, fit un étrange vacarme contre Pontchartrain, comme
d'une entreprise formelle sur sa charge, dont justice lui était due\,;
que la nouvelle n'étant point maritime, il n'en devait pas avoir eu de
courrier, beaucoup moins ne la pas tenir secrète, et avoir osé la porter
au roi\,; et il prétendit qu'au moins aurait-il dû la lui mander à lui,
se taire, et lui laisser faire sa fonction et l'apprendre au roi. Jamais
on ne vit mieux qu'en cette occasion la folie universelle, et qu'on ne
juge jamais des choses par ce qu'elles sont, mais par les personnes
qu'elles regardent. Il ne faut point dire que la cour se partialisa
là-dessus entre les deux secrétaires d'État\,; Pontchartrain n'eut pas
une seule voix pour lui, et Chamillart, qui dans ce fait méritait pis
que d'être sifflé, les eut toutes. Ami des deux, mais ami de la personne
de Chamillart par mille raisons les plus fortes, ami de l'autre à cause
de son père, de sa mère et de sa femme, mais le trouvant d'ailleurs tel
qu'il était, et souffrant de la nécessité de son commerce, j'étais
affligé de l'étrange déraison de celui que j'aimais pour lui-même,
épouvanté de l'iniquité publique exercée sur celui avec qui je n'étais
uni que par ricochet. Ce ne fut pas seulement blâmer ce dernier, ce fut
un cri public, violent, redoublé en tous lieux par toutes personnes,
comme d'un attentat qui méritait punition. Malgré les affres où l'on
était, on ne put supporter d'en avoir été délivré plus tôt presque d'une
journée entière, parce qu'on {[}ne{]} l'avait été que par Pontchartrain,
et on ne s'en avisa que lorsque Chamillart osa s'en plaindre.
Monseigneur, si réservé, éclata, et Pontchartrain fut traité comme un
usurpateur avide, parce qu'il était détesté\,; Chamillart comme celui à
qui il arrachait son bien, parce qu'il était aimé, et qu'il était dans
une faveur déclarée. Personne n'eut le sens de faire réflexion sur la
juste colère où un maître entrerait contre un valet qui aurait de quoi
le tirer d'une inquiétude extrême, qui l'y laisserait tranquillement
ainsi pendant huit ou dix heures, et qui s'en excuserait froidement
après sur ce que cela était du devoir d'un autre valet qu'il avait
attendu.

Le plus rare est que le roi, que cela regardait de plus près, et pour
l'inquiétude dont il avait été délivré huit ou dix heures plus tôt, et
pour des cas semblables si aisés à se retrouver en des occasions
différentes d'une guerre allumée partout et de tous les côtés, n'eut pas
la force de se déclarer entre les deux, ni de dire une seule parole. Le
torrent fut si impétueux que Pontchartrain n'eut qu'à baisser la tête,
se taire et le laisser passer. Telle était la faiblesse du roi pour ses
ministres. On avait déjà vu, en 1702, le duc de Villeroy apporter à
Marly l'importante nouvelle de la bataille de Luzzara, s'y cacher, parce
que Chamillart n'y était pas, laisser le roi et toute la cour dans
l'inquiétude sans oser aborder, aller chercher le ministre, et ne venir
avec lui que longtemps après que la nouvelle de son arrivée s'était
répandue et avait mis tout le monde en l'air, sans que le roi l'eût
trouvé mauvais, ni seulement témoigné là-dessus la moindre chose, et fit
au contraire le duc de Villeroy lieutenant général avant de le renvoyer.
Par cette heureuse délivrance, le voyage des princes fut rompu. Ils
étaient prêts à partir\,; ils ne devaient avoir avec eux que six chevaux
de main, et n'être accompagnés que de Razilly et Denonville, qui avaient
été leurs sous-gouverneurs, et d'O et de Gamaches que le roi avait
attachés à Mgr le duc de Bourgogne, et du fils de Chamillart. Le duc de
Berwick reçut ordre par un courrier de rebrousser chemin vers M. le duc
d'Orléans.

Mais voici une autre sorte d'extravagance qu'il faut que je raconte
avant de quitter l'affaire de Provence. Tessé s'en trouvait chargé\,:
c'était la plus capitale de l'État dans un pays où rien n'était préparé,
et où on manquait de tout, parce qu'on ne s'y était pas attendu\,; des
secours en tout genre fort éloignés, la flotte des ennemis et une armée
sur les bras commandée par les deux plus habiles capitaines, les plus
audacieux, les plus grands ennemis du roi, et s'ils réussissaient le
royaume pris à revers dans des provinces mécontentes, tout ouvert de là
jusque dans Paris et les armées ennemies à toutes les frontières qui
n'attendaient que le signal. Un général chargé de parer un si grand coup
et dans une situation aussi pressée a bien des soins et peu d'envie de
rire. Ce ne fut pas le sentiment de Tessé. Il n'en vit pas apparemment
ces grandes suites si palpables, il ne voyait pas apparemment qu'avec
Toulon la marine du Levant et son commerce étaient perdus, que la
Provence ne l'était pas moins, qu'Arles était un passage sur le Rhône,
et une ville ouverte, où M. de Savoie pouvait faire sa place d'armes en
l'accommodant et se porter de là en Languedoc fumant encore de
fanatiques, à Lyon, et dans les entrailles de la France\,; ou s'il le
vit, comme toutes ces suites-là sautaient aux yeux, en grand homme
supérieur à tout, il y trouva le mot pour rire, et ce qui est
incomparable, apparemment Pontchartrain aussi. Gardant pour soi la clef
des champs pour y être plus libre que dans les retranchements de Toulon,
où il ne fit que passer et où il ne s'arrêta que pour emporter, comme je
l'ai dit, ceux de M. de Savoie, il trouvait le temps d'écrire à
Pontchartrain tous les ordinaires jusqu'aux plus petits détails des
nouvelles des ennemis, et de tout ce qui arrivait et se passait parmi
nous, dans le style de don Quichotte, dont il se disait le triste écuyer
et le Sancho, et tout ce qu'il mandait il l'adaptait aux aventures de ce
roman. Pontchartrain me montrait ses lettres, il mourait de rire, il les
admirait, et il faut dire en effet qu'elles étaient très plaisantes, et
qu'il rendait un compte exact en termes, en style et en aventures de ce
roman avec une suite et plus d'esprit que je ne lui en aurais cru. Moi
cependant j'admirais un homme farci de ces fadaises en faire son capital
pour rendre compte à un secrétaire d'État de l'affaire la plus
importante et la plus délicate de l'État, dans la position si critique
où il se trouvait, et l'admiration même de ce secrétaire d'État qui
trouvait cela admirable\,; et la prosopopée fut soutenue jusque tout à
la fin de l'affaire. Cela me paraîtrait incroyable si je ne l'avais pas
vu.

Les détachements des différentes armées pour la Provence retournèrent
les joindre presque aussitôt qu'ils en furent partis. Marlborough ne
pouvait ajouter foi au mauvais succès de M. de Savoie. Il avait bâti sur
ce projet les plus grands desseins, qui tombèrent d'eux-mêmes. M. de
Savoie ne songea plus qu'à rétablir ses troupes fort diminuées, et qui
avaient beaucoup souffert\,; et au mois d'octobre, il prit Suse
abandonné à une très faible garnison qu'il eut prisonnière de guerre. Ce
fut à quoi se terminèrent tous ses exploits. Un mois après le maréchal
de Tessé arriva à la cour. Sa réception y fut au-dessous du médiocre.
Nous étions à table à Meudon avec Monseigneur lorsqu'il vint lui faire
sa révérence. Je ne vis jamais si maigre accueil, mais ses souterrains
ne mirent guère à le rejeter en selle. Médavy demeura seul en chef en sa
place.

La joie de la naissance du prince des Asturies vint en cadence augmenter
celle de la délivrance de la Provence. Le marquis de Brancas qui servait
lors en Espagne, eut la commission d'y en faire les compliments du roi.
Le duc d'Albe, à cette occasion, donna chez lui, à Paris, une superbe
fête qui dura trois jours de suite, et toujours variée. Elle dut être
tempérée par la perte du royaume de Naples et de Sicile. Le marquis de
Bedmar, vice-roi de cette île, sentant peut-être l'impossibilité de la
conserver, avait obtenu son rappel, et le marquis de Los Balbazès avait
été nommé en sa place. Le marquis de Villena, autrement le duc
d'Escalone, qui avait été vice-roi de Catalogne, et que nous y avons vu
battu par M. de Noailles père, puis par M. de Vendôme, était vice-roi de
Naples, et y avait magnifiquement reçu le roi d'Espagne. Il ne put
soutenir cette ville contre les troupes impériales, qui, n'ayant plus
d'occupation dans toute l'Italie, étaient venues à la facile conquête de
ce royaume qui manquait de troupes et de tout, et dont les habitants,
seigneurs et autres, ne respirent continuellement que les changements de
maîtres.

Ces troupes ne trouvèrent donc aucune résistance à entrer dans Naples,
où elles eurent le plaisir de voir briser aussitôt après la statue de
Philippe V par les mêmes mains qui l'y avaient élevée. Le duc de Tursis
mena le vice-roi sur son escadre à Gaëte, et la ramena après avec celle
de Naples à Livourne. Le siège de Gaëte fut formé bientôt après. C'était
la seule place du royaume de Naples qui tint pour le roi d'Espagne.
Escalone, dénué de tout, y fit des prodiges de patience, de capacité, de
valeur, et mit, les Impériaux en état d'en recevoir l'affront. La
trahison suppléa à la force les habitants, lassés de si longs travaux,
entrèrent en intelligence avec le comte de Thun qui commandait au siège.
Ils lui livrèrent la place. Escalone ou Villena, car il était connu sous
les deux noms, ne s'étonna point. Il se barricada et se défendit de rue
en rue avec tout ce qu'il put ramasser autour de lui, et ne se voulut
jamais rendre. Succombant enfin dans un dernier réduit au nombre et à la
force, il fut pris. Le procédé des Impériaux fut indigne. Au lieu
d'admirer une si magnanime défense, ils n'écoutèrent que le dépit de ce
qu'elle leur avait coûté\,; ils envoyèrent le généreux vice-roi
prisonnier, les fers aux pieds, à Pizzighettone, contre toutes les lois
de la guerre et, de l'humanité, où il demeura très longtemps cruellement
resserré. Martinitz, d'abord nommé vice-roi par l'empereur, fut rappelé
à Vienne, le comte de Thun fait vice-roi par intérim, et le général
Vanbonne, qui avait tant fait parler de lui à la guerre, grand et hardi
partisan, fut du nombre de ceux qui moururent des blessures reçues à ce
siège. Ce fut un ingénieur qui ouvrit une porte aux Impériaux, lesquels
allèrent d'abord égorger tout ce qu'ils purent trouver d'officiers et de
soldats espagnols, demeurés en petit nombre de trois mille qu'ils y
étaient. Les galères n'étaient point dans le port\,; elles étaient
allées chercher en Sicile des vivres pour la place.

On découvrit en septembre une conspiration dans Genève, que M. de Savoie
y avait tramée pour s'en rendre le maître. Plusieurs magistrats de cette
petite république y trempèrent. Beaucoup furent exécutés. Il y en eut
d'assez ennemis de leur patrie pour encourager les conjurés de dessus
l'échafaud, et leur crier de ne rien craindre, qu'ils n'avaient rien
avoué ni nommé personne, et qu'ils poussassent hardiment leur pointe. Ce
n'était pas la première tentative que ce prince eût faite pour s'emparer
de Genève\,; imitateur en cela de ses pères qui en ont toujours
considéré l'acquisition comme une des plus importantes qu'ils pussent
faire.

J'allai cet été à Forges, qui est la saison de ces eaux, pour essayer de
m'y défaire d'une fièvre tierce que le quinquina ne faisait que
suspendre. Je dirai pour une curiosité de médecine que
M\textsuperscript{me} de Pontchartrain y était aussi pour une perte
continuelle de sang, puis d'eau, qui durait depuis longtemps malgré tous
les remèdes. Fagon, à bout, voulut tenter un essai jusqu'alors sans
exemple\,: ce fut de la faire baigner dans l'eau de la fontaine la plus
forte et la plus vitriolée des trois qui y sont, dont on boit le moins,
et qui, du cardinal de Richelieu qui en a pris, a retenu le nom de
\emph{Cardinale}. Jamais personne ne s'était baigné dans l'eau d'aucune,
et M\textsuperscript{me} de Pontchartrain n'y trouva rien moins que du
soulagement. Ce fut là que j'appris une nouvelle entreprise des princes
du sang, qui, dans l'impuissance et le discrédit où le roi les tenait,
profitaient sans mesure de son désir de la grandeur de ses bâtards qu'il
leur avait assimilés, pour s'acquérir de nouveaux avantages qui leur
étaient soufferts pour les partager avec eux. La supériorité et les
différences de rang, si marquées au-dessus d'eux des petits-fils de
France, leur était toujours fâcheux à supporter. Une de ces distinctions
se trouvait aux communions du roi.

On poussait après l'élévation de la messe un ployant au bas de l'autel
au lieu où le prêtre la commence, on le couvrait d'une étoffe, puis
d'une grande nappe qui traînait devant et derrière. Au \emph{Pater},
l'aumônier de jour se levait et nommait au roi à l'oreille tous les ducs
qui se trouvaient dans la chapelle. Le roi lui en nommait deux qui
étaient toujours les deux plus anciens, à chacun desquels aussitôt après
le même aumônier s'avançant allait faire une révérence. La communion du
prêtre se faisant, le roi se levait et s'allait mettre à genoux sans
tapis ni carreau derrière ce ployant et y prenait la nappe\,; en même
temps les deux ducs avertis, qui seuls avec le capitaine des gardes en
quartier s'étaient levés de dessus leurs carreaux et l'avaient suivi,
l'ancien par la droite, l'autre par la gauche, prenaient en même temps
que lui chacun un coin de la nappe qu'ils soutenaient à côté de lui à
peu de distance, tandis que les deux aumôniers de quartier soutenaient
les deux autres coins de la même nappe du côté de l'autel, tous quatre à
genoux, et le capitaine des gardes aussi, seul derrière le roi. La
communion reçue et l'ablution prise quelques moments après, le roi
demeurait encore un peu en même place, puis retournait à la sienne,
suivi du capitaine des gardes et des deux ducs qui reprenaient les
leurs. Si un fils de France s'y trouvait seul, lui seul tenait le coin
droit de la nappe et personne de l'autre côté\,; et quand M. le duc
d'Orléans s'y rencontrait sans fils de France, c'était la même chose. Un
prince du sang présent n'y servait pas avec lui\,; mais s'il n'y avait
qu'un prince du sang, un duc, au lieu de deux, était averti à
l'ordinaire, et il servait à la gauche comme le prince du sang à la
droite. Le roi nommait les ducs pour montrer qu'il était maître du choix
entre eux, sans être astreint à l'ancienneté\,; mais il ne lui est
pourtant jamais arrivé de préférer de moins anciens\,; et je me souviens
que, marchant devant lui un jour de communion qu'il allait à la
chapelle, et voyant le duc de La Force, je le vis parler bas au maréchal
de Noailles\,; et un moment après le maréchal me vint demander qui était
l'ancien de M. de La Force ou de moi. Il ne l'avait pu dire
certainement, et le roi le voulut savoir pour ne s'y pas méprendre.

Les princes du sang, blessés de cette distinction de M. le duc
d'Orléans, qu'ils avaient essuyée assez peu encore avant qu'il allât en
Espagne, s'en voulurent dédommager en usurpant sur les ducs la même
distinction. Ils firent leur affaire dans les ténèbres\,; et à
l'Assomption de cette année, M. le Duc servit seul à la communion du
roi, sans qu'aucun duc fût averti. Je l'appris à Forges\,; je sus que la
surprise avait été grande, et que le duc de La Force, qui aurait dû
servir et le maréchal de Boufflers, étaient à la chapelle. J'écrivis à
ce dernier que cela n'était jamais arrivé, que moi-même j'avais servi
avec les princes du sang et avec M. le Duc lui-même, et il n'y avait pas
même longtemps\,; que cela était aisé à vérifier sur les registres de
Desgranges, maître des cérémonies, et ce que je crus enfin qu'il fallait
faire pour ne pas faire cette perte nouvelle. On visita le registre et
on le trouva écrit et chargé de ce que j'avais mandé et de quantité
d'autres pareils exemples. Mais la mollesse et la misère des ducs n'osa
branler. Je m'en étais douté, et j'avais en même temps écrit à M. le duc
d'Orléans, en Espagne, tout ce que je crus le plus propre à le piquer,
et par rapport à la conservation de sa distinction sur les princes du
sang, à ne pas souffrir cette usurpation sur les ducs pour s'égaler par
là à lui en ce qu'il était possible. À son retour je fis qu'il en parla
au roi\,; le roi s'excusa, M. le Duc dit qu'il n'y avait point eu de
part. M. le duc d'Orléans pressa, tout timide qu'il était avec le roi,
qui répondit que c'étaient les ducs qui d'eux-mêmes ne s'y étaient pas
présentés. Mais comment l'eussent-ils fait sans être avertis\,? et
comment le roi lui-même l'eût-il trouvé\,? Bref, il n'en fut autre
chose, et cela est demeuré ainsi.

Piqué, et peu pressé de retourner à la cour, je m'en allai de Forges à
la Ferté, où M\textsuperscript{me} de Saint-Simon me vint trouver de
Rambouillet, où M\textsuperscript{me} la duchesse d'Orléans l'avait
engagée d'aller avec elle et quelques autres dames. Nous demeurâmes
trois semaines à la Ferté. La cour était à Fontainebleau, où je ne
voulais point aller. Plus sage que moi, M\textsuperscript{me} de
Saint-Simon m'y entraîna. Je n'allai faire ma révérence au roi que le
surlendemain de mon arrivée, et dans l'instant je me retirai et sortis.
Apparemment il remarqua l'un et l'autre. C'était l'homme du monde qui
était le plus attentif à toutes ces petites choses, et il était
exactement informé chaque jour des gens de la cour qui arrivaient à
Fontainebleau, où il aimait surtout à l'avoir grosse et distinguée. Le
jour suivant, passant par son antichambre, allant ailleurs
l'après-dînée, je le rencontrai qui passait chez M\textsuperscript{me}
de Maintenon. À l'instant il me demanda de mes nouvelles. Je répondis
avec respect et brièveté, et, sans le suivre, je continuai mon chemin.
Aussitôt je m'entendis rappeler. C'était le roi qui me parlait encore. À
cette fois, je n'osai plus quitter, et je le suivis jusqu'où il allait.
Il sentait quand il avait fait, peine ou injustice, et quelquefois même
assez souvent il cherchait à faire distinction, et ce qui dans un
particulier supérieur s'appellerait honnêteté. Ce narré m'a conduit à
Fontainebleau plus tôt que de raison, il faut retourner un peu en
arrière. Mais auparavant je dirai que, quoique pressé souvent de me
trouver aux communions du roi depuis, et en des temps où il n'y avait
point de princes du sang à la cour, car les bâtards ne s'y étaient pas
encore présentés, je ne pus jamais m'y résoudre, et jamais je n'y ai été
depuis.

Il arriva une aventure à Marly, peu avant Fontainebleau, qui fit grand
bruit par la longue scène qui la suivit, plus étonnante qu'on ne se le
peut imaginer à qui a connu le roi. Toutes les dames du voyage avaient
alors l'honneur de manger soir et matin, à la même heure, dans le même
petit salon qui séparait l'appartement du roi et celui de
M\textsuperscript{me} de Maintenon. Le roi tenait une {[}table{]} où
tous les fils de France et toutes les princesses du sang se mettaient,
excepté M. le duc de Berry, M. le duc d'Orléans et M\textsuperscript{me}
la princesse de Conti, qui se mettaient toujours à celle de Monseigneur,
même quand il était à la chasse. Il y en avait une troisième plus petite
où se mettaient, tantôt les unes, tantôt les autres\,; et toutes trois
étaient rondes, et liberté à toutes de se mettre à celle que bon leur
semblait. Les princesses du sang se plaçaient à droite et à gauche en
leur rang\,; les duchesses et les autres princesses comme elles se
trouvaient ensemble, mais joignant les princesses du sang et sans
mélange entre elles d'aucunes autres\,; puis les dames non titrées
achevaient le tour de la table, et M\textsuperscript{me} de Maintenon
parmi elles vers le milieu\,; mais elle n'y mangeait plus depuis assez
longtemps. On lui servait chez elle une table particulière où quelques
dames, ses familières, deux ou trois, mangeaient avec elle, et presque
toujours les mêmes. Au sortir de dîner le roi entrait chez
M\textsuperscript{me} de Maintenon, se mettait dans un fauteuil près
d'elle dans sa niche, qui était un canapé fermé de trois côtés, les
princesses du sang sur des tabourets auprès d'eux, et, dans
l'éloignement, les dames privilégiées, ce qui pour cette entrée-là était
assez étendu. On était auprès de plusieurs cabarets de thé et de café\,;
en prenait qui voulait. Le roi demeurait là plus ou moins, selon que la
conversation des princesses l'amusait, ou qu'il avait affaire, puis il
passait devant toutes ces dames, allait chez lui, et toutes sortaient,
excepté quelques familières de M\textsuperscript{me} de Maintenon. Dans
l'après-dînée, à la suite de M\textsuperscript{me} la duchesse de
Bourgogne, personne n'entrait où était le roi et M\textsuperscript{me}
de Maintenon que M\textsuperscript{me} la duchesse de Bourgogne et le
ministre qui venait travailler. La porte était fermée, et les dames qui
étaient dans l'autre pièce n'y voyaient le roi que passer pour souper,
et elles l'y suivaient, après souper, chez lui, avec les princesses
comme à Versailles. Il fallait cet exposé pour entendre ce qui va être
raconté.

À un dîner, je ne sais comment il arriva que M\textsuperscript{me} de
Torcy se trouva auprès de Madame, au-dessus de la duchesse de Duras, qui
arriva un moment après. M\textsuperscript{me} de Torcy, à la vérité, lui
offrit sa place, mais on n'en était déjà plus à les prendre, cela se
passa en compliments, mais la nouveauté du fait surprit Madame et toute
l'assistance qui était debout et Madame aussi. Le roi arrive et se met à
table. Chacun s'allait asseoir, comme le roi, regardant du côté de
Madame, prit un sérieux et un air de surprise qui embarrassa tellement
M\textsuperscript{me} de Torcy qu'elle pressa la duchesse de Duras de
prendre sa place, qui n'en voulut rien faire encore une fois\,; et pour
celle-là, elle aurait bien voulu qu'elle l'eût prise, tant elle se
trouva embarrassée. Il faut remarquer que le hasard fit qu'il n'y avait
que la duchesse de Duras de titrée de ce même côté de la table\,; les
autres, apparemment avaient préféré {[}être{]}, ou par hasard s'étaient
trouvées du côté de M\textsuperscript{me} la duchesse de Bourgogne et de
M\textsuperscript{me} fa Duchesse, les deux princes étant ce jour-là à
la chasse avec Monseigneur. Tant que le dîner fut long le roi n'ôta
presque point les yeux de dessus les deux voisines de Madame, et ne dit
presque pas un mot, avec un air de colère qui rendit tout le monde fort
attentif, et dont la duchesse de Duras même fut fort en peine.

Au sortir de table, on passa à l'ordinaire chez M\textsuperscript{me} de
Maintenon. À peine le roi y fut établi dans sa chaise, qu'il dit à
M\textsuperscript{me} de Maintenon, qu'il venait d'être témoin d'une
insolence (ce fut le terme dont il se servit) incroyable et qui l'avait
mis dans une telle colère qu'elle l'avait empêché de manger, et raconta
ce qu'il avait vu de ces deux places\,; qu'une {[}telle{]} entreprise
aurait été insupportable d'une femme de qualité, de quelque haute
naissance qu'elle fût\,; mais que d'une petite bourgeoise, fille de
Pomponne, qui s'appelait Arnauld, mariée à un Colbert, il avouait qu'il
avait été dix fois sur le point de la faire sortir de table, et qu'il ne
s'en était retenu que par la considération de son mari. Enfilant
là-dessus la généalogie des Arnauld qu'il eut bientôt épuisée, il passa
à celle des Colbert qu'il déchiffra de même, s'étendit sur leur folie
d'avoir voulu descendre d'un roi d'Écosse\,; que M. Colbert l'avait tant
tourmenté de lui en faire chercher les titres par le roi d'Angleterre,
qu'il avait eu la faiblesse de lui en écrire\,; que la réponse ne venant
point, et Colbert ne lui donnant sur cela aucun repos, il avait écrit
une seconde fois, sur quoi enfin le roi d'Angleterre lui avait mandé
que, par politesse, il n'avait pas voulu lui répondre, mais que,
puisqu'il le voulait, qu'il sût donc que, par pure complaisance, il
avait fait chercher soigneusement en Écosse, sans avoir rien trouvé,
sinon quelque nom approchant de celui de Colbert dans le plus petit
peuple, qu'il l'assurait que son ministre était trompé par son orgueil,
et qu'il n'y donnât pas davantage\footnote{On trouve, dans les Mémoires
  contemporains et principalement dans les \emph{Mémoires de l'abbé de
  Choisy}, des détails sur cette faiblesse de Colbert. Voy. ces
  Mémoires, collection Petitot, 2° série, t. LXIII, p.~215-222.}. Ce
récit, fait en colère, fut accompagné de fâcheuses épithètes, jusqu'à
s'en donner à lui-même sur sa facilité d'avoir ainsi écrit\,; après quoi
il passa tout de suite à un autre discours plus surprenant encore à qui
l'a connu. Il se mit à dire qu'il trouvait bien sot à
M\textsuperscript{me} de Duras (car ce fut son terme) de n'avoir pas
fait sortir de cette place M\textsuperscript{me} de Torcy par le bras,
et s'échauffa si bien là-dessus, que M\textsuperscript{me} la duchesse
de Bourgogne et les princesses à son exemple, ayant peur qu'il ne lui en
fit une sortie, se prirent à l'excuser sur sa jeunesse, et à dire qu'il
seyait bien toujours à une personne de son âge d'être douce et facile,
et d'éviter de faire peine à personne. Là-dessus le roi reprit qu'il
fallait qu'elle fût donc bien douce et bien facile en effet de l'avoir
souffert de qui que ce fût sans titre\,; plus encore de cette petite
bourgeoise, et que toutes deux ignorassent bien fort, l'une ce qui lui
était dû, l'autre le respect (ce fut encore son terme) qu'elle devait
porter à la dignité et à la naissance\,; qu'elle devait se sentir bien
honorée d'être admise à sa table et soufferte parmi les femmes de
qualité\,; qu'il avait vu les secrétaires d'État bien éloignés d'une
confusion semblable\,; que sa bonté et la sottise des gens de qualité
les avait laissés mêler parmi eux\,; que ce honteux mélange devait bien
leur suffire à ne pas entreprendre ce que la femme de la plus haute
naissance n'eût pas osé songer d'attenter (ce fut encore l'expression
dont il se servit) mais encore pour respecter les femmes de qualité sans
titre, et ne pas abuser de l'honneur étrange et si nouveau de se trouver
comme l'une d'elles, et se bien souvenir toujours de l'extrême
différence qu'il y avait, et qui y serait toujours\,; qu'on voyait bien
à cette impertinence (ce fut le mot dont il se servit) le peu d'où elle
était sortie, et que les femmes de secrétaires d'État qui avaient de la
naissance, se gardaient bien de sortir de leurs bornes, comme par
exemple, M\textsuperscript{me} de Pontchartrain qui, par sa naissance se
pouvait mêler davantage avec les femmes de qualité, prenait tellement
les dernières places, et cela si naturellement et avec tant de
politesse, que cette conduite ajoutait infiniment à sa considération, et
lui procurait aussi des honnêtetés qui, depuis son mariage, étaient bien
loin de lui être dues.

Après ce panégyrique de M\textsuperscript{me} de Pontchartrain, sur
lequel le roi prit plaisir à s'étendre, il acheva de combler
l'assistance d'étonnement\,; car, reprenant sa première colère que le
long discours semblait avoir amortie, il se mit à exalter la dignité des
ducs et fit connaître pour la première fois de sa vie qu'il n'en
ignorait ni la grandeur, ni la connexité de cette grandeur à celle de sa
couronne et de sa propre majesté. Il dit que cette dignité était la
première de l'État\,; la plus grande qu'il pût donner à son propre sang,
le comble de l'honneur et de la récompense de la plus haute noblesse. Il
s'abaissa jusqu'à avouer que, si la nécessité de ses affaires et de
grandes raisons l'avaient quelquefois obligé d'élever à ce fait de
grandeur (ce fut encore sa propre expression) quelques personnes d'une
naissance peu proportionnée, ç'avait été avec regret\,; mais que la
dignité en soi n'en était point avilie ni en rien diminuée de tout ce
qu'elle était, qu'elle demeurait toujours la même, et tout aussi
respectable à chacun, aussi entière d'ans tous ses rangs, ses
distinctions, ses privilèges, ses honneurs en ces sortes de ducs,
considérables et vénérables à tous, dès là qu'ils étaient ducs, comme
ceux de la plus grande naissance, puisque leur dignité était la même, le
soutien de la couronne, ce qui la touchait de plus près, et à la tête de
toute la haute noblesse, de laquelle elle était en tout séparée et
infiniment distinguée et relevée\,; et qu'il voulait bien qu'on sût que
leur refuser les honneurs et les respects qui leur étaient dus, c'était
lui en manquer à lui-même. Ce sont là exactement les termes de son
discours. De là passant à la noblesse de la maison de Bournonville, dont
était la duchesse de Duras, et à celle de la maison de son mari, sur
lesquelles il s'étendit à plaisir, il vint à déplorer le malheur des
temps qui avait réduit tant de ducs à la mésalliance, et se mit à nommer
toutes les duchesses de peu\,; puis renouvelant de plus belle en sa
colère, il dit qu'il ne fallait pas que les femmes de la plus haute
qualité parleurs maris et par elles-mêmes prissent occasion de la
naissance de ces duchesses de leur rendre quoi que ce fût moins qu'à
celles dont la condition répondait à leur dignité, laquelle méritait en
toutes, qui qu'elles fussent par elles-mêmes, le même respect. (ce fut
encore son terme), puisque leur rang était le même\,; et que ce qui leur
était dû ne leur était dû que par leur dignité, qui ne pouvait être
avilie par leurs personnes, rien ne pouvait excuser aucun manquement
qu'on pouvait faire à leur égard\,; et cela avec des termes si forts et
si injurieux qu'il semblait que le roi ne fût pas le même, et encore par
la véhémence dont il parlait. Pour conclusion, le roi demanda qui des
princesses se voulait charger de dire à M\textsuperscript{me} de Torcy à
quel point il l'avait trouvée impertinente. Toutes se regardèrent et pas
une ne se proposa, sur quoi le roi, se fâchant davantage, dit que si
fallait-il pourtant qu'elle le sût, et là-dessus s'en alla chez lui.

Alors les dames, qui avaient bien vu de loin qu'il y avait eu beaucoup
de colère dans la conversation, et qui pour cela même s'étaient tenues
encore plus soigneusement à l'écart, s'approchèrent un peu par
curiosité, qui augmenta fort en voyant l'espèce de trouble des
princesses qui s'ébranlaient pour s'en aller, lesquelles, après quelque
peu de discours entre elles, se séparèrent et contèrent le fait chacune
à ses amies, M\textsuperscript{me} de Maintenon à ses favorites,
M\textsuperscript{me} la duchesse de Bourgogne à ses dames et à la
duchesse de Duras, en sorte que la chose se répandit bientôt à l'oreille
et courut après partout. On crut que cela était fini\,; mais sitôt que
le roi eut passé, le même jour, de son souper dans son cabinet, la
vesperie recommença encore avec plus d'aigreur, tellement que
M\textsuperscript{me} la Duchesse, craignant enfin pis, conta tout en
sortant à M\textsuperscript{me} de Bouzols pour qu'elle en avertît
Torcy, son frère, et que sa femme prit bien garde à elle. Mais la
surprise fut, extrême quand le lendemain, au sortir du dîner, le roi ne
put, chez M\textsuperscript{me} de Maintenon, parler d'autre chose, et
encore sans aucun adoucissement dans les termes\,; si bien que, pour
l'apaiser un peu, M\textsuperscript{me} la Duchesse lui dit qu'elle
avait averti M\textsuperscript{me} de Bouzols, n'osant le dire à
M\textsuperscript{me} de Torcy elle-même\,; sur quoi le roi, comme
soulagé, se hâta de lui répondre qu'elle lui avait fait grand plaisir,
parce que cela lui épargnait la peine de bien laver la tête à Torcy,
qu'il avait résolu de le faire plutôt que sa femme manquât de recevoir
ce qu'elle méritait. Il ne laissa pas de poursuivre encore les mêmes
propos et de même façon jusqu'à ce qu'il repassât chez lui.

Torcy et sa femme, outrés, furent quelques jours à ne paraître presque
point. Ils firent l'un et l'autre de grandes excuses et force
compliments à la duchesse de Duras, qui elle-même était, surtout devant
le roi, fort embarrassée, lequel quatre jours durant ne cessa de parler
toujours sur ce même ton dans ses particuliers. Torcy, craignant une
sortie, écrivit une lettre au roi de plainte et de douleur respectueuse
d'une tempête dont la source n'était qu'un hasard qu'il n'avait pas tenu
à sa femme de corriger, mais à la duchesse de Duras, qui poliment, quoi
qu'elle eût pu faire, n'avait pas voulu prendre sa place. Toutes sortes
d'aveux de ce qui était dû, et dont sa femme n'avait jamais songé à
s'écarter, et toutes sortes de respects et de traits délicats de
modestie étaient adroitement glissés dans cette lettre. Le roi lui
témoigna en être content à son égard\,; il ménagea les termes sur sa
femme, mais il lui fit entendre qu'elle ferait bien d'être attentive et
mesurée dans sa conduite, tellement que cela fut fini de manière que
Torcy ne sortit pas trop mécontent de la conversation. On peut imaginer
le bruit que fit cette aventure, et jusqu'à quel point les secrétaires
d'État et les ministres si haut montés la sentirent. Le rare fut qu'il y
eut des femmes de qualité qui se sentirent piquées de ce qui avait été
dit sur elles. Toutes affectèrent une grande attention à rendre aux
femmes titrées. Le roi, qui le remarqua, le loua, mais avec aigreur sur
le contraire, et s'est toujours montré depuis le même à cet égard des
femmes titrées et non titrées, et des hommes pareillement. Pour ce qui
est d'ailleurs du rang et de la dignité des dues, son règne entier,
avant et depuis, s'est passé à y donner les plus grandes atteintes.
J'appris l'affaire en gros par ce qu'on m'en écrivit\,; je la sus à mon
retour dans le dernier détail, et le plus précis, par plusieurs
personnes instruites dès les premiers moments, surtout par les dames de
M\textsuperscript{me} la duchesse de Bourgogne, à qui cette princesse
l'avait contée à mesure et à la chaude, et qui, n'étant pas duchesses,
me furent encore moins suspectes de ne rien grossir.

M\textsuperscript{me} la duchesse de Bourgogne, huit jours avant d'aller
à Fontainebleau, fit avec Mgr le duc de Bourgogne et beaucoup de darnes
une grande cavalcade au bois de Boulogne, où il se trouva une infinité
de carrosses de Paris pour la voir. À la nuit, elle mit pied à terre à
la Muette\footnote{Saint-Simon, comme on l'a déjà remarqué, écrit
  toujours \emph{la Meute}. On a suivi l'orthographe moderne.}, où
Armenonville donna un souper magnifique. Les dames de la cavalcade
soupèrent avec Mgr {[}le duc{]} et M\textsuperscript{me} la duchesse de
Bourgogne, laquelle pendant tout le repas fut servie par
M\textsuperscript{me} d'Armenonville debout derrière elle. Au sortir de
table, il parut tout à coup une illumination très galante\,; on entendit
des violons et de toutes sortes d'instruments, on dansa ou on se promena
jusqu'à deux heures après minuit. M\textsuperscript{me} de Fourcy, femme
d'un conseiller d'État, lors prévôt des marchands, et fille de
Boucherat, chancelier de France, avait servi de même
M\textsuperscript{me} la Dauphine de Bavière au dîner que le roi fit à
l'hôtel de ville, avec beaucoup de dames à sa table, au sortir du
\emph{Te Deum} qu'il avait été entendre à Notre-Dame, lorsqu'il fut
guéri de sa grande opération. Il voulut témoigner à Paris qu'il lui
savait gré du zèle qu'elle avait témoigné en cette occasion, et il fut
fort remarqué que, pour l'unique fois de sa vie, il demanda ce repas à
l'hôtel de ville, auquel il ne voulut pas qu'aucun de ses officiers
travaillassent, ni que pas un de ses gardes entrassent dans l'hôtel de
ville. Il n'y fut pas question que M\textsuperscript{me} de Fourcy se
mît à table, non plus que M\textsuperscript{me} d'Armenonville à la
Muette. C'est un honneur auquel la robe la plus distinguée n'a jamais
osé prétendre.

Deux jours, après le roi fit souper avec lui Mademoiselle, fille de M.
le duc d'Orléans, à son grand couvert à Versailles, et entrer après avec
lui dans son cabinet. Cette distinction fit du bruit\,; les princesses
du sang ne mangent point au grand couvert, c'est un honneur réservé aux
fils, filles, petits-fils et petites-filles de France, excepté des
festins de noces dans la maison royale, et dans des cérémonies fort
rares. Il est pourtant arrivé quelquefois que, entre la mort de la
dauphine de Bavière et le mariage de celle de Savoie, les enfants de
Monseigneur trop jeunes pour souper avec le roi, et Monsieur et Madame à
Paris ou à Saint-Cloud, le roi, pour ne pas souper seul, ou tête à tête
avec Monseigneur, fit quelquefois venir au grand couvert
M\textsuperscript{me} la Duchesse et M\textsuperscript{me} la princesse
de Conti, ses filles, mais nulle autre princesse du sang, et cela sans
suite et sans conséquence\,; mais j'ai vu quelquefois ces mêmes
princesses y manger avec Madame à Fontainebleau, quelquefois la cour
d'Angleterre y étant, et quelquefois aussi, mais très peu,
M\textsuperscript{me} la Princesse et M\textsuperscript{me} la princesse
de Conti, sa fille aussi, à Fontainebleau, avec la même cour
d'Angleterre, le soir au grand couvert, jamais à Versailles. C'était une
faveur que le roi faisait quelquefois à ses filles, qui fit crier M. le
Prince fort haut, M\textsuperscript{me} la Princesse étant à
Fontainebleau, qui n'y était pas admise, tandis que
M\textsuperscript{me} la Duchesse, sa belle-fille, et
M\textsuperscript{me} du Maine, sa fille, l'étaient. Le roi ne voulut
pas pousser ce dégoût, et y fit manger quelque peu M\textsuperscript{me}
la Princesse et M\textsuperscript{me} la princesse de Conti, puis n'y en
fit plus manger pas une, et se restreignit au droit\,; apparemment que,
ces princesses ayant mangé au grand couvert quelquefois, il voulut faire
la même grâce à celle-ci qui était sa petite-fille, pour que cela n'eût
pas plus de suite ni de droit que pour les autres.

\hypertarget{chapitre-vi.}{%
\chapter{CHAPITRE VI.}\label{chapitre-vi.}}

1707

~

{\textsc{Tonnerre tue à la chasse le second fils d'Amelot.}} {\textsc{-
Duel de deux capitaines aux gardes\,; Saint-Paul tué et Sérancourt
cassé.}} {\textsc{- Le roi, allant à Fontainebleau, passe pour la
première foi à Petit-Bourg.}} {\textsc{- Prodiges de courtisan.}}
{\textsc{- Mort de Sourdis.}} {\textsc{- Son gouvernement d'Orléanais à
d'Antin.}} {\textsc{- Quel était Bartet\,; sa mort.}} {\textsc{-
Conduite, fortune et mort du cardinal Le Camus.}} {\textsc{- Mort du
comte d'Egmont, dernier de la maison d'Egmont\,; son caractère et sa
succession.}} {\textsc{- Équipée de la comtesse de Soissons.}}
{\textsc{- Retour de Fontainebleau par Petit-Bourg.}} {\textsc{- Mort de
Revel\,; son mariage\,; maréchaux de Broglio.}} {\textsc{- Mort de la
maréchale de Tourville.}} {\textsc{- Faux-saulnage.}} {\textsc{- Étrange
sorte d'escroquerie de Listenais.}} {\textsc{- Cause de la brouillerie
de Catinat et de Chamillart\,; le roi les réconcilie.}} {\textsc{-
Bay\,; son extraction\,; est fait chevalier de la Toison d'or.}}
{\textsc{- Mort du comte d'Auvergne\,; son caractère\,; sa dépouille.}}
{\textsc{- Dépit du comte d'Évreux.}} {\textsc{- Mariage du prince de
Talmont, qui surprend un tabouret de grâce.}}

~

Le fils aîné du feu comte de Tonnerre, étant à la chasse à la plaine
Saint-Denis avec le second fils d'Amelot, conseiller d'État, lors
ambassadeur en Espagne, le tua d'un coup de fusil, le 6 septembre.
M\textsuperscript{me} de Tonnerre fit prendre le large à son fils, et
vint demander sa grâce au roi, l'assurant que le fusil avait parti sans
que son fils y pensât, et que le jeune Amelot était fort son ami. En
même temps, M\textsuperscript{me} de Vaubecourt, soeur d'Amelot, vint
demander au roi de ne point donner grâce à l'assassin de son neveu, qui
l'avait couché en joue, et assura qu'il l'avait tué de propos délibéré.
Ce jeune Amelot était toute l'espérance de sa famille, ayant le corps et
l'esprit aussi bien faits que son aîné les avait disgraciés, qui devint
pourtant président à mortier. Tonnerre était une manière d'hébété fort
obscur et fort étrange. Il eut sa grâce un mois après, il entra pour un
an à la Bastille, donna dix mille livres aux pauvres, distribuables par
le cardinal de Noailles, et eut défense sous de grandes peines de se
trouver jamais en nul lieu public ni particulier où M. Amelot serait, et
obligé de sortir de tous ceux où Amelot le trouverait. Il a peu servi,
quoique avec de la valeur, a épousé une fille de Blansac, et passe sa
vie tout seul dans sa chambre, ou à la campagne, en sorte qu'on ne le
voit jamais.

Ce malheur me fait souvenir que Saint-Paul et Sérancourt se battirent en
duel à l'armée de Flandre, à la tête du camp, sans autre façon, allant
tous deux à pied dîner chez le duc de Guiche. Ils étaient tous deux
capitaines aux gardes et anciens. Saint-Paul fut tué, Sérancourt se
retira au quartier de l'électeur de Bavière. Il fut cassé aussitôt
après, et il fallut ne plus se montrer en France. Son frère, autrefois
intendant de Bourges, employa auprès du roi tout ce qu'il put
inutilement. Il vit encore, à près de cent ans, dans une santé parfaite
de corps et d'esprit et dans la société des hommes, mangeant, marchant
et vivant comme à soixante ou soixante-dix ans.

La disgrâce du maréchal de Villeroy par chez lequel le roi passait
souvent pour aller et venir de Fontainebleau, et la mort de
M\textsuperscript{me} de Montespan, produisirent une nouveauté qui eut
de grandes suites. M\textsuperscript{me} de Maintenon ne craignit plus
son fils\,; elle cessa de ce moment de le haïr comme le fils d'une
ennemie dont elle craignait les retours, et à qui elle ne pouvait
pardonner ce qu'elle lui avait été, ce qu'elle lui devait, le salaire
dont elle l'avait payé. Elle commença à vouloir du bien à ce fils comme
au frère de ces bâtards qui lui étaient si chers, et avec qui il avait
toujours vécu dans une si parfaite dépendance. Cette raison le rendit,
dès qu'il eut perdu sa mère, un homme, dans l'esprit de
M\textsuperscript{me} de Maintenon, à approcher du roi, qu'on tiendrait
toujours par ses vices, de la bassesse desquels rien n'était à craindre
et tout au contraire à profiter. Il fut donc déclaré que le roi irait
coucher chez d'Antin à Petit-Bourg, le 12 septembre.

C'est un prodige que les détails jusqu'où d'Antin porta ses soins pour
faire sa cour de ce passage, et pour la faire jusqu'aux derniers valets.
Il gagna ceux de M\textsuperscript{me} de Maintenon, pendant qu'elle
était à Saint-Cyr, pour entrer chez elle. Il y prit un plan de la
disposition de sa chambre, de ses meublés, jusqu'à ses livres, à
l'inégalité dans laquelle ils se trouvaient rangés ou jetés sur sa
table, jusqu'aux endroits des livres qui se trouvèrent marqués. Tout se
trouva chez elle à Petit-Bourg précisément comme à Versailles, et ce
raffinement fut fit remarqué. Ses attentions pour tout ce qui était
considérable en crédit, maîtres ou valets, et valets principaux de
ceux-là, furent à proportion, et pareillement les soins, la politesse,
la propreté pour tous les autres meuble, commodités de toutes les
sortes, abondance et délicatesse dans un grand nombre de tables,
profusion de toute espèce de rafraîchissements, service prompt et à la
main sitôt que quelqu'un tournait la tête, prévention\footnote{Ce mot
  est pris ici dans le sens de \emph{prévenance}.}, prévoyance,
magnificence en tout, singularités différentes, musique excellente,
jeux, bidets et calèches nombreuses et galantes pour la promenade, en un
mot tout ce que peut étaler la profusion la plus recherchée et la mieux
entendue. Il trouva moyen de voir tout ce qui était dans Petit-Bourg,
chacun dans sa chambre, souvent jusqu'aux valets, et de faire à tous les
honneurs de chez lui, comme s'il n'y eût eu que la personne à qui il les
faisait actuellement. Le roi arriva de bonne heure, se promena fort et
loua beaucoup. Il fit après entrer d'Antin chez M\textsuperscript{me} de
Maintenon avec lui qui lui montra le plan de tout Petit-Bourg. Tout en
fut approuvé, excepté une allée de marronniers qui faisait merveilles au
jardin et à tout le reste, mais qui ôtait la vue de la chambre du roi.
D'Antin ne dit mot, mais le lendemain matin le roi, à son réveil, ayant
porté la vue à ses fenêtres, trouva la plus belle vue du monde, et non
plus d'allée ni de traces que s'il n'y en eût jamais eu où elle était la
veille\,; ni de traces de travail ni de passage dans toute cette
longueur, ni nulle part auprès, que si elle n'eût jamais existé.
Personne ne s'était aperçu d'aucun bruit, d'aucun embarras, les arbres
étaient disparus, le terrain uni au point qu'il semblait que ce ne
pouvait être que l'opération de la baguette de quelque fée bienfaisante
du château enchanté. Les applaudissements récompensèrent la galanterie.
On remarqua fort aussi le motet de la messe du roi, qui convenait à un
bon courtisan.

Avec tout cela il en fit tant que M\textsuperscript{me} de Maintenon ne
put s'empêcher de lui faire une plaisanterie un peu amère, en partant le
lendemain pour Fontainebleau. Après avoir fait le tour des jardins en
calèche, elle lui dit, et devant le monde, qu'elle se trouvait bien
heureuse de n'avoir pas déplu au roi le soir, chez lui, parce qu'elle
était très assurée par tout ce qu'il venait de faire, qu'en ce cas-là il
l'eût envoyée coucher sur le pavé du grand chemin. Il répondit en homme
d'esprit, et n'en augura pas plus mal de sa fortune, d'autant qu'il
voyait par ce passage chez lui pointer ce qu'il avait toujours espéré de
la mort de sa mère. Quinze jours après il en fut certain. Sourdis, dont
j'ai assez parlé pour n'avoir plus rien à en dire, mourut dans sa
retraite en Guyenne. Il était le dernier Escoubleau, et ne laissait
qu'une fille mariée au fils de Saint-Pouange, et il avait le
gouvernement d'Orléanais, qui est fort étendu et où d'Antin avait
plusieurs terres. Il le demanda et l'obtint aussitôt. Il en fut si
transporté qu'il s'écria qu'il était dégelé\,; que le sort était levé\,;
que, puisque le roi commençait à lui donner, il n'était plus en peine de
sa fortune. Sa femme, plus bête et plus sotte qu'on n'en vit jamais, se
mit à bavarder partout que son mari désormais allait cheminer beau
train. Ces enthousiasmes édifièrent d'autant moins la cour qu'elle
commença à en craindre le pronostic qui par la suite eut un
accomplissement entier.

En même temps mourut Bartet à cent cinq ans, sans avoir jamais été
marié. C'était un homme de peu, qui avait de l'esprit, de l'ardeur et
beaucoup d'audace, et qui avait été fort dans le grand monde, et
longtemps en beaucoup d'intrigues et de manèges avec le cardinal de
Mazarin qui l'avait fait secrétaire du cabinet du roi, dont il était
fort connu et de la reine mère. Il avait été fort gâté comme sont ces
sortes de gens qui peuvent beaucoup servir et nuire. Il en était devenu
fort insolent et s'était rendu redoutable. Des impertinences qui lui
échappèrent souvent sur M. de Candale lui attirèrent enfin de sa part
une rude bâtonnade qu'il lui fit donner, et qu'il avoua
hautement\footnote{Voy. les notes à la fin du volume.}. Bartet, outré au
point qu'on le peut juger à ce portrait, fit les haut cris, et ce qui
mit le comble à son désespoir, c'est qu'il n'en fut autre chose. Là
commença son déclin, qui fut rapide et court. Dès qu'on ne le craignit
plus, il sentit combien ses insolences avaient révolté tout le monde\,;
on fut ravi de son aventure, on trouva qu'il l'avait bien méritée\,; les
ministres, les courtisans du haut parage furent ravis d'en être
délivrés\,; chacun, au lieu de le protéger, contribua à sa chute\,; et
quand de dépit il se fut retiré, ils se gardèrent bien de le faire
revenir\footnote{Bartet ne quitta pas la cour immédiatement. Ses lettres
  à Mazarin prouvent, au contraire, que plusieurs années après
  l'événement dont parle Saint-Simon, il était encore le confident
  intime du cardinal. Voy. les notes placées à la fin de ce volume.}.
Accoutumé à nager dans le grand, il n'avait fait aucuns retours sur
lui-même, ne doutant pas d'une fortune proportionnée à l'importance de
ce qui lui passait par les mains. Tout à coup il se trouva tombé de
tout, et sans autre bien que la rage dans le coeur. Le vieux maréchal de
Villeroy, grand courtisan du cardinal Mazarin, et qui avait fort
pratiqué Bartet chez lui, en eut plus de pitié que ce ministre qui
survécut M. de Candale deux ans. Quand Bartet ne sut plus où donner de
la tête, il le retira chez lui auprès de Lyon dans un beau lieu, sur le
bord de la Saône, qu'ils avaient acheté et appelé Neuville\,; il lui
fournit là quelque subsistance, que l'archevêque de Lyon et le second
maréchal de Villeroy continuèrent jusqu'à sa mort. Il eut là tout
loisir, pendant plus de quarante ans, de réflexion et de pénitence.

En ce même mois de septembre mourut à Grenoble le cardinal Le Camus, à
soixante-seize ans, également connu par son esprit, ses débauches, son
impiété, sa pénitence, la fortune qui en résulta, l'ambition avec
laquelle il la reçut et en usa, et le châtiment qu'il en porta jusqu'au
dernier jour de sa vie. Il n'est guère de problème qui présente plus de
choses opposées que la conduite de ce prélat, depuis le commencement
jusqu'à la fin. Il était bienfait, galant, avait mille grâces dans
l'esprit, d'une compagnie charmante. Il était savant, gai, amusant
jusque dans sa pénitence. Il acheta une charge d'aumônier du roi pour se
fourrer à la cour, et se frayer un chemin à l'épiscopat. Ses débauches
et ses impiétés éclatèrent. Il se crut perdu et s'enfuit dans une
retraite profonde, où il se mit à vivre dans toutes les austérités de la
plus dure pénitence. Sa famille avait des amis et des protecteurs. Cette
pénitence fut vantée\,; elle avait duré des années, elle durait encore,
elle fut couronnée de l'évêché de Grenoble. Il s'en crut indigne et eut
grand'peine à l'accepter. Il s'y confina et s'y donna tout entier au
gouvernement de son diocèse, sans quitter ce qu'il put retenir de sa
pénitence. Il s'était condamné aux légumes pour le reste de sa vie. Il
les continua et mangeait chez lui en réfectoire avec tous ses
domestiques, sa livrée même, et la lecture s'y faisait pendant tout le
repas.

Innocent XI, qui aimait la vertu, fut touché de la sienne, et le fit de
son propre mouvement cardinal dans la promotion de septembre 1686, de
vingt-sept cardinaux, qui fut sa dernière, et qui fut aussi pour les
couronnes et les nonces. Le courrier qui apporta la nouvelle et les
calottes au célèbre évêque de Strasbourg Fürstemberg, nommé par le roi,
et à Ranuzzi, nonce en France, passa par Grenoble pour Le Camus. Sa joie
fut telle qu'il en oublia son devoir. Il se mit la calotte rouge sur la
tête, que le courrier lui présenta, puis écrivit au roi une lettre fort
respectueuse, au lieu d'envoyer sa calotte au roi par ce même courrier,
de lui mander qu'étant son sujet il ne voulait rien tenir que de sa
main, et qu'il attendait ses ordres sur la conduite qu'il lui plairait
de lui prescrire. S'il en eût usé ainsi, il n'est pas douteux que le roi
lui aurait mandé de la venir recevoir de sa main, ou la lui aurait
renvoyée avec la permission de la porter et d'accepter\,; mais, piqué de
ce qu'il l'avait prise de lui-même, et d'un pape avec qui il était
brouillé, il fut sur le point de lui défendre de la porter et
d'accepter, et de se porter aux extrémités, s'il n'obéissait pas.
Néanmoins, réflexion faite sur les suites de cet engagement, il se
contenta pour toute réponse de lui défendre de sortir de son diocèse. Il
n'est rien que le cardinal n'ait fait alors et depuis pour se
raccommoder, et pour qu'il lui fût permis de venir montrer sa calotte à
Paris et à la cour. Mais le roi tint ferme jusqu'à sa mort. Il ne lui
permit pas même d'aller à Rome pour le conclave qui suivit la mort
d'innocent XI\,; il obtint d'aller aux deux suivants, mais à condition
de ne s'arrêter nulle part, et de revenir sitôt que le pape serait élu
et couronné. Il ne laissa pas de s'y conduire extrêmement bien, et tout
à fait à la satisfaction des cardinaux français.

On a vu, à l'occasion du passage des princes à Grenoble, à quel point il
fut toute sa vie enivré de sa dignité. Elle lui attira des remontrances
sur sa santé et sur ses légumes\,: «\,Oh\,! mes chers légumes,
s'écria-t-il, je vous ai trop d'obligation pour vous abandonner
jamais.\,» En effet, il leur fut fidèle jusqu'au bout et à son
réfectoire, où il faisait servir à ses domestiques de la viande et des
nourritures ordinaires. Il fut jusqu'à la mort bourrelé de sa disgrâce,
et toujours d'excellente compagnie. Il voulait savoir toutes les petites
intrigues de sa ville, il en parlait fort plaisamment. Il embarrassait
souvent les intéressés. On lui reprochait sa langue, il avouait qu'elle
était plus forte que lui\,; et en effet, il lui refusait peu de choses.
Quoiqu'il n'eût presque de bénéfices que son évêché, qui n'est pas gros,
et cent mille écus de patrimoine, quoiqu'il donnât beaucoup aux pauvres,
et qu'il eût fait de bons établissements à ses dépens, l'énormité de son
testament surprit et scandalisa à sa mort. Il donna fort gros en bonnes
oeuvres, et laissa plus de cinq cent mille livres à sa famille. Il était
frère du premier président de la cour des aides de Paris et du
lieutenant civil de la même ville.

Le comte d'Egmont mourut à Fraga, en Catalogne, ce mois de septembre
1707, à trente-huit ans, sans enfants de la nièce de l'archevêque d'Aix,
Cosnac, élevée chez la duchesse de Bracciano, à Paris, comme sa nièce,
depuis princesse des Ursins, desquels j'ai tant parlé. Il fut le dernier
de ces fameux Egmont, et le dernier mâle de cette grande maison. Il
avait la Toison, ainsi que ses pères, et il était général de la
cavalerie et des dragons d'Espagne et brigadier de cavalerie en France.
C'était un homme fort laid, de peu d'esprit, de beaucoup de valeur,
d'honneur et de probité, et qui s'appliquait fort à la guerre. Son
trisaïeul était frère de ce célèbre Lamoral, comte d'Egmont, à qui le
duc d'Albe fit couper la tête. Celui-ci avait succédé à son frère aîné,
mort sans enfants d'une Aremberg, veuve du marquis de Grana, gouverneur
des Pays-Bas. Il fit peu de jours avant sa mort un testament par lequel
il légua au roi d'Espagne toutes ses prétentions et ses droits sur les
duchés de Gueldre et de Juliers, sur les souverainetés d'Arkel, de
Meurs, Horn, les seigneuries d'Alkmaer, Purmerend, etc., et tous ses
biens à sa sœur, qui avait épousé Nicolas Pignatelli, duc de Bisaccia,
gouverneur des armes du royaume de Naples, retiré à Paris, dont le fils
aîné a épousé la seconde fille du fendue de Duras, fils et frère aîné
des maréchaux-ducs de Duras. Ce comte d'Egmont avait une soeur, cadette
de celle-là, mariée au vicomte de Trasignies, mais tous les biens avec
la grandesse ont passé au fils de la duchesse de Bisaccia dont je viens
de parler, et qui porte le nom de comte d'Egmont et les armes.

La comtesse de Soissons, veuve de celui qui fut tué devant Landau, frère
aîné du prince Eugène, était dans un couvent à Turin. Elle tint des
propos, je ne sais sur quoi, qui la firent chasser par M. de Savoie de
ses États. Arrivée à Grenoble, elle écrivit à M\textsuperscript{me} de
Maintenon pour la prier de lui accorder Saint-Cyr pour retraite.
Chamillart lui manda par ordre du roi de n'entrer pas plus avant dans le
royaume. Elle n'en dit mot et arriva à Nemours, tout auprès de
Fontainebleau, où le roi était. Il envoya lui commander d'en partir
sur-le-champ, de s'aller mettre dans un couvent à Lyon, où elle alla.

La cour de Saint-Germain vint à Fontainebleau le 23 septembre et y
demeura jusqu'au 6 octobre. Le roi y demeura jusqu'au 25 octobre, qu'il
s'en retourna à Versailles par Petit-Bourg, comme il avait fait en
venant.

Revel, que la surprise et la reprise de Crémone avait fait chevalier de
l'ordre, mourut en ce même temps. Il avait épousé, au commencement de
juillet dernier, une soeur du duc de Tresmes, dont il ne laissa point
d'enfants et fort peu de biens. Il était frère de Broglio, que M. le
Duc, de sa grâce, fit en son temps maréchal de France, par la raison que
le Roule est devenu faubourg de Paris. Sa dernière campagne de guerre
avait été celle où le maréchal de Créqui avait été battu à Consarbrück.
Il y était maréchal de camp et n'avait pas servi depuis. Nous voyons son
second fils maréchal de France à meilleur titre. Puységur eut le
gouvernement de Condé qu'avait Revel.

La maréchale de Tourville mourut aussi à peu près en ce même temps. Elle
n'était rien, veuve de La Popelinière, homme d'affaires et riche.
Quoiqu'elle en eût des enfants, elle était assez riche pour que
Tourville eût envie de l'épouser. Langeois, homme d'affaires, fort
riche, donna beaucoup à sa fille pour ce mariage et les logea. Cela ne
dura guère, le mariage ne fut pas heureux. Il en resta un fils, tué dès
sa première campagne, et une fille fort belle, qui a épousé M. de
Brassac, et que la petite vérole, sans la défigurer, a rendue
méconnaissable. Elle a été dame de M\textsuperscript{me} la duchesse de
Berry.

Le faux-saunage continua à causer force désordres. Des cavaliers, des
dragons, des soldats, par bandes de deux ou trois cents hommes, le
firent à forée ouverte, pillèrent les greniers à sel de Picardie et de
Boulonnais, et se mirent à le vendre publiquement. Il y fallut envoyer
des troupes et on détacha deux cents hommes du régiment des gardes,
qu'on y fit marcher sous des sergents sages et entendus. Il y eut de
grands désordres en Anjou et en Orléanais. On résolut de décimer ces
faux-sauniers, et on envoya à leurs régiments les colonels qui avaient
des gens de ce métier dans leurs troupes.

Listenais, qui était un fou sérieux, aussi fou que ceux qu'on enferme,
et dont le frère, Beaufremont, ne l'est pas moins, imagina un moyen
d'escroquer douze cents pistoles à la comtesse de Mailly, sa belle-mère,
qui fit grand bruit par le tour de l'invention. Il signa une lettre
écrite d'une main inconnue à son homme d'affaires, en Franche-Comté, par
laquelle il lui mandait que, revenant de l'armée du Rhin, il avait été
pris entre Benfeld et Strasbourg\,; qu'il ne peut avertir du lieu ni des
mains entre lesquelles il est, mais qu'en payant comptant douze cents
pistoles à un homme qu'il enverra les recevoir à Besançon, il sera mis
en liberté. M\textsuperscript{me} de Mailly, qui apprit cette nouvelle
par cet homme d'affaires, fit remettre la somme, et, avec une sage
défiance, n'en dit mot. Mais le bruit qu'en avait fait l'homme
d'affaires s'était répandu dans cette province, et de là était parvenu à
Paris et à la cour. La date de cette capture était antérieure au départ
de Strasbourg du maréchal de Villars, qui n'en avait pas ouï parler, ni
depuis son arrivée. Aucune lettre de la frontière depuis n'en faisait
mention. L'aventure parut des plus extraordinaires. Quinze jours après,
un valet de chambre de Listenais arriva à Versailles pour chercher
l'argent demandé qu'il se défiait avoir été rendu à Besançon. Il dit
avoir été toujours avec lui depuis sa prise. Il assura que, dès qu'il
aurait touché l'argent, son maître serait mis en liberté. On voulut le
faire suivre, mais il s'écria qu'on s'en gardât bien, parce qu'au
moindre soupçon qu'auraient ceux qui le tenaient d'être découverts, ils
le tueraient. Ce voyage et ce propos mirent l'affaire au net, et
M\textsuperscript{me} de Mailly en fut pour son argent.

Autres quinze jours après, on apprit que Listenais était chez lui en
fort bonne santé à Besançon. Huit jours ensuite, il arriva à l'Étang. Il
dit à Chamillart qu'il avait été pris par des officiers ennemis, que
tous les bruits qui avaient couru depuis sur lui étaient faux\,; qu'il
lui donnerait par écrit le récit de toute son aventure\,; qu'il le
priait d'en faire examiner la vérité\,; que, quand il en serait
suffisamment éclairci, il le priait d'en rendre compte au roi, et que,
s'il s'y trouvait la moindre fausseté, il méritait d'être rigoureusement
puni. On entendit bien ce que tout cela voulait dire. Il n'en coûtait
rien au roi, il n'y avait que M\textsuperscript{me}, de Mailly
d'attrapée, qui aimait mieux perdre son argent que son gendre. Elle
était nièce de M\textsuperscript{me} de Maintenon, elle était en place
et fort amie de Chamillart\,; Listenais reparut à la cour et il n'en fut
pas parlé davantage, mais personne ne s'y méprit, et Listenais n'y
perdit rien, parce qu'il n'avait rien à perdre.

On a vu (t. III, p.~391 et suiv.) ce qui se passa entre le roi, Catinat
et Chamillart, quand le roi voulut se resservir de Catinat, après
l'avoir fait honteusement revenir d'Italie pour y envoyer son maréchal
de Villeroy réparer les torts d'un général si différent de lui.
L'anecdote en est extrêmement curieuse. Quelque sagesse au-dessus de
l'homme que Catinat eût fait paraître en cette occasion, où il eut tant
d'avantage en résistant au roi, qui le pressait de nommer et de lui
parler à coeur ouvert sur l'Italie, Chamillart qui avait eu toute la
frayeur d'être chassé, et Tessé d'être perdu sans ressource ne purent la
lui pardonner, ni se résoudre à retomber une autre fois sous sa coupe,
quelque généreux et chrétien qu'il se fût montré alors. Tessé, valet à
tout faire de Chamillart tant qu'il fut en faveur, n'omit rien pour
l'engager à perdre Catinat, et le mettre hors de toute portée
d'inquiéter leur fortune. Ce n'était pas qu'il ne dût la sienne tout
entière à Catinat qui l'avait toujours distingué dans la guerre de 1688
en Italie, et qui le produisit pour être chargé de la négociation de la
paix particulière de Savoie et du mariage de M\textsuperscript{me} la,
duchesse de Bourgogne. Son patron Louvois était mort alors, Barbezieux,
à peine en fonction, n'avait pas encore les reins assez forts pour
porter bien haut personne, et ce fut au seul Catinat à qui Tessé dut la
confiance de ce traité qui lui valut sa charge, le poussa rapidement au
grand, et acheva sa fortune. On a vu qu'il la trouva trop lente, et de
quelle ingratitude il paya son bienfaiteur en cette même Italie, sans
aucune autre, cause que de l'accélérer à ses dépens, combien il y fut
trompé et Vaudemont aussi dont il avait fait son nouveau maître par
l'envoi du maréchal de Villeroy, et toutes ses souplesses avec celui-ci
qui ne furent pas capables de l'empêcher de l'arrêter sur ses excès à
l'égard de Catinat. Je l'ai dit plus d'une fois, et je le répète, parce
que c'est une expérience infaillible\,: les injures que l'on a faites se
pardonnent infiniment moins que celles qu'on a reçues\,; et c'est ce qui
engagea Tessé à ne garder aucune mesure avec Catinat, qui en avait gardé
avec lui de si difficiles, et qui, ayant de quoi le perdre et pressé par
le roi de parler, ne l'avait pas voulu. Ce risque commun d'alors de lui
et de Chamillart qui l'avait échappé si belle, excita Tessé pour s'en
mettre à l'abri pour toujours, de pousser Chamillart à mettre Catinat
hors de portée, et c'est ce que ce ministre exécuta si bien en
dépouillant ce général de toutes ses troupes sur le Rhin, pour faire
tomber dans le néant en élevant Villars sur le pavais. On a vu depuis
Catinat enveloppé de sa gloire, de sa sagesse, de son mérite, retiré en
silence à Saint-Gratien, refuser l'ordre, et se tenir dans le silence et
l'éloignement.

L'affaire de Provence effraya intérieurement le roi au point de sortir
de son caractère pour chercher du remède partout. Il fit secrètement
consulter Catinat, qui fit un mémoire là-dessus, qu'il envoya au roi. Le
roi le goûta. Je ne sais si l'envie lui reprit de se servir encore de
Catinat qui n'en eut aucune, mais il lui fit dire de venir à Versailles.
Il n'avait pas vu Chamillart depuis son dernier retour du Rhin dont je
viens de parler, qui était en 1702\,; et quoique M. de Beauvilliers fût
fort ami de Chamillart, il l'était beaucoup aussi de Catinat, dont il
connaissait et respectait la vertu. C'était par lui qu'avait passé cette
dernière consultation et l'ordre de venir à Versailles. Il s'y présenta.
C'était à la fin de novembre, comme le roi achevait de s'habiller. Dès
que le roi l'aperçut, il lui dit qu'il lui voulait parler, et le fit
entrer dans son cabinet. Il lui loua son mémoire, en raisonna avec lui,
et lui fit beaucoup d'honnêtetés. C'était un guet-apens. La conclusion
fut de lui dire en propres termes qu'il avait une prière à lui faire,
qu'il espérait qu'il ne lui refuserait pas. Le maréchal se confondit, le
roi reprit la parole, et lui dit\,: «\,Monsieur le maréchal, votre
mésintelligence avec Chamillart m'embarrasse, je voudrais vous voir
raccommodés. C'est un homme que j'aime et qui m'est nécessaire, je vous
aime et vous estime fort aussi.\,» Le maréchal répondit qu'il s'en
allait à l'heure même chez lui. «\,Non, lui dit le roi, cela n'est pas
nécessaire, il est là derrière, je vais l'appeler.\,» Il l'appela
aussitôt, et la réconciliation devant le roi fut bientôt faite. Dès que
Chamillart fut retourné chez lui, Catinat alla lui rendre visite. En
sortant, Chamillart le conduisit, comme il le devait, jusqu'au dernier
bout de son appartement, long et vaste, sans que Catinat l'en pût
empêcher. En se séparant le maréchal lui dit\,: «\,Vous avez voulu,
monsieur, faire cette façon, mais je vous supplie que ce soit pour la
dernière fois, afin que vous me regardiez comme un ami et un serviteur
particulier, et que le public le sache.\,» C'eût été là pour un autre un
trait de courtisan. En Catinat qui n'en voulait faire aucun usage, c'en
fut un d'une rare modestie et d'une parfaite soumission pour ce que le
roi désira de lui, et fort au delà de ce qu'il lui avait demandé. Telle
était sa faiblesse pour ses ministres. Très peu de jours après cette
réconciliation, le roi fut assez longtemps le soir chez
M\textsuperscript{me} de Maintenon avec Chamillart et Tessé. On sut
après que ce maréchal ne servirait plus\,: il se dit en soupçon d'avoir
besoin de la grande opération. On n'ajouta pas grande foi à une
incommodité si subite et si cachée.

Le roi d'Espagne montra une autre sorte de faiblesse qui scandalisa
étrangement tous les grands seigneurs. Ce fut de donner la Toison au
marquis de Bay, qu'il n'avait point encore avilie, mais qu'il avilit
souvent depuis. Ce prétendu marquis de Bay était fils d'un cabaretier de
Gray, en Franche-Comté, qui s'était poussé à la guerre, et qui en effet
la fit fort heureusement et fort utilement, cette campagne, en
Estrémadure.

Le comte d'Auvergne mourut enfin à Paris, le 23 novembre, d'une longue
et fort singulière maladie, où les médecins ne connurent rien peut-être
pour y connaître trop. Il vit avant de mourir l'abbé d'Auvergne son
fils, aujourd'hui cardinal, qu'il avait chassé de chez lui, et avec qui
il était horriblement brouillé. C'était un fort gros homme, qui vint à
rien avant qu'être arrêté dans sa chambre. Il ne ressemblait pas mal à
un sanglier, et toujours amoureux. C'était le meilleur homme du monde à
qui n'avait que faire à lui, le plus difficile quand on y avait affaire.
Il était pointilleux même dans le commerce, aisé à blesser, difficile à
revenir\,; honnête homme pourtant, mais père qui eut bien du tracas dans
sa famille avec ses enfants pour le bien de leur mère\,; glorieux à
l'excès et toujours embarrassé de sa princerie.

Il ne jouit pas longtemps du plaisir de savoir le prince d'Auvergne
(celui qui avait déserté et qui avait pris le service de Hollande) marié
à la soeur du duc d'Aremberg. Le comte d'Évreux, qui avec sa charge de
colonel général de la cavalerie qu'il avait eue de lui, se crut toute sa
dépouille due, n'eut point son logement à Versailles qui fut donné au
maréchal de Villars, ni son gouvernement de Limousin qui fut donné au
duc de Berwick. Il ne le pardonna à l'un ni à l'autre, se plaignit d'eux
amèrement, surtout du dernier, et n'a jamais vécu depuis avec lui qu'en
froideur tout à fait marquée. C'est ainsi qu'on essaye de tourner les
grâces en patrimoine.

Le mariage du prince de Talmont, frère du duc de La Trémoille, malgré la
mésalliance et les cris de Madame, étendit personnellement pour lui les
commencements d'avantages que leur grand'mère avait habilement saisis,
qui donneront lieu ici à une curiosité historique pour en expliquer le
rare prétexte\,; mais il faut reprendre la chose d'un peu loin.

\hypertarget{chapitre-vii.}{%
\chapter{CHAPITRE VII.}\label{chapitre-vii.}}

1707

~

{\textsc{Digression sur la chimère de Naples\,; les trois maisons de
Laval, et l'origine et la nature des distinctions dont jouissent les
ducs de La Trémoille.}} {\textsc{- Mort de Moreau\,; son caractère.}}
{\textsc{- Transcendant et singulier éloge de la piété de Mgr le duc de
Bourgogne.}} {\textsc{- Mort de l'archevêque de Rouen, Colbert\,; son
caractère\,; sa dépouille.}} {\textsc{- Époque de la conservation du
rang, et honneurs aux évêques-pairs transférés en autres sièges.}}
{\textsc{- Mort de l'archevêque d'Aix, Cosnac.}} {\textsc{- Mort et
caractère du chevalier de Lauzun.}} {\textsc{- Mort de Valsemé.}}
{\textsc{- Mort de M\textsuperscript{me} d'Armagnac\,; son caractère.}}
{\textsc{- Époque de visiter en manteau et en mante les princes et
princesses du sang pour les deuils de famille.}} {\textsc{- M. le Grand
veut épouser M\textsuperscript{me} de Châteauthiers, qui le refuse.}}
{\textsc{- Son caractère et sa fin.}} {\textsc{- Mort de Villette.}}
{\textsc{- Ducasse et d'O lieutenants généraux des armées navales.}}
{\textsc{- D'O et Pontchartrain raccommodés.}} {\textsc{- Le roi
s'entremet entre le duc de Rohan et son fils.}} {\textsc{- Caractère du
prince de Léon.}} {\textsc{- Chute d'un plancher du premier président.}}
{\textsc{- Retour du duc de Noailles.}} {\textsc{- Villars à
Strasbourg.}} {\textsc{- Quatre cent mille livres de brevet de retenue
au de Tresmes.}} {\textsc{- Retour de M. le duc d'Orléans.}}

~

Sans entrer dans une digression trop longue des droits et des guerres
des deux branches d'Anjou et de la maison d'Aragon légitime, puis
bâtarde, pour les royaumes de Naples et de Sicile, il suffit de se
rappeler que Jeanne II, reine de Naples et de Sicile, mit le feu, par
ses diverses adoptions, entre les deux branches d'Anjou. Cette couronne
tomba à Jeanne II, après diverses cascades et de grandes guerres.
Celle-ci ne fut ni plus chaste ni plus heureuse que la première Jeanne,
ni plus avisée en mariages et en adoptions. Celle qu'elle fit en faveur
d'Alphonse V, roi d'Aragon, combla tous ses malheurs, et, par les
événements, ôta les royaumes de Naples et de Sicile à la maison de
France, qui demeurèrent, après maintes révolutions, à la maison
d'Espagne.

Pierre le Cruel, tué et vaincu par son frère bâtard, Henri, comte de
Transtamare, aidé par le célèbre du Guesclin et par la France, fut roi
de Castille en sa place, et laissa cette couronne à Jean, son fils,
gendre de Pierre IV, roi d'Aragon. Jean, roi de Castille, laissa deux
fils, Henri le Valétudinaire et Jean. Le Valétudinaire mourut à
vingt-sept ans, et laissa son fils, Jean II, âgé de vingt-deux mois. La
couronne de Castille fut déférée à Jean, son oncle paternel, qui la
refusa constamment, et servit de père à son neveu. Ce neveu, qui devint
un grand roi, fut le père d'Henri III, dit l'Impuissant, et de la
fameuse Isabelle, après son frère reine de Castille qui par son mariage
avec Ferdinand le Catholique, roi d'Aragon, réunit toutes les Espagnes,
excepté le Portugal qu'ils firent passer à leur postérité assez connue.

Ce généreux Jean, qui refusa et conserva la couronne de Castille à son
neveu, en fut tôt après récompensé. Jean Ier Martin, frère de sa mère,
et l'un après l'autre rois d'Aragon, moururent, le premier sans enfants,
le second sans postérité masculine\,; ses filles furent méprisées, et ce
généreux Jean de Castille, leur cousin germain, fut élu roi d'Aragon par
les états. Il régna paisiblement, et il laissa sa couronne à son fils,
Alphonse V, qui fut adopté par Jeanne Il, reine de Naples et de Sicile.
Cet Alphonse V n'eut point d'enfants légitimes. Il fit roi de Naples et
de Sicile, par son abdication et par le consentement de son parti,
Ferdinand son bâtard Jean II, son frère, lui succéda à la couronne
d'Aragon, et fut père de Ferdinand le Catholique, qui, par son mariage
avec Isabelle, reine de Castille, réunit toutes les Espagnes comme je
viens de le dire\,; et, comme on le voit, Isabelle et Ferdinand le
Catholique étaient issus de germains et de même maison, c'est-à-dire que
le comte de Transtamare était également de mâle en mâle leur trisaïeul.

Alphonse, bâtard d'autre Alphonse susdit roi d'Aragon, par l'abdication
duquel il devint roi de Naples et de Sicile, comme on vient de le dire,
y régna trente-sept ans, toujours en guerre ou en troubles, laissa sa
couronne à Alphonse VI, son fils, qui ne la posséda pas plus
tranquillement. Il l'abdiqua en faveur de Jean II son fils, qui mourut à
la fleur de son âge sans enfants. Frédéric II, son oncle paternel, lui
succéda. Ferdinand le Catholique, dont son père était, par bâtardise,
cousin germain, ne laissa pas de le dépouiller de concert avec Louis
XII, qu'il trompa ensuite cruellement, et acquit ainsi à soi et à sa
postérité les royaumes de Naples et de Sicile. Frédéric II vint mourir
de chagrin en France\footnote{Ce roi, qui régna de 1496 à 1501, est
  ordinairement désigné sous le nom de Frédéric III. En effet, il y
  avait eu antérieurement, en Sicile, deux rois du nom de Frédéric\,: au
  XIIe siècle, Frédéric Ier (1197-1250), et Frédéric II (1355-1374).
  Cependant, comme ce dernier ne régna que sur la Sicile, alors séparée
  du royaume de Naples, on a quelquefois donné, comme le fait ici
  Saint-Simon, le nom de Frédéric II au prince qui régnait à la fin du
  XVe siècle.}. Ainsi finit, à Naples et en Sicile, le règne de ces
bâtards d'Aragon.

Ce Frédéric II, dépouillé et mort en France en 1509, avait épousé une
fille d'Amédée IX, duc de Savoie, puis Isabelle des Baux, fille du
prince d'Altamura. Il laissa trois fils et trois filles. Je ne
m'arrêterai point aux trois fils, parce qu'ils moururent tous trois sans
enfants, et finirent ainsi ces célèbres bâtards d'Aragon. La seconde des
filles mourut jeune, sans avoir été mariée\,; la cadette épousa
Jean-Georges, marquise de Montferrat\,; l'aînée, dont il est question
ici, le comte de Laval, et fut mère de la dame de La Trémoille. Après
avoir expliqué ces droits et cette bâtarde descendance d'Aragon,
éclaircissons un peu ces comtes de Montfort, où cette race bâtarde
fondit avec ces prétentions, et de là dans la maison de La Trémoille.

Trois maisons de Laval, qu'il ne faut pas confondre\,: celle de Laval
proprement dite, fondue par l'héritière dans la maison de Montmorency\,;
le second connétable Matthieu II de Montmorency l'épousa en secondes
noces, ayant des fils de sa première femme, de Gertrude de Soissons\,;
il en eut deux de la seconde, dont l'aîné, Guy, prit le nom de Laval, et
brisa la croix de Montmorency de cinq coquilles. Il fut chef de la
branche de Montmorency-Laval, qui dure encore depuis cinq cents ans\,;
c'est elle qu'on connaît sous le nom impropre de la seconde maison de
Laval. Le cinquième petit-fils de ce chef de la branche de
Montmorency-Laval, d'aîné en aîné, ne laissa qu'un fils et une fille. Le
fils, déjà fiancé avec une fille de Pierre II, comte d'Alençon, tomba à
la renverse dans un puits découvert de la grande rue de Laval, où il
jouait à la paume, en 1413, et en mourut huit jours après, et sa soeur
fut son héritière.

Elle avait épousé en 1404, en présence de Jean, duc de Bretagne, Jean de
Montfort, fils aîné de Raoul, sire de Montfort en Bretagne, de Lohéac et
de La Roche-Bernard et de Jeanne, dame de Kergorlay. Par un des articles
du contrat de mariage, Jean de Montfort fut obligé à prendre les noms,
armes et cri de Laval\footnote{Tous les gentilshommes n'avaient pas de
  cri de guerre. C'était un privilège réservé aux seigneurs bannerets,
  ou ayant droit de porter bannière et de marcher à la tête d'une troupe
  de vassaux qui se ralliaient à leur cri de guerre.}, et de céder les
siennes à Charles de Montfort son frère puîné. Jean de Montfort et toute
sa postérité y furent si fidèles, que tous les pères de sa femme, depuis
le puîné du connétable, ayant eu pour nom de baptême Guy, tous les
Laval-Montfort, à cet exemple des Laval-Montmorency, prirent tous le nom
de baptême de Guy, jusqu'à changer le leur quand de cadets ils devinrent
aînés, et prirent le nom de Guy en même temps que celui de comtes de
Laval. C'est cette maison de Montfort, en Bretagne, qui a fait la
troisième maison de Laval. Avant ce mariage, elle portait d'argent à la
croix de gueules, givrée\footnote{\emph{Givré}, en terme de blason,
  signifie portant un serpent dans ses armes.} d'or. Il ne faut pas la
confondre avec les Montfort-l'Amaury de la croisade des Albigeois, qui
étaient bâtards de France. Ceux-ci étaient originaires de Bretagne, où
on ne voit pas même qu'ils aient figuré avant cette riche alliance\,;
mais depuis, bien que fort inférieurs en tout à la maison de
Montmorency, ils l'égalèrent bientôt en biens et en établissements, et
la surpassèrent de beaucoup en rang et en alliances, et figurèrent très
grandement jusqu'à leur extinction. Cette grandeur des Montfort a
continuellement été prise par les gens peu instruits, qui font la
multitude, pour des grandeurs des Laval-Montmorency, dont, pendant la
régence de M. le duc d'Orléans, le comte de Laval, qui fut mis à la
Bastille, chercha à s'avantager avec aussi peu de bonne foi que de
succès.

Trois générations de ces Laval-Montfort, depuis ce mariage de
l'héritière\,; la première fut de trois frères\,; l'aîné épousa
Isabelle, fille de Jean VI, duc de Bretagne, et de Jeanne de France,
fille et soeur de Charles VI et Charles VII. Les ducs de Bretagne,
François Ier et Pierre II, étaient les frères de cette comtesse de
Laval. Laval fut érigé en comté pour son mari\,; les Montmorency ne
l'avaient eu que baronnie. Le maréchal de Lohéac et le seigneur de
Châtillon furent ses frères. Le dernier eut successivement les
gouvernements de Dauphiné, Gennes, Paris, Champagne et Brie, fut
chevalier de Saint-Michel et grand maître des eaux et forêts de France.
D'une de leurs soeurs, mariée à Louis de Bourbon, est issue la branche
qui règne depuis Henri IV. Jean VI, duc de Bretagne, avait accordé sa
fille avec Louis III, depuis duc d'Anjou, et roi de Sicile\,; il préféra
le comte de Laval, et rompit un si grand mariage et si avancé. Le
seigneur de Châteaubriant, amiral de Bretagne, qui donna tant de biens
au connétable Anne de Montmorency, était petit-fils de ce comte de Laval
et de sa seconde femme, héritière de Dinan, dont le père était grand
bouteiller de France. Ce seigneur de Châteaubriant était beau-frère sans
enfants du fameux Lautrec, maréchal de France, dit le maréchal de
Foix\,; et c'est de la dame de Châteaubriant, sa femme, dont, malgré
l'anachronisme du temps de sa mort très avéré\,; on a conté le roman des
amours tragiques du roi François Ier et d'elle.

La seconde génération fut entre autres des deux frères, car je laisse de
grandes alliances et beaucoup d'autres illustrations, pour abréger dans
toutes les trois, Guy XV, comte de Laval, et le seigneur de La
Roche-Bernard, et une soeur entre autres qui fut la seconde femme du bon
roi René, de Naples et de Sicile titulaire, mais en effet duc d'Anjou et
comte de Provence, dont elle n'eut point d'enfants. Guy XV, comte de
Laval, fut grand maître de France, après le Chabannes, comte de
Dammartin. Le fameux seigneur de Chaumont Amboise lui succéda. Il mourut
sans enfants de la fille et soeur de Jean II et de René, ducs d'Alençon,
si connus par leurs procès criminels, et tante paternelle de Charles,
dernier duc d'Alençon, en qui finit cette branche royale.

La troisième génération fut du fils unique du seigneur de La
Roche-Bernard, mort longtemps avant son frère aîné, le comte de Laval,
dont je viens de parler. Ce fils du cadet hérita de son oncle, et c'est
Guy XIV, gouverneur et amiral de Bretagne, en qui finit cette maison
troisième de Laval-Montfort, si brillante. Il mourut en 1531, et laissa
des enfants de ses trois femmes, dont aucun des mâles n'eut, de
postérité ni ne figura.

Sa première femme fut Charlotte d'Aragon, fille aînée de ce Frédéric,
mort en France, dépouillé des royaumes de Naples et de Sicile par Louis
XII et Ferdinand le Catholique. La mère de cette Charlotte d'Aragon
était fille d'Amédée IX, duc de Savoie, comme on le voit en la page 133,
et ses frères, morts sans enfants, furent les derniers mâles de cette
bâtardise couronnée d'Aragon. Ce mariage apporta au comte de
Montfort-Laval, et aux enfants qu'il en eut les chimériques droits et
les prétentions sur Naples et Sicile tels qu'on les a vus expliqués en
la page précédente, avec le vain nom de prince de Tarente, titre affecté
aux héritiers présomptifs de la couronne de Naples. De ce mariage, je ne
parle point des fils, parce qu'outre qu'il n'y en eut qu'un de cette
Aragonaise, qui fut tué en 1522, au combat de la Bicoque, aucune des
autres femmes n'eut postérité\,; ainsi je ne parlerai que des deux
filles de celle-ci. L'aînée mariée à Claude de Rieux, comte d'Harcourt,
dont la fille unique Renée de Rieux succéda à son oncle maternel, et au
père de sa mère, fut comtesse de Laval et marquise de Nesle\,; elle
quitta même son nom de baptême de Renée, pour prendre celui de Guyonne.
Elle mourut sans enfants en 1567, de Louis de Sainte-Maure (Précigny),
marquis de Nesle, en qui finit cette branche de Sainte-Maure, parce que
les deux fils qu'il eut de sa seconde femme, fille du chancelier
Olivier, ne vécurent pas. M\textsuperscript{me} de La Trémoille hérita
de tous les biens de Montfort-Laval de sa soeur aînée, et des chimères
de Naples en même temps\,: elles se trouvent assez expliquées aux pages
précédentes pour n'avoir à y revenir.

Du mariage de François de La Trémoille, vicomte de Thouars, avec Anne de
Montfort-Laval, héritière par accident de sa maison, longtemps après son
mariage, vinrent entre autres enfants trois fils. Louis III de La
Trémoille qui fut lainé, et premier duc de Thouars, par l'érection sans
pairie qu'il en obtint de Charles IX, et les deux chefs des branches de
Royan et de Noirmoutiers. Ce premier duc de La Trémoille, gendre du
connétable Anne de Montmorency, fut père du second duc de La Trémoille,
qui se fit huguenot, dont bien lui valut pour ce monde\,; cela lui fit
épouser une fille du fameux Guillaume de Nassau, prince d'Orange,
fondateur de la république des Provinces-Unies, et marier sa soeur au
prince de Condé, chef des huguenots\,; après son père, tué à la bataille
de Jarnac. La mère de la duchesse de La Trémoille était
Bourbon-Montpensier, cette fameuse abbesse de Jouars qui en sauta les
murs. Henri IV fit pair de France ce second duc de La Trémoille. Son
fils, troisième duc de La Trémoille, épousa M\textsuperscript{lle} de La
Tour, sa cousine germaine, enfants des deux soeurs\,; elle était fille
du maréchal de Bouillon et soeur de M. de Bouillon, et de M. de Turenne,
de la comtesse de Roye, de la marquise de Duras, mère des maréchaux de
Duras et de Lorges, et de la marquise de La Moussaye-Goyon. Ce duc de La
Trémoille, ou touché de la grâce, ou frappé de la décadence du parti
huguenot, avec qui il n'y avait plus guère à gagner avec les chefs qui
lui restaient, prit habilement {[}pour abjurer{]} le temps du siège de
la Rochelle, et le cardinal de Richelieu pour son apôtre. Ce premier
ministre, qui se piquait de savoir tout, et qui en effet savait
beaucoup, avait beaucoup écrit sur la controverse dans les temps de sa
vie où il n'avait pas eu mieux à faire. Il se trouva flatté de la
confiance du duc de La Trémoille en ce genre, et il ne fut pas
insensible à trouver du temps au milieu des soins de ce grand siège, et
de toutes les autres affaires, pour l'instruire et recevoir publiquement
son abjuration. La récompense en fut prompte il le fit mestre de camp
général de la cavalerie, et lui donna son amitié pour toujours. Sa femme
était digne fille de son père, et digne soeur de ses frères, elle se
garda bien de laisser faire son fils catholique\,: le père l'était,
c'était assez. Il porta le nom de prince de Tarente, dont aucun ne
s'était avisé depuis cette Charlotte d'Aragon, comtesse de
Laval-Montfort\,; sa mère eut ses raisons, et le mit au service de
Hollande, que nous protégions alors ouvertement, dans lequel il devint
général de la cavalerie, gouverneur de Bois-le-Duc, et chevalier de la
Jarretière. Son habile mère, par ses frères et par elle-même, leurs
alliances, leurs intelligences, leur religion, trouva le moyen de lui
faire épouser Émilie, fille du feu landgrave Guillaume V de
Hesse-Cassel, et d'Amélie-Élisabeth d'Hanau, cette célèbre héroïne du
siècle passé si attachée à la France. La soeur de la princesse de
Tarente épousa l'électeur palatin, et fut mère de Madame. Leur frère
Guillaume VI, grand-père du roi de Suède d'aujourd'hui, maria ses
filles, l'une au feu roi de Danemark, Christiern V, grand-père de celui
d'aujourd'hui, l'autre à l'électeur de Brandebourg, Frédéric III\,; et
cette princesse de Tarente était mère du duc de La Trémoille gendre du
duc de Créqui et du prince de Talmont, sur le mariage duquel se fait
toute cette digression.

M. de La Trémoille, quoique catholique, s'était mêlé dans les troubles
de la minorité de Louis XIV à l'appui de ses beaux-frères, mais sans y
figurer comme sa femme l'eût bien voulu. Ils avaient été continuellement
nourris par ses frères\,; ils avaient su en tirer tout le fruit. La
frayeur que le cardinal Mazarin conçut de leur capacité politique et
militaire, de leurs alliances au dedans, surtout au dehors, de leurs
appuis, lui inspira une passion extrême de se les réconcilier, de se les
attacher, et de pouvoir compter personnellement sur eux. Il y parvint
enfin, et eux à tout ce qu'ils voulurent, et enfin à leur prodigieux
échange qui ne se fit qu'en 1651, en mars\,; mais longtemps auparavant
l'union se négociait du cardinal avec eux, et ils savaient en tirer les
partis les plus avantageux, en attendant qu'elle fût scellée. La
duchesse de La Trémoille, leur soeur, qui était de tout avec eux, était
ravie de les voir si proches de ce qu'ils s'étaient toujours proposé en
agitant si continuellement la France, mais, parmi la joie des avantages
si immenses que ses frères étaient sur le point d'obtenir pour eux et
pour leur, maison, elle ne laissait pas d'être peinée de voir son mari
demeuré en arrière, et ne pas devenir prince comme eux. Elle se jeta,
faute de mieux, sur la prétention de Naples, qu'il se peut dire qu'elle
enfanta, parce qu'aucun des Laval-Montfort n'y avait jamais pensé, ni
leur héritière, ni sa fille, d'où elle était tombée, comme on l'a vu, à
la grand'mère de son mari, dont la maison n'y avait jamais songé non
plus jusqu'à elle. Elle fit faire des écrits sur cette chimère, et
s'appuya de la naissance de sa belle-fille et des services que la
landgrave, sa mère, dont l'importance et la fidélité devaient toucher,
et qui ne mourut qu'en août 1651 après l'échange, et mit son espérance
dans le crédit où étaient ses frères, qui, dans l'opinion où était le
cardinal Mazarin que son salut, dans la situation où il était alors, se
trouvait attaché à leur réconciliation sincère et entière avec lui,
étaient en effet à même de toutes les conditions qu'ils lui voudraient
prescrire. Elle était bien informée\,; les choses en étaient là en
effet, mais elle se trompa sur ses frères, dont l'amitié ne put
surmonter l'orgueil.

Ce même orgueil qui, depuis le mariage de l'héritière de Sedan par la
protection d'Henri IV, n'avait cessé de bouleverser la France par le
père et par les deux fils contre Henri IV, leur bienfaiteur, contre
Louis XIII et contre Louis XIV jusqu'alors, ne leur permit pas de
communiquer à leur beau-frère le principal fruit qu'ils en allaient
tirer, mais il exigea d'eux de faire parade de leur puissance jusque
hors de leur maison, en procurant des avantages au duc de La Trémoille
qui n'égalassent pas les leurs. Ils ne voulurent donc pas que, comme
eux, il devînt prince, mais ils exigèrent qu'il aurait des distinctions.
Ils firent valoir combien il serait dur de laisser debout la fille de la
landgrave de Hesse et la soeur de l'électrice palatine\,; de là ils
obtinrent non seulement qu'elle serait assise mais que tous les fils
aînés seulement les ducs de La Trémoille à l'avenir auraient le même
rang, et que M\textsuperscript{lle} de La Trémoille, qui épousa depuis
un sixième cadet de Saxe-Weimar, s'assoirait aussi, avec la même
extension pour toutes les filles aînées seulement dés ducs de La
Trémoille, ce qui leur est demeuré depuis. Ils exigèrent, outre ce
solide, deux bagatelles qu'ils donnèrent à leur soeur pour pierres
d'attente, le \emph{pour} aux ducs et duchesses de La Trémoille
seulement. J'ai expliqué ce que c'est (t. II, p.~186), et la permission
d'envoyer réclamer le droit de Naples aux traités de paix, ce que MM. de
La Trémoille n'ont pas manqué de pratiquer depuis, non plus que les
plénipotentiaires de s'en moquer, et de ne point reconnaître ni admettre
ceux qu'ils y ont envoyés. Telles sont les distinctions de MM. de La
Trémoille, et telle leur origine. Revenons maintenant au mariage du
prince de Talmont.

Il avait quitté ses bénéfices et le petit collet assez tard, ennuyé de
n'en avoir pas de plus riches. Grand et parfaitement bien fait, mais
avec l'air allemand au possible\,; son peu de bien l'avait rendu
avare\,; il en chercha et en trouva avec la fille de Bullion. L'embarras
fut Madame, qui traitait le duc de La Trémoille et lui avec grande
amitié, et ne les appelait jamais que \emph{mon cousin}, et ils étaient
germains. Elle et Monsieur même avaient vécu avec toutes sortes d'égards
les plus marqués pour la princesse de Tarente, leur mère, dans les
courts intervalles qu'elle avait passés à Paris, où elle avait paru à la
cour sans prétention aucune, et parmi les femmes, assise comme l'une
d'entre elles. Monsieur et Madame lui obtinrent la permission très
singulière, à la révocation de l'édit de Nantes, non seulement de
demeurer librement à Paris, à la cour, dans ses terres et partout en
France, mais d'avoir un ministre à elle et chez elle partout à sa suite,
pour elle et pour sa suite, et de faire dans sa maison partout, mais à
porte fermée, l'exercice de sa religion. Son mari, qui il avait presque
jamais demeuré en France, s'était retiré à Thouars, chez son père, en
1669, s'y fit catholique un an après, ne vécut que deux ans depuis sans
sortir de Thouars, et mourut quinze mois avant son père. Sa veuve mourut
à Francfort en février 1693, à soixante-huit ans, où elle s'était enfin
retirée depuis quelques années. Au premier mot du mariage du prince de
Talmont, Madame entra en furie. Bullion était petit-fils du surintendant
des finances, et fils d'un président à mortier qui s'était laissé
prendre sa charge pour celle de greffier de l'ordre, et qui n'avait pas
laissé, pour ses grands biens, d'épouser M\textsuperscript{lle} de Prie,
soeur aînée de la maréchale de La Mothe.

Madame n'avait pas oublié la peine qu'elle avait eue à laisser gagner
deux mille pistoles à M\textsuperscript{me} de Ventadour pour admettre
une seule fois M\textsuperscript{me} de Bullion dans son carrosse, qui
espéra par là entrer après en ceux de M\textsuperscript{me} la duchesse
de Bourgogne, manger et aller à Marly, à aucune desquelles {[}choses{]}
elle ne put parvenir. Madame fit tout ce qu'elle put pour détourner le
prince de Talmont d'une alliance si disproportionnée de celles que sa
maison avait\,; elle déclara qu'elle ne verrait jamais ni lui ni sa
femme, et défendit à M. et à M\textsuperscript{me} la duchesse d'Orléans
de signer le contrat de mariage. Elle et Monsieur avaient été aux noces
du duc de La Trémoille, à l'hôtel de Créqui\,; elle n'oublia rien pour
l'engager à rompre avec son frère. Lui, tira sur le temps tant il est
vrai qu'un grand intérêt donne de l'esprit pour ce qui le regarde. Il
tenait au roi par l'estime, par une conduite décente, et par une grande
assiduité, qui était la chose que le roi aimait le plus, même dans les
gens sans charge et le moins à portée de lui. Il lui refusait
obstinément sa survivance pour son fils, par la loi qu'il s'était faite
ou cru faire. Il ne laissait pas d'en être peiné. M. de La Trémoille le
sentait\,; il profita de tout, et de la colère même de Madame. Il
représenta au roi son embarras avec elle, lui insinua que le tabouret de
sa belle-fille aînée et de sa fille aînée devait s'étendre jusqu'à
l'aîné de ses frères\,; qu'il n'avait pas voulu importuner le roi
là-dessus jusqu'alors, espérant que ce seul frère qu'il avait ne se
marierait point\,; qu'il n'avait pas même voulu le tenter par un
tabouret, parce que, n'ayant que peu de bien, il ne pouvait que faire
une alliance désagréable\,; mais que, venant à la faire, il ne pouvait
s'empêcher de demander le tabouret, ou comme justice ou comme grâce, qui
de plus serait le moyen d'adoucir Madame, s'il en pouvait rester
quelqu'un. Le roi le lui accorda, mais uniquement pour sa vie, et non
pour ses enfants, et il s'en expliqua même publiquement. Cette nouveauté
fit du bruit et déplut à bien des gens. Mais l'estime, la considération,
l'amitié que M. de La Trémoille s'était conciliées à force d'honneur, de
probité et de bienséance fit passer la chose avec moins de scandale.
Madame n'en fut point apaisée, mais le mariage se fit avec le tabouret,
et, après bien des années, Madame s'est laissé fléchir. Ce commencement
de succès a fait, en ces derniers temps, le mariage du fils unique du
prince de Talmont, uniquement pour obtenir en se mariant un brevet de
duc\,; et, à la mort de son père, la chimère et le désir de la faire
surnager lui a fait quitter le nom de duc de Châtellerault, pour prendre
celui de prince de Talmont. Il n'a eu aucun bien de sa femme, ni aucune
autre protection que ce brevet pour la parenté de la reine\,; les
humeurs, qui d'avance se pouvaient soupçonner, n'ont pas été
concordantes. Il se peut dire que ce brevet de duc lui coûte fort cher,
et en plus d'une manière.

Moreau, premier valet de chambre de Mgr le duc de Bourgogne, mourut à
Versailles. Il était un des quatre premiers valets de garde-robe du roi,
qui ne mit auprès de ce jeune prince que lui seul et laissa la
disposition de tout le reste au duc de Beauvilliers. Moreau avait été un
des hommes des mieux faits de son temps\,; de l'air le plus noble, d'un
visage agréable. Il était encore tel à soixante-dix-sept ans. À le voir,
il n'est personne qui ne le prît pour un seigneur. Il avait été en
subalterne des ballets du roi et de ses plaisirs dans sa jeunesse, qui
l'aima toujours depuis avec estime et considération marquée. Il avait
été galant, il le fut très longtemps, il eut des fortunes distinguées,
et quantité, que sa figure et sa discrétion lui procurèrent. Il eut
beaucoup d'amis et plusieurs considérables, il passa sa vie à la cour,
et toujours fort instruit de tout. Avec de l'esprit, beaucoup de sens,
c'était un vrai répertoire de cour, et un homme gai, et, quoique sage,
naturellement libre avec un grand usage du meilleur monde qui l'avait
mis au-dessus de son état, et rendu d'excellente compagnie. Avec tant de
choses si propres à gâter un homme de cette sorte, jamais aucun ne
demeura plus en sa place, et ne fut plus modeste, plus mesuré, plus
respectueux. Il était plein d'honneur, de probité et de
désintéressement, et vivait uniment, et moralement bien. Il avait
entièrement l'estime et la confiance de Mgr le duc de Bourgogne et du
duc de Beauvilliers. Il n'aimait ni les dévots ni les jésuites, et il
lâchait quelquefois au jeune prince des traits libres et salés, justes
et plaisants sur sa dévotion, et surtout sur ses longues conférences
avec son confesseur. Quand il se vit près de sa fin, il se sentit si
touché de tout ce qu'il avait vu de si près dans Mgr le duc de
Bourgogne, qu'il envoya le supplier de lui accorder ses prières, et une
communion dès qu'il serait mort, et déclara en même temps qu'il ne
connaissait personne de si saint que ce prince. C'était un homme
entièrement éloigné de toute flatterie, qui n'avait jamais pu s'y ployer
ni la souffrir dans les autres.

Mgr le duc de Bourgogne, sur ce message, monta chez lui et fit ses
dévotions pour lui dès qu'il fut mort. Ce témoignage d'un homme de ce
caractère et dans cet emploi fit grand bruit à la cour. Aussi jamais
prince de cet âge et de ce rang n'a peut-être reçu d'éloges si complets
ni si exempts de flatterie. Moreau fut regretté de tout le monde, et ne
fut jamais marié. Le roi laissa le choix d'un autre premier valet de
chambre à Mgr le duc de Bourgogne. Il choisit Duchesne, premier valet de
chambre de M. le duc de Berry. C'était un homme fort modeste et fort
pieux, qui ne manquait ni de sens ni de monde, discret et fidèle, mais
qui ne fit pas souvenir de Moreau.

Deux grands prélats fort différents l'un de l'autre le suivirent de fort
près. L'un fut l'archevêque de Rouen, Colbert, frère des duchesses de
Chevreuse et de Beauvilliers, qui en furent fort affligées. C'était un
prélat très aimable, bien fait, de bonne compagnie, qui avait toujours
vécu en grand seigneur, et qui en avait naturellement toutes les
manières et les inclinations. Avec cela savant, très appliqué à son
diocèse, où il fut toujours respecté et encore plus aimé, et le plus
judicieux et le plus heureux au choix des sujets pour le gouvernement.
Doux, poli, accessible, obligeant, souvent en butte aux jésuites, par
conséquent au roi, sans s'en embarrasser et sans donner prise, mais ne
passant rien. Il vivait à Paris avec la meilleure compagnie, et de celle
de son état la plus choisie\,; souvent et longtemps dans son diocèse où
il vivait de même, mais assidu au gouvernement, aux visites, aux
fonctions. C'est lui qui a mis ce beau lieu de Gaillon, bâti par le
fameux cardinal d'Amboise, au degré de beauté et de magnificence où il
est parvenu, et où la meilleure compagnie de la cour l'allait voir. Sa
dépouille ne tarda guère à être donnée. M. de La Rochefoucauld, dont la
famille regorgeait de biens d'Église, eut sur-le-champ pour son
petit-fils, qui avait dix-neuf ans, la riche abbaye du Bec, dont il se
repentit bien dans la suite\,; et d'Aubigné, ce parent factice de
M\textsuperscript{me} de Maintenon, dont j'ai suffisamment parlé quand
il fut évêque de Noyon, fut transféré à Rouen, avec une grâce sans
exemple. Ce fut un brevet pour lui conserver le rang et les honneurs
d'évêque, comte et pair de France de Noyon, exemple, dont on a bien
abusé depuis.

L'autre prélat fut l'archevêque d'Aix, Cosnac, mort fort vieux dans son
diocèse, mais la tête entière et toujours le même. J'ai assez parlé de
cet homme, qui peut passer pour illustre, pour n'avoir plus rien à y
ajouter.

M. de Lauzun perdit aussi le chevalier de Lauzun, son frère, à qui il
donnait de quoi vivre, et presque toujours mal ensemble. C'était un
homme de beaucoup d'esprit et de lecture, avec de la valeur\,; aussi
méchant et aussi extraordinaire que son frère, mais qui n'en avait pas
le bon\,; obscur, farouche, débauché, et qui avait achevé de se perdre à
la cour par son voyage avec le prince de Conti en Hongrie. C'était un
homme qu'on ne rencontrait jamais nulle part, pas même chez son frère,
qui en fui fort consolé.

Valsemé, lieutenant général, mourut aussi en Provence où on l'avait
envoyé commander sous M. de Grignan. Il était pauvre, estimé et fort
honnête homme. Je pense qu'il serait un peu, surpris, s'il revenait au
monde, de trouver son fils marié à la comtesse de Claire, fille du feu
comte de Chamilly, faire l'important au Palais-Royal sous, le nom de
Graville, en rejeton de cet amiral.

M\textsuperscript{me} d'Armagnac mourut à la grande écurie à Versailles
le jour de Noël, et laissa peu de regrets. C'était, avec une vilaine
taille grosse et courte, la plus belle femme de France jusqu'à sa mort,
à soixante-huit ans\,; sans rouge, sans rubans, sans dentelles, sans or,
ni argent, ni aucune sorte d'ajustement, vêtue de noir ou de gris en
tout temps, en habit troussé comme une espèce de sage-femme, une
cornette ronde, ses cheveux couchés sans poudre ni frisure, un collet de
taffetas noir et une coiffe courte et plate chez elle comme chez le roi,
et en tout temps. Elle était soeur du maréchal de Villeroy, avait été
dame du palais de la reine, avait été exilée pour s'être trouvée dans
l'affaire qui fit chasser la comtesse de Soissons, Vardes et le comte de
Guiche, dont j'ai parlé ailleurs\,; et que la faveur de son mari n'avait
jamais pu raccommoder avec le roi, qui ne la souffrit qu'avec peine, et
qui, tant que Marly demeura un peu réservé, et même quelque temps après,
ne l'y mena point. C'était une femme haute, altière, entreprenante, avec
peu d'esprit toutefois et de manége, qui de sa vie n'a donné la main ni
un fauteuil chez elle à pas une femme de qualité, qui menait haut à la
main les ministres et leurs femmes, qui passait sa vie chez elle à tenir
le plus grand état de la cour, qui la faisait assez peu, et qui ne
visitait presque jamais personne qu'aux occasions. Tout occupée de son
domestique, également avare et magnifique, elle menait son mari comme
elle voulait, qui ne se mêlait ni d'affaires, ni de dépenses, ni de la
grande écurie que pour le service, et elle de tout despotiquement\,;
impérieuse et dure, tirait la quintessence de sa charge, du gouvernement
et des biens de son mari, traitait ses enfants comme des nègres et leur
refusait tout, excepté ses filles, dont la beauté l'avait apprivoisée,
sur laquelle elle ne les tint pas de fort près, ayant conservé et mérité
toute sa vie elle-même une réputation sans ombre sur la vertu. Tout ce
qui avait affaire à elle la redoutait. Elle noya son fils l'abbé de
Lorraine, parce qu'il voulut partager au moins avec elle le revenu de
ses bénéfices, et en ayant de gros, {[}ne pas{]} les lui laisser toucher
en entier, et dépendre d'elle comme un enfant. Il avait la nomination de
Portugal que le duc de Cadaval lui avait procurée\,; elle avait eu
l'agrément du roi et de Rome. Cette considération n'arrêta point sa
mère\,; elle s'en prit à ses moeurs, qui en effet n'étaient pas bonnes,
elle força M. le Grand à demander au roi de l'enfermer à Saint-Lazare.
Le roi y résista par bonté. Il représenta à M. le Grand que son fils
étant déjà prêtre, il le perdrait sans ressource par cet éclat. M. le
Grand, poussé par sa femme, insista. L'abbé de Lorraine fut mis à
Saint-Lazare, et demeura perdu sans qu'il fût plus question de sa
nomination, dont Rome ne voulut plus ouïr parler, et que le Portugal
retira. Il fut assez longtemps à Saint-Lazare, et n'en sortit qu'en
capitulant avec sa mère sur le revenu de ses bénéfices. Il vécut depuis
obscur, et bien des années sans oser paraître. C'est lui qui est mort
évêque de Bayeux, qu'il eut pendant la régence.

Cette mort donna lieu à une nouvelle usurpation des princes du sang. Une
des distinctions des petits-fils de France et d'eux était que les
personnes qui, à l'occasion des grands deuils de famille, saluaient le
roi en manteau long pour les hommes, et pour les femmes en mante,
visitaient dans le même habit les petits-fils et les petites-filles de
France, mais non les princes ni les princesses du sang. Ceux-ci toujours
blessés de ces différences, s'attirèrent peu à peu des visites en mante
et en manteau des personnes de qualité qui par attachement voulurent
bien avoir cette complaisance, bientôt après laissèrent entendre qu'ils
ne trouvaient pas bon qu'on y manquât, enfin l'établirent en prétention
et y soumirent beaucoup de gens. Dès qu'ils s'y crurent affermis, ils se
mirent à prétendre la même déférence des maréchaux de France, et peu à
peu les y amenèrent comme ils avaient fait les gens de qualité. Une des
choses qui y contribua le plus fut la prostitution où tombèrent les
mantes et les manteaux. La protection publiquement donnée à la confusion
en tout par l'intérêt, le crédit et l'adresse des ministres, les étendit
à chaque occasion douteuse par des permissions expresses, puis par
exemples\,; enfin y alla qui voulut. Beaucoup de gens de qualité,
plusieurs titrés, choqués d'un mélange qui ne laissait plus de
distinction, crurent en reprendre en faisant demander permission au roi
de paraître devant lui sans manteau et sans mante. Ceux qui usurpaient
d'en porter n'étaient pas en état de disputer rien aux princes du sang.
Tout est exemple et mode\,: tels et tels l'ont fait, il faut donc le
faire aussi\,; c'est ce qui aida le plus aux succès des princes du sang.
Quand après les gens considérables, titrés et non titrés, se mirent à se
faire dispenser de saluer le roi en manteau et en mante, plusieurs
firent dire aux princes du sang comme aux fils et petits-fils de France
que le roi les avait dispensés. «\, C'est une honnêteté, disaient-ils,
qui ne coûte rien, nous n'irons point en manteau et en mante chez les
princes du sang\,; qu'importe de ne leur pas faire cette civilité\,?» De
l'un à l'autre elle s'introduisit. Les princes du sang la reçurent, et
comme un devoir et comme une reconnaissance de l'obligation de les voir
en manteau et en mante quand on y avait vu le roi, puisque les voyant
sans cet habillement on les avertissait que le roi en avait dispensé
pour lui, comme il était vrai qu'en ce cas il le fallait faire dire aux
fils et petits-fils de France. Ainsi peu à peu les princes du sang le
prétendirent de tous les gens titrés, mais toutefois sans oser se fâcher
lorsqu'ils y manquaient, comme il arrivait souvent à plusieurs ducs et
duchesses, et surtout aux princes étrangers et à ceux qui en ont le
rang, toujours si attentifs à l'accroître avec qui ils peuvent, et à se
conserver au moins à faute de mieux.

J'ai vu tout cela naître, et à la mort de mon père je me souviens
qu'ayant vu le roi presque sur-le-champ et sans deuil, et Monsieur qui
se trouva dans ce moment-là avec lui par le hasard que j'ai raconté, en
parlant de la perte de mon père, je ne fis rien dire à personne, parce
que la vue de Monsieur lui avait tout dit pour lui et pour les siens,
sinon à M\textsuperscript{me} la grande-duchesse et à
M\textsuperscript{me} de Guise, filles de Gaston. À la mort de
M\textsuperscript{me}, d'Armagnac, M. le Duc, en curée de l'usurpation
du service seul de la communion du roi, crut le temps favorable pour
emporter celle-ci\,; l'intérêt de l'assimilation des bâtards du roi avec
les princes du sang eut pour celle-ci le même ascendant qu'il avait eu
pour l'autre, quoiqu'il s'agît de M. le Grand. Le roi, après quelque
répugnance, lui ordonna d'aller avec ses enfants en manteau chez les
princes et les princesses du sang, et d'y faire aller ses filles en
mante. M. le Grand résista, représenta, tout fut inutile, il en sauta le
bâton par force\,; et c'est l'époque de l'établissement de ce nouveau
droit. Il a fait que presque tout le monde s'est fait dispenser depuis
de voir le roi en manteau et en mante, mais en le faisant dire après aux
princes et princesses du sang, ce qui à présent revient au même, et
n'affranchit plus que de l'importunité du vêtement.

Le grand écuyer, qui n'aimait que lui dans le monde, n'eut pas plutôt
perdu une femme qui avait si bien vécu avec lui, et si utilement pour sa
famille, qu'il songea à se remarier. La figure et la conduite de
M\textsuperscript{me} de Châteauthiers, dame d'atours de Madame, lui
avait toujours plu. Quoique éloignée de l'âge de la beauté, elle en
avait encore, et grand air par sa taille et son maintien, et toujours
une vertu sans soupçon dans le centre de la corruption\,; la probité
était pareille dans un lieu qui n'y était pas moins opposé, tout cela au
moins du temps de la cour de Monsieur, qui était celui de sa jeunesse et
de sa beauté\,; avec cela beaucoup d'esprit et de grâces, aimable au
possible dans la conversation, quand elle le voulait bien et que
l'humeur ne s'y opposait pas. M. le Grand, un mois après être veuf, lui
fit parler. C'était une très bonne demoiselle toute simple, dont le nom
était Foudras. Ils étaient d'Anjou et avaient des baillis dans l'ordre
de Malte. Elle n'avait rien vaillant que ce que lui donnait Madame, et
n'en savait pas même tirer, parce qu'elle était tout à fait noble et
désintéressée. M. le Grand lui fit sentir le rang et les biens qu'elle
trouverait avec lui, et le soin qu'il prendrait en l'épousant de lui
assurer après lui une subsistance convenable au nom qu'elle porterait.
Elle résista et répondit comme elle devait sur une proposition aussi
flatteuse\,; mais elle ajouta qu'elle ne voulait point faire cette peine
aux enfants de M. le Grand. Eux qui virent l'empressement de leur père,
et qui craignirent qu'éconduit de celle-là il n'en épousât quelque
autre, furent trouver M\textsuperscript{me} de Châteauthiers et la
conjurèrent de consentir au mariage. Ils l'en firent presser par leurs
amis. M. le Grand ne se rebuta point. Mais la sage et modeste résistance
de M\textsuperscript{me} de Châteauthiers fut la plus forte, jamais elle
n'y voulut consentir. Toute la France l'admira et ne l'en estima que
davantage, M. le Grand lui-même et toute sa famille. Elle préféra son
repos\,; et sa modestie fut telle qu'elle n'en prit aucun avantage, et
qu'elle évitait même depuis de s'en laisser parler. M. le duc d'Orléans
dans sa régence lui donna plus qu'elle ne voulut avec quoi elle se
retira, après la mort de Madame, dans une maison qu'elle loua dans
Paris, d'où elle ne sortit que pour aller à l'église, et n'y reçut qu'un
très petit nombre d'amis. D'une sage retraite elle s'en fit une de
piété, elle s'y donna tout entière, et elle y est morte depuis deux ou
trois ans, ne voyant plus presque personne, à soixante-dix-sept ou
soixante-dix-huit ans.

Villette, lieutenant général des armées navales, mourut en ce même
temps. Il était cousin germain de M\textsuperscript{me} de Maintenon,
traité d'elle comme tel, et père de Murcé et de M\textsuperscript{me} de
Caylus dont j'ai parlé plus d'une fois. Sa mort fit une promotion dans
la marine\,; au lieu d'un lieutenant général, il y en eut deux. Le
mérite fit Ducasse, la faveur fit d'O, qui de capitaine tout nouveau, et
tout au plus lorsqu'il fut mis auprès du comte de Toulouse, monta à ce
grade si rare et si réservé dans la marine sans être sorti de
Versailles, ni s'en être absenté qu'avec M. le comte de Toulouse. On a
vu qu'il en coûta de ne pas donner une seconde bataille sûrement gagnée,
et Gibraltar repris, malgré la volonté de l'amiral et de toute la
flotte. C'est ainsi que la protection puissante tient lieu de tout à la
cour. Pontchartrain qui la craignait, et qui remis auprès du comte de
Toulouse par la considération du mérite de sa femme, et raccommodé après
avec le maréchal d'Estrées, n'avait pu se rapprocher celui-ci, essaya la
conjoncture, et lui manda, au sortir du travail avec le roi, qu'il était
lieutenant général. La joie de l'être, et l'orgueil flatté du message
d'un ministre ennemi, le disposa à s'en ôter l'épine. Un moment après il
vint le remercier, et ils se raccommodèrent comme on se raccommode
d'ordinaire dans les cours.

{[}L'orgueil{]} de M\textsuperscript{me} de Soubise fit mêler le roi
d'une affaire particulière assez ridicule, contre sa coutume, entre des
gens qu'il n'aimait point, et avec qui il n'avait aucune familiarité. Le
duc de Rohan, qui alternait avec le duc de La Trémoille la présidence de
la noblesse aux états de Bretagne, avait cédé la sienne depuis quelque
temps, avec l'agrément du roi, à son fils aîné que, pour accoutumer le
monde peu à peu à quelque chimère dont j'ai expliqué la moderne vue, il
faisait appeler le prince de Léon, et arborer le manteau ducal à tous
ses enfants avec d'autant plus de facilité que, n'ayant point l'ordre,
leurs carrosses passaient pour être les siens. Le prince de Léon était
un grand garçon élancé, laid et vilain au possible, qui avait fait une
campagne en paresseux, et qui, sous prétexte de santé, avait quitté le
service pour n'en pas faire davantage. On ne pouvait d'ailleurs avoir
plus d'esprit, de tournant, d'intrigue, ni plus l'air et le langage du
grand monde où d'abord il était entré à souhait. Gros joueur, grand
dépensier pour tous ses goûts, d'ailleurs avare\,; et tout aimable qu'il
était, et avec un don particulier de persuasion, d'intrigues, de
souterrains et de ressources de toute espèce, plein d'humeur, de
caprices et de fantaisies, opiniâtre comme son père, et ne comptant en
effet que soi dans le monde.

Il était devenu fort amoureux de Florence, comédienne que M. le duc
d'Orléans avait longtemps entretenue, dont il eut l'archevêque de
Cambrai d'aujourd'hui, et la femme de Ségur, lieutenant général, fils de
celui dont j'ai parlé, avec l'abbesse de La Joye, soeur de M. de
Beauvilliers. M. de Léon dépensait fort avec cette créature, en avait
des enfants, l'avait menée avec lui en Bretagne, mais non pas dans Dinan
même, où il avait présidé aux états, et il arrivait avec elle en
carrosse à six chevaux avec un scandale ridicule. Son père mourait de
peur qu'il ne l'épousât. Il lui offrit d'assurer cinq mille livres de
pension à cette créature, et d'avoir soin de leurs enfants s'il voulait
la quitter, à quoi il ne voulait point entendre. Quelque mal qu'il eût
été toute sa vie avec M\textsuperscript{me} de Soubise, qui de son côté
ne l'aimait pas mieux, et qu'on a vue prendre si amèrement le parti des
Rohan contre lui dans ce procès du nom et des armes que j'ai raconté (t.
V, p.~277 et suiv.), et qu'il gagna malgré ses charmes, elle était fort
peinée de voir son propre neveu, et qui devait être si riche, dans de
pareils liens. Elle fit donc en sorte, avec ces billets dont j'ai parlé,
qui mouchaient si ordinairement entre le roi et elle, qu'il parlât au
fils, puis au père, à qui séparément il donna des audiences et longues
dans son cabinet. Le fils prit le roi par ses deux faibles, les respects
et l'amour, et avec tant d'esprit, de grâces et de souplesse, que le roi
en fit l'éloge, plaignit son coeur épris et le malheur du père, qu'il
entretint après aussi fort longtemps dans son cabinet. La Florence fut
pourtant enlevée aux Ternes, jolie maison dans les allées du Roule, où
le prince de Léon la tenait, et mise dans un couvent. Il devint furieux,
ne voulut plus voir ni ouïr parler de père ni de mère\,; et ce fut pour
consommer la séparation d'avec Florence et raccommoder le fils avec ses
parents, et lé rendre traitable à un mariage, que le roi manda le prince
de Léon, puis le duc de Rohan. Cela se passa à la fin de décembre.

Le 18 du même mois, le premier président étant à dîner chez lui au
palais avec sa famille et quelques conseillers, le plancher fondit tout
à coup, et tous tombèrent dans une cave où il se trouva des fagots qui
les empêchèrent de tomber tout en bas, et même de se blesser. Il n'y eut
que le précepteur des enfants qui le fut. La première présidente se
trouva placée de manière qu'elle fut la seule qui ne tomba point.
L'effroi fut grand, et tel, dans le premier président, que depuis il n'a
jamais été ce qu'il était auparavant.

Le duc de Noailles qui, pour consolider son état de commandant et de
petit général d'armée, s'était tenu tant qu'il avait pu en Roussillon,
arriva pour servir son quartier de capitaine des gardes, et le maréchal
de Villars prit congé pour aller passer le reste de l'hiver à Strasbourg
avec sa femme qu'il ne quittait pas volontiers. En ce même temps, le duc
de Tresmes, qui n'avait point encore de brevet de retenue sur sa charge
depuis qu'il l'avait en titre par la mort de son père, en obtint un de
quatre cent mille livres.

M. le duc d'Orléans arriva d'Espagne le 30 décembre au lever du roi,
après lequel il demeura longtemps seul avec lui dans son cabinet. La
réception et du roi et du monde fut telle que le méritait son heureuse
et agréable campagne. Comme il devait retourner bientôt en ce pays-là,
il y avait laissé presque tous ses équipages. Il en était fort content,
et on l'y était fort de lui. Le duc de Berwick eut ordre de l'y
attendre.

\hypertarget{chapitre-viii.}{%
\chapter{CHAPITRE VIII.}\label{chapitre-viii.}}

1708

~

{\textsc{Année 1708.}} {\textsc{- Cent cinquante mille livres de brevet
de retenue à Chamillart.}} {\textsc{- Deux cent mille livres de brevet
de retenue au maréchal de Tessé.}} {\textsc{- Trois mille livres de
pension à Albéroni.}} {\textsc{- Du Luc, évêque de Marseille, passe à
Aix.}} {\textsc{- Rois et force bals à la cour.}} {\textsc{- Comédies de
M\textsuperscript{me} du Maine.}} {\textsc{- Duc de Villeroy capitaine
des gardes sur la démission de son père.}} {\textsc{- Vaudémont
souverain de Commercy, etc.}} {\textsc{- Mort du marquis de Thianges\,;
son caractère.}} {\textsc{- Courte digression sur sa mère.}} {\textsc{-
Mariage de Seignelay et de M\textsuperscript{lle} de Fürstemberg.}}
{\textsc{- Vilenie des serments chez le roi.}} {\textsc{- Chamillart,
fort languissant, songe à se soulager et à marier son fils.}} {\textsc{-
Réflexions importantes sur les choix.}} {\textsc{- Mariage de Cani avec
une fille de Mortemart.}} {\textsc{- Mesures sur la place des
finances.}} {\textsc{- Desmarets contrôleur général des finances\,; ma
conversation avec lui.}} {\textsc{- Directeurs généraux des finances
abolis.}} {\textsc{- Chute d'Armenonville.}} {\textsc{- Poulletier
intendant des finances.}} {\textsc{- Colère du conseil et du
chancelier.}} {\textsc{- Duchesse du Maine refuse de signer après
M\textsuperscript{lle} de Bourbon le contrat de mariage de Cani.}}
{\textsc{- Mort, extraction et caractère du chevalier de Nogent.}}
{\textsc{- Mort de Langlée.}} {\textsc{- Mort du comte d'Oropesa.}}
{\textsc{- Mort, extraction, fortune et caractère de Montbron\,; sa
dépouille.}} {\textsc{- Oran pris par les Maures.}} {\textsc{- Mort de
Tésut\,; sa charge donnée à son frère par l'exclusion de l'abbé
Dubois.}} {\textsc{- Caractère des deux frères.}} {\textsc{- Caractère
de Nancré, exclu par le roi de suivre M. le duc d'Orléans en Espagne.}}
{\textsc{- Plaisante exclusion et plus rare inclusion de Fontpertuis\,;
son caractère.}}

~

L'année 1708 commença par les grâces, les fêtes et les plaisirs. On ne
verra que trop tôt qu'elle ne continua pas longtemps de même. Chamillart
obtint sur sa charge de l'ordre cent cinquante mille livres de brevet de
retenue, et le maréchal Tessé sur la sienne, de M\textsuperscript{me} la
duchesse de Bourgogne, une autre de deux cent mille livres. M. de
Vendôme procura à son Albéroni trois mille livres de pension, à qui nous
verrons faire dans quelque temps une fortune et une figure si
prodigieuse. L'évêque de Marseille, frère du comte du Luc, passa à
l'archevêché d'Aix. Je le remarque parce qu'il devint, longues années
après, le triste successeur à Paris du cardinal de Noailles. Le roi fit
à Versailles de magnifiques Rois avec beaucoup de dames, où la cour de
Saint-Germain se trouva. Il y eut après le festin un grand bal chez le
roi, qui en donna plusieurs parés et masqués tout l'hiver à Marly et à
Versailles, où il y en eut aussi chez Monseigneur et dans l'appartement
de M\textsuperscript{me} la duchesse de Bourgogne. Les ministres lui en
donnèrent, M\textsuperscript{me} la duchesse du Maine encore, laquelle
se donna en spectacle tout l'hiver, et joua des comédies à Clagny en
présence de toute la cour et de toute la ville. M\textsuperscript{me} la
duchesse de Bourgogne les alla voir souvent, et M. du Maine, qui en
sentait tout le parfait ridicule et le poids de l'extrême dépense, ne
laissait pas d'être assis au coin de la porte et d'en faire les
honneurs.

Le maréchal de Villeroy, fatigué des dégoûts d'une cour où il avait tant
brillé et où il n'espérait plus de se pouvoir reprendre, flottait depuis
quelque temps dans l'incertitude sur sa charge entre le dépit journalier
de la faire avec des désagréments continuels, accoutumé de longue main à
trouver des distinctions partout, et la crainte du vide et de l'ennui.
Il y avait longtemps que le duc et la duchesse de Villeroy m'avaient dit
qu'il leur en avait parlé. Ils ne laissaient pas de s'ennuyer de la
lenteur de sa résolution, et ils s'en consolaient dans la crainte d'un
refus qui deviendrait une exclusion. L'espérance, fondée sur un reste de
bonté pour le maréchal, était légère après tout ce qui s'était passé. Le
duc de Villeroy, dans toute la faveur de son père, n'avait jamais cessé
de sentir que ses lettres en Hongrie n'étaient point effacées\,; il ne
s'apercevait pas moins que M\textsuperscript{me} de Maintenon n'était
jamais bien revenue pour lui depuis l'affaire de M\textsuperscript{me}
de Caylus. Parmi ces angoisses, le maréchal de Villeroy, qui depuis
quelque temps ne leur parlait plus de rien, prit enfin sa résolution, et
la veille des Rois, au retour de la messe du roi, il s'approcha de lui
dans son cabinet pour lui demander à se démettre de sa charge en faveur
de son fils. À peine en eut-il commencé la proposition, que le roi, qui
vit d'abord où elle tendait, l'interrompit, et se hâta de lui accorder
sa demande, tant il se sentit soulagé de se défaire de lui comme que ce
fût, dans une fonction si intime et si continuelle pendant le quartier,
et néanmoins si fréquente encore dans les autres quartiers par mille
détails. Ainsi, ce que la faveur du maréchal la plus déclarée n'avait pu
obtenir de lui-même, ce qu'elle n'eût peut-être pas arraché du roi avec
son goût pour le père et ses anciennes répugnances pour le fila, que les
nouvelles n'avaient pas raccommodées, tout céda à la disgrâce du
maréchal de Villeroy, et à la peine que le roi avait à le supporter.

Le duc de Villeroy était ce jour-là avec Monseigneur qui courait le daim
au bois de Boulogne. La nouvelle lui fut portée sans qu'il voulût la
croire avant d'en avoir reçu des avis redoublés. Je ne vis jamais de
gens si aises que la duchesse de Villeroy et lui, et nous nous
rappelâmes avec plaisir ce souper si plein de larmes de la duchesse, et
des soupirs de son mari, qui crut ses peines, ses services et sa fortune
perdus par le caprice de son père à persévérer de lui défendre de voir
Chamillart. La maréchale de Villeroy, avec son bon et sage esprit, fut
ravie, mais le maréchal, après avoir joui vingt-quatre heures des
compliments de la cour, sentit avec horreur tout son vide, et qu'il ne
tenait plus à rien. Cette situation lui devint insupportable.
Jusqu'alors il avait été le roi de Lyon, il se voulut rejeter sur cette
partie d'existence et y aller régner, mais ce gouvernement était dans le
département de Chamillart. Il en craignit tout, il chercha à s'en
délivrer. Torcy était de ses amis, qui avait le Dauphiné dans le sien\,;
il lui proposa de troquer avec Chamillart, qui n'aurait pas bonne grâce
de refuser le gouvernement de son gendre, pour se conserver les
occasions de tourmenter le maréchal dans le sien. Torcy y consentit,
Chamillart aussi, et le roi y donna son approbation pour éviter les
querelles sur Lyon, et les importunités qu'il en aurait essuyées. Voilà
donc le maréchal en repos\,; mais quand de là il voulut profiter du troc
pour s'en aller à Lyon la permission lui en fut refusée, ce qui
renouvela et combla ses désespoirs.

Ce fut en ce temps-ci que M. de Vaudemont obtint la souveraineté sur
Commercy, et la préséance en Lorraine sur tous ceux de cette maison, qui
le brouilla avec eux sans retour comme je l'ai raconté d'avance\,; il
eut en même temps à Versailles le petit logement que la mort du marquis
de Thianges laissa vacant.

Thianges était Damas et de grande naissance, fort brave, avec de
l'esprit et des lettres, beaucoup d'honneur et de probité, mais si
particulier, si singulier, qu'il vécut toujours à part, et ne tira aucun
parti de se trouver fils de la soeur de M\textsuperscript{me} de
Montespan, et d'une soeur par elle-même si bien avec le roi, et si
grandement distinguée tant qu'elle a vécu. Elle n'était morte qu'en
1693, dans un magnifique logement de plain-pied et contigu à celui de
Monseigneur, où les enfants du roi, et de sa soeur, qui l'aimaient et la
craignaient, la visitaient continuellement, ainsi que tout ce qui était
de plus distingué à la cour. Monsieur y allait souvent, et il n'y avait
point de ministre qui ne comptât avec elle. Tout jeune que j'étais
alors, j'étais admis chez elle avec bonté, par la parenté et l'amitié de
ma mère. Je me souviens qu'elle était au fond de son cabinet, d'où elle
ne partait pour personne, et même ne se levait guère. Elle avait les
yeux fort chassieux, avec du taffetas vert dessus, et une grande bavette
de linge qui lui prenait sous le menton. Ce n'était pas sans besoin\,:
elle bavait sans cesse et fort abondamment. Dans cet équipage\,; elle
semblait à son air et à ses manières la reine du monde\,; et tous les
soirs, avec sa bavette et son taffetas vert, elle se faisait porter en
chaise au haut du petit escalier du roi, entrait dans ses cabinets, et y
était avec lui et sa famille assise dans un fauteuil, depuis la fin du
souper jusqu'au coucher du roi. On prétendait qu'elle avait encore plus
d'esprit que M\textsuperscript{me} de Montespan, et plus méchante. Là
elle tenait le dé et disputait, et souvent aigrement contre le roi qui
aimait à l'agacer. Avec des choses fort plaisantes, elle était
impérieuse et glorieuse au dernier point. Elle vantait toujours sa
maison au roi, en effet grande et ancienne\,; et le roi, pour la piquer,
la rabaissait toujours. Quelquefois de colère elle lui disait des
injures, et plus le roi en riait, plus sa furie augmentait. Un jour
étant là-dessus, le roi lui dit qu'avec toutes ses grandeurs, elle n'en
avait aucune de celles de la maison de Montmorency, ni connétables ni
grands maîtres, etc. «\,Cela est plaisant, répondit-elle, c'est que ces
messieurs-là d'auprès de Paris étaient trop heureux d'être à vous autres
rois, tandis que nous, rois dans nos provinces, nous avions aussi nos
grands officiers comme eux, des gentilshommes d'autour de nous.\,»
C'était la personne du monde qui demeurait le moins court, qui
s'embarrassait le moins, et qui très souvent embarrassait le plus la
compagnie. Elle ne sortait presque jamais de Versailles, si ce n'était
pour aller voir M\textsuperscript{me} de Montespan.

M. de La Rochefoucauld était son ami intime, et Mademoiselle aussi.
Toutes deux étaient fort propres pour leur manger. Le roi prenait
plaisir à leur faire mettre des cheveux dans du beurre et dans des
tourtes, et à leur faire d'autres vilenies pareilles. Elles se mettaient
à crier, à vomir, et lui à rire de tout son coeur. M\textsuperscript{me}
de Thianges voulait s'en aller, chantait pouille au roi, mais sans
mesure, et quelquefois à travers la table, faisait mine de lui jeter ces
saletés au nez. Elle fut de toutes les parties, et de tous les voyages,
tant qu'elle le voulut bien, et le roi l'en pressa souvent depuis que sa
santé l'eut rendue plus sédentaire. Elle parlait aux enfants de sa soeur
avec un ton et une autorité de plus que tante, et eux avec elle dans les
recherches et les respects. Elle avait été belle, mais non comme ses
soeurs. Elle était mère de M\textsuperscript{me} de Nevers et de
M\textsuperscript{me} Sforce et du marquis de Thianges, duquel elle ne
fit jamais grand cas. Il était menin de Monseigneur, lieutenant général
et depuis longtemps, fort homme de bien. Il ne laissa point d'enfants de
la nièce de l'archevêque de Paris, Harlay, personne fort extraordinaire,
qui avec de la beauté ne fit jamais parler d'elle, et qui avait passé
longues années fille d'honneur de Mademoiselle, avec qui elle se
querellait souvent.

Seignelay épousa une fille de la princesse de Fürstemberg avec peu de
bien, mais trop pour une si grande alliance. À la mort de son père,
ministre et secrétaire d'État, il avait eu en payant gros la survivance
de la charge de maître de la garde-robe du roi, de La Salle, qui n'était
point marié, et qui avait très peu ou point de bien.

Le comte d'Évreux qui n'avait pas encore prêté son serment de colonel
général de la cavalerie, le prêta les premiers jours de cette année, et
encourut l'indignation des valets de la chambre. Le monopole des
serments était toujours allé croissant. D'une libéralité légère à ceux
qui prennent et rendent l'épée et le chapeau, cela s'était tourné en
droit par l'usage, et le droit avait toujours grossi par la sottise des
uns et l'intérêt des autres. Depuis plusieurs années, il y en avait
quantité montés à sept ou huit mille livres. Il ne fallait pas se
brouiller avec des valets que le roi croyait et aimait mieux que
personne, sans exception d'aucuns, si ce n'était de ses bâtards, et qui
par la fréquence des heures rompues qu'ils passaient seuls avec le roi
tous les jours, pouvaient quelquefois servir, mais incomparablement plus
nuire, et qui ont bien rompu des fortunes. Le comte d'Évreux paya en
argent blanc. Ils s'offensèrent, ils dirent qu'ils ne recevaient qu'en
or, et firent grand vacarme.

On a vu ci-devant, en plus d'un endroit, combien Chamillart, accablé
sous le poids des affaires, désirait d'être déchargé des finances, qui
de jour en jour devenaient plus difficiles. À la fin sa santé y
succomba. Les vapeurs lui firent traîner une vie languissante qui
ressemblait à une longue mort. Une petite fièvre fréquente, un
abattement universel, presque aucuns aliments indifférents, le travail
infiniment pénible, des besoins de lit et de sommeil à des heures
bizarres, en fin mot, un homme à bout, et qui se consumait peu à peu.
Dans ce triste état, qui le forçait souvent à manquer des conseils, et
quelquefois son travail avec le roi, il se sentit pressé de se décharger
du détail du trésor royal. Ce ne pouvait être qu'entre les mains d'un
des deux directeurs des finances. Armenonville, avec de l'esprit, de la
douceur, de la capacité et de l'expérience, même avec du monde, ne
s'était pu défaire d'une fatuité qu'une fortune prématurée donne aux
gens de peu, et il avait quelquefois hasardé jusqu'à des airs
d'indépendance dont Chamillart l'avait fait repentir. Le choix tomba
donc sur Desmarets. Quoique cette nouvelle confiance ne fût rien en
effet qu'une augmentation de travail, comme il s'en expliqua lui-même,
on pressentit dès lors son élévation\,; et on s'empressa chez lui, comme
si déjà il eût été déclaré contrôleur général.

Chamillart, instruit par l'affaiblissement de sa santé, songeait en même
temps à solider, son fils dans sa charge par une alliance qui pût l'y
soutenir. Les Noailles, ancrés partout par leurs filles, en voulaient
mettre une dans cette maison toute-puissante pour tenir tout\,; ils y
travaillaient, et M\textsuperscript{me} de Maintenon se laissait
entendre que ce mariage lui serait fort agréable. Mais la famille de
Chamillart y répugnait. Il s'était mis dans la cour de
M\textsuperscript{me} la duchesse de Bourgogne une jalousie entre les
filles de Chamillart et les Noailles, qui de la part des premières
allait jusqu'à l'antipathie. Gâtées comme elles l'étaient par une
prodigieuse fortune, et non moins encore par père et mère, elles ne se
contraignaient pas, et se croyaient tout permis. La duchesse de Lorges
était fort au gré de M\textsuperscript{me} la duchesse de Bourgogne\,;
elle était souvent admise en des confidences. C'était moissonner le
champ de la maréchale d'Estrées, et un peu dans celui de ses jeunes
soeurs. C'en était plus qu'il ne fallait pour qu'elles ne pussent se
souffrir. M\textsuperscript{me} Chamillart, ardente à conserver l'air de
gouverner chez elle, quelque peu et quelque mal qu'elle y gouvernât,
craignait le joug des Noailles. Son mari, qui l'éprouvait souvent, le
redoutait bien plus encore. Il s'éloignait donc beaucoup de leur donner
toutes sortes de droits chez lui en prenant leur fille pour son fils. Le
roi même, qui les appréhendait souvent, n'avait pas paru de goûter cette
affaire. Pour moi, qui voyais tout ce qu'il y avait à voir sur la santé
de ce ministre, sur les calamités de son administration, sur les cabales
naissantes, sur son peu de précaution fondée sur une excessive
confiance, je ne cessais d'inculquer à ses filles l'alliance des
Noailles, qui, par elle-même infiniment honorable aux Chamillart, était
la seule qui embrassât toutes les cours et tous les âges et qui par
conséquent fût un soutien pour tous les temps. Elle fixait
M\textsuperscript{me} de Maintenon par la considération du duc de
Noailles, elle dont les changements de goût avaient été si funestes à
des gens avec qui elle avait été autant ou plus intimement unie et plus
longuement qu'avec Chamillart Monseigneur, pour d'autres temps, leur
était assuré par, tous ses entours. M\textsuperscript{lle} Choin, à qui
les Noailles faisaient une cour servile, les ménageait à cause de
M\textsuperscript{me} de Maintenon, dont ils étaient le canal de
communication avec elle\,; M\textsuperscript{me} la Duchesse déjà leur
amie, et d'Antin d'un autre côté\,; d'un troisième, La Vallière, et
M\textsuperscript{me} la princesse de Conti, quelque peu considérable
qu'elle fût devenue. Enfin les liens secrets qui attachaient ensemble
M\textsuperscript{me} la duchesse de Bourgogne et lès jeunes Noailles,
ses dames du palais, répondaient de cette princesse pour le présent et
pour le futur\,; et par eux-mêmes auprès de Mgr le duc de Bourgogne ils
étaient sûrs des ducs de Chevreuse et de Beauvilliers. Ils y gagnaient
encore la duchesse de Guiche, dont l'esprit, le manège et la conduite
avait tant de poids dans sa famille, chez M\textsuperscript{me} de
Maintenon, et auprès du roi même, et qui imposait tant à la cour et au
monde. Je n'avais avec aucun des Noailles nulle sorte de liaison, sinon
assez superficiellement avec la maréchale, qui ne m'en avait jamais
parlé. Mais je croyais voir tout là pour les Chamillart, et c'était ce
qui m'engageait à y exhorter les Pillés, et ceux de leur plus intime
famille qui pouvaient être consultés.

Le duc de Beauvilliers était ami intime de, Chamillart. Il pouvait
beaucoup sur lui, mais non assez pour le ramener sur des choses qu'il
estimait capitales au bien de l'État. Il espéra vaincre cette
opiniâtreté en se l'attachant de plus en plus par les liens d'une proche
alliance. Je n'entreprendrai pas de justifier la justesse de la pensée,
mais la pureté de l'intention, parce qu'elle m'a été parfaitement
connue. Lui et la duchesse, sa femme, qui ne pensèrent jamais
différemment l'un de l'autre, prirent donc le dessein de faine le
mariage de la fille de la duchesse de Mortemart, qui n'avait aucun bien,
qui était auprès de sa mère et ne voulait point être religieuse. Au
premier mot qu'ils en touchèrent à la duchesse de Mortemart, elle bondit
de colère, et sa fille y sentit tant d'aversion, que plus d'une année
avant qu'il se fît, la marquise de Charost, fort initiée avec eux, lui
ayant demandé sa protection en riant lorsqu'elle serait dans la faveur,
pour la sonder là-dessus\,: «\,Et moi la vôtre, lui répondit-elle,
lorsque par quelque revers je serai redevenue bourgeoise de Paris.\,» M.
et M\textsuperscript{me} de Chevreuse, quoique si intimement unis avec
M. et M\textsuperscript{me} de Beauvilliers, car unis est trop peu dire,
rejetèrent tellement cette idée qu'ils ne furent plus consultés. J'ai su
d'eux-mêmes et de la duchesse de Mortemart, que, si sa fille l'eût voulu
croire, jamais ce mariage ne se serait fait.

De tout cela je compris que M. et M\textsuperscript{me} de Beauvilliers,
résolus d'en venir à bout, gagnèrent enfin leur nièce, et que, sûrs de
leur autorité sur M\textsuperscript{me} de Mortemart et sur le duc et la
duchesse de Chevreuse, ils poussèrent leur pointe vers les Chamillart,
qui, peu enclins aux Noailles, ne trouvant point ailleurs de quoi se
satisfaire, saisirent avidement les suggestions qui leur furent faites.
Une haute naissance avec des alliances si proches de gens si grandement
établis flatta leur vanité. Un goût naturel d'union qu'ils voyaient si
grande dans toute cette parenté les toucha fort aussi. Une raison
secrète fut peut-être la plus puissante à déterminer Chamillart\,; en
effet, elle était très spécieuse à qui n'envisageait point les
contredits. Personne ne sentait mieux que lui-même l'essentielle
incompatibilité de ses deux charges et l'impossibilité de les conserver
toutes deux. Il périssait sous le faix, et avec lui toutes les affaires.
Il ne voulait ni ne pouvait quitter celle de la guerre\,; mais, étant
redevable du sommet de son élévation aux finances, il comprenait mieux
que personne qu'elles emporteraient avec elles toute la faveur et la
confiance, et combien il lui importait en les quittant de se faire {[}de
son successeur{]} une créature reconnaissante qui l'aidât, non un ennemi
qui cherchât à le perdre, et qui en aurait bientôt tout le crédit. Le
comble de la politique lui parut donc consister dans la justesse de ce
choix, et il crut faire un chef-d'oeuvre en faisant tomber les finances
sur un sujet de soi-même peu agréable au roi, et par là peu à portée de
lui nuire de longtemps\,; il se le lia encore par des chaînes si fortes,
qu'il lui en ôta le vouloir et le pouvoir.

La personne de Desmarets lui parut faite exprès pour remplir toutes ces
vues. Proscrit avec ignominie à la mort de Colbert son oncle, revenu à
Paris à grande peine après vingt ans d'exil, suspect jusque par sa
capacité et ses lumières, silence imposé sur lui à Pontchartrain,
contrôleur général, qui n'obtint qu'à peine de s'en servir tacitement
dans l'obscurité et comme sans aveu ni permission\,; la bouche fermée
sur lui à tous ses parents en place qui l'aimaient\,; poulié\footnote{On
  a déjà vu plus haut ce mot, qui signifie hissé avec une poulie. Les
  précédents éditeurs l'ont remplacé par le mot \emph{poussé}.} à force
de bras et de besoins par Chamillart, mais par degrés, jusqu'à celui de
directeur des finances, mal reçu même alors du roi, qui ne put
s'accoutumer à lui tant qu'il fut dans cette place, redevable de tout à
Chamillart, c'était bien l'homme tout tel que Chamillart pouvait
désirer. Restait de l'enchaîner à lui par d'autres liens encore que ceux
de la reconnaissance, si souvent trop faibles pour les hommes\,; et
c'est ce qu'opérait le mariage de M\textsuperscript{lle} de Mortemart,
qui rendait encore-les ducs de Chevreuse et de Beauvilliers témoins et
modérateurs de la conduite de Desmarets si proche de tous les trois, et
si étroitement uni et attaché aux deux ducs. Tant de vues si sages et si
difficiles à concilier, remplies avec tant de justesse, parurent à
Chamillart un coup de maître\,; mais il en fallait peser les contredits
et comparer le tout ensemble.

Il ne tint pas à moi de les faire tous sentir, et je prévis aisément,
par la connaissance de la cour et des personnages, le mécompte du duc de
Beauvilliers et de Chamillart. Celui-ci était trop prévenu de soi, trop
plein de ses lumières, trop attaché à son sens, trop confiant pour être
capable de prendre en rien les impressions d'autrui. Je ne crus donc pas
un moment que l'alliance acquit sur lui au duc de Beauvilliers le plus
petit grain de déférence ni d'autorité nouvelle\,; je ne crus pas un
instant que M\textsuperscript{me} de Maintenon, indépendamment même de
son désir pour les Noailles, pût jamais s'accommoder de ce mariage. Sa
haine pour M. de Cambrai était aussi vive que dans le fort de son
affaire. Son esprit et ses appuis le faisaient tellement redouter à ceux
qui l'avaient renversé, et qui possédaient M\textsuperscript{me} de
Maintenon tout entière, que, dans la frayeur d'un retour, ils tenaient
sans cesse sa haine en haleine. Maulevrier, aumônier du roi, perdu pour
son commerce avec lui, avait eu besoin des longs efforts du P. de La
Chaise, son ami intime, pour obtenir une audience du roi, afin de s'en
justifier, il n'y avait que peu de jours. La duchesse de Mortemart
était, après la duchesse de Béthune, la grande âme du petit troupeau, et
avec qui, uniquement pour cela, on avait forcé la duchesse de Guiche, sa
meilleure et plus ancienne amie, de rompre entièrement et tout d'un
coup. La duchesse de Mortemart, franche, droite, retirée, ne gardait
aucun ménagement sur son attachement pour M. de Cambrai. Elle allait à
Cambrai, et y avait passé souvent plusieurs mois de suite. C'était donc
une femme que M\textsuperscript{me} de Maintenon ne haïssait guère moins
que l'archevêque\,; on ne le pouvait même ignorer.

J'étais de plus effrayé du dépit certain qu'elle concevrait de voir
Chamillart, sa créature et son favori, lui déserter pour ainsi dire, et
passer du côté de ses ennemis, comme il lui échappait quelquefois de les
appeler, je veux dire, dans la famille des ducs de Chevreuse et de
Beauvilliers, qu'elle rougissait encore en secret de n'avoir pu réussir
à perdre. Je n'étais pas moins alarmé sur son intérêt que sur son goût.
Elle en avait un puissant d'avoir un des ministres au moins dans son
entière dépendance, et sur le dévouement sans réserve duquel elle pût
s'assurer. On voit comme elle était avec les ducs de Chevreuse et de
Beauvilliers. Elle n'aimait guère mieux Torcy, et par lui-même et comme
leur cousin germain, qui s'était toujours dextrement soustrait à sa
dépendance, et ne s'en maintenait pas moins bien avec le roi. Elle était
tellement mal avec le chancelier dès le temps qu'il avait les finances,
qu'elle contribua, pour s'en défaire dans cette place, à lui faire
donner les sceaux\,; et depuis qu'il les eut, ses démêlés avec M. de
Chartres, et par lui avec les évêques pour leurs impressions et leurs
prétentions à cet égard, à voient de plus en plus aigri
M\textsuperscript{me} de Maintenon contre lui.

Son fils était un homme tout de travers, tout insupportable, duquel elle
ne pouvait ni ne se voulait aider. Chamillart, l'unique de tous
entièrement à elle, lui manquant entièrement aussi à son sens par, ce
mariage, il ne lui en demeurerait plus aucun. Je prévis bien que le
fruit, et prompt, de ce mariage serait de donner les finances à
Desmarets\,; qu'elle n'en pourrait parer le coup\,; qu'il en résulterait
qu'elle se résoudrait à défaire son propre ouvrage, désormais subsistant
sans elle et lié à ses ennemis\,; et que, son intérêt excitant sa
vengeance, elle entreprendrait tout pour le chasser, et par ce moyen
mettre en sa place une créature entièrement affidée, dont elle pût
entièrement disposer. Croire M\textsuperscript{me} de Maintenon
toute-puissante, on avait raison\,; mais la croire telle sans art et
sans contours, ce n'était pas connaître le roi ni la cour. Jamais prince
ne fut plus jaloux que lui de son indépendance et de n'être point
gouverné, et jamais pas un, ne le fut davantage. Mais, pour le
gouverner, il ne fallait pas qu'il pût le soupçonner\,; et c'est pour
cela que M\textsuperscript{me} de Maintenon avait besoin d'un ministre
dans un entier abandon à elle, et auquel elle se pût parfaitement fier.
Par lui, elle faisait tout ce que le roi croyait faire, et qu'il aurait
refusé par jalousie d'être gouverné si elle y eût paru. Ce curieux
détail, qui mènerait trop loin ici, pourra se développer ailleurs\,; il
suffit de le marquer ici en gros pour faire comprendre comment
M\textsuperscript{me} de Maintenon était toute-puissante, et l'extrême
besoin d'un ministre tout à elle pour l'être. Elle en trouva toujours,
parce que c'était le moyen sûr de primer tous les autres en faveur, en
`autorité, en confiance, et que le tout-puissant Louvois qu'elle avait
tué à terre, et qui allait à la Bastille, s'il n'était mort la veille de
cette exécution résolue, était une formidable leçon\,; et pour le duc de
Beauvilliers contre lequel ses poursuites n'étaient pas finies, on verra
ailleurs ce qui l'y déroba.

Ni lui ni Chamillart n'envisagèrent donc pas assez ce que je prévis de
ce mariage. Ils aimèrent mieux se croire que ces frayeurs. Dès qu'ils
l'eurent conclu entre eux, Chamillart en parla à M\textsuperscript{me}
de Maintenon qui d'abord se hérissa, et qui en éloigna le roi. Le
ministre s'en aperçut bien lorsqu'il lui en parla. Mais, malheureusement
accoutumé à marier ses enfants contre le gré de la puissance souveraine,
comme on l'a vu de La Feuillade, il retourna à la charge. Il obtint donc
un consentement dépité de sa bienfaitrice, et forcé du roi, à qui,
contre sa coutume, il échappa de dire que, puisque Chamillart voulait
absolument une quiétiste, au bout du compte cela ne lui faisait rien. De
cette façon s'accomplit le mariage au cuisant déplaisir de toute la
famille des Mortemart qu'ils ne prirent pas soin de trop cacher. Les
bâtards, qui se sont toujours piqués de prendre part en eux tous, ne se
cachèrent pas non plus d'entrer sur cela dans leur sentiment, et cette
conduite put confirmer ce qui vient d'être expliqué du dépit qu'en
conçut M\textsuperscript{me} de Maintenon, leur ancienne gouvernante,
qui tenait si tendrement à eux, et eux à elle avec tant de dépendance.
La noce se fit à l'Étang avec joie et magnificence, mais sans rien
d'outré, et la nouvelle marquise de Cani jouit environ six semaines de
toute la splendeur de son beau-père. Mais sa santé devenant tous les
jours plus mauvaise et son crédit plus tombé, faute d'avoir pu tenir
tous les engagements que la nécessité des affaires lui avait fait
contracter, et que cette même nécessité l'empêchait de remplir, il
songea tout de bon à tirer de ce mariage le principal avantage qu'il
s'en était proposé.

De longue main, Chamillart avait préparé sa besogne en faisant valoir
celle de Desmarets en toute occasion, et en se déchargeant sur lui des
affaires les plus importantes que sa santé ne lui permettait pas de
suivre d'assez près. Il avait de plus commencé à sentir que la nécessité
des affaires s'était enfin montrée au roi de manière à le laisser
abdiquer, et il connaissait trop M\textsuperscript{me} de Maintenon pour
n'avoir pas remarqué du changement en elle depuis la proposition du
mariage de son fils. Il en jugea, mais trop tard, qu'il était tellement
temps de remettre les finances, qu'elles lui seraient arrachées pour peu
qu'il différât à lui en donner la satisfaction. Cette découverte le
dégoûta de telle sorte, qu'il fut extrêmement tenté de se défaire de
tout à la fois, et d'en laisser démêler la fusée à son fils. Il le fut
au point qu'il n'en put être détourné qu'à peine par toute l'autorité de
la famille à laquelle il venait de s'allier, et par les désespoirs de sa
femme. C'est un secret que je sus dès lors par la duchesse de Mortemart,
que cela ne consola pas du mariage auquel elle s'était laissé entraîner
malgré elle. Le roi était alors à Marly. Il était piqué de ce que
M\textsuperscript{me} de Saint-Simon et moi avions quitté la danse qu'il
nous avait fait continuer d'autorité jusqu'à cette année. Je ne crus pas
qu'à trente-quatre ans que j'avais lors, elle me parât du ridicule de la
pousser si loin. On dansait à Marly, et nous ne fûmes point du voyage.
J'étais à l'Étang, où Chamillart, presque toujours au lit, et presque
point au travail, s'amusait avec sa famille. M'étant trouvé seul avec
lui, il me confia ce qu'il allait faire, mais sans aller jusqu'à me dire
ses desseins sur un successeur. Le mariage était fait\,; la haine en
était encourue\,; en cette situation il fallait au moins profiter de ce
qu'il se pouvait.

J'étais ami de Desmarets, je connaissais les désirs des ducs de
Chevreuse et de Beauvilliers\,; je voyais l'intérêt de Chamillart.
Quoique je me doutasse bien que son choix tombait sur lui, je craignis
la défaillance des moribonds qui leur fait si souvent changer leur
testament. Sans lui nommer Desmarets pour ne le point mettre en garde,
et ne l'irriter point aussi d'avoir pénétré ses vues, je lui représentai
son extrême intérêt d'avoir un successeur à lui qu'il eût le crédit de
faire\,; que ce successeur ne pût douter qu'il ne tînt son élévation que
de lui, et s'il était possible encore, qu'il fût tel que d'autres
engagements, outre ceux de la reconnaissance, l'unissent étroitement à
lui. Je le fortifiai surtout à n'être pas, dans une affaire pour lui si
capitale, la dupe des complaisances et des respects, mais à nommer, et à
faire, s'il en était besoin, un effort de crédit pour que son choix
l'emportât. J'appuyai fortement sur ce dernier article, parce que je
craignis les ruses de M\textsuperscript{me} de Maintenon, la faiblesse
et l'indécision du roi, et, plus que tout, la confiance de Chamillart
qui s'y pourrait trouver trompée. Le soir même j'allai à Paris, j'y vis
en arrivant Desmarets chez lui à qui je parlai franchement, et qui me
parla de même. Je trouvai un homme qui voyait les cieux ouverts, et qui
bien informé de toutes les démarches, bien appuyé des ducs de Chevreuse
et de Beauvilliers, comptait pour le lendemain le changement de sa
fortune.

M. le duc d'Orléans qui était sur son départ pour l'Espagne, m'avait
donné rendez-vous pour le lendemain matin au Palais-Royal. Nous y fûmes
enfermés longtemps tête à tête à discuter ses affaires, après quoi je le
mis en propos de celle des finances. Il savait tout par
M\textsuperscript{me} de Maintenon avec qui il était bien alors. Il me
la dit embarrassée et si peinée de l'état des choses, qu'elle l'avait
assuré que tout homme lui serait bon, pourvu que ce fût le plus habile,
et que, l'ayant pressée par curiosité sur Desmarets, elle ne lui en
avait point dit de mal, mais l'avait trouvée froide, et avait su d'elle
que le roi y avait un grand éloignement, sans quoi sa déclaration eût
été déjà faite. Je voulus pénétrer davantage sur les prétendants, mais
je n'en vis aucun de formel sinon Voysin, porté par
M\textsuperscript{me} de Maintenon, mais faiblement, parce qu'à ce coup
elle ne se trouvait pas la plus forte\,; qu'elle sentait que Chamillart
obtiendrait qui il voudrait, auquel elle ne s'ouvrait plus, et qu'elle
s'attendait bien qu'il ferait tout pour Desmarets. Là-dessus, je
retournai du Palais-Royal chez lui, et lui donnai une vive alarme. Il
m'assura cependant qu'il avait des lettres de Marly de ce même matin, et
il était lors midi, qui l'assuraient que les mesures étaient si bien
prises qu'il n'était pas possible qu'elles manquassent. Nous raisonnâmes
sur ce qui se pouvait faire. Je l'exhortai à presser vivement les deux
ducs de faire terminer la chose, l'un qui était à Paris en poussant son
beau-frère, l'autre, par lui-même pour ne pas donner le temps à
M\textsuperscript{me} de Maintenon de gagner du terrain, et au roi de
s'affermir trop dans sa répugnance. Je lui recommandai de se garder bien
de faire part de ce que je venais de découvrir au duc de Beauvilliers,
de peur de le ralentir sur la chose même en armant sa faiblesse
naturelle, surtout de bien confirmer Chamillart à le nommer nettement et
fortement sans se cacher sous des ambages, ni laisser au roi à le
deviner, ni la liberté de lui résister en face, ni de différer la
nomination à une autre fois.

Je laissai Desmarets dans ces agitations, quoique pleines d'espérance.
J'y étais moi-même pour lui, et pour l'intérêt de Chamillart. C'était le
dimanche gras. Je devais souper à l'hôtel de Chevreuse. On y fut gai en
apparence, inquiet en effet de n'avoir point de nouvelles que nous nous
promîmes de nous envoyer dès que nous en aurions. Le lundi matin je fus
chez le chancelier sur le midi, qui était à Paris, qui m'apprit que
Desmarets était contrôleur général. Je le mandai à l'instant à l'hôtel
de Chevreuse, où Goesbriant arrivait dans le même moment de la part de
son beau-père, lequel était à Marly, et en vint descendre le soir chez
le chancelier, auprès duquel il logeait, et avec qui il avait toujours
conservé une grande liaison. Lorsqu'il fut employé aux finances, il
demeura plusieurs jours sans en être directeur, sur quoi le chancelier
lui dit plaisamment que l'enfant était baptisé et en sûreté, mais non
encore nommé. Il avait beaucoup de traits comme celui-là, tous plaisants
et fort justes.

Le mardi gras, lendemain de cette déclaration, j'allai le matin chez
Desmarets. Je le trouvai dans son cabinet, au milieu des compliments, et
déjà des affairés. Il quitta tout dès qu'il me vit, et commença son
remercîment par des excuses de n'avoir pu venir lui-même chez moi me
donner part de sa nouvelle fortune, lesquelles il assaisonna de tout ce
qu'il put de mieux, puis me tirant à part dans une fenêtre, il me
raconta pendant plus d'une grosse demi-heure tout ce qui s'était passé.
Il me dit que Chamillart, qui n'avait pu sortir de l'Étang le samedi,
était allé à Marly le dimanche, et avait parlé au roi, qui, ayant
accepté sa démission des finances sans y faire de difficulté, avait
longtemps raisonné avec lui sur le successeur, sans témoigner de goût
particulier pour personne\,; que ce ministre, pressé à diverses reprises
de proposer qui il croyait le plus capable de bien remplir ses pénibles
fonctions, prononça enfin son nom, après avoir vainement essayé par
beaucoup de contours et de propos vagues, de le désigner et d'y faire
venir le roi\,; que le roi n'en fit encore nulle difficulté, et
l'accepta aussitôt, et lui ordonna de le lui amener le lendemain matin
lundi\,; qu'étant retourné tard à l'Étang, il ne lui put mander que fort
tard aussi de se rendre de bon matin le lendemain lundi à l'Étang, sans
ajouter rien de plus\,; qu'arrivé à sept heures, Chamillart lui apprit
lui-même l'heureux changement de sa fortune\,; qu'aussitôt après il le
mena à Vaucresson, petite maison de campagne du duc de Beauvilliers
assez proche, où, après avoir conféré assez longtemps, ils s'en allèrent
tous trois ensemble à Marly pour arriver à l'issue de la messe du roi.
Chamillart et Desmarets entrèrent dans son cabinet, où il consomma
l'affaire, et prévint Desmarets en lui expliquant lui-même l'état
déplorable de ses finances, tant pour lui faire voir qu'il savait tout,
que pour lui épargner peut-être l'embarras de lui en rendre un compte
exact, comme cela ne se pouvait éviter à l'entrée d'une
administration\,; le roi ajouta que, les choses en cet état, il serait
très obligé à Desmarets s'il y pouvait trouver quelque remède, et point
du tout surpris si tout continuait d'aller de mal en pis\,; ce qu'il
assaisonna de toutes les grâces dont il avait coutume de flatter ses
nouveaux ministres en les installant. Desmarets alla ensuite rendre ses
hommages à M\textsuperscript{me} de Maintenon, qui le reçut honnêtement,
sans rien de plus. Il revint de là à Paris par où il en était venu. Il
me dit que le roi l'avait infiniment surpris et soulagé, en lui disant
si nettement l'état de ses finances\,: surpris, parce qu'il n'imaginait
pas qu'il en sût le quart\,; soulagé, en lui ôtant la peine
indispensable de lui rendre un compte affligeant, et qui était
désagréable pour son prédécesseur, duquel il tenait son retour et sa
place.

Il me fit ensuite un plan abrégé de la conduite qu'il prétendait garder,
qui me parut très bonne. Il se proposait de ne se point engager comme
Chamillart en des paroles impossibles à tenir, de rétablir la bonne foi
qui est l'âme de la confiance et du commerce, de rendre au roi un compte
si net et si journalier, que, profitant des connaissances qu'il lui
avait montrées, il ne lui en laissât pas perdre le souvenir, soit pour
être disculpé des impossibilités qui se trouveraient dans les affaires,
soit aussi pour profiter auprès de lui des ressources qu'il pourrait
trouver. Comme il me parla avec beaucoup de confiance, et qu'il ne
laissa pas de me laisser entrevoir qu'il n'estimait pas tout ce qu'avait
fait Chamillart, je me licenciai à lui bien représenter les obligations
qu'il lui avait, et sur ce qu'il en voulut mettre quelque chose sur le
compte du chancelier, je ne le marchandai pas, et je lui remis bien
expressément devant les yeux que celui-là n'avait que désiré, mais que
l'autre avait effectué\,; que du néant d'une disgrâce obscure et
douloureuse par son prétexte et sa longueur, il l'avoir à force de bras
ramené sur l'eau pour l'honneur et pour la fortune, et lui avait enfin
donné sa propre place. Je m'échappai même jusqu'aux considérations de
reconnaissance et d'ingratitude. Desmarets les reçut bien. À ce propos
il me dit que, s'il se trompait désormais en amis, ce serait bien sa
faute, puisque vingt ans de disgrâce lui avaient appris à les bien
démêler. J'en pris occasion de lui toucher un mot de quelques personnes
considérables sur lesquelles je lui trouvai une mémoire nette et
présente.

Je lui dis en même temps que, depuis qu'il était rentré dans les
finances, il devait savoir les gens qui y faisaient des affaires\,; que
j'étais bien assuré qu'il n'y trouverait M\textsuperscript{me} de
Saint-Simon et moi pour rien\,; que nous avions toujours abhorré ces
sortes de moyens d'avoir, et que, du temps de Pontchartrain et de celui
de Chamillart, nous n'avions jamais voulu nous salir les mains
d'aucune\,; que tout ce que je lui demanderais serait accès facile,
payement de mes appointements et marques de considération et d'ancienne
amitié dans les affaires qu'on ne pouvait éviter d'avoir avec la
finance, depuis que tout l'était devenu, et qu'il n'y avait patrimoine
qui ne passât souvent devant messieurs des finances, à raison des taxes,
des impositions, des droits qui s'imaginaient tous les jours, tellement
qu'il fallait leur être redevable du peu qui en demeurait aux
propriétaires de plusieurs siècles. Il ne se put rien ajouter, à tout ce
qu'il me répondit là-dessus. Il me dit qu'il n'était pas à savoir
combien nous étions éloignés, M\textsuperscript{me} de Saint-Simon et
moi, de faire des affaires, et de là se lâcha sur les prostitutions en
ce genre de gens du plus haut parage, sur les trésors que MM. de Marsan
et de Matignon, unis ensemble, avaient amassés sans nombre et sans
mesuré, et sur tout ce que la maréchale de Noailles et sa fille, la
duchesse de Guiche, ne cessait de tirer, qui tous les quatre entre
autres avaient fait grand tort à Chamillart. Je l'arrêtai sur les
dernières, et lui contai que M\textsuperscript{me} de Saint-Simon,
fatiguée à la fin de tout ce qu'elle entendait contre Chamillart, à
l'occasion de ces deux dames, l'en ayant averti, il s'était mis à
sourire en avouant les choses en leur entier, et lui apprit qu'il avait
un ordre du roi pour leur donner part, à toutes les deux, dans toutes
les affaires qui se faisaient et se feraient, ce qui surprit extrêmement
Desmarets. Il le fut bien plus encore de ce que Chamillart se lavait les
mains des autres qui faisaient leurs affaires par le canal
d'Armenonville à son insu, mais avec certitude qu'il ne le trouverait
pas mauvais, bien qu'il ignorât le nombre prodigieux et les détails de
ces exactions.

Ces propos lui ouvrirent le champ sur Armenonville, indigné toujours que
son premier retour n'eût abouti qu'à le faire, pour son argent, confrère
cadet d'un homme dont la comparaison lui était odieuse. Il s'en était
souvent ouvert à moi dans ces temps-là. Jamais il n'avait été bien avec
lui qu'à l'extérieur. J'étais content d'Armenonville dans tout ce qui
s'était présenté à juger devant lui pour des taxes de terres et d'autres
semblables misères qui ne sont que trop continuelles. Il aimait
naturellement à obliger, surtout les personnes de qualité. Il me contait
souvent aussi ses griefs sur Desmarets dont il me savait ami, et plus
d'une fois, tandis qu'ils furent directeurs des finances, je fus arbitre
de leurs pointilleries. Desmarets n'était pas de meilleure condition
qu'Armenonville. Si l'un était neveu de Colbert, l'autre était
beau-frère de Pelletier le ministre. Mais le cruel compliment de ce
dernier en congédiant Desmarets, que, j'ai rapporté (t. II, p.~407),
était sans doute le germe de cette haine qu'il ne put retenir avec moi
dans ce moment de prospérité, quoiqu'il ne pût ignorer que je fusse de
ses amis, et la joie de pouvoir l'humilier et s'en défaire. Je quittai
Desmarets l'esprit rempli de réflexions sur les étranges mutations de ce
monde, et de doute d'une grande et indissoluble union entre Chamillart
et Desmarets.

L'instant de l'élévation d'un contrôleur général libre de tout autre
emploi fut celui de la suppression des deux directeurs des finances, qui
n'avaient été faits que pour le soulagement de Chamillart. Le roi voulut
que Desmarets fût remboursé de {[}sa charge{]}\,; et pour Armenonville,
on chercha quelqu'un qui voulût acheter bien cher une nouvelle place
d'intendant des finances. Le roi acheva le payement par l'érection d'une
capitainerie nouvelle du bois de Boulogne, avec la jouissance du château
de la Muette, et la survivance pour son fils, et une pension de douze
mille livres. Il lui conserva aussi son logement au château de
Versailles\,; mais en même temps il le priva de l'entrée au conseil des
finances, et le réduisit à la sèche fonction de simple conseiller
d'État\,: encore lui donna-t-il un dégoût inusité. La moitié des
conseillers d'État est ordinaire, l'autre moitié semestre\footnote{Les
  conseillers d'État ordinaires étaient en fonction toute l'année\,; les
  semestres, pendant six mois seulement.}. Cette différence est plutôt
un nom qu'une chose\,; mais les semestres, sont touchés de monter à
ordinaires, et le roi avait toujours coutume de faire monter l'ancien.
Armenonville l'était\,: Fourcy mourut, il demanda à monter\,; Voysin,
son cadet, fut préféré. Ce pauvre homme, si entêté du monde et de la
cour, vit disparaître en un moment celle qui remplissait ses
antichambres, congédia ses bureaux, et nettoya son cabinet de papiers de
finance pour y faire place aux factums des plaideurs. Il était à l'Étang
pour son travail ordinaire, un jour avant que Desmarets y fût mandé pour
devenir son maître. Il y était encore le matin qu'il y arriva\,; il l'y
vit arriver de Marly contrôleur général. Rien ne le surprit davantage,
tant on aime à se flatter. Il était fort répandu dans le monde, il avait
des amis, il voyait que les finances allaient changer de main, il
connaissait les appuis de Desmarets, il devait être averti. Il ne put
désespérer de sa fortune, il ne crut pas le coup de foudre si imminent.
Tout étourdi qu'il en fut, il le supporta en galant homme, et il fut
regretté. Je l'allai voir, et je me fis toujours un plaisir de lui
marquer la même considération et la même amitié.

Le nouvel intendant des finances fut Poulletier, très riche financier
qui avait passé sa vie dans les partis. Chamillart, à qui il était fort
attaché, lui voulut faire cette fortune inouïe pour un financier
qu'aucune magistrature n'avait encore décrassé. Ce fut ce que le
chancelier appela le testament de Chamillart, la honte de ces charges,
la flétrissure du conseil où ces intendants s'assoient, jugent, ont rang
de conseillers d'État, et quand ils le deviennent, en fixent
l'ancienneté à leur date d'intendants des finances. Cela fit grand
bruit. Le chancelier cria bien haut, le conseil députa pour faire des
oppositions, puis de très-horribles remontrances\,; ce n'en était plus
le temps\,: rien ne fut écouté. Desmarets se tint neutre pour plus d'une
raison. Chamillart tint ferme, et le roi maintint le changement d'un
financier en juge de la finance et des autres procès. Un jour que, dans
la chaleur de cette lutte, le chancelier s'emportait sur cette tache
seul avec moi, qu'il disait si livide et qui déshonorait tout un corps
illustre, je me mis à sourire et à lui demander froidement si ces
charges d'intendants des finances étaient héréditaires\,: il fut surpris
de la question. Je lui demandai ensuite s'il les comparait à nos
dignités, et le corps du conseil à notre collège\,; il fut encore plus
étonné. Après qu'il m'eut répondu à ces deux questions\,: «\,Ne vous
émerveillez donc pas, lui dis-je, si vous m'avez vu si outré lorsque ce
pied plat de Villars, sorti du greffe de Condrieu, est devenu duc
héréditaire.\,» À cela le chancelier n'eut pas un mot à répliquer. Il
baissa la tête, il m'avoua que j'avais grande raison, et il se lâcha
avec moi sur cet avilissement incroyable où, avec tant de soin, on prend
plaisir à tout confondre. Jamais depuis je ne l'ouïs dire un mot du
conseil et de Poulletier. Je me suis un peu étendu sur ce mariage du
fils de Chamillart, sur le changement de contrôleur général et sur ce
qui se passa alors entre Desmarets et moi. L'application de toutes ces
choses trouvera sa place en son temps.

Il n'est pas croyable combien on en prit occasion de crier contre le duc
de Beauvilliers. Avec sa dévotion, sa modestie, sa retraite, il
sacrifiait, disait-on, sa nièce, d'un sang illustre, à la passion de
dominer dans le conseil, et de se rendre l'arbitre des affaires par
Chamillart, dont le fils devenait son neveu par Desmarets et par Torcy,
ses cousins germains. La pureté de ses intentions n'était pas à portée
d'une cour si ambitieuse, où les envieux de ses places et de sa faveur
ne pouvaient comprendre qu'elles fussent si parfaitement soumises en lui
à la plus sévère vertu. M\textsuperscript{me} de Maintenon, enragée de
n'avoir pu le perdre, y donnait secrètement le ton par ses
confidentes\,; Harcourt et sa cabale, qui dévoraient ses emplois,
déployèrent une éloquence agréable à leur protectrice\,; les Noailles,
si outrés d'avoir manqué leur coup, ne se ménagèrent pas, et c'était une
tribu qui entraînait bien des gens\,; M. de La Rochefoucauld, qui ne les
aimait pas ni M\textsuperscript{me} de Maintenon, mais envieux né jusque
d'une cure de village, ne clabauda pas moins. Il n'y avait pas moyen
d'expliquer à cette multitude des raisons secrètes et qu'ils étaient si
peu capables de croire et de goûter. Il fallut donc se taire et laisser
écouler le torrent, qui passa aussi vite qu'il s'était formé, et dont la
sage tranquillité du duc de Beauvilliers ne put être seulement émue.

Le contrat de mariage de Cani (c'est le nom que prit le fils de
Chamillart en se mariant) fit naître une difficulté qui eut des suites
dont il n'est pas temps de parler. M\textsuperscript{lle} de Bourbon le
signa au-dessous de M\textsuperscript{me} la Duchesse sa mère\,;
M\textsuperscript{me} la duchesse du Maine s'en scandalisa et refusa de
signer\,; pour lors il n'en fut autre chose.

Le chevalier de Nogent mourut fort vieux, et s'était marié par une
ancienne inclination, il n'y avait pas longtemps, à une
M\textsuperscript{me} de La Jonchère à qui et à ses enfants il avait
donné tout son bien, et ne laissa point d'enfants. C'était une manière
de cheval de carrosse qui était de tout temps ami intime de
Saint-Pouange et favori de M. de Louvois. Cela l'avait fait aide de camp
du roi en toutes ses campagnes, et donné une sorte de considération.
Pendant une de celles-là, M. de Louvois, qui était magnifique pour ses
amis, lui fit bâtir et meubler la plus jolie maison du monde sous la
terrasse de Meudon, avec des jardins fort agréables, qu'il trouva prête
à habiter à son retour. On peut juger du plaisir de la surprise\,; c'est
la même que M\textsuperscript{me} de Verue a eue depuis et qu'elle a
tant embellie. Ce chevalier de Nogent était assez familièrement avec le
roi, mais depuis longtemps fort peu à la cour et dans le monde. Tout son
mérite était son attachement à M. de Louvois. Il était frère de Nogent,
tué au passage du Rhin, maître de la garde-robe, beau-frère de M. de
Lauzun, de Vaubrun, tué lieutenant général au combat d'Altenheim, cette
admirable retraite que fit M. de Lorges à la mort de M. de Turenne, et
de la princesse de Montauban. Leur père était capitaine de la porte, qui
par son esprit s'était bien mis à la cour, et fort familièrement avec le
cardinal Mazarin et la reine-mère. Leur nom était Bautru, de la plus
légère bourgeoisie de Tours.

Langlée mourut aussi en même temps sans avoir jamais été marié. J'ai
suffisamment parlé de ce bizarre personnage (t. II, p.~385 et suiv.). Le
monde y perdit du jeu, des fêtes et des modes, et les femmes beaucoup
d'ordures. Il laissa plus de quarante mille livres de rente, sa belle
maison meublée et d'autres effets à M\textsuperscript{lle} de Guiscard,
fille unique de sa soeur.

En même temps mourut encore le comte d'Oropesa, retiré auprès de
l'archiduc de Barcelone, duquel aussi j'ai suffisamment parlé (t. III,
p.~88).

Fort peu après mourut Montbron, que le servage à Louvois avait élevé et
porté même dans la familiarité du roi par la petitesse des détails.
C'était un petit homme de mine chétive, d'esprit médiocre, mais tout
tourné à faire, grand vanteur, parleur impitoyable, toutefois point
malhonnête homme, assez bon officier et brave, que le roi eût volontiers
fait maréchal de France, s'il eût osé par la comparaison de Montal, du
duc de Choiseul et d'autres qu'il ne voulut pas faire. Montbron portait
en plein le nom et les armes de cette grande et ancienne maison fort
tombée depuis longtemps, et qui le laissa faire, parce qu'on fait
là-dessus tout ce qu'on veut en France. Il venait de père en fils d'un
chevalier de Montberon, général des finances en 1539, qui était son
trisaïeul, et qui portait de Montberon brisé d'un filet en barre. Cette
marque, qui est d'un bâtard, et son emploi, sont parlants dans un homme
de ce nom. Sa postérité ne fit guère plus de figure en biens ni en
emplois. Le père de celui dont il s'agit ici fit ériger son méchant
petit fief de Sourdun en vicomté sous le nom de Montberon en 1654,
servit en de petits emplois, fut gouverneur de Bray-sur-Seine, et
parvint à faire deux de ses fils chevaliers de Malte. L'aîné, dont on
parle ici, se fourra dans la confiance de M. de Louvois, qui lui fit
donner la seconde compagnie des mousquetaires, dont le roi s'amusait
fort alors. Il devint lieutenant général et successivement gouverneur
d'Arras, Gand, Tournai et Cambrai et seul lieutenant général de Flandre,
où il demeurait toujours. M. de Louvois le fit chevalier de l'ordre en
la promotion de 1688, où il mit tant de militaires et tant de gens de
bas aloi. Montbron conserva toute sa vie ses cheveux verts, avec une
grande calotte, qui figurait fort mal avec son cordon bleu par-dessus.
Il venait voir le roi tous les ans, et en était toujours bien traité et
distingué. Il s'avisa d'être médecin et chimiste\,; il mit un remède à
la mode qui tua la plupart de ceux qui en usèrent, tous par des cancers.
Il lui en vint un à la main dont il mourut aussi. Un peu auparavant il
se démit de sa lieutenance générale de Flandre, dont le roi lui fit
donner cent cinquante mille livres par le chevalier de Luxembourg, et, à
sa mort, il donna Cambrai à Besons, et Gravelines, qu'avait celui-ci, à
Chamerault, favori de M. de Vendôme.

M. le duc d'Orléans n'avait voulu partir que mains garnies. Il savait ce
qu'il avait coûté à sa gloire et aux succès de la guerre, la campagne
précédente, du dénuement extrême de l'Espagne. Lorsqu'il arrangeait tout
pour son départ, on apprit que les Maures avaient pris Oran et accordé
une honnête capitulation à la garnison qui s'était retiré à Muzalquivir.
Tésut, fils d'un conseiller au parlement de Bourgogne, des amis de mon
père, et qui prenait soin de sa provision de vin, mourut subitement. Il
était secrétaire des commandements de M. le duc d'Orléans. C'était un
garçon de beaucoup d'esprit et de connaissances, fort singulier et fort
atrabilaire, et cependant assez répandu dans le monde, où il était
estimé et considéré au-dessus de son état. Il avait été en même qualité
à Monsieur, et quoique bien avec tout ce qui le gouvernait, il ne
laissait pas d'être fort honnête homme. L'abbé Dubois, que nous verrons
cardinal et maître du royaume, brigua fort la charge de Tésut, et M. le
duc d'Orléans, avec ce faible qu'il a toujours eu pour lui, et qui
semble devenu une plaie fatale aux princes pour leurs précepteurs,
mourait d'envie de la lui donner. M\textsuperscript{me} la duchesse
d'Orléans, dont pourtant il avait achevé le mariage, ne craignait rien
davantage, parce qu'elle le connaissait, et le roi, qui le connaissait
encore bien mieux, s'y opposa si décisivement que son neveu n'osa passer
outre. Il donna donc la charge à l'abbé de Tésut, frère de celui qui
venait de mourir, tout aussi honnête homme que lui, mais tout aussi
atrabilaire, et qui avait été employé en Hollande, en Allemagne et à
Rome pour les affaires de la succession palatine entre Madame et
l'électeur palatin. L'abbé Dubois ne put digérer cette exclusion. Ne
pouvant s'en prendre au roi ni guère à M\textsuperscript{me} la duchesse
d'Orléans, son désespoir se tourna contre l'émule qui l'avait emporté
sur lui. Jamais il ne lui pardonna, non pas même après que la fortune
aveugle l'eut élevé sur le plus haut pinacle. Il n'est pas temps de
s'étendre sur cet étrange compagnon.

Le roi voulut savoir les gens qui devaient suivre M. le duc d'Orléans en
Espagne, et ne voulut pas permettre que Nancré en fût. Le voyage de sa
belle-mère avec M\textsuperscript{me} d'Argenton l'avait gâté auprès du
roi. Il avait obtenu une audience pour s'en justifier à son retour de
Dauphiné, comme je l'ai dit alors\,; il crut y avoir réussi et se trouva
bien étonné de ce coup de caveçon. Il ploya les épaules\,; mais en
compère adroit, plein d'esprit, de fausseté et de manéges, à qui les
moyens quels qu'ils fussent ne coûtaient rien, il espéra bien de se
relever.

Parmi ceux qui devaient être de la suite du voyage M. le duc d'Orléans
nomma Fontpertuis. À ce nom, voilà le roi qui prend un air austère\,:
«\,Comment, mon neveu, lui dit le roi, Fontpertuis, le fils de cette
janséniste, de cette folle qui a couru M. Arnauld partout\,! je ne veux
point de cet homme-là avec vous. --- Ma foi, sire, lui répondit M. le
duc d'Orléans, je ne sais pas ce qu'a fait la mère, mais pour le fils,
il n'a garde d'être janséniste, et je vous en réponds\,; car il ne croit
pas en Dieu. --- Est-il possible, mon neveu\,? répliqua le roi en se
radoucissant. --- Rien de plus certain, sire, reprit M. d'Orléans\,; je
puis vous en assurer. --- Puisque cela est, dit le roi, il n'y a point
de mal, vous pouvez le mener.\,» Cette scène, car on ne peut lui donner
d'autre nom, se passa le matin\,; et l'après-dînée même, M. le duc
d'Orléans me la rendit pâmant de rire, mot pour mot, telle que je
l'écris. Après en avoir bien ri tous deux, nous admirâmes la profonde
instruction d'un roi dévot et religieux, et la solidité des leçons qu'il
avait prises de trouver sans comparaison meilleur de ne pas croire en
Dieu que d'être ce qu'on lui donnait pour janséniste, celui-ci dangereux
à suivre un jeune prince à la guerre, l'autre sans inconvénient par son
impiété. M. le duc d'Orléans ne se put tenir d'en faire le conte, et il
n'en parlait jamais sans en rire aux larmes. Le conte courut la cour et
puis la ville\,; le merveilleux fut que le roi n'en fut point
fâché\footnote{Ce passage se trouve déjà plus haut (t. V\,; p.~349).}.
C'était un témoignage de son attachement à la bonne doctrine, qui, pour
ne lui pas déplaire, éloignait de plus en plus du jansénisme. La plupart
en rirent de tout leur coeur\,; il s'en trouva de plus sages qui en
eurent plus d'envie de pleurer que de rire, en considérant jusqu'à quel
excès d'aveuglement le roi était conduit. Ce Fontpertuis était un grand
drôle, bien fait, ami de débauche de M. de Donzi, depuis duc de Nevers,
grand joueur de paume. M. le duc d'Orléans aimait aussi à y jouer, et de
tout temps aimait M. Donzi qu'il avait vu d'enfance avec nous au
Palais-Royal\,; et beaucoup plus en débauche lorsqu'il s'y fut livré.
Donzi lui produisit ce Fontpertuis pour qui il prit de la bonté.
Longtemps après, dans sa régence, il lui donna moyen de gagner des
trésors au trop fameux Mississipi, toujours sous la protection de M. de
Nevers. Mais quand ils furent gorgés de millions, Fontpertuis sans
proportion plus que l'autre, ils se brouillèrent, dirent rage l'un de
l'autre, et ne se sont jamais revus.

\hypertarget{chapitre-ix.}{%
\chapter{CHAPITRE IX.}\label{chapitre-ix.}}

1708

~

{\textsc{Projet d'Écosse.}} {\textsc{- Duc de Chevreuse ministre d'État
incognito.}} {\textsc{- Projet de faire révolter les Pays-Bas
espagnols.}} {\textsc{- Soupçons injustes de Chamillart éclaircis par
Boufflers.}} {\textsc{- Retour sincère de Chamillart pour Bergheyck.}}
{\textsc{- Ignorance et opiniâtreté surprenantes de Vendôme avec
Bergheyck devant le roi.}} {\textsc{- Principaux de la suite du roi
d'Angleterre en Écosse\,; leur état et leur caractère.}} {\textsc{-
Middleton et sa femme\,; leur état, leur fortune, leur caractère.}}
{\textsc{- Officiers généraux français de l'expédition.}} {\textsc{-
Gacé désigné maréchal de France\,; son caractère.}} {\textsc{- Départ du
roi d'Angleterre, que la rougeole arrête à Dunkerque.}} {\textsc{- Il
met à la voile.}} {\textsc{- Belle action du vieux lord Greffin.}}
{\textsc{- Espions à Dunkerque.}} {\textsc{- Le roi d'Angleterre battu
d'une grande tempête.}} {\textsc{- Attente et désirs des Écossais.}}
{\textsc{- Le roi d'Angleterre, chassé en mer et combattu par la flotte
anglaise, déclare Gacé maréchal de France et revient à Dunkerque.}}
{\textsc{- Gacé prend le nom de maréchal de Matignon.}} {\textsc{-
Middleton et Forbin causes du retour, et très suspects.}} {\textsc{-
Belle action du chevalier de La Tourouvre.}} {\textsc{- Prisonniers sur
le Salisbury bien traités.}} {\textsc{- Lévi lieutenant général.}}
{\textsc{- Grandeur de courage de Greffin.}} {\textsc{- Époque des noms
de chevalier de Saint-Georges et de Prétendant demeurés enfin au roi
Jacques III.}} {\textsc{- Entrevue du roi et de la cour débarquée et
revenue à Marly.}} {\textsc{- Sage conduite de la reine Anne et de ses
alliés.}}

~

Depuis longtemps un projet des plus importants frappait secrètement à
toutes les portes pour se faire écouter. Son heure arriva enfin au
dernier voyage de Fontainebleau où il fut résolu, où les promoteurs que
je devinai à leurs démarches, me l'avouèrent sous le dernier secret, où
j'en découvris un qui n'a été su que de bien peu de personnes intimes\,:
c'est que le duc de Chevreuse était en effet ministre d'État sans en
avoir l'apparence et sans entrer au conseil. À la fin je m'en doutai\,;
ses conférences si fréquentes à Fontainebleau avec Pontchartrain, l'aveu
qu'ils me firent l'un et l'autre de ce qui s'y traitaient, les suites de
cette affaire dans ce même voyage achevèrent de me persuader que je ne
me trompais pas en croyant le duc de Chevreuse ministre. Je me hasardai
de le dire nettement au duc de Beauvilliers, qui dans sa surprise me
demanda avec trouble d'où je le savais, et qui enfin me l'avoua sous le
plus profond secret. Dès le jour même, je me donnai le plaisir de le
dire au duc de Chevreuse. Il rougit jusqu'au blanc des yeux, il
s'embarrassa, il balbutia, il finit par me conjurer de garder sur cela
un secret impénétrable, qu'il ne put me dissimuler plus longtemps.

Je sus enfin par eux-mêmes qu'il y avait plus de trois ans, même quatre,
que les ministres des affaires étrangères, de la guerre, de la marine et
des finances avaient ordre de ne lui rien cacher, les deux premiers de
lui communiquer tous les projets et toutes les dépêches, et tous quatre
de conférer de tout avec lui. Il entrait très souvent chez le roi par
les derrières, souvent aux heures ordinaires. Il avait des audiences du
roi longues dans son cabinet, tantôt retenu par le roi, tantôt y restant
de lui-même quand tous en sortaient.

Quelquefois au dîner, mais presque tous les soirs au milieu du souper,
il venait au coin du fauteuil du roi. On se rangeait alors pour les
seigneurs. Le roi, qui entendait le mouvement, ne manquait guère à se
tourner pour voir qui arrivait, et quand c'était M. de Chevreuse, la
conversation se liait bientôt, puis se faisait à l'oreille, ou par M. de
Chevreuse de lui-même, ou par le roi qui l'appelait et lui parlait bas.
J'en fus Longtemps la dupe avec toute la cour, qui admirait qu'un détail
des chevau-légers pût fournir à des conversations si longues, si
fréquentes et si fort à l'oreille, et qui s'en étonna bien plus quand ce
prétexte eut cessé par la démission de cette compagnie à son fils. À la
fin je me doutai d'autre chose, et j'en découvris tout le mystère à
Fontainebleau. C'était d'affaires d'État qu'il s'agissait dans ces
conversations, et d'affaires d'État que le duc de Chevreuse s'occupait
si assidûment dans son cabinet, où personne ne pouvait comprendre que
ses affaires domestiques ni celles des chevau-légers le pussent tenir si
habituellement. Il avait toujours été, au goût du roi. C'était peut-être
le seul homme d'esprit et savant qu'il ne craignît point. Il était
rassuré par sa douceur, sa mesure, sa modestie, et par ce tremblement
devant lui qui fit toujours son grand mérite et celui du duc de
Beauvilliers. Personne ne parlait plus juste, plus nettement, plus
facilement, plus conséquemment, ni avec plus de lumière, avec une
douceur et un tour aisé en tout. Le roi l'aurait volontiers mis dans le
conseil, mais M\textsuperscript{me} de Maintenon, Harcourt, jusqu'à M.
de La Rochefoucauld qu'il craignit là-dessus, l'en empêchèrent. Il prit
donc le parti de cet incognito, que je crois avoir été unique en ce
genre, et dont personne peut-être, hors le duc de Chevreuse, ne se
serait accommodé, surtout avec la certitude de l'obstacle qui le
réduisait à cette sorte de ténèbres subsisterait toujours, et toujours
lui fermerait la portée du conseil. Il était un avec le duc de
Beauvilliers, et ils passaient presque tout leur loisir ensemble\,; ils
étaient en liaison et cousins germains de Torcy, et maintenant de
Desmarets, et amis intimes de Chamillart dès son entrée au ministère.
Quoique le chancelier fût ennemi de Beauvilliers, il aimait le duc de
Chevreuse, et celui-ci en avait été si content lors de ses divers
échanges avec Saint-Cyr qu'il en était demeuré de ses amis. Par
conséquent Pontchartrain, quoiqu'il n'aimât pas les amis de son père,
n'osait, avec les ordres qu'il avait, n'être pas en grande mesure avec
lui\,; et de cette façon, les commerces continuels d'affaires des
ministres avec lui, et de lui avec eux, étaient couverts des liaisons de
parenté, d'amitié et de société.

Ce fut par lui que le projet fut admis. Hough, gentilhomme anglais,
plein d'esprit et de savoir, et qui surtout possédait les lois de son
pays, y avait fait divers personnages. Ministre de profession et furieux
contre le roi Jacques, puis catholique et son espion, il avait été livré
au roi Guillaume qui lui pardonna. Il n'en profita que pour continuer
ses services à Jacques. Il fut pris plusieurs fois, et s'échappa
toujours de la Tour de Londres et d'autres prisons. Ne pouvant plus
demeurer en Angleterre, il vint en France, où, vivant en officier, il
s'occupa toujours d'affaires, et fut payé pour cela par le roi et par le
roi Jacques, au rétablissement duquel il pensait sans cesse. L'union de
l'Écosse avec l'Angleterre lui parut une conjoncture favorable par le
désespoir de cet ancien royaume de se voir réduit en province sous le
joug des Anglais. Le parti jacobite s'y était conservé\,; le dépit de
cette union forcée l'accrut dans le désir de la rompre par un roi qu'ils
auraient rétabli. Hough, qui conservait partout des intelligences, fut
averti de cette fermentation\,; il y fit des voyages secrets, et, après
avoir frappé longtemps ici à diverses portes de ministres, Caillières, à
qui il s'ouvrit, en parla au duc de Chevreuse, puis au duc de
Beauvilliers, qui y trouvèrent de la solidité. C'était un moyen sûr de
faire une diversion puissante, de priver les alliés du secours des
Anglais occupés chez eux, et les mettre dans l'impuissance de soutenir
l'archiduc en Espagne, et dans l'embarras partout ailleurs dénués des
forces anglaises. Les deux ducs gagnèrent Chamillart, puis Desmarets
tout à la `fin, dès qu'il fut en place. Mais le roi était si rebuté des
mauvais succès qu'il avait si souvent éprouvés de ces sortes
d'entreprises, que pas un d'eux n'osa la lui proposer. Chamillart ne
faisait qu'y consentir. Épuisé de corps et d'esprit, accablé d'affaires,
il n'était pas en situation de devenir le promoteur de cette affaire.
Chevreuse en parla au chancelier pour voir s'il la goûterait et s'il
voudrait persuader son fils dont le ministère devenait principal en ce
genre. Le chancelier y entra. Pontchartrain n'osa rebuter, mais il
essaya de profiter de la lenteur naturelle de M. de Chevreuse et de sa
facilité à raisonner sans fin pour allonger et le rebuter à force de
difficultés. C'est ce qui me fit découvrir l'affaire à Fontainebleau.
J'y logeais chez Pontchartrain au château, et j'étais fort souvent chez
M. de Chevreuse. Leurs visites continuelles, leurs longues conférences
me mirent en curiosité, et je sus enfin dès Fontainebleau, de quoi il
s'agissait entre eux, que Caillières après me mit au net à mesure du
progrès.

C'était cependant à qui attacherait le grelot. Le duc de Noailles leur
parut propre à gagner M\textsuperscript{me} de Maintenon qui en était
coiffée, et qui lui parlait de tout. M. de Chevreuse, nonobstant tout ce
que le maréchal avait fait et tenté contre eux dans l'affaire de M. de
Cambrai, était toujours en liaison avec eux, parce que, tantôt par ordre
du roi, et quelquefois à la prière des parties, il avait essayé de les
accommoder avec les Bouillon dans l'affaire de la vassalité de Turenne,
qui avait été poussée extrêmement loin entre eux et qui n'était rien
moins que finie ni qu'amortie. Ils attendirent donc le retour du duc de
Noailles de Roussillon, et s'ouvrirent à lui du projet d'Écosse. Flatté
de la confiance, du besoin de son secours et d'une occasion d'entrer de
plus en plus avec M\textsuperscript{me} de Maintenon en affaires
importantes, il se chargea volontiers de lui parler de celle-ci et de la
lui faire approuver. Elle était alors pour le duc de Noailles en
admiration continuelle\,; elle n'eut donc pas de peine à approuver ce
qu'il lui présenta comme faisable. Ces mesures prises, il ne fut plus
question que d'y amener le roi. Il ne fallait pas moins pour y réussir
que M\textsuperscript{me} de Maintenon avec tous les ministres. Encore
était-il si dégoûté de toutes ces sortes d'entreprises, dont pas une
n'avait réussi, qu'il ne donna dans celle-ci que par complaisance et
sans avoir pu la goûter. Dès qu'il y eut consenti, on mit tout de bon la
main à l'oeuvre\,; mais en même temps, on se proposa une autre
entreprise de cadence et de suite à celle-ci.

On crut pouvoir profiter du désespoir dans lequel les traitements des
Impériaux avaient jeté les Pays-Bas espagnols, tombés entre leurs mains
après la bataille de Ramillies, et les faire révolter dans le temps que
l'affaire d'Écosse étourdirait les alliés, les priverait de tout secours
d'Angleterre, et les engagerait peut-être à y en envoyer. Bergheyck,
dont j'ai eu assez souvent occasion de parler pour n'avoir plus à le
faire connaître, fut mandé comme l'homme le plus instruit de l'état de
ces pays, par les amis et les intelligences qu'il y avait toujours
conservés, et dont la capacité, le grand sens et la connaissance des
personnes et des lieux seraient les plus capables d'éclairer, tant pour
la résolution à prendre que pour la manière d'exécuter. Il arriva donc
chez Chamillart. Ce ministre, séduit dans tous les commencements par
ceux dont il se servait à Bruxelles, qui pour conserver et accroître
leur autorité voulurent ruiner celle de Bergheyck, avait conçu des
soupçons auxquels il donna trop d'essor. Boufflers, qui commandait alors
à Bruxelles et dans tous les Pays-Bas français et même espagnols, par
son union avec le marquis de Bedmar, suivit de près Bergheyck, et à
force de s'en informer et de l'éclairer il reconnut qu'il n'y avait
point d'homme plus capable, plus fidèle, plus désintéressé. Sa conduite
avec nos généraux, nos officiers, nos intendants confirma si pleinement
le témoignage que Boufflers ne cessa d'en rendre, que Chamillart,
n'osant plus attaquer son autorité, entra enfin en concert avec lui de
toutes choses, et s'en trouva si excellemment bien qu'il lui donna toute
sa confiance, et devint pour toujours son ami particulier. On confia
donc à Bergheyck le projet résolu d'Écosse, et on lui proposa celui des
Pays-Bas\,; il ne le jugea pas impossible. L'embarras était que les
Espagnols étaient les moins forts dans toutes les places. Mais
Bergheyck, après y avoir bien pensé, crut pouvoir pratiquer si bien les
principaux des villes que tout réussirait sans peine dans ce premier
étonnement de l'entreprise d'Écosse, avec l'appui de la combustion de
l'Angleterre, de nos armées en Flandre, et en même temps de quelque
expédition sur le Rhin, pour tenir partout les ennemis en incertitude et
en haleine.

Avant de congédier Bergheyck, il fallut examiner, dans la supposition du
succès, les mouvements à faire faire aux armées de Flandre, selon les
divers cas et les diverses ouvertures qui se pourraient présenter. Pour
cela il fallut raisonner avec celui qui les devait commander. C'était le
duc de Vendôme, que le goût du roi mettait volontiers dans ce secret.
Lui et Bergheyck en raisonnèrent devant le roi, Chamillart présent.
Parcourant les différentes choses qui se pourraient exécuter, selon que
la facilité s'en présenterait par un côté ou par un autre-, il fut
question de Maestricht. Vendôme, ne doutant de rien, expliquait comment
il prétendait s'y prendre\,; Bergheyck contestait. Vendôme, indigné
qu'un homme de plume osât disputer de mouvements de guerre et
d'entreprises sur des places avec lui, s'échauffa\,; l'autre, froid et
respectueux, demeura ferme. À la fin ils comprirent que le cours de la
Meuse formait la dispute. Vendôme se moqua de Bergheyck comme d'un
ignorant qui ne savait pas la position des lieux. Bergheyck, toujours
modeste, se rabattit à ne se point mêler des dispositions que Vendôme
prétendait faire, mais à maintenir qu'elles seraient inutiles, parce
qu'il mettait la Meuse entre lui et Maestricht. Vendôme plus échauffé
soutint que c'était le contraire, que la Meuse ne coulait point là, mais
d'un autre côté, et qu'elle n'était point entre lui et Maestricht de la
manière qu'il proposait de se mettre. De cette façon il pouvait avoir
raison\,; de l'autre, à se placer comme il voulait, l'entreprise était
non seulement impossible, mais ne se pouvait imaginer. Dans ce contraste
de facilité ou d'impossibilité physique, le fait en décidait. Vendôme
eut beau répondre qu'il était sûr de ce qu'il avançait, et crier en
maître de l'art avec mépris de cet homme de plume qui voulait savoir
mieux que lui la situation des lieux, le roi, lassé d'une pure question
de fait, prit des cartes. On chercha celle où était Maestricht, et elle
prouva que Bergheyck avait raison. Un autre que le roi eût senti à ce
trait quel était ce général de son goût, de son coeur, de sa
confiance\,; un autre que Vendôme eût été confondu\,; mais ce fut
Bergheyck qui le demeura de cette scène, et qui ne cessa depuis de
trembler de plus en plus de voir les armées en de telles mains, et
l'aveuglement du roi pour elles. Il fut renvoyé très promptement en
Flandre pour travailler au projet de révolte, et il le fit si utilement
qu'on put compter bientôt après sur un solide succès, mais ce succès
était si dépendant de celui d'Écosse, par lequel il fallait commencer
avant que de remuer rien en Flandre que, le premier ayant avorté, ce ne
fut que par la spéculation qu'on put juger de ce qui serait résulté des
intelligences et des pratiques de Bergheyck.

On avait caché dans le village de Montrouge, près Paris, des députés
écossais, chargés des pouvoirs des principaux seigneurs du pays et d'une
infinité d'autres signatures. Ils pressaient fortement l'expédition. Le
roi en donna tous les ordres. On arma trente vaisseaux à Dunkerque et
dans les ports voisins, en comptant les bâtiments de transport. Le
chevalier de Forbin, qui s'était signalé, comme on l'a vu en son temps,
dans la mer Adriatique, dans celle du Nord, et sur les côtes
d'Angleterre et d'Écosse, fut choisi pour commander l'escadre destinée
pour l'Écosse. On envoya quatre millions en Flandre pour le payement des
troupes dont on fit avancer six mille hommes sur les côtes vers
Dunkerque. Ce qui s'y passait fut donné pour armements de particuliers,
et le mouvement des troupes pour changements de garnisons. Le secret fut
observé très entier jusqu'au bout\,; mais le mal fut que tout fut très
lent. La marine ne fut pas prête à temps\,; ce qui dépendit de
Chamillart encore plus tard.

Lui et Pontchartrain, de longue main aigris l'un contre l'autre, se
rejetèrent mutuellement la faute avec beaucoup d'aigreur. La vérité est
que tous deux y étaient, mais que Pontchartrain fut plus qu'accusé d'y
avoir été par mauvaise volonté et l'autre par impuissance. On eut grand
soin qu'il ne parût aucun mouvement à Saint-Germain. On couvrit le peu
d'équipages qu'on tint prêts au roi d'Angleterre d'un voyage à Anet pour
des parties de chasse. Il ne devait être suivi, comme en effet il ne le
fut, que du duc de Perth qui avait été son gouverneur, de Scheldon qui
avait été son sous-gouverneur, des deux Hamilton, de Middleton, et de
fort peu d'autres.

Perth était Écossais\,; il avait été longtemps chancelier d'Écosse, qui
est la première dignité et la plus autorisée du pays, et qui est aussi
militaire, toujours remplie par les premiers seigneurs. Ses gendres, ses
neveux, ses plus proches y occupaient encore les premiers emplois, y
avaient le principal crédit, et étaient tous dans le secret et les plus
ardents promoteurs de l'entreprise. Le sous-gouverneur était un des plus
beaux, des meilleurs et des plus étendus esprits de toute l'Angleterre,
brave, pieux, sage, savant, excellent officier, et d'une fidélité à
toute épreuve. Les Hamilton étaient frères de la comtesse de Grammont,
des premiers seigneurs d'Écosse, braves et pleins d'esprit, fidèles.
Ceux-là, par leur soeur, étaient fort mêlés dans la meilleure compagnie
de notre cour\,; ils étaient pauvres et avaient leur bon coin de
singularité. Middleton était le seul secrétaire d'État, parce qu'il
avait coulé à fond le duc de Melford, frère du duc de Perth, qui était
l'autre, qui n'en avait plus que le nom depuis les exils où fort
injustement, à ce que les Anglais de Saint-Germain prétendaient,
Middleton l'avait fait chasser. Il n'habitait plus même Saint-Germain.
La femme de Middleton était gouvernante de la princesse d'Angleterre, et
avait toute la confiance de la reine. C'était une grande femme, bien
faite, maigre, à mine dévote et austère. Elle et son mari avaient de
l'esprit et de l'intrigue comme deux démons\,; et Middleton, par être de
fort bonne compagnie, voyait familièrement la meilleure de Versailles.
Sa femme était catholique, lui protestant, tous deux de fort peu de
chose, et les seuls de tout ce qui était à Saint-Germain qui touchassent
tous leurs revenus d'Angleterre. Le feu roi Jacques, en mourant, l'avait
fort exhorté à se faire catholique. C'était un athée de profession et
d'effet, s'il peut y en avoir, au moins un franc déiste\,; il s'en
cachait même fort peu. Quelques mois après la mort de Jacques, il fut un
matin trouver la reine, et comme éperdu lui conta que ce prince lui
était apparu la nuit, lui déclara avec grande effusion de coeur qu'il
devait son salut à ses prières, et protesta qu'il était catholique. La
reine fut assez crédule pour s'abandonner au transport de sa joie.
Middleton fit une retraite qu'il termina par son abjuration, se mit dans
la grande dévotion, et à fréquenter les sacrements. La con fiance de la
reine en lui n'eut plus de bornes\,; il gouverna tout à Saint-Germain.
La Jarretière lui fut offerte qu'il refusa par modestie, mais pour tout
cela ses revenus d'Angleterre ne lui étaient pas moins fidèlement remis.
Plus d'une fois le projet d'Écosse, proposé d'abord à Saint-Germain,
avait été rejeté par lui, et méprisé par la reine qu'il gouvernait.
Quand il se vit pleinement ancré, il quitta peu à peu la dévotion, et
peu à peu reprit son premier genre de vie sans que son crédit en reçût
de diminution. Cette fois, comme les précédentes, il fut de tout le
secret\,; mais, comme notre cour y entrait avec efficace, il n'osa le
contredire, mais il s'y rendit mollement. Tel fut le seul et véritable
mentor que la reine donna au roi son fils pour l'expédition d'Écosse.

L'affaire était au point qu'elle ne pouvait plus être retardée\,; le
secret commençait à transpirer. On avait embarqué une prodigieuse
quantité d'armes et d'habits pour les Écossais\,; les mouvements de
terre et de mer étaient nécessairement devenus trop visibles sur la
côte. Chamillart fit nommer pour lieutenants généraux Gacé, frère de
Matignon, et Vibraye\,: le premier bon et honnête homme, mais sans
esprit, sans capacité, sans réputation quelconque à la guerre\,;
Vibraye, brave et fort débauché, c'était tout. M. de Chevreuse voulut
que Lévi, son gendre, fût l'ancien des deux maréchaux de camp\,; Ruffey,
mort sous-gouverneur du roi, fut l'autre. Chamillart, intime des
Matignon, saisit cette occasion pour faire Gacé maréchal de France. Le
roi eut la complaisance pour son ministre de faire expédier par Torcy
des patentes à Gacé d'ambassadeur extraordinaire auprès du roi
d'Angleterre, et de trouver bon que Chamillart remît au roi d'Angleterre
un paquet cacheté, qui contenait les provisions de maréchal de France
pour le même Gacé, à qui ce prince le devait remettre lorsqu'il aurait
mis pied à terre en Écosse.

Enfin, le mercredi 6 mars, le roi d'Angleterre partit de Saint-Germain.
Tant de lenteurs ne permirent pas de douter qu'on ne fût enfin instruit
en Angleterre. On comptait que {[}les Anglais{]} n'auraient pas de quoi
s'y opposer, parce que le chevalier Leake avait emmené presque tout ce
qui leur restait de vaisseaux de guerre à l'escorte d'un grand convoi
pour le Portugal. On fut surpris de voir arriver, le dimanche 11 mars,
le chevalier de Fretteville à Versailles avec la nouvelle que Leake,
repoussé par les vents contraires à Torbay (où on sut depuis qu'il
s'était tenu caché), était venu bloquer Dunkerque, sur quoi on avait
débarqué nos troupes. Il apportait une lettre du roi d'Angleterre, qui
criait fort contre ce débarquement, et qui voulait tout forcer, et à
quelque prix que ce fût, tenter de passer et de se rendre en Écosse. Il
en fit tant de bruit à Dunkerque que le chevalier de Forbin ne put
s'empêcher d'envoyer reconnaître cette flotte par les chevaliers de
Tourouvre et de Nangis, sur le rapport desquels on espéra de pouvoir
passer\,; et tout de suite on fit rembarquer les troupes. Mais voici le
contretemps, supposé que l'entreprise ne fût pas déjà échouée longtemps
avant le départ de Saint-Germain. La princesse d'Angleterre avait eu la
rougeole\,; elle commençait à peine à entrer en convalescence lors du
départ du roi son frère. On l'avait empêché de la voir, de peur qu'il ne
gagnât ce mal, sur le point de l'entreprise. Il se déclara à Dunkerque,
à la fin de l'embarquement des troupes. Voilà un homme au désespoir, qui
veut qu'on l'enveloppe dans des couvertures et qu'on le porte au
vaisseau. Les médecins crièrent que c'était le tuer avec certitude\,; il
fallut demeurer. Deux des cinq députés écossais, cachés chez le bailli à
Montrouge, avaient été renvoyés, il y avait plus de quinze jours, pour
annoncer en Écosse l'arrivée imminente de leur roi avec des armes et des
troupes. Le mouvement que cela devait produire donnait encore plus
d'impatience du départ. Enfin le roi d'Angleterre, à demi guéri et fort
faible, se voulut déterminément embarquer le samedi 19 mars, malgré les
médecins et la plupart de ses domestiques. Les vaisseaux ennemis
s'étaient retirés\,; à six heures du matin, ils mirent à la voile par un
bon vent et par une brume qui les fit perdre de vue sur les sept heures.

Il y avait à Saint-Germain un vieux milord Greffin\,; fort borné, fort
protestant, mais fort fidèle, que la passion de la chasse et sa bonté à
voit attaché à M. le comte de Toulouse, à M. de La Rochefoucauld, et aux
chasseurs de la cour qui tous l'aimaient. Il n'avait rien su du tout que
par le départ du roi d'Angleterre\,; il fut sur-le-champ trouver la
reine. Avec la liberté anglaise, il lui reprocha son peu de confiance en
lui, malgré ses services et sa constante fidélité\,; celle qu'elle
témoignait à d'autres qui, sans les nommer, ne le valaient en rien\,; le
peu de bonté qu'elle lui avait montré en tous les temps, finit par
l'assurer que son âge, sa religion, ni la douleur de se voir si
maltraité, ne l'empêcheraient pas de suivre le roi, et de le servir
jusqu'au dernier moment de sa vie, de manière à faire honte à la reine,
et de ce pas vint à Versailles demander un cheval et cent louis à M. le
comte de Toulouse\,; et tout de suite piqua droit à Dunkerque, où il
s'embarqua avec les autres.

On arrêta, en divers endroits de Dunkerque, onze hommes que le
gouverneur d'Ostende y avait envoyés pour être exactement informés de
tout. Il y en avait un douzième qui se cacha si, bien dans la ville
qu'on ne le put trouver\,; mais, lors de cette capture, le roi
d'Angleterre était à la voile. Il essuya le soir même une furieuse
tempête, après laquelle il mouilla derrière les bancs d'Ostende.

Deux fois vingt-quatre heures après le départ de notre escadre,
vingt-sept vaisseaux de guerre anglais parurent devant Dunkerque.
Beaucoup de troupes anglaises marchèrent vers Ostende, et des
Hollandaises vers la Brille pour se mettre en état de passer la mer.
Rambure, lieutenant de vaisseau, qui commandait une frégate, fut séparé
de l'escadre par la tempête. Il fut obligé de relâcher aux côtes de
Picardie, d'où, dès qu'il le put, il se remit après l'escadre qu'il crut
déjà en Écosse. Il fit donc route sur Édimbourg, et ne trouva aucun
vaisseau dans toute sa traversée. Comme il approchait de l'embouchure de
la rivière, il vit la mer couverte de barques et de petits bâtiments
qu'il ne crut pas pouvoir éviter, et dont il aima mieux s'approcher de
bonne grâce. Les patrons lui dirent que leur roi devait être arrivé\,;
qu'ils n'eh avaient point de nouvelles\,; qu'il était attendu avec
impatience\,; que ce grand nombre de bâtiments venait au-devant de lui
et à sa découverte\,; qu'ils lui amenaient des pilotes pour le faire
entrer dans la rivière et le conduire à Édimbourg, où tout était dans
l'espérance et la joie. Rambure, également surpris que l'escadre qui
portait le roi d'Angleterre n'eût point encore paru, et de la publicité
de son arrivée prochaine, remonta vers Édimbourg toujours de plus en
plus environné de barques qui lui tenaient le même langage. Un
gentilhomme du pays passa d'un de ces bâtiments sur la frégate. Il lui
apprit la signature des seigneurs principaux qu'il lui nomma\,; que ces
seigneurs étaient assurés de plus de vingt mille hommes du pays prêts à
prendre les armes, et de toute la ville qui n'attendait que son arrivée
pour le proclamer. Rambure se mit ensuite à descendre la rivière pour
chercher à rejoindre {[}l'escadre{]}, dont il était d'autant plus en
peine que ce qu'il venait de voir et d'apprendre était plus
satisfaisant. Approchant de l'embouchure, il entendit un grand bruit de
canon à la mer, et, peu après, il aperçut beaucoup de vaisseaux de
guerre. Approchant de plus en plus, et, sortant de la rivière, il
distingua l'escadre de Forbin poursuivie par vingt-six gros navires de
guerre, et de quantité d'autres bâtiments, dont il perdit bientôt de vue
tant notre escadre que de l'avant-garde des ennemis. Il continua de
hâter sa route pour joindre, mais il ne put arriver que tout n'eût
dépassé l'embouchure. Alors, après avoir évité les plus reculés de
l'arrière-garde anglaise, il remarqua que leur flotte donnait une rude
chasse au roi d'Angleterre, qui longeait cependant la côte parmi le feu
du canon et souvent de la mousqueterie. Rambure essaya longtemps de
profiter de la légèreté de sa frégate pour gagner la tête, mais toujours
coupé par des vaisseaux ennemis et toujours en danger d'être pris, il
prit le parti de revenir à Dunkerque, d'où il fut aussitôt dépêché à la
cour pour y porter ces tristes et inquiétantes nouvelles. Elles furent
suivies, cinq ou six jours après\,; du retour du roi d'Angleterre, qui
rentra le 7 avril à Dunkerque avec peu de ses vaisseaux, fort
maltraités.

Ce prince, après la tempête qu'il essuya d'abord, ayant repris sa route
avec son escadre rassemblée, se perdit de son chemin deux fois
vingt-quatre heures, ce qui, sans la violence des vents qui était
cessée, n'est pas aisé à comprendre dans la traversée de la hauteur des
bancs d'Ostende, où ils s'étaient jetés pendant la tempête, à la rivière
d'Édimbourg. Cette méprise donna le temps aux Anglais de les joindre,
sur quoi le roi d'Angleterre tint conseil sans y appeler personne des
autres vaisseaux. On perdit beaucoup de temps et fort précieux en
délibérations. Middleton, qui avait seul toute la confiance, y prévalut.
Ils perdirent le temps d'entrer dans la rivière. Les Anglais étaient si
proches qu'il n'y avait pas moyen de prendre le tour pour entrer, et
d'éviter le combat, ou en entrant, ou dans la rivière même, tout au plus
d'être suivis d'assez près pour être brûlés au débarquement. On résolut
donc de dépasser la rivière d'Édimbourg, de longer la côte, et de gagner
le port d'Inverness à quinze ou vingt lieues plus loin. Mais Middleton
cria si haut que le roi d'Angleterre n'était attendu qu'à Édimbourg, et
qu'ils ne trouveraient aucune disposition ailleurs, et le chevalier de
Forbin le seconda si puissamment, et d'une manière si équivoque que,
malgré le duc de Perth, malgré les deux Hamilton, malgré tous les
officiers principaux du vaisseau, et sans y en appeler des autres
navires, il fut décidé qu'on reprendrait la route de France. Ils ne
longèrent donc presque point la côte, et revirèrent.

Dans ce mouvement, la flotte ennemie, forçant de voiles, joignit, par
son avant-garde, l'arrière-garde de l'escadre, avec qui elle engagea un
combat fort opiniâtre. Le chevalier de Tourouvre s'y distingua beaucoup
et, avec son vaisseau couvrit toujours celui du roi d'Angleterre, du
salut duquel il fut uniquement cause. Les Anglais prirent deux vaisseaux
de guerre et quelques bâtiments. Sur l'un de ces deux vaisseaux étaient
le marquis de Lévi, le lord Greffin et les deux fils de Middleton, qui,
tous, après divers mauvais traitements, furent conduits à Londres.
Greffin, condamné promptement à mort, insulta ses juges, demeura ferme à
ne répondre jamais un mot qui pût intéresser personne, méprisa la mort,
et fit tant de honte à ses juges qu'ils suspendirent l'exécution. La
reine lui envoya un répit, puis un autre, sans que jamais il en
demandât, et finalement il demeura libre dans Londres sur sa parole. Il
eut toujours de nouveaux répits, et bien reçu partout, vécut là comme
dans sa patrie\,; averti enfin que {[}les répits{]} ne cesseraient
point, il y vécut ainsi plusieurs années, déjà fort vieux, et il y
mourut de sa mort naturelle. Les deux fils de Middleton ne furent ni
arrêtés, ni poursuivis, mais partout fort accueillis. M. de Lévi fut
envoyé à Nottingham tenir compagnie au maréchal de Tallard et aux autres
prisonniers\,; le reste de ceux de ce vaisseau fut renvoyé en France sur
leur parole. Le parti pris de revirer de bord sur Dunkerque, dans le
vaisseau du roi d'Angleterre, ce prince ouvrit le paquet que Chamillart
lui avait remis cacheté. Il en savait le contenu, et très apparemment
Gacé aussi. Il lui remit sa patente et le déclara maréchal de France. Il
était difficile de l'être à meilleur marché. Il prit sur-le-champ le nom
de maréchal de Matignon, en mémoire de son bisaïeul qui a fait l'honneur
de leur maison. Lévi fut en même temps déclaré lieutenant général\,;
c'était pour cela que son beau-père l'avait fait embarquer.

Ce fut la première fois que le roi d'Angleterre prit, pour être
incognito, le nom de chevalier de Saint-Georges, et que ses ennemis lui
donnèrent celui de Prétendant, qui lui sont enfin demeurés tous deux. Il
montra beaucoup de volonté et de fermeté, qu'il gâta par une docilité
qui fut le fruit d'une mauvaise éducation, austère et resserrée, que la
dévotion mal entendue en partie, en partie le désir de le maintenir dans
la crainte et la dépendance, lui fit donner par la reine, sa mère, qui
voulut toujours dominer avec toute sa Sainteté. Il écrivit de Dunkerque
pour demeurer en quelque ville voisine, en attendant l'ouverture de la
campagne qu'il demanda à faire en Flandre. Cette dernière partie fut
accordée, mais on le fit revenir à Saint-Germain. Hough le précéda avec
les journaux du voyage et celui de Forbin, à qui le roi donna mille écus
de pension et dix mille de gratification, que lui valut Pontchartrain
qu'il avait si bien servi à sa mode. Hough avait été fait pair d'Irlande
avant partir.

Le roi d'Angleterre arriva à Saint-Germain le vendredi 20 avril, et vint
avec la reine le dimanche suivant à Marly, où le roi était. Je fus
curieux de l'entrevue. Il faisait fort beau. Le roi, suivi de tout le
monde, sortit au-devant. Comme il allait descendre les degrés de la
terrasse, et que nous voyions la cour de Saint-Germain au bout de cette
allée de la Perspective, qui s'avançait lentement, Middleton seul
s'approcha du roi d'un air fort remarquable, et lui embrassa la cuisse.
Le roi le reçut gracieusement, lui parla à trois ou quatre reprises, le
regardant à chaque fois fixement, à en embarrasser un autre, puis
s'avança dans l'allée. En approchant les uns des autres, ils se
saluèrent, puis les deux rois se détachèrent en même temps, chacun, de
sa cour, doublèrent un peu le pas assez également l'un et l'autre, et
avec la même égalité s'embrassèrent étroitement plusieurs fois. La
douleur était peinte sur les visages de tous ces pauvres gens. Le duc de
Perth fit après sa révérence au roi, qui le reçut honnêtement, mais
seulement comme un grand seigneur. On s'avança après vers le château
avec quelques mots indifférents qui mouraient sur les lèvres. La reine
avec les deux rois, entrèrent chez M\textsuperscript{me} de Maintenon,
la princesse demeura dans, le salon avec M\textsuperscript{me} la
duchesse de Bourgogne et toute la cour. M. le prince de Conti, saisi de
sa curiosité naturelle, s'empara de Middleton\,; le duc de Perth prit le
duc de Beauvilliers et Torcy. Le peu d'autres Anglais, plus accueillis
que d'ordinaire pour les faire causer, se dispersèrent parmi les
courtisans, qui ne tirèrent rien de leur réserve qu'une ignorance
affectée qui disait beaucoup, et des plaintes générales du sort et des
contretemps. Les deux rois furent longtemps tête à tête, pendant que
M\textsuperscript{me} de Maintenon entretenait la reine. Ils sortirent
au bout d'une heure\,; une courte et triste promenade suivit, qui
termina la visite.

Middleton fut violemment soupçonné d'avoir bien averti lés Anglais. Ils
ne firent pas, semblant de se douter de rien, mais ils prirent sans
bruit toutes leurs précautions, cachèrent leurs forces navales, firent
semblant d'en envoyer la plus grande partie escorter un convoi en
Portugal, tinrent prêtes le peu de troupes qu'ils avaient en Angleterre,
qu'ils firent approcher de l'Écosse où ils envoyèrent des gens affidés
en attendant mieux\,; et la reine, sous divers prétextes de confiance et
d'amitié, retint à Londres le duc d'Hamilton, le plus accrédité seigneur
d'Écosse, sur le point, d'y retourner, et qui était l'âme et le chef de
toute cette affaire. Elle n'eh donna part à son parlement que
lorsqu'elle fut devenue publique\,; et après qu'elle fut avortée, elle
ne voulut rechercher personne, et elle évita sagement de jeter l'Écosse
dans le désespoir. Toute cette conduite augmenta fort son autorité chez
elle, lui attacha les coeurs, et ôta toute envie de remuer davantage par
n'avoir plus d'espérance de succès. Ainsi avorta un projet si bien et si
secrètement conduit jusqu'à l'exécution, qui fut pitoyable, et avec ce
projet celui de la révolte des Pays-Bas, auquel il ne fut plus permis de
penser.

Les alliés firent sonner bien haut cette tentative d'une puissance qu'on
avait lieu de croire aux abois, qui ne le dissimulait pas même pour les
mieux tromper, et qui, ne cessant de faire des démarches humiliantes
pour obtenir la paix, par des émissaires obscurs qu'elle envoyait de
tous côtés avec des propositions spécieuses, ne songeait à rien moins
qu'à envahir la Grande-Bretagne, et par contrecoup à pousser ses
conquêtes partout. L'effet en fut grand pour resserrer et irriter de
plus en plus cette formidable alliance. Heinsius, pensionnaire de
Hollande, le plus accrédité qu'aucun autre dans cette grande place ne
l'avait été dans sa république, avait hérité de tout l'esprit, de toutes
les vues et de toute la haine du prince d'Orange. On verra ailleurs que
le prince Eugène, Marlborough et lui n'étaient qu'un, et que ce
formidable triumvirat menait tout. Les deux généraux étaient déjà en
conférence avec le Pensionnaire à la Haye. Le prince Eugène avait refusé
d'aller en Espagne, ce que l'archiduc ne lui pardonna jamais, et
l'accusa toujours d'avoir empêché la cour de Vienne de le secourir
autant et aussi à temps qu'il aurait fallu pour assurer ses succès.
Staremberg alla commander, l'armée d'Espagne. J'ai voulu raconter de
suite toute cette expédition manquée d'Écosse\,; retournons maintenant
un peu en arrière.

\hypertarget{chapitre-x.}{%
\chapter{CHAPITRE X.}\label{chapitre-x.}}

1708

~

{\textsc{Mariage de Béthune et d'une soeur du duc d'Harcourt\,; de
Fervaques et de M\textsuperscript{lle} de Bellefonds\,; de Gassion et
d'une fille d'Armenonville\,; de Monasterol et de la veuve de La
Chétardie.}} {\textsc{- Le chancelier de Pontchartrain refuse un riche
legs de Thevenin.}} {\textsc{- Mort et substitution du vieux marquis de
Mailly.}} {\textsc{- Mort de la duchesse d'Uzès.}} {\textsc{- Retraite,
caractère et traits de Brissac, major des gardes du corps.}} {\textsc{-
Cardinal de Bouillon perd un procès devant le roi contre les réformés de
Cluni.}} {\textsc{- Mariage et grandesse de M. de Nevers
d'aujourd'hui.}} {\textsc{- Extraction et caractère de Jarzé, qui
succède à Puysieux en Suisse.}} {\textsc{- Tentative d'un capitaine de
vaisseau, qui avait pris le nom et les armes de Rouvroy, d'être reconnu
de ma maison.}} {\textsc{- M\textsuperscript{me} la duchesse de
Bourgogne blessée.}} {\textsc{- Mot étrange du roi.}} {\textsc{-
Anecdote oubliée sur l'abbé de Polignac, depuis cardinal.}} {\textsc{-
Voyage de Chamillart vers l'électeur de Bavière en Flandres.}}
{\textsc{- Mgr le duc de Bourgogne secrètement destiné à l'armée de
Flandre, et le duc de Vendôme sous lui}}

~

Il se fit plusieurs mariages\,: Béthune, neveu de la reine de Pologne,
qui n'a voit presque rien vaillant, plus touché de l'alliance que du
bien, épousa une sœur du duc d'Harcourt, qui n'eut que quatre-vingt
mille livres. C'est dommage que le bout du projet de ces Mémoires
n'atteigne pas le temps de la mort du dernier prince de la maison
d'Autriche\footnote{Le dernier empereur de la maison de
  Habsbourg-Autriche fut Charles VI, qui mourut le 20 octobre 1740.}. On
verra dans ce mariage si indifférent en apparence, et si, fort ignoré
des puissances de l'Europe, le germe dont la Providence avait destiné la
faiblesse à les remuer toutes, à anéantir cette fameuse pragmatique qui
avait enrôlé toute l'Europe pour son soutien, et à mettre sur la tête
d'un prince de Bavière, qui n'était pas prêt à nuire, le diadème
impérial, la couronne de Bohème, et partager encore d'autres provinces
avec d'autres provinces aux dépens de l'héritière qui se les croyait
toutes si assurées, avec l'empire pour son époux, et qui avait de si
puissants défenseurs, dont les intérêts avec les siens étaient les
mêmes. À qui considère les événements que racontent les histoires dans
leur origine réelle et première, dans leurs degrés, dans leurs progrès,
il n'y a peut-être aucun livre de piété (après les divins et après le
grand livre toujours ouvert du spectacle de la nature) qui élève tant à
Dieu, qui en nourrisse plus l'admiration continuelle, et qui montre avec
plus d'évidence notre néant et nos ténèbres. Cette réflexion m'échappe à
cette occasion qui aurait la même application sous de bons yeux à une
infinité d'autres, mais non pas avec la même évidence et la même clarté,
pour qui a connu de source le ressort unique de ce grand événement, et
les jeux différents de ce ressort unique.

Fervaques, fils de Bullion, épousa la fille de la marquise de
Bellefonds\,; et Gassion une fille d'Armenonville. Il était petit-fils
du frère aîné du maréchal de Gassion, et sert actuellement de lieutenant
général avec réputation. Monasterol, envoyé de l'électeur de Bavière,
tout à fait dans sa confiance, qui recevait ici ses subsides, gros
joueur, grand dépensier et fort dans les belles compagnies, devint
amoureux de la veuve de La Chétardie, gouverneur de Béfort, frère de ce
curé de Saint-Sulpice, directeur de M\textsuperscript{me} de Maintenon,
duquel elle avait des enfants, dont l'aîné a été ambassadeur en Prusse
où il a fort bien servi, et l'est maintenant à Pétersbourg, où il a eu
part à la révolution qui a mis la tzarine Élisabeth, fille de ce célèbre
czar Pierre Ier, sur le trône. Cette M\textsuperscript{me} de La
Chétardie était faite à peindre et grande, fort belle, sans esprit, mais
très galante et fort décriée, grande dépensière et fort impérieuse\,;
elle subjugua Monasterol qui fit la folie de l'épouser, et qui fut après
bien honteux de le déclarer.

Thevenin, riche partisan, mourut sans enfants. Il devait sa fortune au
chancelier, tandis qu'il était contrôleur général. Il avait une fort
belle maison joignant la sienne, magnifiquement meublée, qu'il lui donna
avec les meubles par son testament. Le chancelier ne voulut point
prendre le legs, quoique le roi lui conseillât de l'accepter. Cette
action de désintéressement fut fort approuvée, d'autant qu'après que le
roi lui en eut parlé il n'en parla plus pendant six semaines, en sorte
qu'on croyait qu'il l'accepterait. Au bout de ce temps il représenta au
roi ses raisons, et fit après sa renonciation.

Le vieux marquis de Mailly mourut à quatre-vingt-dix-huit ans dans la
belle maison qu'il avait bâtie au bout du pont Royal, et laissa plus de
soixante mille écus de rente en fonds de terre. Sa femme, qui avait lors
quatre-vingts ans et qui le survécut encore longtemps, était devenue
héritière de tous les biens de sa maison qui était Montcavrel, par la
mort du fils de son frère, jeune garçon de douze ou quatorze ans, dont
elle prenait soin depuis la mort de son frère et de sa belle-soeur
qu'elle avait plaidés toute sa vie. Ces Montcavrel étaient la branche
aînée de la maison de Monchy, dont était cadet le maréchal
d'Hocquincourt, frère du grand-père de M\textsuperscript{me} de Mailly.
Sa tante paternelle avait épousé le frère aîné de son mari. De ce
mariage une fille mariée à Montcavrel, frère unique de
M\textsuperscript{me} de Mailly. À force de procès et d'épargnes, de
mariés chacun avec fort peu de bien, {[}avec{]} l'héritage de la branche
de Montcavrel, et une très longue vie tout appliquée à former une
opulente maison, ils y parvinrent. Le mariage de leur second fils avec
la parente de M\textsuperscript{me} de Maintenon, qu'elle fit dame
d'atours de M\textsuperscript{me} la duchesse de Bourgogne, leur fit
obtenir en 1701 des lettres patentes dérogeant en leur faveur à tous
édits, déclarations et coutumes, qui autorisèrent la substitution qu'ils
firent du marquisat de Nesle et d'autres terres pour plus de quarante
mille écus de rente en faveur des mâles à perpétuité. À tout ce qui est
arrivé depuis au marquis de Nesle, leur petit-fils, qui leur a
immédiatement succédé, il n'a pas paru que Dieu ait béni ou
l'acquisition de ces biens, ou la vanité d'avoir laissé sans aucune
sorte de portion, même viagère, les filles et les cadets sur cette
substitution.

Le duc d'Uzès perdit aussi sa grand'mère paternelle depuis longtemps
retirée, fort vieille. C'était une femme de grand mérite et de beaucoup
de piété. Elle était d'Apchier, c'est-à-dire de la branche aînée de la
maison de Joyeuse, grande et fort ancienne, dont la diversité du nom et
des armes que portent ses diverses branches les font souvent méconnaître
pour sorties masculinement de la même tige. Le nom de la maison est
Châteauneuf, seigneur de Randon.

Brissac, major des gardes du corps, qui n'était ni ne se prétendait rien
moins que des Cossé, mais un fort simple gentilhomme tout au plus, se
retira dans ce temps-ci de la cour chez lui à la campagne, où il mourut
bientôt après d'ennui et de vieillesse à plus de quatre-vingts ans.
C'était, de figure et d'effet, une manière de sanglier qui faisait
trembler les quatre compagnies des gardes du corps, et compter avec lui
les capitaines, tout grands seigneurs et généraux d'armée qu'ils
fussent. Le roi s'était servi de lui pour mettre ses gardes sur ce grand
pied militaire où ils sont parvenus, et pour tous les détails intérieurs
de dépense, de règle, de service et de discipline\,; et il s'était
acquis toute la confiance du roi par son inexorable exactitude, par la
netteté, de ses mains, par son aptitude singulière en ce genre de
service. Avec tout l'extérieur d'un méchant homme, il n'était rien
moins, mais serviable sans vouloir qu'on le sût, et a souvent paré bien
des choses fâcheuses, mais tout cela avec des manières dures et
désagréables. Il avait de la valeur, mais ses fonctions qui
l'attachaient auprès du roi ne le laissaient jamais sortir de la cour,
où il devint lieutenant général et gouverneur de Guise. Le roi, parlant
un jour de service des majors dans les troupes, qui pour être bons
majors les en faisait haïr\,: «\,S'il faut être parfaitement haï pour
être bon major, répondit M. de Duras, qui avait le bâton derrière le
roi, voilà, sire, le meilleur qui soit en France,\,» tirant Brissac par
le bras qui en fut confondu\,; et le roi à rire, qui l'eût trouvé fort
mauvais de tout autre, mais M. de Duras s'était mis sur un tel pied de
liberté qu'il ne se contraignait sur rien ni sur personne devant le roi,
ce qui le faisait fort redouter, et il en disait souvent de fort salées.
Ce major avait une santé très robuste, et se moquait toujours des
médecins, et très souvent de Fagon en face devant le roi, que personne
autre n'eût osé attaquer. Fagon payait de mépris, souvent de colère, et
avec tout son esprit en était embarrassé. Ces courtes scènes étaient
quelquefois très plaisantes.

Brissac, peu d'années avant sa retraite, fit un étrange tour aux dames.
C'était un homme droit qui ne pouvait souffrir le faux. Il voyait avec
impatience toutes les tribunes bordées de dames l'hiver au salut les
jeudis et les dimanches où le roi ne manquait guère d'assister, et
presque aucune ne s'y trouvait quand on savait de bonne heure qu'il n'y
viendrait pas\,; et sous prétexte de lire dans leurs heures, elles
avaient toutes de petites bougies devant elles pour les faire connaître
et remarquer. Un soir que le roi devait aller au salut, et qu'on faisait
à la chapelle la prière de tous les soirs qui était suivie du salut,
quand il y en avait, tous les gardes postés et toutes les dames placées,
arrive le major vers la fin de la prière, qui, paraissant à la tribune
vide du roi, lève son bâton et crie tout haut\,: «\,Gardes du roi,
retirez-vous, rentrez dans vos salles\,; le roi ne viendra pas.\,»
Aussitôt les gardes obéissent, murmures tout bas entre les femmes, les
petites bougies s'éteignent, et les voilà toutes parties excepté la
duchesse de Guiche, M\textsuperscript{me} de Dangeau et une ou deux
autres qui demeurèrent. Brissac avait posté des brigadiers aux débouchés
de la chapelle pour arrêter les gardes, qui leur firent reprendre leurs
postes, sitôt que les dames furent assez loin pour ne pouvoir pas s'en
douter. Là-dessus arrive le roi qui, bien étonné de ne point voir de
dames remplir les tribunes, demanda par quelle aventuré il n'y avait
personne. Au sortir du salut, Brissac lui conta ce qu'il avait fait, non
sans s'espacer sur la piété des dames de la cour. Le roi en rit
beaucoup, et tout ce qui l'accompagnait. L'histoire s'en répandit
incontinent après\,; toutes ces femmes auraient voulu l'étrangler.

Le cardinal de. Bouillon, dans son exil vide d'occupations meilleures,
travaillait à s'assujettir les moines réformés de la congrégation de
Cluni. Comme cardinal et abbé général il avait assujetti les non
réformés, parce que les cardinaux ont usurpé tous les droits d'abbés
réguliers, et par cette raison il les voulait étendre sur les réformés.
Ceux-ci disaient que cet abus des cardinaux ne se pouvait tolérer qu'à
l'égard de moines qui n'avaient point d'autre supérieur général, mais
que pour eux, qui dépendaient du général particulier de leur réforme, et
du régime de leur congrégation, ils n'avaient que des honneurs et des
respects à rendre au cardinal de Bouillon, dont l'autorité
bouleverserait tout chez eux, et n'y avait jamais été reconnue depuis
qu'ils étaient réformés et rassemblés en congrégation subsistante. Cela
fit un procès au grand conseil où les causes de l'ordre de Cluni sont
commises, qui fut soutenu de part et d'autre avec grande chaleur. Le
cardinal le perdit en entier, et entra en furie. Sa famille renouvela
les clameurs qu'on a vu ailleurs qu'ils firent sur la manière dont fut
dressé l'arrêt de la coadjutorerie de Cluni pour l'abbé d'Auvergne\,;
les plaintes furent portées au roi qui fut pressé, de manière que,
contre toute règle, il voulut bien que l'affaire fût portée devant lui
pour y être jugée de nouveau. Elle fut examinée par un bureau de trois
conseillers d'État, devant qui elle fut rapportée par un maître des
requêtes, et tous quatre vinrent un samedi après dîner chez le roi, où
le conseil de finances se trouva, pour avoir des magistrats. Le cardinal
de Bouillon n'eut que trois voix pour lui. L'affaire dura quatre heures,
et l'arrêt du grand conseil confirmé en tous ses points. Il est
difficile d'exprimer la rage qu'il en conçut lorsqu'il apprit cette
nouvelle, qui lui tourna tellement la tête qu'elle eut une part
principale à ce qu'il exécuta depuis.

M. de Donzi, hors d'espérance d'être duc, avait cherché à y suppléer par
un mariage. Il le trouva dans la fille aînée de J. B. Spinola,
gouverneur d'Ath et lieutenant général des armées de Charles II, roi
d'Espagne, qui en 1677 le fit faire prince de l'empire, et le fit enfin
grand d'Espagne, de la première classe pour un gros argent qu'il paya.
Il n'eut point de fils, il n'eut que deux filles dont l'aînée eut sa
grandesse après lui, et que Donzi épousa, et prit d'elle, en se mariant,
le nom de prince de Vergagne. Il fallait craindre, à la vie qu'il
menait, de se méprendre et de dire \emph{Vergogne}. L'autre fille épousa
le frère de Seignelay. Ni l'une ni l'autre ne furent heureuses. Le
prince de Chimay, beau-frère alors de Vergagne, fut fait en ce temps-ci
grand aussi de première classe.

Puysieux, lieutenant général, gouverneur d'Huningue, à qui l'ambassade
de Suisse avait valu l'ordre, comme on l'a vu, et une des trois places
de conseiller d'État d'épée, se lassa d'un emploi qui ne pouvait plus le
conduire à rien, et où il s'ennuyait malgré l'estime, l'affection, la
considération qu'il s'y était universellement acquises. On chercha qui y
envoyer, et on trouva peu de gens qui s'y offrissent. Il fallait la
singularité de l'éducation de Puysieux avec le roi, celle de sa
grand'mère, l'alliance de sa mère, pour en tirer avec tout son esprit
tout le parti qu'il en tira. Faute de mieux, Jarzé fut nommé à la
surprise de tout le monde. C'était un gentilhomme d'Anjou fort riche et
fort avare, avec de l'esprit, de la lecture et quelques amis, mais fort
peu répandu, et tout appliqué à ses affaires et à amasser quoique sans
enfants. Il avait perdu un bras il y avait plus de trente ans à la
guerre, et n'avait pas servi depuis, ni presque vu la cour. Apparemment
qu'il s'ennuya, et qu'il voulut enfin tenter quelque fortune. Il n'était
connu que par son père, qui est ce Jarzé qui, par l'aventure des
capitaines des gardes aux Feuillants\,; fut un moment capitaine des
gardes du corps à la place du vieux Charost, à qui la charge fut rendue
tôt après. Cette aventure entre autres est très bien détaillée dans les
Mémoires de M\textsuperscript{me} de Motteville, et celle encore des
folles amours du même Jarzé pour la reine mère, qui le
chassa\footnote{Voy. les notes à la fin du volume.}, et dont il perdit
sa fortune.

La promotion des deux lieutenants généraux des armées navales en fit
faire une autre en descendant quelque temps après, dont Rouvroy ne fut
pas content. C'était un capitaine de vaisseau bon officier et brave
homme, qui serait vice-amiral il y a longtemps, si son humeur
incompatible, ses folles hauteurs, et son audace à piller partout ne
l'avaient fait honnêtement chasser près de toucher au but. Je dis
honnêtement, mais toutefois, malgré ses plaintes et ses cris, sans
aucune récompense. C'était un homme dont le père ou le grand-père obscur
avait apparemment trouvé le nom et les armes de Rouvroy meilleures à
prendre dans le choix qu'il s'en proposait, puisqu'il les prit sans en
être. Le peu qu'ils étaient le fit longtemps ignorer. Ce Rouvroy-ci
avait deux soeurs. La beauté de l'une a fait longtemps du bruit. Elle
avait été fille d'honneur de Madame, et Saint-Vallier, capitaine de la
porte du roi alors, l'épousa. L'autre suppléa par l'intrigue à la
beauté. Elle fut aussi fille d'honneur de Madame\,; elle épousa un riche
gentilhomme d'auprès de Cambrai qui avait la terre d'Oisy, dont il
portait le nom\,; et toutes deux ont eu des enfants. Elles s'étaient
données à Monsieur et à Madame pour être de même maison que nous. Leur
frère se maria mal à leur gré\,; elles firent ce qu'elles purent pour
l'en empêcher. Ne sachant plus qu'y faire, elles s'avisèrent de venir
trouver mon père, dans l'espérance qu'il ne les désavouerait pas en
face, et qu'elles en tireraient protection pour empêcher ce mariage tout
près de se célébrer. Elles lui dirent qu'elles avaient recours à lui
pour se plaindre de leur frère, et pour lui demander s'il souffrirait
qu'un homme qui avait l'honneur d'être de sa maison se mariât de la
sorte.

Mon père, qui n'avait jamais eu aucun commerce avec pas un d'eux, et qui
était vif, prit feu, leur répondit tout net qu'il ne reconnaissait ni
lui ni elles\,; que jamais il n'avait ouï parler de cette parenté\,;
qu'il les défiait de la prouver\,; et que partant il ne se mêlerait
point de leurs affaires. Il ajouta que c'était bien assez qu'il ne dit
mot au nom de Rouvroy et à la croix de ses armes qu'ils portaient, sans
lui venir parler impudemment d'une fausse parenté. Une abondance de
larmes fut toute leur réponse, et elles s'en allèrent interdites,
confuses, et enragées de l'affront qu'elles se venaient d'attirer. La
scène se passa dans la chambre de ma mère, qui ne dit mot\,; j'y étais,
et cela me frappa tellement, que je m'en souviens comme d'hier,
maintenant que je l'écris. M\textsuperscript{me} de Saint-Vallier était
lors mariée, dans la force de sa beauté, fort du grand monde, fort
galantisée, et elle avait tout l'esprit et le tour à profiter de tant
d'avantages. Sa soeur était fille de Madame. Elles s'allèrent plaindre à
Monsieur, qui se trouva à Paris, et firent grand bruit de leur aventure,
que mou père méprisa parfaitement. Monsieur l'envoya prier de passer au
Palais-Royal. Il y raconta à lui et à Madame le fait, et ce qui s'était
passé entre lui et ces femmes, de manière que l'un et l'autre en
demeurèrent satisfaits, et leur conseillèrent de se taire dès qu'elles
n'avaient point de preuves à montrer. Cela finit tout court de la sorte,
et leur frère se maria.

Ce serait ici le lieu d'expliquer mon nom et mes armes, et comment avec
un nom que je ne porte point et la moitié des armes que j'écartèle,
c'était prétendre en effet être de ma maison\,; la parenthèse en serait
trop longue\,: elle se trouvera mieux placée parmi les Pièces, pour ne
pas interrompre le fil de la narration. Bien des années se passèrent
sans plus en entendre parler. La personne que Rouvroy avait épousée
était fille de la sous-gouvernante des filles de Monsieur, et de feu
Madame sa première femme. Elle se trouva une personne d'esprit, de
vertu, de douceur, et d'un véritable mérite, extrêmement bien avec
M\textsuperscript{me} la princesse de Conti, et ne bougeant de chez
elle, sur un pied d'amitié, d'estime et de confiance, et tout aussi
aimée et comptée de M\textsuperscript{lle} de Lislebonne, de
M\textsuperscript{me} d'Espinoy et de M\textsuperscript{me} d'Urfé, et
très bien avec M\textsuperscript{me}s de Villequier, puis d'Aumont, et
de Châtillon, sa soeur. Monseigneur même, qui, dans ces temps-là, ne
bougeait de chez M\textsuperscript{me} la princesse de Conti, prit de la
bonté pour elle, et elle fut toujours de tout avec eux. À la fin le mari
ou la femme s'ennuyèrent d'un état agréable à Versailles et à
Fontainebleau, mais non à la cour. Pour en être, c'est-à-dire, des fêtes
et des voyages de Marly, il fallait pouvoir être admise à table et
dans-les carrosses, comme les femmes de qualité\,; c'est ce qui manquait
à l'agrément solide de sa vie, et c'est ce qui eût été de plain pied son
mari étant de ma maison. Il se mit donc à me faire sa cour dans les
galeries, puis à venir quelquefois chez moi les mâtins, en homme qui me
faisait sa cour comme à un ami de M. de Pontchartrain, pour son
avancement dans la marine. Je le recevais civilement\,; je lui fis même
plaisir utilement, et autant que je le pus, néanmoins toujours attentif
à ses propos et à ses démarches, dans le souvenir très présent de ce qui
s'était passé de ses soeurs avec mon père. Cette conduite dura ainsi
quelques années sans aucune mention que d'avancement, et moi toujours
poli et serviable, mais toutefois en garde de l'attirer chez moi.

Enfin, cette année, sur la fin du carême, piqué de la promotion de
marine dont j'ai parlé, il me vint faire ses plaintes avec vivacité,
s'applaudit d'avoir tiré son fils de la marine pour le mettre dans le
régiment des gardes, et ajouta que, par tout ce qui lui en revenait du
duc de Guiche et de tous les officiers, il espérait qu'il ne me ferait
pas déshonneur, ni au nom qu'il portait. J'entendis ce français. Nous
descendions le degré, moi pour aller dîner à Paris, et lui
m'accompagnant. Pour toute réponse, je lui demandai s'il n'y voulait
rien mander, et me séparai de lui à la galerie, qui me parut fort
embarrassé. Avant de monter en carrosse, j'allai chez
M\textsuperscript{me} d'Urfé, à qui je contai ce qui venait de
m'arriver, l'aventure de mon père, et la priai de vouloir bien dire à
Rouvroy et à sa femme que, tant que les politesses n'avaient été que
douteuses, je les avais reçues avec la civilité qu'ils pouvaient
désirer, mais qu'au propos qui me, venait d'être tenu, je ne pouvais
dissimuler que je ne connaissais nulle parenté avec eux\,; que je n'en
avais jamais ouï parler autrement à mon père et aux trois autres
branches de notre maison, dont je ne suis que la quatrième\,; que je
croyais Rouvroy tout aussi bon qu'il le pouvait souhaiter, mais
nullement de ma maison\,; que ces choses-là consistaient en preuves, que
je serais ravi qu'il m'en montrât qui me le fissent reconnaître, mais
que jusque-là je n'en ferais rien, et que lui-même, s'il n'en avait
point, aurait mauvaise grâce de le vouloir prétendre, et le prétendrait
inutilement. J'ajoutai que je la priais d'en rendre compte à
M\textsuperscript{me} la princesse de Conti, et de lui dire que, sans
l'amitié qu'elle avait pour sa femme, je n'aurais pas entendu le propos
de parenté si patiemment, et qu'il se devait contenter de ce que je lui
laissais faire ce que bon lui semblait sur le nom et les armes qu'il
prenait, sans vouloir encore être reconnu pour être ce qu'il n'était
pas, et ce qu'il ne pouvait prouver qu'il fût, puisqu'il n'avait pas
encore tenté de le faire.

Revenu à Versailles, je trouvai le duc d'Aumont sortant de chez le
chancelier comme j'y entrais. Il m'arrêta dans l'antichambre, et me fit
un grand préambule du désespoir de Rouvroy, et qu'il n'était pas permis
d'attaquer les gens sur leur naissance, et du bruit que cela faisait. Je
me mis à rire et à lui dire que j'attaquais si peu cet homme sur sa
naissance, que je ne m'étais pas seulement donné la peine de savoir qui
il était et de quel droit il prenait le nom et les armes qu'il
portait\,; mais de penser qu'à force de bruit, de plaintes et de
langages, il me ferait ou l'avouer, ou consentir tacitement qu'on le
crût de ma maison, il pouvait être bien persuadé que je n'en ferais
rien. M. d'Aumont me répondit que ces sortes d'affaires étaient toujours
délicates et désagréables\,; que c'était par amitié et par intérêt pour
moi qu'il me parlait\,; qu'il ne fallait pas avoir toujours tant de
délicatesse sur les parentés\,; que Rouvroy était enragé et résolu de
porter\,; ses plaintes au roi. Je répondis encore avec le même
sang-froid que, si Rouvroy était assez fou pour se plaindre au roi de ce
que je ne le voulais pas reconnaître, j'aurais l'honneur de lui en dire
les raisons, qu'il goûterait, je croyais, autant que celles de
Rouvroy\,; qu'en un mot, ce n'était point là une affaire de crierie,
mais de preuves, à quoi je reviendrais toujours\,; que tout ce bruit ne
m'émouvrait pas le moins du monde, mais que je me persuadais qu'il
nuirait fort à qui y avait recours, faute de preuves si aisées à
montrer, s'il en avait, et si ridicules à prétendre, s'il n'en avait
pas. Je laissai ainsi M. d'Aumont peu content de la commission qu'il
avait apparemment prise par amitié pour lime de Rouvroy, et de l'effet
de son éloquence. Je ne laissai pas de prendre mes précautions du côté
de Monseigneur et du roi, après quoi je me mis peu en peine des
clabauderies que je ne payai que de mépris.

Je sus que Rouvroy avait été à nos autres branches, dont il ne fut pas
plus content que de moi. Il fut à divers généalogistes qui ne le
satisfirent pas mieux, Clérembault entre autres qui l'assura qu'il ne
trouverait jamais ombre de la moindre preuve, ni même de remonter bien
haut. À ma grande surprise, M\textsuperscript{lle} de Lislebonne et
M\textsuperscript{me} d'Espinoy lui conseillèrent de se taire, par le
tort irréparable que lui faisait une prétention rejetée qu'il ne pouvait
prouver. Sa femme pleurait sans cesse une folie qu'elle faisait tout ce
qu'elle pouvait pour arrêter. Enfin las de crier et d'aboyer à la lune,
sans toutefois qu'il lui échappât que des plaintes et des cris, dont
rien ne pouvait me blesser, il prit le parti de se taire, et je n'en ai
pas ouï parler depuis.

Je n'ai pas cru devoir omettre cette aventure, pour ne pas laisser dans
l'erreur ceux que le nom et les armes que ces gens-là ont pris y
pourraient induire. Je l'ai déjà dit à propos de Maupertuis et de la
maison de Melun, on fait en France tout ce que l'on veut là-dessus,
nulle voie de l'empêcher, nulle justice à entendre. Un garde-marine qui
n'était point Rochechouart en prit le nom et les armes. Il trouva M. de
Vivonne prêt à s'embarquer pour la révolte de Sicile\,; il le sut, et ne
le pouvant empêcher, il l'appela devant tout le monde, et le remercia de
la bonne opinion qu'il avait de sa maison, dont il ne pouvait donner une
plus sûre marque que de l'avoir préférée à tant d'autres pour en choisir
pour soi le nom et les armes. Venons maintenant à quelque chose de plus
intéressant.

M\textsuperscript{me} la duchesse de Bourgogne était grosse\,; elle
était fort incommodée. Le roi voulait aller à Fontainebleau contre sa
coutume, dès le commencement de la belle saison, et l'avait déclaré. Il
voulait ses voyages de Marly en attendant. Sa petite-fille l'amusait
fort, il ne pouvait se passer d'elle, et tant de mouvements ne
s'accommodaient pas avec son état. M\textsuperscript{me} de Maintenon en
était inquiète, Fagon en glissait doucement son avis. Cela importunait
le roi, accoutumé à ne se contraindre pour rien, et gâté pour avoir vu
voyager ses maîtresses grosses, ou à peine relevées de couches, et
toujours alors en grand habit. Les représentations sur les Marlys le
chicanèrent sans les pouvoir rompre. Il différa seulement à deux
reprises celui du lendemain de la Quasimodo, et n'y alla que le mercredi
de la semaine suivante, malgré tout ce qu'on put dire et faire pour l'en
empêcher, ou pour obtenir que la princesse demeurât à Versailles.

Le samedi suivant, le roi se promenant après sa messe, et s'amusant au
bassin des carpes entre le château et la Perspective, nous vîmes venir à
pied là duchesse du Lude toute seule, sans qu'il y eût aucune dame avec
le roi, ce qui arrivait rarement le matin. Il comprit qu'elle avait
quelque chose de pressé à lui dire, il fut au-devant d'elle, et quand il
en fut à peu de distance, on s'arrêta, et on le laissa seul la joindre.
Le tête-à-tête ne fut pas long. Elle s'en retourna, et le roi revint
vers nous, et jusque près des carpes sans mot dire. Chacun vit bien de
quoi il était question, et personne ne se pressait de parler. À la fin
le roi, arrivant tout auprès du bassin, regarda ce qui était là de plus
principal, et sans adresser la parole à personne, dit d'un air de dépit
ces seules paroles\,: «\,La duchesse de Bourgogne est blessée. » Voilà
M. de La Rochefoucauld à s'exclamer, M. de Bouillon, le duc de Tresmes
et le maréchal de Boufflers à répéter à basse note, puis M. de La
Rochefoucauld à se récrier plus fort que c'était le plus grand malheur
du monde, et que s'étant déjà blessée d'autres fois, elle n'en aurait
peut-être plus. «\,Eh\,! quand cela serait, interrompit le roi tout d'un
coup avec colère, qui jusque-là n'avait dit mot, qu'est-ce que cela me
ferait\,? Est-ce qu'elle n'a pas déjà un fils\,? et quand il mourrait,
est-ce que le duc de Berry n'est pas en âge de se marier et d'en
avoir\,? et que m'importe qui me succède des uns ou des autres\,? Ne
sont-ce pas également mes petits-fils\,?» Et tout de suite avec
impétuosité\,: «\,Dieu merci, elle est blessée, puisqu'elle avait à
l'être, et je ne serai plus contrarié dans mes voyages et dans tout ce
que j'ai envie, de faire par les représentations des médecins et les
raisonnements des matrones. J'irai et viendrai à ma fantaisie et on me
laissera en repos.\,» Un silence à entendre une fourmi marcher succéda à
cette espèce de sortie. On baissait les yeux, à peine osait-on respirer.
Chacun demeura stupéfait. Jusqu'aux gens de bâtiments et aux jardiniers
demeurèrent immobiles. Ce silence dura plus d'un quart d'heure.

Le roi le rompit, appuyé sur la balustrade, pour parler d'une carpe.
Personne ne répondit. Il adressa après la parole sur ces carpes à des
gens des bâtiments qui ne soutinrent pas la conversation à
l'ordinaire\,; il ne fut question que de carpes avec eux. Tout fut
languissant, et le roi s'en alla quelque temps après. Dès que nous
osâmes nous regarder hors de sa vue, nos yeux se rencontrant se dirent
tout. Tout ce qui se trouva là de gens furent pour ce moment les
confidents les uns des autres. On admira, on s'étonna, on s'affligea, on
haussa les épaules. Quelque éloignée que soit maintenant cette scène,
elle m'est toujours également présente. M. de La Rochefoucauld était en
furie, et pour cette fois n'avait pas tort. Le premier écuyer en pâmait
d'effroi\,; j'examinais, moi, tous les personnages, des yeux et des
oreilles, et je me sus gré d'avoir jugé depuis longtemps que le roi
n'aimait et ne comptait que lui, et était à soi-même sa fin dernière.
Cet étrange propos retentit bien loin au delà de Marly.

Avant d'aller plus loin, j'ai besoin de retourner un moment sur mes pas
pour ne pas oublier une anecdote qui aurait dû être écrite dès la fin de
1705 ou le commencement de 1706 tout au plus tard\footnote{Passage omis
  par les précédents éditeurs depuis *Avant d'aller plus**loin jusqu'à
  le hasard* (p.~218). L'anecdote racontée par Saint-Simon se trouve
  déjà plus haut, t. V, p.~96 et suiv., mais on n'a pas cru devoir
  supprimer les redites de l'auteur dans une édition complète de ses
  Mémoires.}. Cette transposition au moins servira de préliminaire à une
autre plus importante. On se souviendra de ce qui a été dit en son lieu
de l'abbé de Polignac, de sa figure, de son caractère, de son brillant à
la cour depuis, son retour d'exil et de sa dangereuse galanterie. Je le
vis, dès les commencements de ce temps-là, courtiser fort le duc de
Chevreuse, le mettre sur des points de science, laisser des queues aux
questions pour y revenir, enfin s'introduire chez lui\,; {[}ce{]} qui
n'était pas une chose facile. Cette conduite attira mes réflexions. Le
bel air et M. de Chevreuse n'allaient point ensemble, beaucoup moins les
allures de l'abbé de Polignac ni de pas un des gens de la cour avec qui
il s'était particulièrement lié. Je crus voir son dessein\,; je crus
aussi en apercevoir le danger. Je m'y confirmai de plus en plus, et je
pris enfin la résolution de le montrer à celui qu'il regardait de plus
près. Un soir, à Marly, causant avec le duc de Beauvilliers au coin de
son feu tête à tête, je lui témoignai ma surprise de cette liaison si
nouvelle du duc de Chevreuse et de l'abbé de Polignac si peu faits l'un
pour l'autre. M. de Beauvilliers me dit que cela était tout naturel\,;
que tous deux savaient beaucoup, tous deux gens d'esprit\,; qu'à Marly
on était plus rassemblé qu'à Versailles, et qu'on se trouvait plus
souvent chez le roi à différentes heures\,; qu'il était tout naturel que
ce hasard les eût mis aux mains sur quelques questions de belles-lettres
ou de science\,; que je savais comme ils étaient l'un et l'autre\,; que
de question en question ils s'étaient accoutumés et plu à raisonner
ensemble, que cela avait formé la liaison.

Je lui dis que cela était tout simple de la part de M. de Chevreuse,
mais que, du côté de l'abbé de Polignac, je croyais apercevoir du
dessein\,; que ma pensée était qu'il en voulait faire un pont pour
l'aborder lui-même. «\,Eh\,! bien, interrompit le duc, quand cela
serait, où est le mal\,? il est vrai que M. de Chevreuse m'en a parlé\,;
je l'ai vu chez lui, et il l'a amené chez moi. C'est un homme de
qualité, de beaucoup d'esprit et de fort bonne compagnie, avec qui il y
a mille choses agréables à apprendre. --- Eh\,! monsieur, voilà le
point, lui dis-je. Vous le trouvez tel, et cela est vrai. Ce qu'il veut,
c'est de vous-même d'en faire un autre pont pour pénétrer jusqu'à Mgr le
duc de Bourgogne. --- Eh\,! pourquoi\,? répliqua-t-il, ne le lui pas
faire voir, s'il y a de l'instruction et de l'utilité à trouver dans une
conversation agréable pour Mgr le duc de Bourgogne\,? Je ne vois à cela
aucun inconvénient. Et moi, lui dis-je, j'en vois beaucoup, et tel que
vous ne le sentirez que quand il n'en sera plus temps.\,»

Il s'altéra un peu et me pria de lui développer ce qui ne se présentait
pas à lui, avec un petit air de doux défi. «\,Voilà, lui dis-je, votre
charité qui déjà s'effarouche. Mais vous me pardonnerez de vous dire
que, avec une charité si délicate, on ignore tout, et on tombe en
beaucoup d'inconvénients dans une cour. Puisque j'ai commencé à
l'effaroucher, j'irai jusqu'au bout. Tachez, monsieur, de connaître vos
gens. L'abbé de Polignac est une sirène enchanteresse, et qui en fait
métier et profession. C'est un homme faux, ambitieux, qui entreprendra
tout et à qui aucun moyen ne coûtera pour arriver à ses fins. Toute sa
vie jusqu'à présent n'a été que cela. Ses moeurs, ses liaisons, sa
conduite n'ont aucun rapport avec M. de Chevreuse ni avec vous. Il n'a
été à lui que pour arriver à vous\,; il ne veut vous capter que pour
parvenir par vous à Mgr le duc de Bourgogne, qu'il enchantera par son
esprit, par son jargon, par son savoir. Il s'y ancrera par soi-même, et
une fois ancré le voudra dominer pour faire sa fortune, ne pensera
conséquemment qu'à vous écarter pour être seul possesseur\,; et
souvenez-vous, monsieur, que je vous prédis qu'il en viendra à bout, si
vous avez la simplicité de l'introduire.\,»

M. de Beauvilliers se fâcha tout de bon. Il me dit qu'il n'y avait plus
moyen de raisonner avec moi\,; que je soupçonnais tout\,; que je jugeais
mal tout le monde\,; qu'en un mot tout ce qui me passait par la tête je
croyais le voir\,; que rien ne me coûtait, charité, jugements
téméraires, imputations de desseins impossibles\,; que je ne lui
persuaderais pas que l'abbé de Polignac eût ni la pensée, ni la volonté,
ni, quand cela serait, le pouvoir de le débusquer, quelque bien qu'il
réussît auprès du jeune prince, et qu'enfin il me priait de ne lui
parler jamais de l'abbé de Polignac. «\,Vous serez obéi, lui dis-je, et
très ponctuellement, mais à votre dam, monsieur\,; je ne puis m'empêcher
de vous le répéter pour la dernière fois, et de vous prier de vous en
souvenir.\,» De là nous passâmes à d'autres choses. Il eut
contentement\,; je ne lui nommai plus le nom de l'abbé de Polignac\,; je
cessai aussi d'en parler à M. de Chevreuse. On verra que je fus prophète
et que M. de Beauvilliers le reconnut humblement. Il n'avait pu se
dissimuler lors de ce que je vais raconter. Il ne me l'avait pas avoué
encore\,; mais ce qui lui était arrivé de conforme à ce que je lui avais
prédit aurait dû le rendre pour une autre fois plus docile. Il est vrai
que l'excès de l'énormité le trompa. Reprenons maintenant au temps où
nous étions, c'est-à-dire à Marly au sortir de Pâques.

Le hasard apprend souvent parles valets des choses qu'on croit bien
cachées. Il s'en trouva des miens, amis d'un sellier à Paris, qui
travaillait secrètement aux équipages de Mgr le duc de Bourgogne pour la
guerre, et qui eut l'indiscrétion de le leur dire et de les leur
montrer, en leur recommandant fort le secret que lui-même ne gardait
pas. Ils me le contèrent\,: cela m'ouvrit les yeux sur un voyage fort
bizarre que Chamillart était allé faire en Flandre avec Chamlay et
Puységur. Il partit de Versailles le soir même du jour de Pâques, et il
en arriva à Marly le soir du 20 avril, et fut douze jours en ce voyage.
Sa santé très languissante le rendit remarquable, et plus encore le
temps où il partit. On était lors dans la plus grande inquiétude de
l'entreprise d'Écosse, et le roi d'Angleterre arriva à Saint-Germain le
même soir que Chamillart revint à Marly de Flandre. Ce jour était le
vendredi, veille de celui où la duchesse du Lude vint apprendre au roi à
sa promenade que M\textsuperscript{me} la duchesse de Bourgogne était
blessée, et où se passa ce que j'en ai raconté. Elle accoucha le lundi
suivant. Toutes ces époques méritent d'être marquées.

Je fis mes réflexions sur la destination de Mgr le duc de Bourgogne\,;
je ne vis pour lui que le Rhin ou la Flandre, et ce voyage de Chamillart
me décida pour la Flandre. Il y était allé en effet, comme je le sus
depuis, pour disposer l'électeur de Bavière à aller sur le Rhin, pour
laisser à Mgr le duc de Bourgogne l'armée de Flandre dans une
conjoncture où on espérait la révolte des Pays-Bas espagnols, de la
révolution d'Écosse\,; en quoi on faisait la faute de se priver du
secours qu'on se devait promettre de l'affection de ces provinces pour
l'électeur qui les avait si longtemps gouvernées, qui en était adoré, et
qui eût été l'instrument le plus propre à donner vigueur à cette révolte
une fois commencée. Chamillart rencontra Hough en chemin qui lui apprit
les contretemps de la traversée du roi d'Angleterre, et le peu
d'espérance d'aucun succès, dont le ministre fut tellement touché qu'il
en demeura une partie de la nuit sur son lit immobile sans pouvoir se
remuer. Il dépêcha au roi, et continua son voyage, mais avec d'autres
pensées que celles qu'il avait eues jusqu'alors. Mais ce changement de
face des affaires n'en produisit aucun dans la destination des généraux.

\hypertarget{chapitre-xi.}{%
\chapter{CHAPITRE XI.}\label{chapitre-xi.}}

1708

~

{\textsc{L'électeur de Bavière {[}destiné{]} au Rhin, et le duc de
Berwick sous lui\,; Villars au Dauphiné.}} {\textsc{- Conversation
curieuse avec le duc de Beauvilliers sur la destination de Mgr le duc de
Bourgogne.}} {\textsc{- Déclaration des généraux des armées.}}
{\textsc{- Grand prieur en France, avec défense d'approcher de Paris et
de la cour plus que quarante lieues.}} {\textsc{- Maréchal de Matignon
sert sous le duc de Vendôme.}} {\textsc{- Éclat et réflexion sur cette
nouveauté.}} {\textsc{- Vendôme à Clichy.}} {\textsc{- Son étrange
réception à Bergheyck\,; etc., que le roi lui envoie.}} {\textsc{- Le
roi coupe plaisamment la bourse à Samuel Bernard.}}

~

L'électeur eut grand'peine à quitter la Flandre\,: il y était avec
décence dans les restes de son gouvernement, et par là même il y
commandait avec décence l'armée française. Là, il n'agissait directement
que contre la Hollande et l'Angleterre, les Impériaux n'y étaient
qu'auxiliaires. Sur le Rhin il était dépaysé, hors de son gouvernement,
aux mains directement avec l'empereur et l'empire, dans la situation si
personnellement fâcheuse où il se trouvait, qu'il était de son intérêt
de n'aigrir pas, dans la perspective d'une paix tôt où tard à faire.
C'était de général naturel dans son gouvernement devenir général à gages
et mercenaire, allant où on l'envoyait, et avilir sa dignité, que, dans
ses disgrâces, il avait si fort rehaussée. D'autre part, c'était avilir
encore plus celle de l'héritier nécessaire de la couronne, par montrer,
en déplaçant l'électeur, que ce prince ne voudrait pas lui obéir. Après
bien des représentations d'un prince sans ressources, Chamillart eut
recours à l'argent, quelque court qu'il en fût, et l'électeur, faute de
pouvoir mieux, en prit pour sauter le bâton de l'armée du Rhin. Il eut
huit cent mille livres payées comptant de gratification extraordinaire,
outre ses pensions, ses subsides, et tout ce qu'il tirait du roi, encore
se repentit-il d'avoir cédé. Il dépêcha un courrier après Chamillart
pour se rétracter, qui, dans l'embarras où cela le jeta, le lui renvoya
avec promesses d'autres quatre cent mille livres qui firent les huit,
parce qu'il n'en avait donné d'abord que quatre, et cette augmentation
fixa enfin la résolution forcée de l'électeur.

Berwick était de retour et publiquement destiné à l'armée de Dauphiné,
où Tessé commandait dans ces provinces et pressait fort son retour.
Villars était à Strasbourg, méditant le siège de Philippsbourg, si
l'affaire d'Écosse eût réussi, pour favoriser celle des Pays-Bas. On a
vu à quel point il s'était brouillé en Bavière avec l'électeur. Il en
était demeuré en ces termes depuis, nul moyen par conséquent de les
remettre ensemble\,; aussi Chamillart avait eu ordre de lui proposer
Berwick qu'il accepta, et de lui promettre qu'on allait faire revenir
tout présentement Villars, à qui on donnerait l'armée de Dauphiné.
J'explique ces choses un peu à l'avance\,; je les sus bientôt avant leur
déclaration, et je les préviens ici pour ne pas embarrasser le récit que
je vais faire, dans lequel il aurait fallu mettre ces destinations que
j'y sus. Pour le marché d'argent de l'électeur, je ne l'appris qu'après.

Un des premiers soirs que nous fûmes arrivés à Marly, et qu'il faisait
fort beau, M. de Beauvilliers, qui avait envie de causer avec moi, me
mena dans le bas du jardin, vers l'abreuvoir, où tout est à découvert et
où on ne peut être entendu de personne. J'avais résolu de lui parler de
la destination de Mgr le duc de Bourgogne, et ce fut là où je
l'exécutai. Il fut étonné que je le susse, je lui en dis le comment\,;
il me l'avoua et me demanda si je ne trouvais pas cela fort à propos, et
tout de suite m'en fit l'éloge en gros comme de la seule bonne
résolution à prendre. Ce fut alors que j'appris par lui l'objet du
voyage de Chamillart en Flandre, et la disposition des généraux telle
que je l'ai racontée, et là aussi où je lui fis les objections sur
l'électeur de Bavière que j'ai expliquées, sur quoi il me répondit qu'il
avait fallu tout faire céder à la nécessité d'envoyer Mgr le duc de
Bourgogne en Flandre. De là il se mit à enfiler les raisons en détail.
Il me dit que, dans le découragement des affaires, il était important de
les remonter et de donner une nouvelle vigueur aux troupes par la
présence de l'héritier nécessaire\,; qu'il était indécent qu'il languît
dans l'oisiveté à son âge, tandis que sa maison brûlait de toutes
parts\,; que le roi d'Angleterre allait à la guerre\,; qu'il était plus
que temps que M. le duc de Berry la connût, et qu'il ne serait pas
soutenable de l'y envoyer, et en même temps de retenir son frère\,; que
la licence était montée en Flandre, et, par ceux-là mêmes qui la
devaient le plus empêcher\,; à un point qu'il n'y avait plus de remède à
y espérer que de l'autorité de ce prince\,; que cette licence était la
cause principale de tous les malheurs, puisque la discipline et la
vigilance sont l'âme des armées\,; qu'il était infiniment utile de
profiter de tout ce que ce prince avait montré en ses deux uniques
campagnes de goût et de talent pour la guerre, afin de l'y former et de
l'y rendre capable\,; que le Dauphiné et l'Allemagne n'étant pas dignes
de lui par le rien ou le peu qu'il y avait à y faire, il n'y avait que
la Flandre où il pût aller\,; que ces raisons étaient toutes si fortes
qu'elles avaient enfin très sagement déterminé.

J'approuvai fort ce qu'il me dit sur l'oisiveté des princes et l'utilité
de les former à la guerre, mais j'osai contester tout le reste. Je dis
qu'il eût été fort à souhaiter que Mgr le duc de Bourgogne eût continué
de commander les armées, et je m'étendis là-dessus\,; mais je soutins
qu'après une discontinuation de plusieurs campagnes, après tant de
pertes et de malheurs, dans une nécessité de toutes choses, avec des
troupes si accoutumées à se défier de la capacité de leurs généraux, et
qu'à force de mauvaise conduite on avait mises dans l'habitude de ne
plus tenir devant l'ennemi, et de se croire d'avance toujours battues,
un temps de défensive et si triste ne me semblait pas propre pour
remettre Mgr le duc de Bourgogne à la tête d'une armée qui croirait
beaucoup faire que de ne pas reculer et de n'essuyer pas de fâcheuses
aventures, dont les moindres deviendraient avec lui très embarrassantes
et très affligeantes\,; que ce prince s'était accoutumé à un particulier
qui ne convenait point à la vie de l'armée, et duquel il se déferait
malaisément\,; que la raison contraire y ferait briller M. son frère à
son préjudice, chose infiniment dangereuse\,; mais que le pire de tous
les inconvénients était celui de la présence du duc de Vendôme. «\,Eh\,!
c'est précisément pour cela, interrompit le duc de Beauvilliers, que la
présence de Mgr le duc de Bourgogne est nécessaire. Il n'y a que lui
dont l'autorité puisse animer la paresse de M. de Vendôme, émousser son
opiniâtreté, l'obliger à prendre les précautions dont la négligence a
coûté souvent si cher et a pensé si souvent tout perdre. Il n'y a que la
présence de Mgr le duc de Bourgogne qui puisse réveiller la mollesse des
officiers généraux, tenir en crainte l'exactitude de tous, en respect la
licence effrénée du soldat, rétablir l'ordre et la subordination dans
l'armée, que M. de Vendôme a totalement ruinés depuis qu'il commande en
Flandre.\,» Je ne pus m'empêcher de sourire de tant de confiance, ni de
lui répondre avec assurance que rien de tout cela n'arriverait, mais
bien la perte de Mgr le duc de Bourgogne.

Il serait difficile de rendre quel fut l'étonnement du duc à cette
repartie. Je me laissai interrompre, je demandai après d'être patiemment
entendu, et je m'expliquai ensuite à mon aise.

Je lui dis donc que, pour en juger comme je faisais, il n'y avait qu'à
connaître ces deux hommes, et à cette connaissance joindre celle de la
cour, et d'une armée qui deviendrait cour, au moment que Mgr le duc de
Bourgogne y serait arrivé. Que le feu et l'eau n'étaient pas plus
différents, ni plus incompatibles, que l'étaient Mgr le duc de Bourgogne
et M. de Vendôme, l'un dévot, timide, mesuré à l'excès, renfermé,
raisonnant, pesant et compassant toutes choses, vif, néanmoins, et
absolu, mais avec tout son esprit, simple, retenu, considéré, craignant
le mal, et de former des soupçons, se reposant sur le vrai et le bon,
connaissant peu ceux à qui il avait affaire, quelquefois incertain,
ordinairement distrait et trop porté aux minuties\,; l'autre, au
contraire, hardi, audacieux, avantageux, impudent, méprisant tout,
abondant en son sens avec une confiance dont nulle expérience ne l'avait
pu déprendre, incapable de contrainte, de retenue, de respect, surtout
de joug, orgueilleux au comble en toutes les sortes de genres, âcre et
intraitable à la dispute, et hors d'espérance de pouvoir être ramené sur
rien\,; accoutumé à régner, ennemi jusqu'à l'injure de toute espèce de
contradiction, toujours singulier dans ses avis, et fort souvent
étrange, impatient à l'excès de plus grand que lui, d'une débauche
également honteuse et abominable, également continuelle et publique,
dont même il ne se cachait pas par audace\,; ne doutant de rien, fier du
goût du roi si déclaré pour lui et pour sa naissance, et de la puissante
cabale qui l'appuie, fécond en artifices avec beaucoup d'esprit, et
sachant bien à qui il a affaire, tous moyens bons, sans vérité, ni
honneur, ni probité quelconque, avec un front d'airain qui ose tout, qui
entreprend tout, qui soutient tout, à qui l'expérience de l'état où il
s'est élevé par cette voie confirme qu'il peut tout, et que pour lui il
n'est rien qui soit à craindre. Que cette ébauche de portrait de ces
deux hommes était incontestable, et sautait aux yeux de quiconque avait
un peu examiné l'un et l'autre par leur conduite, et par les occasions
qu'ils ont eues de se montrer tels qu'ils sont. Que cela étant ainsi, il
était impossible qu'ils ne se brouillassent, et bientôt\,; que les
affaires n'en souffrissent, que les événements ne se rejetassent de l'un
sur l'autre, que l'armée ne se partialisât\,; que le plus fort ne perdît
le plus faible, et que ce plus fort serait Vendôme, que nul frein, nulle
crainte ne retiendrait, et qui avec sa cabale perdrait le jeune prince,
et le perdrait sans retour. Que le vice incompatible avec la vertu
rendrait la vertu méprisable sur ce théâtre de vices, que l'expérience
accablerait la jeunesse, que la hardiesse dompterait la timidité, que
l'asile de la licence, et l'asile par art, pour se faire adorer, en
rendrait odieux le jeune censeur, que le génie avantageux, audacieux,
saisirait tout, que les artifices soutiendraient tout, que l'armée, si
accoutumée au crédit et au pouvoir de l'un et à l'impuissance de
l'autre, abandonnerait en foule celui dont rien n'était à espérer ni à
craindre, pour s'attacher à celui dont l'audace serait sans bornes, et
dont la crainte avait tenu glacée toute l'encre d'Italie, tandis qu'il y
avait été.

M. de Beauvilliers, qui avec toute sa sagesse et sa patience commençait
à en être à bout, voulut ici prendre la parole\,; mais je le conjurai de
vouloir bien m'écouter jusqu'au bout sur une affaire qui en entraînait
tant d'autres. «\, Mais est-il possible, me dit-il, qu'il vous reste
encore quelque chose\,? Et quelque chose, répondis-je, de plus important
encore, si vous voulez bien mien donner le temps.\,» Je lui dis qu'après
avoir traité l'armée, il fallait venir à la cour. Mais pour m'entendre
ici, il faut se souvenir de sa situation, et surtout de ce que j'ai
expliqué (t. III, p.~195 et suiv.) de M\textsuperscript{lle} de
Lislebonne, de M\textsuperscript{me} d'Espinoy, des mêmes encore, de
leur oncle de Vaudemont (p. 2 et suiv. de ce volume) de leur union avec
M\textsuperscript{lle} Choin et M\textsuperscript{me} la Duchesse d'une
part, avec MM. du Maine et de Vendôme de l'autre, de leur autorité sur
Chamillart, de M\textsuperscript{me} de Soubise, et de
M\textsuperscript{me} de Maintenon à l'égard de toutes ces personnes.

Je dis donc à M. de Beauvilliers qu'il fallait ajouter à tout ce que je
venais de lui représenter la part qu'y pouvaient prendre les cabales de
la cour. «\,Le roi, monsieur, a soixante-dix ans, et vous savez qu'on se
porte toujours sur le futur, surtout quand on n'espère pas de changer le
présent. M\textsuperscript{lle} Choin n'a que de la sécheresse pour Mgr
{[}le duc{]} et M\textsuperscript{me} la duchesse de Bourgogne. Elle
gouverne Monseigneur entre M. le prince de Conti et M. de Vendôme, qui
ont toute leur vie été les deux émules de l'amitié de ce prince. Vous
jugez bien pour, qui elle est, après ce qui lui est arrivé.
M\textsuperscript{me} la Duchesse le veut aussi gouverner, et vous voyez
tout ce qu'elle fait, et combien elle réussit auprès de lui. Vous
n'ignorez pas aussi qu'elle ne peut souffrir M\textsuperscript{me} la
duchesse de Bourgogne\,; M\textsuperscript{lle} de Lislebonne et
M\textsuperscript{me} d'Espinoy sont les dominantes à Meudon\,;
Monseigneur passe presque tous les matins seul chez elles, vous pensez
bien qu'elles le veulent gouverner et M. de Vaudemont par elles. Quant à
présent, toutes ces personnes vivent entre elles dans la plus intime
union\,; c'est un groupe qui ne fait qu'un. C'est leur intérêt pour
posséder seuls Monseigneur et en écarter tout, autre pour le solide, et
cet intérêt subsistera tant que le roi vivra, sauf après que Monseigneur
sera sur le trône à tirer chacun pour soi aux dépens des liaisons
anciennes, et ce sera à qui demeurera en principale possession d'un
prince trop borné pour choisir, et plus encore pour voir rien par
soi-même\,; mais en attendant l'union subsistera par le même intérêt de
n'y laisser ancrer personne. Excepté M\textsuperscript{me} la Duchesse,
qui n'a jamais aimé que pour le plaisir, vous n'ignorez pas les liaisons
de tous ces autres personnages avec M. de Vendôme, vous en avez eu les
plus grandes preuves d'Italie et depuis. Voilà donc des personnages sur
qui il peut solidement compter aujourd'hui\,; et lui par lui-même, et
chacun de ces autres personnages chacun par soi, à plus forte raison
tous ensemble sont les maîtres de Chamillart, et vous ne pouvez vous
dissimuler à vous-même qu'ils lui feront voir tout dans le point précis
qu'ils voudront, et que leur autorité sur lui et leur artifice prévaudra
sur lui, et à vous et à toute autre considération. Chamillart de plus
est livré à M. du Maine, et M. du Maine par Vendôme est à eux\,; mais ce
n'est pas tout.

Mgr le duc de Bourgogne touche à vingt-six ans. À cet âge son esprit, sa
vertu, son application lui ont acquis une réputation en Europe, et les
plus grandes espérances des François. Il a réussi en ses deux seules
campagnes. Il réussit plus encore dans le conseil. La cour le regarde
avec une vénération dont elle ne se peut défendre, quoiqu'en crainte de
l'austérité de ses moeurs, laquelle a déjà importuné le roi en plus
d'une occasion, et qui met avec lui Monseigneur en une sorte de malaise
qui se fait souvent sentir. Un héritier de la couronne devenu dauphin
avec ces avantages, et continuant de réussir comme il a commencé, initié
dans tous les conseils et dans toutes les affaires, n'est-il pas tout
naturellement l'âme du gouvernement et de la distribution des grâces
sous un père devenu roi vieux sans s'être jamais instruit ni appliqué\,?
Qui des ministres, des princes, des courtisans osera être son émule\,?
Qui d'eux, au contraire, n'en dépendra pas pour le présent et osera
tirailler rien contre lui auprès du roi son père\,? Qui, de plus, à la
taille et à l'âge de ce père, ne redoutera pas une prompte fin de son
règne qui mettra entre les mains du fils la souveraine puissance à
découvert, et les livrera tous à son bon plaisir\,? Je conviens que
cette dernière raison devrait retenir tout le monde\,; mais que ne peut
point l'audace et l'ambition qui veut toujours agir, parvenir, acquérir,
gouverner\,; qui s'enivre du présent, qui espère et s'étourdit sur
l'avenir\,; qui se mécompte sur sa puissance et sur l'étroit et le
timide d'une vertu dont ils ignorent l'étendue et la lumière\,; en un
mot de gens entraînés par la violence de leurs désirs\,! Tels sont ceux
dont il s'agit ici qui, pour gouverner Monseigneur devenu roi, ont
l'intérêt le plus pressant d'empêcher que son fils ne le gouverne, qui
n'en seront plus à temps si la mort du roi trouve ce prince dans la
réputation où nous le voyons, et qui pour cela n'ont d'autre ressource
qu'à tout hasarder pour la lui arracher du vivant du roi, et pour le
mettre dans le plus triste état où il leur soit possible de le réduire.
Je pense, monsieur, continuai-je, avoir démontré leur intérêt\,; ce ne
serait pas les connaître que de douter de leurs désirs quand leur
conduite explique si parfaitement leurs vues\,; et ce serait être
aveugle sur l'intérêt de tout ce qui est monstrueux à l'égard de Dieu et
même des hommes, que de douter du tremblement des bâtards, à l'égard
d'un prince aussi religieux que Mgr le duc de Bourgogne, pour leurs
rangs qui blasphèment, et leurs établissements qui effrayent. Vous
connaissez l'esprit, le manège, les artifices, l'application continuelle
de M. du Maine. Elles n'ont de contradictoire que la timidité, la
passion pour lui de M\textsuperscript{me} de Maintenon, et le faible du
roi pour l'un et l'antre\,; les ténèbres, de plus, de ses manèges, la
rassurent\,; l'audace et l'esprit, la position, les succès de M. de
Vendôme le fortifient\,; la fougue et l'impétuosité de sa femme le
pousse. Toutes ces vérités sont si claires que vous n'en sauriez nier
pas une. Vous n'avez qu'un retranchement, c'est la possibilité d'une
exécution aussi étrange à concevoir qu'un anéantissement d'un prince tel
en tout genre qu'est Mgr le duc de Bourgogne.

«\,Le monstrueux, monsieur, est qu'un tel projet se puisse présenter à
l'esprit. Quelque difficile qu'en soit l'exécution, elle l'est moins que
d'oser se la mettre dans la tête. Il faut pour arriver à ce but des
conjonctures qui ne se peuvent rencontrer dans l'uni de la vie ordinaire
de la cour\,; mais à la guerre, à la tête de troupes découragées, sans
discipline, manquant de force choses, dans la funeste habitude des plus
tristes revers, avec un général dont la licence, la puissance,
l'habitude lui ont acquis le coeur du soldat et du bas officier, la
terreur des autres, et personnellement intéressé à perdre le jeune
prince, avec toute l'audace et les appuis qui le peuvent assurer, les
occasions s'en peuvent trouver, et creuser de ces abîmes auxquels il
n'est guère naturel de s'attendre et qui font l'étonnement des nations.
Rendre la vertu importune, puis ridicule dans une armée, où personne ne
la connaît plus\,; montrer en odieux le jeune censeur de la licence qui
a lié à soi les officiers généraux et particuliers\,; faire redouter les
exemples sans lesquels on ne peut arrêter les désordres, et les donner
comme cruauté\,; tourner l'application et l'exactitude si nécessaires en
petitesse, en ignorance, en défaut des premières notions et de toute
lumière\,; présenter les précautions comme timidité, comme crainte
déplacée, qui dispose à mal juger du courage d'esprit et du caractère du
jugement\,; proposer des partis téméraires qu'on serait bien fâché qu'on
prît, mais dont on dispute avec opiniâtreté pour s'en avantager avec les
ignorants et les sots qui font le plus grand nombre, pour ne pas dire le
total à fort peu près en ces matières, et rejeter sur le jeune prince
les conseils qu'on appelle timides, et qu'on donne bientôt pour lâches,
avec le contraste du bouillant de l'âge et du désir de gloire d'un jeune
homme qui devrait avoir besoin d'être retenu, et qui retient au
contraire un général plein de capacité et d'expérience\,; avoir des
émissaires qui, sans être dans le secret, débitent tout ce qu'on veut,
écrivent, crient\,; en avoir à la ville, à la cour, qui font l'écho\,;
susciter des disputes, des contrariétés qui produisent des dits, des
contredits, des procès pour ainsi dire, qui se répètent et se déguisent
avec artifice en se débitant\,; en un mot, vouloir toujours le contraire
de ce que veut le prince, pour se plaindre, pour jeter toute faute sur
lui, pour faire crier\,; et surtout vouloir se battre contre toute
raison, et en manquer l'occasion quand elle se présente pour affubler le
prince de poltronnerie\,; et le déshonorer après y avoir préparé par
tout ce que je viens d'exposer, et ne se pas mettre en peine des suites
pour l'armée et pour l'État, afin d'écraser mieux le prince sous le
poids, voilà, monsieur, ce qui se présente à moi de très possible à un
homme aimé, gâté, révéré, appuyé, maître passé en audace, en artifice et
en sacrifices de tout à soi-même. Alors le cri de l'armée retentira dans
la ville, dans le royaume, dans la cour. Monseigneur sera paqueté contre
son fils, et le premier à lui jeter la pierre\,; le courtisan, qui
craint déjà son austérité, sera ravi de pousser de main en main cette
pierre qu'il ne craindra plus, maniée par Monseigneur même. Si cela
arrive, que jugez-vous que feront les personnes que j'ai nommées\,? Quel
parti n'en tireront-elles pas\,? et avec quel art ne feront-elles pas
jouer tous leurs ressorts de derrière les tapisseries\,?
M\textsuperscript{me} la duchesse de Bourgogne pleurera, mais il faudra
des raisons, non des larmes\,; qui les produira contre ce torrent\,? qui
osera se montrer à la cabale pour en être sûrement la victime tôt ou
tard\,? M\textsuperscript{me} de Maintenon sera affligée pour sa
princesse, mais persuadée par M. du Maine. Le roi outré écoutera les
traits adroits, ménagés, obscurs de ce cher fils de ses amours, et les
principaux valets intérieurs {[}seront{]} séduits par la familiarité de
Vendôme, par les caresses de M. du Maine, et de tout temps blessés du
sérieux du jeune prince avec eux, si fort en contraste avec les manières
du roi et de Monseigneur pour eux. La mode, le bel air sera d'un côté
avec un flux de licence, le silence de l'autre et la solitude. Tout
cela, monsieur, ne me paraît ni impossible ni éloigné, et si,
indépendamment de tant de machines manifestement dressées par l'intérêt
le plus pressant, il arrive une aventure malheureuse en Flandre, de
celles dont l'Italie, l'Allemagne, la Flandre même n'ont que trop et
trop fraîchement donné les plus cruelles expériences, vous verrez M. de
Vendôme en sortir glorieux, et Mgr le duc de Bourgogne perdu, et perdu à
la cour, en France et dans toute l'Europe.\,»

M. de Beauvilliers, avec toute sa douceur et sa patience, eut
grand'peine à me laisser dire jusqu'à la fin\,; puis, avec une gravité
sévère, me reprit de me laisser aller de la sorte à des idées bizarres
et sans possibilité, dont le fondement n'était en moi que le dégoût des
défauts de M. de Vendôme, l'aversion de son rang et de sa naissance, et
l'impatience de la faveur dans laquelle je le voyais\,; que tel qu'il
pût être, il ne s'aveuglerait pas assez pour se risquer en lutte contre
l'héritier nécessaire de la couronne, dont la réputation était la
consolation des Français, l'espérance de la cour, la surprise du monde,
tout ennemi qu'il est de la vertu, que le roi, malgré ce que j'avais
remarqué, aimait avec quelque chose de plus encore que de l'estime, et
que tous respectaient, dont l'épouse faisait tout son plaisir intérieur
et celui de M\textsuperscript{me} de Maintenon, un prince enfin que tout
le monde ne pouvait s'empêcher de respecter, et dont ce peu qu'il disait
dans le conseil ou dans des occasions était recueilli avec une attention
surprenante, et portait un véritable poids. Le duc revint encore, et
avec un peu d'amertume, sur mes préventions, sur l'excès où non
imagination et mes aversions les portaient, et sur non pas l'ineptie,
car il était trop mesuré pour employer ce terme, mais il m'en fit bien
sentir la valeur, de se laisser aller à l'idée qu'il fût possible de
concevoir le projet, et plus encore de pouvoir l'exécuter, de perdre le
fils aîné et héritier de la maison, qui le demeurerait toujours, quoi
qu'on pût faire, et qui régnerait à son tour. Je lui répondis que, sans
être persuadé par ses raisons contre les miennes, je me soumettais à ses
lumières, surtout pour un parti pris et arrêté, et sur lequel il n'y
avait plus à délibérer, mais que je me serais reproché de ne lui avoir
pas confié mes craintes, que personne ne souhaitait plus ardemment que
moi qui n'eussent pas lieu. Il se rasséréna et se mit à me parler de la
conduite que Mgr le duc de Bourgogne devait se proposer à l'armée, dont
nous convînmes aisément comme très importante, comme de s'appliquer et
de s'instruire beaucoup, mais hors de son cabinet, par la conversation
avec les meilleurs officiers généraux\,; des promenades pour reconnaître
les pays, les marches, les fourrages, les camps, les positions des
gardes et des postes\,; se communiquer fort aux officiers, parler
aisément à tous\,; distinguer ceux qui le méritaient à divers égards\,;
entrer dans le détail des troupes, avec un grand soin d'éviter le petit
et là minutie\,; se montrer familièrement et souvent à elles\,; être
gracieux en tout temps\,; et à table être gai sans donner lieu à une
liberté peu respectueuse, et à la tenir trop longtemps\,; témoigner à M.
de Vendôme toutes sortes d'égards et de confiance, l'apprivoiser, ne
rien voir de ce qui ne devait pas être aperçu, beaucoup moins en ouvrir
la bouche, ni la laisser ouvrir en sa présence, mais conserver, parmi
ces manières, dignité, gravité, supériorité et autorité.

Nous déplorâmes le plus que pitoyable accompagnement de ces princes\,:
d'O et Gamaches pour Mgr le duc de Bourgogne, desquels j'ai suffisamment
parlé ailleurs pour n'avoir rien à y ajouter\,; et pour M. le duc de
Berry Razilly seul, bon homme, droit, vrai, plein d'honneur, mais d'un
esprit médiocre, et qui, élevé pour l'Église, marié par la mort de son
frère aîné trop tard pour entrer dans le service, faisait à la lettre sa
première campagne avec ce prince. Un particulier aurait eu soin de mieux
accompagner ses fils. Nous nous séparâmes de la sorte, moi toujours si
persuadé que je ne pus m'empêcher de témoigner en gros mes craintes au
duc de Chevreuse, je dis en gros en le renvoyant là-dessus à M. de
Beauvilliers, parce qu'à la façon dont j'étais avec eux, parler à l'un
c'était aussi parler à l'autre, aussi le trouvai je plein des mêmes
espérances que son beau-frère, et dans la même conviction que lui sur
cette campagne de Mgr le duc de Bourgogne, et plus encore, s'il se
pouvait, par son penchant naturel à tout voir en bien et à tout espérer.
L'un et l'autre contèrent cette conversation aux duchesses leurs femmes,
pour qui ils avaient peu de secrets, et M. de Beauvilliers, plus
scandalisé encore qu'il n'avait voulu me le paraître, s'en plaignit à la
duchesse de Saint-Simon. Je lui promis pour l'apaiser que je ne lui en
parlerais plus, à condition aussi qu'il me promettrait de n'oublier rien
de tout ce que je lui avais dit là-dessus. Chamillart ne faisait
qu'arriver de Flandre, où sur le courrier de repentir de l'électeur, on
envoya Saint-Frémont l'exorciser avec les quatre cent mille livres de
plus dont j'ai parlé. Enfin il consentit de nouveau\,; le courrier de
Saint-Frémont en arriva la nuit du dimanche au lundi 30 avril.

Chamillart en porta la nouvelle au roi ce même lundi matin à Marly, où
nous étions encore, où le jour même, de peur de variation, le roi
déclara les généraux de ses armées comme je les ai dits ci-dessus, et
fit dépêcher un courrier à Villars pour le faire revenir de Strasbourg
et lui apprendre sa destination nouvelle. Le duc de Noailles retourna en
Roussillon commander une poignée de monde avec le titre de général, et
un seul maréchal de camp sous lui. Le roi déclara en même temps que M.
le duc de Berry, mais comme volontaire seulement, accompagnerait Mgr son
frère, et les trois seuls hommes de leur suite que j'ai dits. Il déclara
aussi que le roi d'Angleterre ferait la campagne en Flandre, mais dans
un entier incognito, sous le nom de chevalier de Saint-Georges. Villars,
attaché à ses sauvegardes, ne se contraignit point sur son déplaisir de
quitter l'Allemagne. Berwick, plus mesuré, n'en eut pas moins de se voir
un maître, et un maître si différent de lui en moeurs, en conduite, en
vie journalière, environné d'une petite cour qu'il fallait ménager, et
l'un et l'autre de fort mauvaise humeur de quitter la Flandre.

Quatre jours avant cette déclaration, M. de Vendôme, qui était dans le
secret et qui avait travaillé deux heures avec Chamillart chez
M\textsuperscript{me} de Maintenon avec le roi, s'en alla passer quatre
jours chez Duchy, frère de Plenoeuf, à Bellesbat, avec ses plus
familiers, d'où il poussa chez lui à la Ferté-Alais, où son frère le
grand prieur se rendit, nouvellement revenu de Gènes, d'où l'ennui
l'avait chassé et le peu de satisfaction sur ses prétentions de rang et
de distinctions. Il avait eu permission de revenir en France où il
voudrait, à condition de n'approcher de Paris ni de la cour plus près de
quarante lieues, excepté pour voir son frère un jour ou deux à la
Ferté-Alais. L'entrevue fut assez fraîche et la séparation avec peu de
satisfaction réciproque\,: ils ne se sont guère revus depuis. M. de
Vendôme revint à Marly le 1er mai et y demeura jusqu'au 4. Ces
bagatelles de dates sont importantes. Dans ce court intervalle, il
travailla plusieurs fois avec Chamillart, tantôt chez Mgr le duc de
Bourgogne, tantôt avec le roi et le même ministre chez
M\textsuperscript{me} de Maintenon, et Puységur fut admis en ces
conférences.

Le 4 mai au matin, le roi, sortant de son cabinet, trouva le maréchal de
Matignon, à qui il dit qu'il commanderait l'armée de Flandre sous le duc
de Vendôme, au nom duquel, comme au sien, il le cajola avec toutes les
flatteries dont il savait si bien assaisonner de si étranges nouveautés.
Ce dix-huitième maréchal de France n'eut pas honte de se répandre en
actions de grâces, et pour combler l'ignominie, en respects pour le
maître qui lui était donné. On peut juger qu'il était arrivé tout
préparé, et que Chamillart, à qui il devait son si léger bâton, lui
avait bien fait sa leçon. Il n'est pas croyable avec quelle liberté on
s'expliqua publiquement sur cette destination. Les maréchaux de France,
ceux qui aspiraient à l'être, les gens même qui ne regardaient que de
loin le bâton, ne purent se retenir. Le fait de Tessé à l'égard de
Vendôme, que j'ai raconté, ne fut pas oublié. On parla de la patente de
M. de Turenne offerte et du billet informe pour l'Italie seulement\,;
Matignon fut maltraité, on parla du bâton comme étant déshonoré, et du
métier qui l'a pour but comme ne pouvant plus mener à rien qu'à la
flétrissure. Les commentaires les plus amers et les plus libres n'y
furent pas épargnés, et tout haut en plein salon. De sept ou huit
maréchaux de France qui étaient ce voyage-là à Marly, aucun, tant qu'il
dura, ne parla au maréchal de Matignon, et, à leur exemple, qui que ce
soit à la lettre\,; son approche dissipait les pelotons et {[}faisait{]}
déserter les sièges\,: je n'ai rien vu de si marqué. Le maréchal de
Noailles, le plus valet de tous les hommes, ne laissa pas de se
recrobiller\footnote{Vieux mot qui est pris ici dans le sens de se
  \emph{regimber}.}. Quoique je ne fusse avec lui que très médiocrement
en mesure, il s'avisa de me demander ce que je pensais d'une si étrange
nouveauté. Je lui répondis froidement que, puisque ces sortes de princes
nous précédaient nous autres pairs depuis quelques années au parlement,
il ne devait plus sembler surprenant qu'ils commandassent les maréchaux
de France dans les armées.

Je sais l'exemple de Louis de La Trémoille qui n'avait aucune prétention
par naissance ni par rang\,; je n'ignore pas ceux de la maison de
Lorraine et de quelque chose de pareil pour M. d'Angoulême\,; mais ces
abus ne doivent pas tourner en règle. Je doute que du temps de Louis de
La Trémoille les maréchaux de France fussent encore bien nettement
officiers de la couronne comme ils le sont devenus depuis. Leur petit
nombre fixé les rendait plus considérables que leurs offices, qui à
peine quittaient leurs premières fonctions militaires au sortir de
l'écurie du roi, et très subalternes au connétable qui en était sorti
avant eux\,; et ces premières fonctions militaires étaient des
chevauchées par le royaume qu'ils partageaient entre eux pour visiter
les troupes, en faire les revues, et pourvoir à leur discipline et à
leur subsistance. L'office de connétable n'était presque jamais
vacant\,; il offusquait étrangement le leur. On sait quels étaient la
faveur, la puissance, les établissements et le mérite personnel de Louis
de La Trémoille sous qui tout ployait alors, et qui s'en prévalut. Pour
la maison de Lorraine, on aura répondu à tout en alléguant la tyrannie
des Guise et de leur formidable Ligue. Qui fait des maréchaux de France
peut bien les commander. M. de Mayenne en fit cinq ou six, parmi
lesquels MM. de La Châtre et de Brissac furent reconnus pour tels par
Henri IV à leur accommodement. Quant à M. d'Angoulême, ce fut le fruit
d'un gouvernement odieux et étranger. Il était confiné en prison pour le
reste de ses jours, en commutation de la perte de sa tête, à quoi il
avait été juridiquement condamné plusieurs années avant la mort d'Henri
IV.

La tyrannie de Marie de Médicis et de son maréchal d'Ancre souleva tout
et arma les princes. Le maréchal d'Ancre éperdu ne put leur opposer que
M. d'Angoulême, qui du cachot passa subitement à la tête de toutes les
forces du roi, et qui s'en prévalut dans les suites. C'est l'exemple qui
blessa M. d'Épernon qui ne voulut plus obéir aux maréchaux de France, et
qui toujours depuis commanda des corps séparés dans une entière
indépendance, et qui, se trouvant avec eux, comme à Saint-Jean d'Angély,
à la Rochelle et ailleurs, eut son quartier et son commandement à part,
sans prendre ni jamais recevoir leurs ordres. Mais entre les disparates
trop familières à notre nation, celle qui regarde l'office des maréchaux
de France est difficile à comprendre\,; c'est le seul qui ait
continuellement acquis, et qui se soutienne dans les honneurs les plus
marqués et les plus délivrés de toute dispute, c'est aussi le seul que
les princes, étrangers ou bâtards, dédaignent comme au-dessous d'eux.
Jusque-là qu'il n'y a point d'exemple d'aucun qui ait été maréchal de
France, tandis qu'ils courent tous après tous les autres offices de la
couronne. En même temps, quelles différences de fonctions\,! Le grand
chambellan n'a plus que celles de servir le roi, quand il s'habille ou
qu'il mange à son petit couvert\,; il est dépouillé de tout le reste, et
n'a nulle part aucun ordre à donner, ni qui que ce soit sous sa charge.
Le grand écuyer met le roi à cheval, et commande uniquement à la grande
écurie, en quoi, pour la réalité, il n'est pas plus que le premier
écuyer. Le colonel général de l'infanterie et le grand maître de
l'artillerie commandent, à la vérité, à des gens de guerre, mais ils se
trouvent dans les armées, ils obéissent sans difficulté aux maréchaux de
France. L'office de ceux-là est plus ancien que ces trois derniers, et
même que celui de l'amiral, et les fonctions des maréchaux de France
sont bien autrement nobles, puisqu'ils n'en ont d'autres que de
commander les armées, de donner l'ordre partout où ils se trouvent avec
des gens de guerre, et d'être les juges de la noblesse sur le point
d'honneur. Jusqu'au grand maître de France, qui depuis longtemps est un
prince du sang, il ne commande qu'aux maîtres d'hôtel, ne se mêle que
des tables\,; et encore depuis Henri III, à cause du dernier Guise qui
l'était, a-t-il perdu toute inspection sur tout ce qui regarde la bouche
du roi, et à cet égard, le premier maître d'hôtel est indépendant de
lui. J'ajoute que les princes du sang même sont colonels, maréchaux de
camp, lieutenants généraux, et servent et roulent par ancienneté avec
ceux qui ont les mêmes grades. À quoi mènent-ils, et que se propose-t-on
en les acquérant\,? le bâton de maréchal de France, et c'est ce bâton
dont aucun prince ne veut. Il faut avouer que c'est une manie, et
qu'elle est tout à fait inintelligible. Les princes allemands, même
souverains, n'ont pas cette fantaisie, ils sont ravis d'être faits
feld-maréchaux, qui est la même chose que nos maréchaux de France, au
jugement près du point d'honneur qu'ils n'ont pas, et toutefois je doute
qu'on fût bien reçu à leur proposer de céder à nos princes bâtards, ni à
pas un de la maison de Lorraine.

Vendôme en usa en cette occasion comme il avait fait lorsqu'il avait
obtenu ce billet informe du roi, pour commander les maréchaux de France
en Italie. Il partit sur-le-champ, \emph{ne varietur}. Le compliment du
roi au maréchal de Matignon lui avait été fait le vendredi matin à
Marly, 4 mai. Ce même jour, Vendôme s'en alla de Marly à Clichy, pour en
partir le lundi suivant pour la Flandre\,; il ne voulut pas être témoin
du vacarme d'une telle nouveauté\,; il n'y eut pas moyen de l'arrêter
jusqu'au lendemain samedi, 5 mai, que Bergheyck, de nouveau mandé pour
prendre avec lui de nouvelles et dernières mesures, devait arriver tout
droit à Marly, pour s'en retourner tout court en Flandre, après avoir
donné seulement un jour à Marly, où il fut logé dans le pavillon où
était Chamillart. Il ne s'agissait plus de la révolte des Pays-Bas,
depuis le malheureux succès d'Écosse. Le roi voulut, dans ce changement
de mesures, consulter Bergheyck sur celles à achever de fixer pour la
campagne, où l'envoi de son petit-fils lui faisait prendre un double
intérêt, et Bergheyck, qui était l'âme de toutes les affaires en
Flandre, ne pouvait s'en absenter en ce point surtout de l'ouverture si
prochaine de la campagne, sans beaucoup d'inconvénients. Il arriva tard
le samedi 5\,; le dimanche 6, il travailla le matin avec le roi et
Chamillart avant le conseil. L'après-dînée, le roi s'amusa à lui faire
les honneurs de ses jardins, et à le promener partout\,; le soir, il
travailla deux heures avec lui et Chamillart chez M\textsuperscript{me}
de Maintenon. Après le travail du matin, le roi envoya à Clichy,
Bergheyck\,; Chamlay et Puységur, conférer avec M. de Vendôme, pour
revenir dîner à Marly à trois heures, se promener ensuite comme je viens
de dire, rendre compte du voyage de Clichy chez M\textsuperscript{me} de
Maintenon, le soir, et y résumer tout avec le roi, et y recevoir ses
derniers ordres pour s'en retourner le lendemain 7 en Flandre. On voit
ici l'excès de la complaisance du roi pour le duc de Vendôme, et
l'orgueil démesuré de celui-ci\,: faire perdre tout ce temps à
Bergheyck, pour l'aller trouver à Clichy, dans le seul jour qu'il a à
demeurer ici, au lieu de retenir à Marly Vendôme vingt-quatre heures de
plus, pour y voir Bergheyck, et y conférer, et résoudre tout sous les
yeux du roi ensemble.

Voilà donc Bergheyck, Puységur et Chamlay courant à Clichy après M. de
Vendôme. Ils l'y trouvèrent dans le salon de la maison de Crosat, au
milieu d'une nombreuse et fort médiocre compagnie, qui se promenait les
mains derrière son dos. Il fut à eux et leur demanda ce qui les amenait.
Ils lui dirent que le roi les envoyait vers lui. Sans les tirer
seulement dans une fenêtre, et sans bouger de la même place, il se fit
expliquer à voix basse de quoi il s'agissait. La réponse du héros fut
courte. Il leur dit tout haut qu'il serait sur la frontière presque
aussitôt que Bergheyck à Mons\,; que, sur les lieux, il travaillerait
avec plus de justesse, et, avec une demi-révérence et une pirouette, il
alla rejoindre sa compagnie, qui s'était tenue éloignée par discrétion.
Leur surprise à tous trois fut sans pareille. Quoiqu'ils le connussent
bien, ils demeurèrent quelques moments immobiles d'un mépris si
audacieux et si public pour des affaires de cette première importance,
et pour des gens comme eux envoyés exprès par le roi pour en conférer
avec lui et en rapporter au roi le résultat le jour même. Le roi, fort
surpris de les voir sitôt de retour, leur en demanda la cause. Ils se
regardèrent. Enfin Puységur, plus hardi, raconta le succès du voyage. Le
roi ne put se contenir de laisser échapper un geste qui fit connaître ce
qu'il pensait, mais ce fut tout, et, après un moment de silence, il les
envoya travailler et dîner chez Chamillart, pour montrer après ses
jardins à Bergheyck. La journée se passa comme je l'ai dit d'abord, et
le lendemain, 7 mai, Bergheyck, dès le matin, repartit pour Mons. Ce
trait de Vendôme fit grand bruit. Enté si frais sur ce qui venait de se
passer du maréchal de Matignon, il en redoubla le vacarme, et à moi
l'intime persuasion de tout ce que j'avais prédit à M. de Beauvilliers.
L'audace de Vendôme à l'égard du roi même et de ses affaires les plus
importantes, et la faiblesse du roi à un trait si public et si marqué,
me devinrent des garants sûrs de tout ce que j'avais prévu. Je laissai à
Puységur les réflexions à faire faire là-dessus au duc de Beauvilliers.
Je n'en voulus même suggérer aucune au premier, et je ne parlai pas même
de Clichy à M. de Beauvilliers ni à M. de Chevreuse. Il n'était plus
temps de rien. M. de Vendôme partit de Clichy pour la Flandre le lundi 7
mai, comme il l'avait résolu.

Je ne veux pas omettre une bagatelle dont je fus témoin à cette
promenade, où le roi montra ses jardins à Marly, et dont la curiosité de
voir les mines et d'ouïr les propos du succès du voyage de Clichy
m'empêchèrent d'en rien perdre. Le roi, sur les cinq heures, sortit à
pied et passa devant tous les pavillons du côté de Marly. Bergheyck
sortit de celui de Chamillart pour se mettre à sa suite. Au pavillon
suivant le roi s'arrêta. C'était celui de Desmarets, qui se présenta
avec le fameux banquier Samuel Bernard, qu'il avait mandé pour dîner et
travailler avec lui. C'était le plus riche de l'Europe, et qui faisait
le plus gros et le plus assuré commerce d'argent. Il sentait ses forces,
il y voulait des ménagements proportionnés, et les contrôleurs généraux,
qui avaient bien plus souvent affaire de lui qu'il n'avait d'eux, le
traitaient avec des égards et des distinctions fort grandes. Le roi dit
à Desmarets qu'il était bien aise de le voir avec M. Bernard, puis, tout
de suite, dit à ce dernier\,: «\,Vous êtes bien homme à n'avoir jamais
vu Marly, venez le voir à ma promenade, je vous rendrai après à
Desmarets.\,» Bernard suivit, et pendant qu'elle dura, le roi ne parla
qu'à Bergheyck et à lui, et autant à lui qu'à d'autres, les menant
partout et leur montrant tout également avec les grâces qu'il savait si
bien employer quand il avait dessein de combler. J'admirais, et je
n'étais pas le seul, cette espèce de prostitution du roi, si avare de
ses paroles, à un homme de l'espèce de Bernard. Je ne fus pas longtemps
sans en apprendre la cause, et j'admirai alors où les plus grands rois
se trouvent quelquefois réduits.

Desmarets ne savait plus de quel bois faire flèche. Tout manquait et
tout était épuisé. Il avait été à Paris frapper à toutes les portes. On
avait si souvent et si nettement manqué à toutes sortes d'engagements
pris, et aux paroles les plus précises, qu'il ne trouva partout que des
excuses et des portes fermées. Bernard, comme les autres, ne voulut rien
avancer. Il lui était beaucoup dû. En vain Desmarets lui représenta
l'excès des besoins les plus pressants, et l'énormité des gains qu'il
avait faits avec le roi, Bernard demeura inébranlable. Voilà le roi et
le ministre cruellement embarrassés. Desmarets dit au roi que, tout bien
examiné, il n'y avait que Bernard qui pût le tirer d'affaire, parce
qu'il n'était pas douteux qu'il n'eût les plus gros fonds et partout\,;
qu'il n'était question que de vaincre sa volonté, et l'opiniâtreté même
insolente qu'il lui avait montrée\,; que c'était un homme fou de vanité,
et capable d'ouvrir sa bourse si le roi daignait le flatter. Dans la
nécessité si pressante des affaires, le roi y consentit, et pour tenter
ce secours avec moins d'indécence et sans risquer de refus, Desmarets
proposa l'expédient que je viens de raconter. Bernard en fut la dupe\,;
il revint de la promenade du roi chez Desmarets tellement enchanté, que
d'abordée il lui dit qu'il aimait mieux risquer sa ruine que de laisser
dans l'embarras un prince qui venait de le combler, et dont il se mit à
faire des éloges avec enthousiasme. Desmarets en profita sur-le-champ,
et en tira beaucoup plus qu'il ne s'était proposé.

\hypertarget{chapitre-xii.}{%
\chapter{CHAPITRE XII.}\label{chapitre-xii.}}

1708

~

{\textsc{Mort, fortune et caractère de Mansart.}} {\textsc{- Place des
bâtiments fort diminuée, et fort singulièrement donnée à d'Antin.}}
{\textsc{- Mort\,; état et caractère de La Frette.}} {\textsc{- Mort de
Montgivrault\,; son caractère, son état, et de son frère Le Haquais.}}
{\textsc{- Mort de la jeune marquise de Bellefonds.}} {\textsc{- Mort,
naissance, conduite, famille et caractère de la comtesse de Grammont.}}

~

Pendant ce même voyage {[}à Marly{]} Mansart mourut fort brusquement. Il
était surintendant des bâtiments, et personnage sur lequel il faut
s'arrêter un moment. C'était un grand homme bien fait, d'un visage
agréable, et de la lie du peuple, mais de beaucoup d'esprit naturel,
tout tourné à l'adresse et à plaire, sans toutefois qu'il se fût épuré
de la grossièreté contractée dans sa première condition. D'abord
tambour, puis tailleur de pierres, apprenti maçon, enfin piqueur, il se
fourra auprès du grand Mansart, qui a laissé une si grande réputation
parmi les architectes, qui le poussa dans les bâtiments du roi, et qui
tâcha de l'instruire et d'en faire quelque chose. On le soupçonna d'être
son bâtard. Il se dit son neveu, et quelque temps après sa mort, arrivée
en 1666, il prit son nom pour se faire connaître et se donner du relief,
{[}ce{]} qui lui réussit. Il monta par degrés, se fit connaître au roi,
et profita si bien de sa familiarité passée des seigneurs aux valets et
aux maçons, que, trouvant en lui les grâces de l'obscurité et du néant,
il crut lui trouver aussi les talents de son oncle, et se hâta d'ôter
Villacerf malgré lui, comme on l'a vu en son lieu, et de mettre Mansart
en sa place. Il était ignorant dans son métier. De Cote, son beau-frère,
qu'il fit premier architecte, n'en savait pas plus que lui. Ils tiraient
leurs plans, leurs dessins, leurs lumières, d'un dessinateur en
bâtiments, nommé L'Assurance, qu'ils tenaient tant qu'ils pouvaient sous
clef.

L'adresse de Mansart était d'engager le roi, par des riens en apparence,
en des entreprises fortes ou longues, et de lui montrer des plans
imparfaits, surtout pour ses jardins, qui tout seuls lui missent le
doigt sur la lettre. Alors Mansart s'écriait qu'il n'aurait jamais
trouvé ce que le roi proposait, il éclatait en admiration, protestait
qu'auprès de lui il n'était qu'un écolier, et le faisait tomber de la
sorte où il voulait, sans que le roi s'en doutât le moins du monde. Avec
ses plans il s'était frayé l'entrée des cabinets, et peu à peu de tous
et partout, et à toutes les heures, même sans plans et sans avoir rien à
dire de son emploi. Il en vint à se mêler dans la conversation en ces
heures privées\,; il y accoutuma le roi à lui adresser la parole sur des
nouvelles et sur toute matière\,; il hasardait quelquefois des
questions\,; mais il savait prendre ses moments\,; il connaissait le roi
en perfection, et ne se méprenait point à se familiariser ou à se, tenir
sur la réserve. Il montra aux promenades des échantillons de cette
privante pour faire sentir ce qu'il pouvait. Il n'en abusa point pour
mal faire à personne, mais il eût été dangereux de le blesser. Il acquit
ainsi une considération qui subjugua non seulement les seigneurs et les
princes du sang, mais les bâtards et les ministres qui le ménageaient,
et jusqu'aux principaux valets de l'intérieur. Sans se méconnaître en
effet, la grossièreté qui lui était demeurée, le rendait ridiculement
familier. Il tirait un fils de France par la manche, et frappait sur
l'épaule d'un prince du sang on peut juger comment il en usait avec
d'autres.

Le roi, qui trouvait fort mauvais que les courtisans malades ne
s'adressassent pas à Fagon et ne se soumissent pas en tout à lui, avait
la même faiblesse pour Mansart, et t'eût été un démérite dangereux à qui
faisait des bâtiments ou des jardins, de ne s'abandonner pas à Mansart
qui aussi s'y donnait tout entier, mais il n'était point habile. Il fit
un pont à Moulins, où il alla plusieurs fois. Il le crut un chef
d'oeuvre de solidité, il s'en vantait avec complaisance. Quatre ou cinq
mois après qu'il fut achevé, Charlus, père du duc de Lévi, vint au lever
du roi, arrivant de ses terres tout proche de Moulins, et il était
lieutenant général de la province. C'était un homme d'esprit, peu
content, et volontiers caustique. Mansart, qui s'y trouva, voulut se
faire louer, lui parla du pont, et tout de suite pria le roi de lui en
demander des nouvelles. Charlus ne disait mot. Le roi, voyant qu'il
n'entrait point dans la conversation, lui demanda des nouvelles du pont
de Moulins. «\,Sire, répondit froidement Charlus, je n'en ai point
depuis qu'il est parti, mais je le crois bien à Nantes présentement. ---
Comment\,! dit le roi, de qui croyez-vous que je parle\,? C'est du pont
de Moulins. --- Oui, sire, répliqua Charlus avec la même tranquillité,
c'est le pont de Moulins qui s'est détaché tout entier la veille que je
suis parti, et tout d'un coup, et qui s'en est allé à vau-l'eau.\,» Le
roi et Mansart se trouvèrent aussi étonnés l'un que l'autre, et le
courtisan à se tourner pour rire. Le fait était exactement vrai. Le pont
de Blois, bâti par Mansart quelque temps auparavant, lui avait fait le
même tour.

Il gagnait infiniment aux ouvrages, aux marchés et à tout ce qui se
faisait dans les bâtiments, desquels il était absolument le maître, et
avec une telle autorité qu'il n'y avait ouvrier, entrepreneur, ni
personne dans les bâtiments qui eût osé parler, ni branler le moins du
monde. Comme il n'avait point de goût ni le roi non plus, jamais il ne
s'est rien exécuté de beau, ni même de commode, avec des dépenses
immenses. Monseigneur ne voulut plus se servir de lui pour Meudon, parce
qu'il s'aperçut enfin, à l'aide d'autrui, qu'il le voulait embarquer en
des ouvrages prodigieux. Le roi, qui en devait savoir bon gré à
Monseigneur et mauvais à Mansart, fit au contraire ce qu'il put pour les
raccommoder, jusqu'à vouloir entrer pour beaucoup extraordinairement
dans cette dépense. Monseigneur était piqué d'avoir été pris pour dupe,
et s'en excusa. C'est de du Mont que j'ai su ce fait, qui en était
toujours en colère. Cette belle chapelle de Versailles, pour la
main-d'oeuvre et les ornements, qui a tant coûté de millions et
d'années, si mal proportionnée, qui semble un \emph{enfeu} par le haut
et vouloir écraser le château, n'a été faite ainsi que par artifice.
Mansart ne compta les proportions que des tribunes, parce que le roi ne
devait presque jamais y aller en bas, et il fit exprès cet horrible
exhaussement par-dessus le château pour forcer par cette difformité à
élever tout le château d'un étage\,; et, sans la guerre qui arriva, cela
se serait fait, pendant laquelle il mourut. Une colique de douze heures
l'emporta et fit beaucoup parler le monde. Fagon, qui s'empara de lui et
qui le condamna assez gaiement, ne permit pas qu'on lui donnât rien de
chaud. Il prétendit qu'il s'était tué à un dîner à force de glace et de
pois, et d'autres nouveautés des potagers dont il se régalait,
disait-il, avant que le roi en eût mangé.

On débita que les fermiers des postes, qui, par un crédit aussi
supérieur qu'inconnu, avaient toujours su parer aux coups portés à leurs
gains immenses, et qui venaient tout nouvellement de faire refuser une
prodigieuse enchère offerte sur par gens très solvables, présentés par
M. le duc d'Orléans dans le court voyage qu'il était venu faire
d'Espagne, furent avertis que Mansart s'était chargé de faire voir au
roi des mémoires contre eux, qu'ils étaient venus à bout depuis peu de
faire rejeter sans autre examen\,; qu'il avait même obtenu sa permission
de tirer un gros argent de l'avis de cette affaire s'il se trouvait bon,
et qu'il avait refusé jusqu'à quarante mille livres de rente, que ces
fermiers avaient offert de lui assurer, pour s'en désister. L'enflure
démesurée de son corps, aussitôt après sa mort, et quelques taches qui
se trouvèrent à l'ouverture, donnèrent cours à ces propos, vrais ou
faux.

Ce qui est certain, c'est que peu de jours avant sa mort il avait fort
pressé le roi sur ses avances dans les bâtiments, et sur celle des
principaux de ceux qui étaient sous sa charge, et sur l'épuisement de
leur crédit et du sien, qu'étant allé faire les mêmes représentations à
Desmarets, celui-ci, qui, comme on vient de voir, ne savait plus de quel
côté se tourner\,; lui déclara qu'il n'aurait point d'argent qu'il n'eût
rendu compte des derniers fonds qu'il avait touchés. Mansart, piqué au
dernier point d'une proposition si nouvelle, qui attaquait la confiance
en lui et le droit de sa charge de surintendant, qui était ordonnateur
et point du tout comptable, se défendit sur cette raison. Desmarets lui
répliqua durement qu'il dirait tout ce qu'il voudrait, mais qu'il
n'aurait pas un sou qu'il n'eût montré en quoi étaient passées les
dernières quatre ou cinq cent mille livres qu'il avait touchées depuis
très peu de temps, sans que la menace de s'en plaindre au roi pût
ébranler la fermeté du contrôleur général. Là-dessus, Mansart fit en
effet sa plainte. Il trouva le roi de même avis, et avec la même fermeté
que le contrôleur général, tellement qu'ayant voulu répliquer, il avait
été rudement tancé. On crut donc que cette première et si dure marque,
d'une chute prochaine, l'embarras où elle le jetait, et l'effort qu'il
se fit deux ou trois jours durant de cacher ses peines, causèrent en lui
la révolution qui le tua. Pendant sa maladie le roi en parut fort en
peine et y envoyait à tous moments. Une heure avant de mourir, Mansart
se confessa et pria le maréchal de Boufflers de recommander au roi sa
famille\,; et la veuve eut une pension. C'était dans le salon un
mouvement indécent pour un particulier de cette espèce. D'Antin y
pleurait et disait que ce n'était pas tant Mansart que l'affliction et
la privation du roi d'un homme de ce mérite. Il sécha et regretta
bientôt ses larmes.

À peine Mansart fut-il mort que le roi envoya chercher Pontchartrain, à
qui il enjoignit bien expressément de faire mettre à l'instant le scellé
partout à Marly, à Versailles, à Paris, et de prendre toutes les
précautions possibles pour empêcher que rien pût être détourné. Deux
heures après il l'envoya quérir encore pour lui réitérer les mêmes
ordres et savoir ceux qu'il avait donnés. Le lendemain samedi, 11 mai,
le chancelier étant venu à l'ordinaire au conseil des finances, le roi
le consulta là-dessus, et lui ordonna de contribuer de son ministère
pour que tout se passât avec la dernière exactitude et vigilance. La
surprise fut grande de voir le roi si dégagé sur une perte qu'une si
grande et si longue faveur devait rendre sensible par celle même du
plaisir et de la commodité, sans mélange d'aucune humeur, ni d'une
condition contraignante, qui lui avait fait trouver du soulagement à la
mort de ses ministres et de ses plus apparents favoris. Il ne se trouva
rien à la levée des scellés qui ternît la mémoire de Mansart. Il était
obligeant et serviable, et, comme je l'ai dit, ne se méconnaissait
point. Mais sa grossièreté, malgré tout son esprit, et la familiarité
qui en est la suite dans un homme de rien, gâté par la faveur, avait
fait en lui un mélange d'impertinence de surface qui empêcha qu'il ne
fût regretté.

Sa place fut un mois sans être remplie, et fit les voeux de quantité de
gens de tous, états. En appointements, logements, droits et commodités
de toutes sortes, sans prendre quoi que ce soit, elle valait à Mansart
plus de cinquante mille écus de rente, et il fut offert trois millions
au roi de cette charge et de celles qui en dépendaient. Le roi la voulut
diminuer et la changer de nature pendant la vacance. Il se déclara
lui-même le surintendant et l'ordonnateur de ses bâtiments, dont il se
réserva les signatures, en petit, comme il avait fait en grand lorsque,
après la chute de Fouquet, il supprima la charge de surintendant des
finances, dont il fit Colbert contrôleur général. Il arriva de l'une
comme de l'autre. Colbert, qui perdit Fouquet, de concert avec Le
Tellier, se servit entre autres grands ressorts, du danger et de l'abus
de la charge de surintendant, à laquelle, d'intendant de la maison du
cardinal Mazarin jusqu'à sa mort, il n'osait prétendre, mais dont il
voulait se réserver toute l'autorité. C'est ce qu'il fit en accablant le
roi des signatures que faisait le surintendant. Il lui fit accroire
qu'il ordonnait de tout par là, tandis que lui-même en conserva toute la
puissance sous la sûreté de ces signatures du roi qu'il fit faire comme
il voulut, et ses successeurs après lui. Il en arriva de même sur les
bâtiments. Le roi déclara qu'il en ferait un directeur général, et ce
directeur, qu'il élagua tant qu'il put, imita en tout Colbert, à la
fidélité près, comme cela n'a que trop paru pendant sa gestion, et comme
son testament l'a mis depuis dans la plus claire évidence.

Plusieurs candidats se présentèrent, ou le furent par le public. Voysin,
porté à tout par M\textsuperscript{me} de Maintenon, qui était fort
occupée de l'approcher du roi pour l'élever à tout ensuite\,;
Chamillart, qui n'y pensa jamais, pour le consoler, disait-on, des
finances\,; Pelletier, comme un emploi qui se marierait si bien avec le
sien des fortifications, qui, par son travail réglé avec le roi, lui
ôterait l'importunité d'une familiarité nouvelle\,; Desmarets, qui, avec
le même avantage, aurait encore celui d'épargner au roi les contrastés
des payements, les trois que je sais qui demandèrent furent le premier
écuyer qui ne s'en cacha pas à moi, La Vrillière qui me le confia, et
d'Antin. Le Premier avait l'estime et la familiarité du roi, et sa
confiance sur des détails d'argent qui n'avaient point de tiers,
indépendamment de ceux de la petite écurie. Il entendait les, bâtiments,
les prix\,; il avait du goût, de l'honneur, de la fidélité, de
l'exactitude. La confiance de Louvois, l'autorité qu'il s'était
conservée dans cette famille, et qui lui était restée de la
considération de son père, toutes ces choses lui en avaient fait oublier
l'origine et la modestie. Il était gendre et beau-frère des ducs
d'Aumont. Avec l'ordre et une belle charge après son père, il s'était
mis dans la tête de se faire duc. Les bâtiments lui donnaient des
entrées et des privantes continuelles, il espérait en profiter pour
cette élévation. La Vrillière avait une charge de secrétaire d'État,
qui, pour parler comme en Espagne, se pouvait appeler
caponne\footnote{C'est-à-dire mutilée, châtrée.}. Il était réduit aux
provinces de son département depuis que la révocation de l'édit de
Nantes et ses suites avait anéanti les affaires de la religion prétendue
réformée, qui avait fait le département particulier de cette charge. Nul
n'y était devenu ministre d'État\,; il était compté pour fort peu, parce
qu'on ne compte guère les gens à la cour, surtout ceux dont tout l'état
n'est que de se mêler d'affaires, que par celles qu'on peut avoir à eux.
Son désir, au défaut d'importance, était donc de relever sa charge par
la privante, et par la relation de toutes les heures avec le roi, qu'il
aurait trouvées en faisant un département à sa charge des bâtiments, et
de tout ce qui en dépend, et qu'un secrétaire d'État en familiarité et
en faveur sait bien étendre. Il avait beaucoup de goût et de
connaissance pour bien faire cette charge, et il la souhaitait
passionnément.

Le premier écuyer et lui contraignirent d'Antin plus que nul autre. Il
voulait s'approcher intimement du roi de quelque façon que ce pût être,
il voulait aller à tout, et son esprit était capable de tout. Il avait
déjà, comme on l'a vu, tâché d'être fait duc à la mort de son père. Sa
naissance ne s'y opposait pas, il n'avait plus M\textsuperscript{me} de
Maintenon contraire depuis la mort de sa mère, elle n'était pas même
éloignée de l'approcher du roi, par rapport aux bâtards. Ceux-ci le
portaient à découvert, et les Noailles aussi, qui étaient lors dans la
plus haute faveur. Chacun d'eux croyait y trouver son compté, et le
passage par Petit-Bourg les encourageait à le servir\,; mais il avait
beaucoup d'esprit, chose, en général, que le roi craignait, et éloignait
de sa personne, et une réputation de prendre comme il pouvait, bien
dangereuse pour les bâtiments. Rien toutefois ne les rebuta, et
Monseigneur, que cette dernière raison devait arrêter, comme on va voir,
plus que personne, se laissa gagner par M\textsuperscript{me} la
Duchesse, et entraîner, parce qu'il compta du crédit qui portait
d'Antin, jusqu'auprès de M\textsuperscript{me} de Maintenon, à oser,
pour la première fois de sa vie, témoigner au roi à son âge qu'il
désirait les bâtiments à d'Antin, l'affaire traînait, et cela même
donnait espérance aux rivaux. Le premier écuyer vint une après-dînée
dans ma chambre, venant de mettre le roi dans son carrosse. Il nous
trouva M\textsuperscript{me} de Saint-Simon et moi seuls\,; ce qui avait
dîné avec nous était déjà écoulé. Dès que la porte fut fermée, il me dit
d'un air de ravissement que pour le coup il croyait d'Antin solidement
exclu, malgré tous ses appuis. Il nous conta qu'il savait, par les
valets intérieurs qui l'avaient vu, que le roi avait dit ce même jour-là
à Monseigneur qu'il avait une question à lui faire, sur laquelle il
voulait savoir la vérité de lui. «\,Est-il vrai, ajouta-t-il, que,
jouant et gagnant gros, vous avez donné votre chapeau à tenir à d'Antin,
dans lequel vous jetiez tout ce que vous gagniez, et que le hasard vous
ayant fait tourner la tête, vous surprîtes d'Antin empochant votre
argent de dedans le chapeau\,?» Monseigneur ne répondit mot\,; mais
regardant le roi en baissant la tête, témoigna que le fait était vrai.
«\,Je vous entends, Monseigneur, dit le roi, je ne vous en demande pas
davantage,\,» et sur cela se séparèrent, et Monseigneur sortit à
l'instant du cabinet. Nous conclûmes, comme le premier écuyer, que cette
question n'était faite que par rapport aux bâtiments, et qu'après cet
éclaircissement, d'Antin en était très certainement revenu. Le
lendemain, La Vrillière me dit la même chose, transporté de joie de se
pouvoir compter délivré d'un compétiteur si dangereux.

Le quatrième, jour, qui était un dimanche, tout à la fin de la matinée,
le premier écuyer vint chez moi, et m'apprit que d'Antin avait les
bâtiments. Il était furieux avec tout son froid et sa sagesse, peut-être
moins de s'en voir éconduit, que de ce qui se pouvait attendre d'une
telle faiblesse, après la réponse de Monseigneur. Et puis raisonnez
conséquemment dans les cours\,! Le roi eut l'égard pour Monseigneur de
vouloir que ce fût de lui que d'Antin apprît sa fortune\,; son transport
de joie fut plus fort que lui\,; il s'y livra, il dit que c'était à ce
coup que le sort était levé, qu'il n'était plus en peine de sa fortune.
Il eut toutes les entrées qu'avait Mansart, il les élargit même, et
bientôt il sut subjuguer le roi, et l'amuser. Il n'en fut pas moins
assidu auprès de Monseigneur, ni moins souvent avec les bâtards, surtout
avec M\textsuperscript{me} la Duchesse\,; il n'en joua pas moins\,; en
un mot, quatre corps n'eussent pas suffi à sa vie de tous les jours. Il
fut plaisant qu'un seigneur comptât, et avec raison, sa fortune assurée
par les restes, en tout estropiés, d'un apprenti maçon, en titre, en
pouvoir, en appointements, réduits à un tiers. Ce fut une sottise\,; il
eut bientôt après plus d'autorité et de revenu que Mansart, mais en s'y
prenant d'une autre manière. En bref, il devint personnage, et le fut
toujours depuis de plus en plus.

La Frette mourut en ce temps-ci fort subitement. J'ai parlé du fameux
duel qui le fit sortir du royaume avec son frère\,; c'étaient peut-être
les deux hommes de France les mieux faits et les plus avantageux\,; leur
nom était Gruel, et des plus minces gentilshommes de France, et La
Frette un des plus légers fiefs du Perche. Leur grand-père s'attacha au
premier comte de Soissons, prince du sang, dont il fut domestique
principal, et qui obtint d'Henri IV le pénultième collier de la première
promotion de l'ordre du Saint-Esprit, qu'Henri IV fit depuis son sacre,
en 1595, aux Augustins à Paris. C'est de lui qu'on fait le conte que
disant, en recevant le collier\,: \emph{Domine, non sum dignus}, qu'on
ne dit plus, et qu'on n'a peut-être jamais dit, Henri IV lui répondit\,:
«\,Je le sais bien, je le sais bien, c'est pour l'amour de mon cousin de
Soissons qui m'en a prié.\,» La Frette le porta vingt ans, et il était
gouverneur de Chartres. Son fils le fut aussi, et du Pont-Saint-Esprit.
Il fut encore capitaine des gardes de Gaston duc d'Orléans, frère de
Louis XIII. Le comte de Saint-Aignan, depuis duc et pair et père du duc
de. Beauvilliers, et lui, épousèrent les deux soeurs de même nom que
Servien, surintendant des finances. Celle que La Frette épousa était
veuve en premières noces d'un Le Ferron, dont une fille unique fort
riche, veuve en premières noces, sans enfants, de Saint-Mégrin, dont
j'ai parlé ailleurs, tué au combat du faubourg Saint-Antoine, {[}fut{]}
remariée au duc de Chaulnes, tellement que ces La Frette dont il est
question ici étaient frères de mère de la duchesse de Chaulnes, et
cousins germains du duc de Beauvilliers, qui les servirent toute leur
vie de tout leur pouvoir, ce qui leur fut d'une grande protection et
considération.

M. de Chaulnes, étant ambassadeur extraordinaire à Rome en 1667 et 1670,
y eut grande part aux élections de Clément IX et Clément X (Rospigliosi
et Altieri), avec qui il fut si bien qu'il le pressa tant de s'employer
pour lui auprès du roi qu'il ne put s'en défendre, et le pria d'obtenir
la grâce des deux La Frette. Le pape le fit de si bonne grâce, et voulut
si fortement dispenser le roi de son serment des duels à leur égard, que
le roi, n'y pouvant consentir pour les conséquences, s'engagea au pape
de les laisser revenir en France sur sa parole, vivre en liberté à Paris
et partout, jouir et disposer entièrement de leurs biens, mais sous
d'autres noms. Ils revinrent donc de la sorte, et allaient partout
annoncés et appelés de leur nom, mais s'abstenant de livrées, d'armes et
de se trouver dans aucun lieu public. On leur écrivait à leur adresse
sous leur nom à Paris, chez eux et partout. Ils vécurent toujours ainsi
sous la protection tacite du roi, qui, pour la forme, fit toujours
semblant de les ignorer. Il arriva une affaire qui fit grand bruit, où
Flamarens, lors premier maître d'hôtel de Monsieur, se trouva si mêlé
qu'on fouilla jusque dans le Palais-Royal pour le trouver. Monsieur se
plaignit au roi de ce manque de respect pour lui, et ajouta aigrement
que cette recherche l'offensait, d'autant plus qu'on ne disait mot aux
deux La Frette qui depuis plusieurs années étaient dans Paris, et qui y
allaient partout à visage découvert. Le roi répondit gravement que cela
ne pouvait être, et sur ce que Monsieur insista., il l'assura qu'il s'en
ferait informer, et les ferait arrêter dans les vingt-quatre heures
s'ils se trouvaient dans Paris. En même temps, il les fit avertir d'en
sortir sur-le-champ pour deux ou trois jours, après quoi ils pourraient
y revenir et vivre à leur ordinaire, et il ordonna qu'on fît d'eux
partout Paris une recherche éclatante. Mais il enjoignit bien
expressément qu'on ne la commençât pas sans être bien assuré qu'ils en
étaient sortis. Il ne tint qu'à Monsieur de voir ensuite que le roi
s'était un peu moqué de lui, en lui donnant cette satisfaction
apparente. L'aîné mourut longtemps avant le cadet. Jamais gens ne surent
mettre à si grand profit une mort civile, l'honneur d'un duel, et cette
tacite protection du roi qui, en effet, en tout son règne a été une
distinction unique, ni vivre si largement de procès et de petites
tyrannies. Ni l'un ni l'autre ne furent mariés, et ce dernier était
vieux.

Il mourut peu de jours après un autre homme extraordinaire. On
l'appelait le chevalier de Montgivrault. M. de Louvois l'avait
scandaleusement chassé du service, où il était ingénieur dans la
première guerre de Flandre en 1667, où il avait acquis beaucoup de bien.
Malgré cette aventure et une réputation peu nette, il sut devenir une
espèce d'important à force d'esprit, de galanterie, de commodité pour
autrui et d'excellente chère. Il se fit ainsi beaucoup d'amis
considérables à la cour et à la ville. Le maréchal de Tessé, le duc de
Tresmes, Caumartin, Argenson entre autres étaient ses intimes. Il avait
acquis par là de la considération, et il avait eu l'art de s'ériger chez
lui un petit tribunal où beaucoup de gens étaient fort aises d'être
reçus. Il avait acheté Courcelles auprès du Mans, qui a été depuis la
retraite de Chamillart qui l'acheta, où Montgivrault dépensa beaucoup,
et où j'ai admiré sa folie d'avoir mis ses armes jusque sur toutes les
portes, les cheminées et les plafonds. Il n'avait jamais été marié et
laissa un gros bien.

Son frère, qui faisait fort peu de cas de lui, s'appelait Le Haquais, et
ne s'était point marié non plus. Il était son aîné et était demeuré fort
pauvre. Il avait été avocat général de la cour des aides, avec une
grande réputation d'éloquence, de savoir et de probité. C'était un homme
parfaitement modeste et parfaitement désintéressé. On ne pouvait avoir
plus d'esprit, un tour plus fin, ni en même temps plus aisé, avec
beaucoup de grâce et de réserve\,; avec cela salé, volontiers caustique,
gai, plaisant, plein de saillies et de reparties, éloquent jusque par
son silence. Ses lettres étaient charmantes, et pour peu qu'il se
trouvât à son aise, de la meilleure compagnie du monde. Le chancelier de
Pontchartrain et lui, à peu près de même âge, avaient été amis intimes
dans leur jeunesse. Galants, chasseurs, mêmes goûts, même sorte d'esprit
et de sentiments en toute leur vie. Lorsque le chancelier fut en
fortune, il fit pour son ancien ami des bagatelles à sa convenance,
parce qu'il ne voulut jamais mieux. Il était de tous les voyages de
Pontchartrain où je l'ai fort connu\,; et ce qui est respectable pour
les deux amis, c'est que sans s'y mêler de rien, ni sortir de son état
de petit bourgeois de Paris, comme il s'appelait souvent lui-même, il y
était comme le maître de la maison\,: tout le domestique en attention et
en respect, et tout ce qui y allait en première considération. Le
chancelier, outre l'amitié et la confiance, lui en témoigna toujours une
extrême et toute sa famille aussi\,; il montrait vouloir que tout le
monde-lui en portât, et Le Haquais était aimé de tous. Il vivait avec
grand respect pour les gens considérables qu'il y voyait, il n'en
manquait point au chancelier ni, à la chancelière, qui l'aimaient autant
l'un que l'autre\,; mais il ne laissait pas de vivre fort en liberté
avec eux, et de laisser échapper des traits de vieil ami qui ne lui
messeyaient pas et qui étaient toujours bien reçus. Dans les dernières
années sa piété s'accrut tellement que le chancelier et sa femme ne
l'avaient plus à Pontchartrain autant qu'ils l'y voulaient. Ils
l'appelaient leur muet, parce que la charité avait mis un cachet sur sa
bouche, auquel on perdait beaucoup. Je m'en plaignais souvent à
lui-même\,; on ne le voyait jamais qu'à Pontchartrain\,; il vivait fort
retiré à Paris.

Le marquis de Bellefonds, petit-fils du maréchal, perdit sa femme toute
jeune et mariée depuis peu\,; elle était Hennequin, fille d'Égvilly, qui
avait le vautrait\footnote{Terme de vénerie. On appelait \emph{vautrait}
  l'équipage de chasse pour le sanglier.}.

Quatre ou cinq jours après, c'est-à-dire le 3 juin, la comtesse de
Grammont mourut à Paris à soixante-sept ans. Elle était Hamilton, de
cette grande maison d'Écosse si puissante, si ancienne, si grandement
alliée et si souvent avec les Stuarts.

Marie, fille de Jacques Stuart II, roi d'Écosse, mariée en 1468 à
Jacques Hamilton, comte d'Arran, fut mère de Jacques II Hamilton, comte
d'Arran, régent d'Écosse sous le roi Jacques Stuart V, et père de
Jacques III Hamilton, régent d'Écosse et tuteur de l'infortunée Marie
Stuart, reine d'Écosse, épouse de notre roi François II dont il fit le
mariage. Il fut fait duc de Châtellerault, terre en Poitou qui lui fut
donnée, et que lui et sa postérité perdirent avec la dignité pour s'être
retiré en Écosse, et y avoir quitté le parti français par l'inimitié des
Guise, qui pour se rendre les maîtres des affaires d'Écosse le voulurent
faire périr et le persécutèrent partout. Sa postérité et lui-même ont
souvent réclamé leur terre et leur dignité. Sa mère était tante
paternelle du cardinal Béton\,; son père l'avait épousée du vivant de sa
première femme, qui s'appelait Humie, qui n'avait point d'enfants, et
qu'il avait répudiée. Ce duc de Châtellerault laissa de sa femme, fille
du comte de Morton, trois fils\,: l'aîné fut insensé, les autres\,;
persécutés en Écosse, se réfugièrent en Angleterre. La reine Élisabeth
les fit rétablir en Écosse par Jacques, roi d'Écosse et depuis
d'Angleterre après elle. L'aîné fut comte d'Arran, et créé marquis
d'Hamilton\,; le cadet marquis de Pasley\,; celui-ci laissa plusieurs
enfants. D'un d'eux, qui fut comte d'Albecorn, et de Marie Boid, sa
femme, plusieurs enfants, dont Georges Hamilton, chevalier, baronnet,
eut d'une Butler, son épouse, la comtesse de Grammont et ses deux
frères, dont il a été parlé plusieurs fois. De l'aîné, Jean Hamilton,
comte d'Arran et marquis d'Hamilton, vint Jacques V, marquis d'Hamilton,
chambellan et sénéchal de Jacques Ier, roi de la Grande-Bretagne, fils
de l'infortunée Marie Stuart, et successeur d'elle en Écosse et
d'Élisabeth en Angleterre. Il donna aussi la Jarretière au marquis
d'Hamilton. Jacques VI, marquis d'Hamilton, son fils, fut fait duc
d'Hamilton et chevalier de la Jarretière par le malheureux roi Charles
Ier, pour lequel il mourut sur un échafaud en 1649. Il ne laissa que des
filles. Anne, l'aînée, épousa Guillaume Douglas, comte de Selkirke, que
Charles II, après son rétablissement, fit duc d'Hamilton\,; et c'est de
lui que descendent les ducs d'Hamilton d'aujourd'hui.

Le père et la mère de la comtesse de Grammont étaient catholiques,
vinrent passer quelque temps en France avec leurs enfants\,; ils mirent
la comtesse de Grammont, toute jeune, à Port-Royal des Champs, où elle
fut élevée, et elle en avait conservé tout le goût et le, bon, à travers
les égarements de la jeunesse, de la beauté, du grand monde et de
quelques galanteries, sans que, comme on l'a vu, la faveur ni le danger
de la perdre l'aient jamais pu détacher de l'attachement intime à
Port-Royal.

C'était une grande femme qui avait encore une beauté naturelle sans
aucun ajustement, qui avait l'air d'une reine, et dont la présence
imposait le plus. On a vu ailleurs comment se fit son mariage, le goût
si marqué et si constant du roi pour elle, jusqu'à inquiéter toujours
M\textsuperscript{me} de Maintenon, pour qui la comtesse de Grammont ne,
se contraignit pas. Elle avait été dame du palais de la reine. C'était
une personne haute, glorieuse, mais sans prétention et sans
entreprise\,; qui se sentait fort, mais qui savait rendre, avec beaucoup
d'esprit, un tour charmant, beaucoup de sel, et qui choisissait fort ses
compagnies, encore plus ses amis. Toute là cour la considérait avec
distinction, et jusqu'aux ministres comptaient avec elle. Personne ne
connaissait mieux qu'elle son mari\,; elle vécut avec lui à merveilles.
Mais, ce qui est prodigieux, c'est qu'il est vrai qu'elle ne put s'en
consoler, et qu'elle-même en était honteuse. Ses dernières années furent
uniquement pour Dieu.

Elle comptait bien, dès qu'elle serait veuve, de se retirer entièrement,
mais le roi s'y opposa si fortement qu'il fallut demeurer. Ce ne fut pas
pour longtemps\,; de grandes infirmités la tirèrent de la cour\,;
{[}ce{]} dont elle fit le plus saint usage et le plus solitaire, et
mourut ainsi avant ses deux années de deuil.

Elle n'avait que deux filles\,: toutes deux de beaucoup d'esprit, fort
dangereuses, fort du grand monde, fort galantes, qui avaient été filles
d'honneur de M\textsuperscript{me} la dauphine de Bavière, et qui
n'avaient rien. L'une épousa un vilain milord Stafford, qui était
Howard, qui passait sa vie à Paris aux Tuileries et aux spectacles, et
que personne ne voulait voir, avec qui elle se brouilla bientôt et s'en
sépara. Depuis sa mort elle alla vivre en Angleterre de ce qu'il lui
avait donné, en l'épousant, et n'en eut point d'enfants. L'autre se fit
chanoinesse et abbesse de Poussay, où elle s'est convertie et a vécu
dans une grande pénitence et bien soutenue. Comme elles n'avaient rien,
leur mère écrivit en mourant au roi et à M\textsuperscript{me} de
Maintenon pour leur demander pour elles sa pension du roi. De ces deux
lettres, l'une fut dédaignée, l'autre négligée\,: Tel est le crédit des
mourants les plus aimés et les plus distingués durant leur vie. Il n'y
eut ni réponse ni pension.

\hypertarget{chapitre-xiii.}{%
\chapter{CHAPITRE XIII.}\label{chapitre-xiii.}}

1708

~

{\textsc{Éclat entre Chamillart et Bagnols, qui en quitte l'intendance
de Flandre et met Chamillart en danger.}} {\textsc{- Mariage de
Courcillon avec la fille unique de Pompadour.}} {\textsc{- Leur
caractère et leur situation.}} {\textsc{- Mariage, état, caractère de
Lanjamet et de sa femme.}} {\textsc{- Mariage de Louville avec la fille
de Nointel, conseiller d'État.}} {\textsc{- Enlèvement de
M\textsuperscript{lle} de Roquelaure par le prince de Léon.}} {\textsc{-
Mariage du prince de Léon et de M\textsuperscript{lle} de Roquelaure.}}

~

Chamillart s'était brouillé avec Bagnols, intendant très accrédité de
Lille et conseiller d'État, dans le court voyage qu'il avait fait en
Flandre. Il chassa d'autorité un principal commis de l'extraordinaire de
la guerre, résidant en Flandre, pour friponnerie. C'était un homme
entièrement à Bagnols, qui fit auprès de Chamillart l'impossible pour le
sauver\,; jusqu'à prendre fait et cause, et déclarer que, si cet homme
avait volé, il fallait qu'il fût de moitié. Chamillart tint bon, l'autre
aussi, qui leva l'étendard et qui entreprit de faire rétablir ce commis
malgré le ministre. Il y eut des lettres fortes. Bagnols en demanda
justice, tous ses amis se remuèrent, et tous les ennemis de Chamillart.
Jamais on ne vit tant de vacarme pour si peu de chose, ni un intendant
le prendre si haut contre un ministre, son supérieur. Chamillart
l'emporta, mais à force de bras, et y usa beaucoup de son crédit. Alors
Bagnols demanda à se retirer\,: nouvel éclat. Le roi qui en était
content voulut le retenir, on lui fit des avances, il y eut force
pourparlers\,; Chamillart même, qui sentit le roi fâché, se prêta. Plus
on en faisait pour Bagnols, plus il en était gâté, et plus il
prétendait. À la fin Chamillart l'emporta encore, mais il s'éreinta, et
Bagnols quitta l'intendance et vint ameuter à Paris. C'était une bonne
tête, débauché, fort au goût de tout ce qui avait servi en Flandre, par
son esprit, sa bonne maison, sa grande chère et délicate, et le soin de
plaire et d'obliger\,; d'excellente compagnie, toute sa vie du grand
monde, avec beaucoup d'amis et considérables, fort proche du chancelier,
des Louvois par sa femme, et fort porté par ce qui en restait, très
capable et supérieur à son emploi, où il avait servi avec une grande
utilité et distinction.

M\textsuperscript{me} de Maintenon ne regardait plus Chamillart depuis
le mariage de son fils que comme un homme qui lui avait manqué.
L'aversion avait succédé à l'amitié. J'ai expliqué ailleurs son intérêt
pressant d'avoir un ministre à elle, et elle n'en avait aucun depuis
qu'elle ne comptait plus sur Chamillart. C'était donc à ses dépens
qu'elle en voulait un autre à elle, et il était tout trouvé en la
personne de Voysin. Le roi, contre toute coutume, alla de Versailles
dîner le 4 juin à Meudon, avec M\textsuperscript{me} la duchesse de
Bourgogne, plusieurs dames et M\textsuperscript{me} de Maintenon, qui y
vit en particulier M\textsuperscript{lle} Choin, et
M\textsuperscript{lle} Choin était outrée contre Chamillart, qui
naturellement opiniâtre, et devenu sujet à l'humeur par le mauvais état
des affaires et de sa santé, n'avait jamais voulu procurer un petit
régiment d'infanterie au frère de M\textsuperscript{lle} Choin, qui
servait depuis longues années, quelque chose que M\textsuperscript{lle}
de Lislebonne et M\textsuperscript{me} d'Espinoy eussent pu lui dire, et
qui, piquées du persévérant refus, et ne voulant pas qu'il tombât sur
elles, expliquèrent à M\textsuperscript{lle} Choin tout ce qu'elles
avaient dit et fait pour résoudre Chamillart. Je sus ce détail pansa
fille Dreux, qui avait de l'esprit, et qui\,; étant la seule de la
maison qui eût du sens, en était fort peinée. Je sus encore par le
maréchal de Boufflers et par le duc et la duchesse de Villeroy les
mouvements de la cabale formée des amis de Bagnols et des ennemis de
Chamillart ralliés au maréchal de Villeroy.

Cette conversation si nouvelle et si recherchée par
M\textsuperscript{me} de Maintenon avec M\textsuperscript{lle} Choin,
jusqu'à aller exprès dîner à Meudon, et s'y couvrir du roi, sans y
coucher, m'effaroucha dans ces circonstances, car l'affaire du commis et
de la rupture s'était passée dès les premiers jours de l'arrivée de
Chamillart en Flandre, et avait éclaté et fait de grands progrès avant
même son retour. Je compris que M\textsuperscript{me} de Maintenon, qui
jusqu'alors n'avait tenu le moindre compte de Monseigneur, ni gardé la
plus petite mesure avec la Choin, voulait profiter de son dépit contre
Chamillart, et qu'elle y était excitée par ce qui se passait entre le
roi et Monseigneur sur les bâtiments, dont elle était informée par les
Noailles. Je craignis un coup de foudre subit pour Chamillart, et je ne
crus pas m'en pouvoir reposer sur personne. Je l'en avertis, je le
trouvai instruit et embarrassé. Il n'était pas temps de contester avec
lui, et de lui reprocher d'avoir pris son parti trop vite et trop haut
sur Bagnols, ni sa folle opiniâtreté sur ce régiment pour Choin\,; il
fallait aller au remède, et à temps. Je lui conseillai de parler dès le
lendemain au roi, de lui dire que, quelque honoré qu'il fût de sa place,
il y tenait peu dans le triste état présent, mais qu'il tenait
infiniment à sa personne par son coeur et par reconnaissance\,; qu'il
n'y avait biens ni fortuné pour lesquels il voulût lui donner une minute
de peine\,; qu'il voyait avec douleur un orage se former contre lui
qu'il n'avait pas mérité, mais que, pour peu que le roi fût embarrassé
de lui, ou qu'il en aimât mieux un autre en sa place, il la lui
remettrait de tout son coeur, uniquement pour lui plaire et pour mériter
la conservation de ses bontés, et de l'honneur de ses bonnes grâces qui
lui étaient plus chères que nuls établissements, et sans lesquels il ne
pourrait vivre. Je l'exhortai à n'en pas dire davantage, et sur ce ton,
et avec cette force et ce dégagement\,; de bien regarder cependant le
roi entre deux yeux, dont le plus léger mouvement serait en ce moment
très significatif\,; de saisir promptement ce qu'il lui répondrait,
quand il ne serait simplement qu'honnête\,; surtout de ne pas insister à
la retraite, et de se bien garder de la sottise de se vouloir faire
prier. J'ajoutai qu'avec cette conduite, et à temps comme il était
encore, j'osais lui répondre, sans être grand clerc à la cour, qu'il
serait bien reçu quand bien même il embarrasserait le roi\,; et que de
cette époque ce serait un nouveau bail passé avec lui, qui, sans en dire
un seul mot, mais laissant faire le roi à l'égard de ceux qui
l'attaqueraient, leur ferait tomber incontinent les armes des mains.

Chamillart goûta ma pensée\,; je n'eus pas besoin de l'exorciser, mais
bien le dépit de se voir réduit là, et par ce dépit, l'envie de ne rien
faire, et de se laisser culbuter, voilà ce que j'eus à combattre, et
j'en vins à bout enfin avant de le quitter. Je lui recommandai bien que
ce compliment se fit dans le cabinet du roi, et point du tout chez
M\textsuperscript{me} de Maintenon, où elle aurait été présente\,; il me
le promit, et que ce serait le lendemain. Il m'embrassa, me remercia, et
me donna rendez-vous chez lui à son retour de cette espèce d'assaut.
Moi-même j'en étais inquiet, quelque bonne espérance que j'en eusse. Je
craignais le roi déjà peut-être circonvenu, de l'incertitude, la
froideur de sa part, le dépit du ministre qui s'empêtrerait en allant
trop loin et qui se ferait prendre au mot.

Le temps me dura fort pendant quinze ou vingt heures que j'allai au
rendez-vous. Je fus soulagé du premier coup d'oeil. Je vis mon homme
gai, léger, qui m'embrassa encore, et qui était assuré et ravi. Il me
dit qu'il avait parlé précisément comme je le lui avais conseillé\,; que
le roi s'était mis à sourire, et lui avait répondu qu'il était bien
simple de penser que tout ce bruit fit sur lui la moindre impression\,;
qu'il continuât à le bien servir, comme il avait toujours fait\,; que,
pour lui, il l'aimerait toujours, qu'il le soutiendrait, et qu'il
voulait qu'il prît confiance en ce qu'il lui disait. Respects,
remercîments, tendresses de Chamillart, bontés encore du roi là-dessus,
et puis parlèrent de leurs affaires. Chamillart en revint rajeuni, et
une maison hors de dessus l'estomac. Il n'en parla à qui que ce soit
qu'aux ducs de Chevreuse et de Beauvilliers, après la chose faite, qui
ne la croyaient pas à ce point de danger, mais qui furent très aisés du
succès. Il est vrai que je m'en sus beaucoup de gré. Très peu de jours
après, tous ces bruits et les menées tombèrent\,; le roi apparemment les
avait nettement éconduits. Mais je crus devoir conjurer Chamillart de
modérer sa confiance, de marcher la sonde à la main, et de comprendre
par cette affaire qu'il n'était pas invulnérable, et que cet avortement
de dessein ne ferait qu'irriter et raffiner davantage les personnes à
qui il venait de le faire péter dans la main. Par ce changement
d'intendant de Lille, il se fit un mouvement qui porta Le Blanc de
l'intendance d'Auvergne à celle d'Ypres. Je le remarque à cause de tout
ce qu'il lui arriva depuis.

Dangeau maria son fils unique à la fille unique de Pompadour qui avait
treize ans, d'une taille et d'une beauté charmante qui dure encore.
Courcillon avait vingt et un ans. J'ai assez parlé de lui et de son père
et de sa mère pour n'avoir rien à y ajouter. Ils ne pouvaient pas
trouver un plus grand parti pour leur fils, ni M. et
M\textsuperscript{me} de Pompadour un plus dans leur goût pour leur
fille qu'ils vendirent. Ils étaient riches, mais fort obérés, et
n'avaient rien à donner à leur fille. Ils étaient sans crédit et dans
l'obscurité. Loin de pouvoir raccommoder leurs affaires, c'étaient des
gens qui, avec de l'esprit l'un et l'autre, avaient sans cesse laissé
tout fondre entre leurs mains, jusqu'aux biens de la fortune, à leurs
alliances, à leur naissance, sans cesser d'être fort glorieux.
Pompadour, avec un esprit orné de beaucoup de lecture, l'avait de
travers et sans justesse, et toute sa vie avait fait autant de sottises
que de pas. Son grand-père, qu'on appelait Laurière, était frère cadet
et oncle des deux marquis de Pompadour, chevaliers de l'ordre en 1633 et
1661, le dernier mort en 1684, père de M\textsuperscript{me} de
Saint-Luc et d'Hautefort en qui la branche aînée finit. Le fils de ce
premier Laurière épousa une soeur de M. de Montausier, depuis duc et
pair et gouverneur de Monseigneur, et de ce mariage vint le marquis de
Pompadour dont il est ici question. Il était cadet et porta longtemps le
petit collet. Son aîné mourut, et M. de Montausier l'approcha de
Monseigneur\,; et lui fit donner un régiment d'infanterie et succéder à
son père qui était sénéchal et gouverneur de Périgord. C'était un homme
bien fait, qui avait même de beaux traits, mais dont la physionomie, le
maintien et toute la figure serrait le coeur de tristesse\,; elle était
toute faite pour être crieur d'enterrement. Cet extérieur ne trompait
pas, rien de si ennuyeux ni de si affligeant que tout le reste. Il se
mit à jouer gros jeu et à perdre\,; il devint amoureux de la troisième
fille de M. et de M\textsuperscript{me} de Navailles, qui ne voulurent
point de lui. Sa persévérance, le désir de la fille qui y répondait, les
instances de ses deux soeurs, celles du duc de Montausier vainquirent
enfin la résistance. La première nuit des noces ne fut pas modeste. Ils
passèrent au lit trois jours et trois nuits, et cela se réitéra souvent
dans la suite. Pompadour abandonna la guerre et puis la cour, fit le
plongeon au grand monde, et s'enterra dans une entière obscurité. Il
vendit son gouvernement et mit ses affaires dans le plus grand désordre.
Sans se lasser l'un de l'autre, l'ennui leur prit enfin de leur état,
leur fille leur parut propre à les en tirer, en la mariant, non pour
elle, mais pour eux.

La duchesse douairière d'Elboeuf, qui les aimait par les respects
infinis qu'ils, lui rendaient, vivait beaucoup avec
M\textsuperscript{me} de Dangeau à la cour, et lui faisait la sienne par
rapport à M\textsuperscript{me} de Maintenon. Elle imagina ce mariage
pour leur plaire et pour s'ancrer de plus en plus. Dangeau, riche et
jouissant de gros du roi, était en état d'attendre les biens d'une
belle-fille dont l'alliance l'honorait infiniment, et à laquelle il ne
serait pas parvenu s'il y avait eu du bien présent. C'était à l'âge de
M\textsuperscript{me} de Maintenon une occasion à ne pas perdre pour
obtenir des grâces qui lui fissent faire un mariage sans s'incommoder.
M\textsuperscript{me} de Maintenon aimait extrêmement
M\textsuperscript{me} de Dangeau, et plût à Dieu qu'elle n'eût approché
d'elle que des femmes de ce caractère\,! Elle n'osait oublier d'avoir
été accueillie par la mère de M\textsuperscript{me} de Navailles, et
chez elle longtemps en arrivant d'Amérique, et elle se piquait d'amitié
pour M\textsuperscript{me} d'Elboeuf. Par la même raison elle ne pouvait
ne pas favoriser M\textsuperscript{me} de Pompadour sa soeur. Le mariage
se fit donc sans rien donner à la fille, seule héritière, en tirant le
père et la mère d'obscurité, qu'on vit naître à la cour à leur âge comme
des champignons. Dangeau avec l'agrément du roi et de Monseigneur céda
sa place de menin à Pompadour, et son gouvernement de Touraine à son
fils, et M\textsuperscript{me} de Dangeau sa place, de dame du palais à
sa belle-fille, que depuis longtemps sa santé et ses privantes ne lui
laissaient plus guère exercer, et le roi lui fit la galanterie de lui
conserver sa pension de six mille livres de dame du palais, sans qu'elle
le demandât, et sans préjudice de celle de sa belle-fille. Voilà donc
les Pompadour initiés tout à coup à la cour, à Marly, à Meudon, chez
M\textsuperscript{me} de Maintenon quelquefois. La femme, qui avait été
belle, avait toujours été désagréable. Jamais elle n'avait ouvert les
yeux qu'à moitié. C'était une précieuse de quartier avec un esprit
guindé et une politique accablante\,; toutefois avec de l'esprit et fort
polie. Ils ne bougèrent de chez Dangeau. L'union entre eux fut
continuelle. Ceux-là y mettaient la protection, les autres les respects
et les adorations jusque des escapades de leur gendre qui se moquait
d'eux avec peu de ménagement. Parmi tout cela leur contentement à tous
fut extrême et durable.

On sut presque en même temps le mariage de Lanjamet avec la fille d'un
procureur à Paris qu'il avait longtemps entretenue, puis épousée il y
avait trois ou quatre ans secrètement. Elle avait eu de la beauté, mais
de l'esprit et de l'intrigue comme quatre démons, de la méchanceté et de
la noire scélératesse comme quatorze diables. Ce Lanjamet avait aussi
beaucoup d'esprit, quelque petite intrigue et de la valeur. Il avait été
longtemps lieutenant au régiment des gardes. C'était de ces insectes de
cour qu'on est toujours surpris d'y voir et d'y trouver partout, et dont
le peu de conséquence fait toute la consistance. C'était un fort petit
homme, vieillot, avec grand nez de perroquet, étrangement élevé et
recourbé qui lui tenait tout le visage\,; qui parlait, s'intriguait,
décidait et se fourrait partout où il trouvait des maisons ouvertes, et
fort peu d'autres le voulaient recevoir.

Je ne sais par quel prodige il avait fait une campagne aide de camp du
roi, qui lui avait donné un petit gouvernement en Bretagne. Il tenait
ses assises chez M\textsuperscript{me} de Ventadour, chez la duchesse du
Lude et chez M. le Grand. Il ne sortait point de ces lieux-là, et
{[}allait{]} fort peu en d'autres. Sa fatuité se rebecquait à l'écart en
insolence, mais ménagée avec art, quand il n'était pas content des gens.
Il était familier à manger dans la main. Avec tout cela, c'était un
Breton qui n'était pas gentilhomme, et à qui les états en firent un jour
l'affront. M. de La Trémoille qui présidait me le conta. Il voulut faire
opiner la noblesse. Les voix s'élevèrent confusément et crièrent qu'on
fit sortir qui n'avait pas droit d'opiner, qu'ont les plus pauvres et
plus jeunes gentilshommes. M. de La Trémoille jeta les yeux partout, et
dit qu'il ne voyait là personne qui n'eût droit d'opiner. À ce mot
toutes les voix se mirent à crier\,: «\,Lanjamet\,! Lanjamet\,! qu'il
sorte ou nous n'opinerons point\,;» et tout de suite Lanjamet sortit
sans se défendre et sans prononcer un mot. Son effronterie de s'être
fourré là pour s'en faire après un titre fut payée de cet affront. Il ne
parut plus depuis aux états, mais il n'en revint pas moins impudent à la
cour\,; c'est-à-dire à Versailles, car il n'était pas sur le pied de
Marly et de Meudon. Cette aventure apprit à M. de La Trémoille qu'il
n'était pas gentilhomme. Sa femme, galante et veuve aussi d'un
procureur, fut pour lui, quelque néant qu'il fût, un mariage honteux. Il
ne laissa pas de la produire chez M. le Grand, dont par la suite elle
brouilla toute la famille, et s'en fit chasser, et de presque partout où
son mari l'avait fourrée. Depuis la mort, du roi, je ne sais ce qu'ils
sont devenus, et je n'en ai ouï parler que sur cette brouillerie qui la
fit chasser avec éclat de chez M. le Grand.

Louville se maria aussi dans ce temps-ci. Depuis son retour d'Espagne,
il n'avait songé qu'à raccommoder ses affaires, se bâtir très
agréablement, mais sagement, à Louville, et vivre à Paris avec ses amis
sans regret à la fortune, et comme si elle ne lui eût jamais présenté
des cours et des royaumes à gouverner. Il chercha à se marier sagement
aussi. Il épousa une fille de Nointel, conseiller d'État, frère de la
duchesse de Brissac et de la femme de Desmarets, contrôleur général, et
dans une grande liaison avec lui. La noce s'en fit à Bercy chez le
gendre de Desmarets, qui, outre les familles, fut honorée de la
meilleure compagnie. Il eut le bonheur d'épouser une femme bien faite,
vertueuse, sensée, gaie, entendue, qui vécut comme un ange avec lui, et
qui ne songea qu'à ses devoirs et à entretenir ses amis, quoique
beaucoup plus jeune, et qui se fit aimer, estimer et considérer partout.
Nointel était {[}fils{]} de Béchameil, surintendant de Monsieur, duquel
j'ai parlé ailleurs\footnote{Phrase omise dans les précédentes éditions.}.

Le prince de Léon n'espérant plus de ravoir sa comédienne, et pris par
famine, non seulement consentit, mais désira se marier. Son père et sa
mère, qui avaient pensé mourir de peur qu'il n'épousât cette créature,
ne le souhaitaient pas moins. Ils songèrent à la fille aînée du duc de
Roquelaure qui devait être extrêmement riche un jour, et qui bossue et
fort laide, ayant dépassé la première jeunesse, ne pouvait guère espérer
un parti de la naissance du prince de Léon qui serait duc et pair, et à
qui cinquante mille écus de rente étaient assurés, sans les autres biens
qui le regardaient. Une si bonne affaire de part et d'autre s'avança
jusqu'à conclusion\,; mais, sur le point de signer, tout se rompit avec
aigreur par la manière altière dont la duchesse de Roquelaure voulut
exiger que le duc de Rohan donnât plus gros à son fils. Il en était
justement très mécontent. Il était taquin encore plus qu'avare\,; lui et
sa femme se piquèrent, tinrent ferme et rompirent. Voilà les futurs au
désespoir\,; le prince de Léon, qui craignait que son père ne traitât
des mariages sans dessein de les faire pour ne lui rien donner\,; la
prétendue, dans la frayeur de l'avarice de sa mère qui ne la marierait
point et la laisserait pourrir dans un couvent. Elle avait plus de
vingt-quatre ans, elle, avait beaucoup d'esprit, de ces esprits hardis,
décidés, entreprenants, résolus. Le prince de Léon en avait plus de
vingt-huit. On a vu, il n'y a pas longtemps, quel était son caractère.

Dalles de Roquelaure étaient au faubourg Saint-Antoine, aux Filles de la
Croix, où M. de Léon avait eu la permission de voir celle qu'il devait
épouser. Dès qu'il sentit leur mariage rompu il courut au couvent, il
l'apprit à M\textsuperscript{lle} de Roquelaure, fit le passionné, le
désespéré\,; lui persuada que jamais leurs pères et mères ne les
marieraient, et qu'elle pourrirait au couvent. Il lui proposa de n'en
être pas les dupes, qu'il était prêt à l'épouser si elle voulait y
consentir\,; que ce n'était point eux qui avaient imaginé leur mariage,
mais leurs parents qui l'avaient trouvé convenable, et que leur avarice
rompait\,; que, dans quelque colère qu'ils entrassent, il faudrait bien
qu'ils s'apaisassent, et qu'ils demeureraient mariés et affranchis de
leurs caprices\,; en un mot, il lui en dit tant qu'il la persuada, et
encore qu'il n'y avait pas un moment à perdre. Ils convinrent de leurs
faits pour que la fille pût recevoir de ses nouvelles, et il s'en alla
donner ordre à l'exécution de ce projet. M\textsuperscript{me} de
Roquelaure et M\textsuperscript{me} de La Vieuville, qui fut depuis dame
d'atours de M\textsuperscript{me} la duchesse de Berry, étaient de tout
temps les deux doigts de la main, et M\textsuperscript{me} de La
Vieuville était l'unique personne à qui, ou à l'ordre de qui
M\textsuperscript{me} de Roquelaure avait permis à la supérieure de la
Croix de confier ses filles, ensemble ou séparément, toutes les fois
qu'elle les irait prendre ou qu'elle les enverrait chercher. M. de Léon,
qui en était instruit, fait ajuster un carrosse de même forme, grandeur
et garniture semblable à celui de M\textsuperscript{me} de la Vieuville,
avec ses armes et trois habits de sa livrée, un pour le cocher, deux
pour les laquais\,; contrefait une lettre de Mine de La Vieuville avec
un cachet de ses armes\,; et envoie cet équipage avec un laquais des
deux bien instruit porteur de la lettre aux Filles de la Croix, le mardi
matin, 29 mai, à l'heure qu'il savait que M\textsuperscript{me} de La
Vieuville les envoyait chercher quand elle les voulait avoir.
M\textsuperscript{lle} de Roquelaure, qui avait été avertie, porte la
lettre à la supérieure, lui dit que M\textsuperscript{me} de La
Vieuville l'envoie chercher seule, et si elle n'a rien à lui mander.

La supérieure accoutumée à cela, et la gouvernante aussi, ne prirent pas
la peine de voir la lettre, et, avec le congé de la supérieure, sortent
sur-le-champ, et montent dans le carrosse qui marcha aussitôt, et qui
s'arrêta au tournant de la première rue, où le prince de Léon attendait,
qui ouvrit la portière, sauta dedans, et voilà le cocher à fouetter de
son mieux, et la gouvernante, presque hors d'elle de ce qui arrivait, à
crier de toute sa force. Mais au premier cri, M. de Léon lui fourra un
mouchoir dans la bouche, qu'il lui tint bien ferme. Ils arrivèrent de la
sorte, et en fort peu de temps, aux Bruyères, près du Ménilmontant,
maison de campagne du duc de Lorges, élevé {[}avec le prince de Léon{]},
et de tout temps son ami intime, qui les y attendait, avec le comte de
Rieux, dont l'âge et la conduite s'accordaient mal ensemble, et qui
était venu là pour servir de témoin avec le maître du logis. Il avait un
prêtre interdit et vagabond, Breton, tout prêt à les marier. Il dit la
messe, et fit la célébration sur-le-champ, puis mon beau-frère mena ces
beaux époux dans une belle chambre. Le lit et les toilettes y étaient
préparées. On les déshabilla, on les coucha, on les laissa seuls deux ou
trois heures, on leur donna ensuite un bon repas, après lequel ils
mirent l'épousée dans le même carrosse qui l'avait amenée, et sa
gouvernante qui se désespérait. Elles rentrèrent au couvent.
M\textsuperscript{lle} de Roquelaure s'en alla tout délibérément dire à
la supérieure tout ce qui venait de se passer\,; et sans la moindre
émotion des cris, qui de la supérieure et de la gouvernante gagnèrent
bientôt toute la maison, s'en alla tranquillement dans sa chambre écrire
une belle lettre à sa mère, pour lui rendre compte de son mariage,
l'excuser et lui en demander pardon.

On peut juger de ce que, la duchesse de Roquelaure put devenir à cette
nouvelle. La gouvernante, tout éperdue qu'elle était, lui écrivit en
même temps tous les faits, la ruse, la violence qu'elle avait soufferte,
sa justification comme elle put, ses désespoirs. M\textsuperscript{me}
de Roquelaure, dans sa première fureur, ne raisonne point, croit que son
amie l'a trahie, court chez elle, la trouve, et dès la porte se met à
hurler les reproches les plus amers. Voilà M\textsuperscript{me} de La
Vieuville dans un étonnement sans pareil, qui lui demande à qui elle en
a, ce qui peut être arrivé, et parmi les sanglots et les furies n'entend
rien et comprend encore moins. Enfin, après une longue et furieuse
quérimonie, elle commence à découvrir le fait, elle le fait répéter,
expliquer, proteste d'injure, qu'elle n'a pas songé à
M\textsuperscript{lle} de Roquelaure, fait venir tous ses gens en
témoignage que son carrosse n'est point sorti de la journée, ni qu'aucun
de ses gens n'est allé au couvent. M\textsuperscript{me} de Roquelaure,
toujours en furie, en reproches, qu'après l'avoir assassinée elle
l'insulte encore et veut se moquer d'elle\,; l'autre à dire et à faire
tout ce qu'elle peut pour l'apaiser, et à se mettre en furie à son tour
de la supercherie qu'on lui a faite. Enfin, après avoir été très
longtemps sans s'entendre, puis sans se calmer, M\textsuperscript{me} de
Roquelaure commença enfin à se persuader de l'innocence de son amie\,;
et toutes deux à jeter feu et flammes contre M. de Léon, et contre ceux
qui l'avaient aidé à lui faire cette injure. M\textsuperscript{me} de
Roquelaure était particulièrement outrée contre M. de Léon, qui pour la
mieux amuser, l'avait continuellement vue depuis la rupture avec des
respects et des assiduités qui l'avaient gagnée, en sorte que,
nonobstant l'aigreur avec laquelle l'affaire s'était rompue, l'amitié
entre elle et lui s'était de plus en plus réchauffée avec promesse
réciproque de durer toujours. Elle était en ragée contre sa fille, non
seulement de ce qu'elle avait commis, mais de la gaieté et de la liberté
d'esprit qu'elle avait marquée aux, Bruyères, et des chansons dont elle
avait diverti le repas.

Le duc et la duchesse de Rohan aussi furieux, mais moins à plaindre,
firent de leur côté un étrange bruit. Leur fils, bien en peiné de se
tirer de ce mauvais pas, eut recours à sa tante de Soubise, pour
s'assurer du roi dans une affaire qui ne pouvait pas lui être
indifférente, quelque mal qu'elle fût avec son frère. Elle l'envoya à
Pontchartrain trouver le chancelier\,; il y arriva le lendemain de ce
beau mariage à cinq heures du matin, comme le chancelier s'habillait, à
qui il demanda conseil `et secours. Il l'exhorta à faire l'impossible
pour fléchir son père, et surtout M\textsuperscript{me} de Roquelaure,
et cependant de tenir le large. À peine avaient-ils commencé à parler,
que M\textsuperscript{me} de Roquelaure lui manda qu'elle était au haut
de la montagne, où elle le priait de lui venir parler. Ils étaient de
tout temps extrêmement amis. Elle avait appris en chemin que le prince
de Léon avait passé pour aller à Pontchartrain. Elle ne voulut pas se
commettre à l'y voir\,; c'est ce qui la fit arrêter à un demi-quart de
lieue où le chancelier vint aussitôt à cheval la trouver. Il monta dans
son carrosse, et y trouva la fureur même. Elle lui dit qu'elle n'était
pas venue lui demander conseil, mais lui rendre compte, comme à son ami,
de ce qu'elle allait faire, et verser sa douleur dans son sein, et comme
au chef de la justice la lui demander tout entière. Le chancelier lui
laissa tout dire, puis voulut lui parler à son tour\,; mais, dès qu'elle
sentit qu'il la voulait porter à quelque raison, elle s'emporta de plus
en plus, et de ce pas s'en alla tout droit à Marly, où le roi était, et
dont elle n'était pas ce voyage. Elle y descendit chez la maréchale de
Noailles\,; la grand'mère paternelle du maréchal de Noailles était fille
du maréchal de Roquelaure, et l'envoya dire son malheur à
M\textsuperscript{me} de Maintenon, et la conjurer qu'elle pût voir le
roi en particulier chez elle. En effet, elle y entra sur la fin du dîner
du roi, par les fenêtres du jardin qui étaient toutes des portes, et
comme au sortir de table le roi y entra à son ordinaire, suivi de ce qui
avait coutume d'y être admis à ces heures-là, M\textsuperscript{me} de
Maintenon alla au-devant de lui contre sa coutume, lui parla bas, et
l'emmena sans s'arrêter dans sa petite chambre, dont elle ferma la porte
aussitôt. M\textsuperscript{me} de Roquelaure se jeta à ses pieds et lui
demanda justice du prince de Léon dans toute son étendue. Le roi la
releva avec la galanterie d'un prince à qui elle n'avait pas été
indifférente, et chercha à la consoler\,; mais, comme elle insistait
toujours à demander justice, il lui demanda si elle connaissait bien
toute l'étendue de ce qu'elle voulait, qui n'était rien moins que la
tête du prince de Léon. Elle redoubla toujours ses mêmes instances, quoi
que le roi lui pût dire, tellement que le roi lui promit enfin que,
puisqu'elle le voulait, elle aurait justice tout entière, et qu'il la
lui promettait. Avec cela, et force compliments, il la quitta et repassa
droit chez lui, d'un air fort sérieux, sans s'arrêter à personne.

Monseigneur, les princesses et ce peu de dames qui étaient dans le
premier cabinet avec lui et elles, qui entraient toujours dans la petite
chambre, et qui cette fois étaient demeurés avec les dames, ne pouvaient
comprendre ce qui causait cette singularité unique, et l'inquiétude se
joignit à la curiosité en voyant repasser le roi comme je viens de dire.
Le hasard avait fait que personne n'avait vu entrer
M\textsuperscript{me} de Roquelaure, et ils, en étaient {[}là{]} lorsque
M\textsuperscript{me} de Maintenon sortit de la petite chambre, et
apprit à Mgr et à M\textsuperscript{me} la duchesse de Bourgogne de quoi
il s'agissait. Cela se répandit incontinent dans la chambre, où la bonté
de la cour brilla incontinent dans tout son lustre. À peine eut-on
plaint un moment M\textsuperscript{me} de Roquelaure, que les uns par
aversion des grands airs impérieux de cette pauvre mère, la plupart
saisis du ridicule de l'enlèvement d'une créature que l'on savait très
laide et bossue par un si vilain galant, s'en mirent à rire et
promptement aux grands éclats, et jusqu'aux larmes avec un bruit tout à
fait scandaleux. M\textsuperscript{me} de Maintenon s'y abandonna comme
les autres, et corrigea tout le mal sur la fin en disant que cela
n'était guère charitable, d'un ton qui n'était pas monté pour imposer.
Elle avait ses raisons pour avoir des égards pour M\textsuperscript{me}
de Roquelaure, et cependant pour ne l'aimer pas\,; du duc de Rohan, ni
de son fils, elle ne s'en souciait, en façon du monde. La nouvelle gagna
incontinent le salon et y reçut tout le même accueil. Néanmoins, après
avoir bien ri, la réflexion et l'intérêt propre (et il y avait là bien
des pères et des mères, et des gens qui le pouvaient devenir) rangea
tout le monde du côté de M\textsuperscript{me} de Roquelaure\,; et, à
travers les moqueries et la malignité, il n'y eut personne qui ne la
trouvât, fort à plaindre, et n'excusât sa première furie.

Nous étions demeurés à Paris, M\textsuperscript{me} de Saint-Simon et
moi, et nous savions avec tout Paris cet enlèvement fait la veille, mais
nous ignorions tout le reste, surtout le lieu où le mariage s'était
fait, et la part que M. de Lorges y avait, lorsque, le surlendemain de
l'aventure, je fus réveillé à cinq heures du matin en sursaut, et vis en
même temps ouvrir mes fenêtres et mes rideaux, et M\textsuperscript{me}
de Saint-Simon et son frère devant moi. Ils me contèrent tout ce que je
viens de dire, au moins pour l'essentiel de l'affaire\,; un homme de
beaucoup d'esprit et de capacité, qui avait soin des nôtres, entra en
robe de chambre, avec qui ils allèrent, consulter, tandis qu'ils me
firent habiller et mettre les chevaux au carrosse. Je ne vis jamais
homme si éperdu que le duc de Lorges. Il avait avoué le fait à
Chamillart qui l'avait envoyé à Doremieu, avocat alors fort à la mode,
qui l'avait extrêmement effrayé. En le quittant, il accourut au logis
pour nous faire aller à Pontchartrain\,; et, comme les choses les plus
sérieuses sont très souvent accompagnées de quelques circonstances
ridicules, il vint frapper de toutes ses forces à un cabinet qui était
devant la chambre de M\textsuperscript{me} de Saint-Simon. Ma fille
était assez malade, elle la crut plus mal, et, dans la pensée qui la
saisit d'abord que c'était moi qui frappais ainsi, elle accourut
m'ouvrir. La vue de son frère l'épouvanta doublement. Elle s'enfuit dans
son lit, où il la suivit pour lui conter sa déconvenue. Elle sonna pour
faire ouvrir ses fenêtres et voir clair, et justement elle avait pris la
veille une jeune fille de la Ferté, de seize ans, qui couchait dans le
cabinet, de l'autre côté, joignant sa chambre. M. de Lorges, pressé de
son affaire, lui dit de se dépêcher d'achever d'ouvrir, de s'en aller et
de fermer sa porte. Voilà une petite créature troublée, qui prend sa
robe et son cotillon, qui monte chez une ancienne femme de chambre qui
l'avait donnée, qui l'éveille, qui veut dire, qui n'ose, et qui enfin
lui conte ce qui lui vient d'arriver, et qu'elle a laissé au chevet du
lit de M\textsuperscript{me} de Saint-Simon un beau monsieur, tout
jeune, tout doré, frisé et poudré, qui l'a chassée fort vite de la
chambre. Elle était toute tremblante et fort étonnée. Elles surent
bientôt qui c'était. On nous en fit le conte en partant, qui nous
divertit fort malgré l'inquiétude.

Le chancelier nous raconta les visites matinales qu'il avait eues la
veille et ce qui s'y était passé. Il nous conseilla fort l'évasion du
prêtre et de tous ceux qui pouvaient témoigner, la soustraction des
signatures, et une négative bien résolue, avec quoi il nous assura que
M. de Lorges n'avait rien à craindre. Delà nous allâmes à l'Étang, où
nous trouvâmes Chamillart fort déplaisant d'une si désagréable affaire,
mais peu alarmé. Le roi avait ordonné qu'on lui rendît compte de tout,
et à mesure, de chaque pas et de chaque procédure. Tout cela passait par
Pontchartrain qui devenait par là un peu le modérateur des juges\,; et
moyennant sa femme qui lui avait écrit, peut-être beaucoup plus par le
mouvement que M\textsuperscript{me} de Soubise s'était donné, nous
étions sûrs de lui. Nous revînmes à Paris descendre chez
M\textsuperscript{me} la maréchale de Lorges, fort persuadés que nous
n'en aurions que la peine\,; nous y apprîmes que le prêtre et les valets
étaient déjà évadés, et qu'on travaillait à faire disparaître l'acte et
les signatures. M\textsuperscript{me} de Roquelaure avait fait partir
Montplaisir, lieutenant des gardes du corps, fort galant homme et leur
ami particulier, pour aller porter cette fâcheuse nouvelle au duc de
Roquelaure à Montpellier, qui fut, s'il se peut, plus furieux que sa
femme. Toutefois, après de grands vacarmes, tant à Paris qu'en
Languedoc, on commença à comprendre que le roi, qui voulait être si
exactement et si continuellement informé de tout sur cette affaire,
n'abandonnerait pas au déshonneur public la fille de
M\textsuperscript{me} de Roquelaure, ni beaucoup moins à l'échafaud, ou
à la mort civile en pays étranger, le propre neveu de
M\textsuperscript{me} de Soubise.

Le duc et la duchesse de Foix, soeur de Roquelaure, commencèrent à
adoucir sa femme et lui ensuite. Eux et leurs amis leur firent peur de
la difficulté des preuves juridiques, des volontés de porter l'affaire à
la dernière extrémité de rigueur, de la honte et de la rage du démenti
après l'avoir entreprise et suivie\,; et peu à peu les rendirent
capables d'entendre dire qu'il valait encore mieux faire un mariage
convenable en soi, qu'eux-mêmes avaient voulu, que de s'exposer à ces
cruels inconvénients et à déshonorer leur fille. Le rare fut que le duc
et la duchesse de Rohan se rendirent les plus épineux. Le mari était
plein de chimères\,; il n'eût pas été fâché de voir son fils, dont il
avait toujours été mécontent, aller tenter fortune et s'établir en
Espagne. La mère, qui avait une grande prédilection pour le second,
aurait été bien aise d'en faire l'aîné. Ils ne se soucièrent donc point
de hasarder le succès ni de hâter la délivrance de leur fils, réduit à
se tenir caché\,; et n'eurent point de honte de chercher à profiter du
malheur de M. et de M\textsuperscript{me} de Roquelaure, et de leur
tenir le pied sur la gorge pour en tirer plus que ce dont ils s'étaient
contentés lorsque le mariage avait pensé être conclu, et qui ne s'était
rompu sur le combien de la dot. Ils voulurent encore exiger des
conditions plus fortes\,; il se fit plusieurs négociations là-dessus. Le
chancelier, ami de M\textsuperscript{me} de Roquelaure, et le duc
d'Aumont, à la prière du prince de Léon, s'étaient mêlés du mariage la
première fois. La même raison les y fit entrer la seconde, mais à bout
avec des gens incapables d'aucune considération, la combustion entre les
deux maisons devenait inévitable, si le roi, à la prière de
M\textsuperscript{me} de Soubise, n'eût fait ce qu'il n'avait fait de sa
vie. Il entra lui-même dans tous les détails particuliers\,; il pria,
puis commanda en maître. Il manda à diverses fois le duc et la duchesse
de Rohan qui n'y voulaient point aller, leur parla tantôt séparément
dans son cabinet, tantôt ensemble et longtemps avec une grande bonté,
quoiqu'il ne les aimât guère, et une grande patience\,; et finalement
leur donna le duc d'Aumont et le chancelier, non plus pour arbitres,
mais pour juges des conditions du mariage qu'il leur déclara vouloir
absolument être fait et célébré avant qu'il allât à Fontainebleau.

Sur le compte que le chancelier et le duc d'Aumont rendirent que le duc
et surtout la duchesse de Rohan ne voulaient demeurer d'accord en rien,
ni finir, le roi envoya chercher M\textsuperscript{me} de Rohan, et lui
déclara, après tout ce qu'il put d'honnête, que les choses n'en étaient
pas venues où elles en étaient pour en demeurer là, et qu'il en eût le
démenti\,; et que, si elle et son mari ne consentaient, il saurait bien
achever validement le mariage sans feux par son autorité souveraine,
dans une conjoncture de cette qualité. Il permit ensuite au prince de
Léon de le venir remercier, et lui demander pardon de toutes ses
fautes\,; et finalement après tant de bruit, d'angoissés et de peines,
le contrat fut signé par les deux familles assemblées chez la duchesse
de Roquelaure, mais fort tristement. Les bans furent publiés, et avec la
permission du cardinal de Noailles, qui ne se donne guère, les deux
familles se rendirent à l'église du couvent de la Croix, où
M\textsuperscript{lle} de Roquelaure était gardée à vue depuis son beau
mariage par cinq ou six religieuses qui se relayaient. Elle sortit du
dedans et entra dans l'église\,; le prince de Léon par une autre porte
en même temps, sans compliments de personne, car cela avait été concerté
ainsi, et qu'ils ne se diraient mot. Le curé dit la messe et les maria.
La cérémonie finie, chacun signa, et sans se dire une parole chacun s'en
alla de son côté. Les mariés montèrent ensemble dans un carrosse pour se
rendre à quelques lieues de Paris chez un financier, des amis du prince
de Léon, en attendant qu'ils eussent une maison dans Paris, où ils
payèrent leur folie d'une cruelle indigence, qui ne finit presque
qu'avec leur vie, n'ayant presque pas survécu ni l'un ni l'autre le duc
de Rohan et M. et M\textsuperscript{me} de Roquelaure. Ils ont laissé
plusieurs enfants.

Pour être correct, il faut ajouter que tout fut signé et consommé avant
Fontainebleau, mais que le duc de Rohan, qui était tombé malade de
dépit, et qui ne voulut jamais donner que douze mille livres de rente à
son fils, quoique M\textsuperscript{me} de Roquelaure en offrît dix-huit
mille si M. de Rohan voulait aller jusque-là, profita de l'empressement
du roi pour en obtenir des lettres patentes, qui, nonobstant toute règle
du royaume et toutes lois et coutumes de Bretagne, qui n'y permettent
aucune substitution, lui permissent d'en faire une graduelle à l'infini
de tous ses biens de Bretagne, où les cadets et les filles seraient fort
maltraités. M\textsuperscript{me} de Soubise et M\textsuperscript{me} de
Roquelaure emportèrent ce consentement, qui ne coûtait rien au roi,
après quoi il fallut faire la substitution. Il se passa encore deux mois
à cet ouvrage, pendant lesquels le roi envoya plus d'une fois le duc
d'Aumont au duc de Rohan pour le presser de finir, et le manda à
Fontainebleau pour l'en presser lui-même. Enfin cet ouvrage fut achevé
au bout de deux mois, les, lettres patentes expédiées et enregistrées
comme il le voulut, et le mariage célébré immédiatement après en la
manière que je l'ai rapportée.

\hypertarget{chapitre-xiv.}{%
\chapter{CHAPITRE XIV.}\label{chapitre-xiv.}}

1708

~

{\textsc{Cardinal de Bouillon à Rouen et à la Ferté.}} {\textsc{- Sa
vanité et ses misères.}} {\textsc{- Baluze publie son Histoire de la
maison d'Auvergne, fondée surtout sur le faux cartulaire de Brioude,
dont le fabricateur se tue dans la Bastille.}} {\textsc{- Départ des
princes pour l'armée de Flandre.}} {\textsc{- Duc de Bourgogne à
Cambrai.}} {\textsc{- Conduite du roi d'Angleterre, incognito à l'armée
de Flandre.}} {\textsc{- Villars à la cour\,; son dépit et sa morale.}}
{\textsc{- Hanovre, général des Impériaux sur le Rhin.}} {\textsc{-
Orage sur la Moselle.}} {\textsc{- Armée de Flandre de Mgr le duc de
Bourgogne.}} {\textsc{- Duc d'Enghien nommé à seize ans chevalier de
l'ordre.}} {\textsc{- Voyage de Fontainebleau par Petit-Bourg.}}
{\textsc{- État désespéré de M\textsuperscript{me} de Pontchartrain\,;
son mari résolu à la retraite.}} {\textsc{- Mort de
M\textsuperscript{me} de Pontchartrain.}} {\textsc{- Folies et faussetés
de son mari.}}

~

Le cardinal de Bouillon, outré de succomber dans toutes les entreprises
qu'il avait tentées pour se soumettre la congrégation réformée de Cluni,
et des insultes qu'il en recevait en personne, ne put durer davantage à
Cluni, à Paray, ni dans ces environs. Il obtint permission d'aller
passer quelque temps à Rouen, où son abbaye de Saint-Ouen lui donnait
des affaires, mais ce fut à condition de prendre sa route de telle sorte
qu'il n'approchât de nulle part plus près de trente lieues de Paris et
de la cour. Il demanda la passade à plusieurs personnes dont les maisons
étaient plus commodes que les méchants cabarets d'une route de traverse.
Il eut le dépit d'être refusé de la plupart, entre autres de La
Vrillière, qui ne crut pas de la politique d'héberger un exilé qui avait
déplu au roi avec tant d'éclat et d'opiniâtreté. Il me lit demander par
l'abbé d'Auvergne d'être reçu à la Ferté. Je ne crus pas devoir être si
scrupuleux. La parenté si proche de M\textsuperscript{me} de Saint-Simon
avec les Bouillon, l'intimité qui avait été entre eux et M. le maréchal
de Lorges toute sa vie, la manière dont ils en avaient usé dans mon
procès au conseil, puis à Rouen, contre le duc de Brissac, les
sollicitations publiques que j'avais faites avec eux au grand conseil
pour la coadjutorerie de Cluni et ses suites, m'engagèrent d'en user
autrement. Ils en furent fort touchés. Le cardinal séjourna chez moi
quelques jours, d'où il s'en alla à Rouen, où la singularité du
caractère et la proximité d'Évreux le fit recevoir avec beaucoup
d'empressement et de respects. Mais sa vanité extrême gâta tout. Il eut
une bonne et grande table où il convia beaucoup de gens, mais il la fit
tenir par deux ou trois personnes qui lui étaient là particulièrement
attachées, et mangea toujours seul sous prétexte de santé\,; mais cette
persévérante diète eu démasqua bientôt l'orgueil. Sa table devint
déserte, bientôt après sa maison, et chacun s'offensa d'une hauteur
inconnue, même aux princes du sang.

En même temps que cette fierté indigna, la faiblesse de ses plaintes ne
lui attira pas l'estime. Sa situation lui était insupportable, et il ne
pouvait s'en cacher. Elle le fit tomber dans un inconvénient tout à fait
misérable. Il s'avisa de se faire peindre, et beaucoup plus jeune qu'il
n'était. Le monde ne l'avait pas encore déserté à Rouen, il y en avait
beaucoup dans sa chambre lorsqu'il dit au peintre qu'il fallait ajouter
le cordon bleu à son portrait, parce qu'il le peignait dans un âge où il
le portait encore. Cette petitesse surprit fort la compagnie. Elle la
fut bien davantage lorsque le cardinal, voyant qu'on se mettait en soin
d'en chercher quelqu'un pour le faire voir au peintre, dit qu'il n'était
pas besoin d'aller si loin, et se déboutonnant aussitôt, en montra un
qu'il portait par-dessous, pareil à celui qu'il portait par-dessus avant
que le roi lui eût fait redemander l'ordre. Le silence des assistants le
fit apercevoir de ce qui se passait en eux. Il en prit occasion d'une
courte apologie pleine de vanité, et d'une explication des droits de la
charge de grand aumônier.

Il prétendit n'en être pas dépouillé, parce qu'il n'en avait pas donné
la démission, que cela était si vrai, que, pour ne pas embarrasser la
conscience des maisons religieuses et hôpitaux soumis à sa juridiction
comme grand aumônier, il avait donné tous ses pouvoirs aux cardinaux de
Coislin et de Janson, comme à ses vicaires, lorsqu'ils étaient entrés
dans sa charge\,; mais il n'ajouta pas qu'ils s'étaient bien gardés
d'agir dans ces maisons en vertu de ces pouvoirs qu'ils n'avaient jamais
demandés, et qu'ils avaient parfaitement méprisés. À l'égard de l'ordre,
il dit que les deux charges de grand aumônier de France et de grand
aumônier de l'ordre étant unies, et ayant prêté le serment des deux, il
ne s'était pas cru délié de l'obligation de porter le cordon bleu et la
croix du Saint-Esprit\,; mais que, par déférence pour lé roi, il se
contentait de les porter par-dessous, et sans que cela parut. Avec cette
délicatesse de conscience, ou plutôt avec cette misère de petit enfant,
que faisait-il donc de la croix brodée\,? La portait-il aussi sur sa
veste et par-dessous\,? Cette platitude et tout son discours acheva de
le faire tomber dans l'esprit de ceux qui en furent témoins et de ceux
qui l'apprirent. La privation de ces marques extérieures était une des
choses du monde qui le touchaient le plus\,; et comme il n'osait
continuer de les mettre à ses armes, il avait cessé depuis d'en avoir
nulle part, en sorte que sa vaisselle et ses carrosses, tout n'était
marqué que par des chiffres et des tours semées, sans écussons. C'était
pour la même raison qu'il n'allait plus qu'en litière, sous prétexte de
commodité. Il en avait une superbement brodée dedans et dehors, qui
avait un étui pour la pluie et pour aller par pays.

Il fut visité à Rouen par fort peu de gens, de sa famille ou de ses
amis. Il s'y occupa des affaires de son abbaye de Saint-Ouen, mais
beaucoup plus du sieur Marsollier, chanoine d'Uzès, à qui la Vie du
cardinal Ximénès avait donné de la réputation, que celle qu'il fit
depuis de M. de la Trappe n'a pas soutenue, et qu'il faisait travailler
à celle de M. de Turenne. Pendant ce séjour à Rouen, il perdit encore un
procès fort important contre lés réformés de Cluni, et fort piquant. Il
ne put se rendre maître de son désespoir, et acheva de se faire mépriser
en Normandie comme il avait fait en Bourgogne. À la fin il eut ordre de
s'y en retourner. Nouvelle rage. Il me fit demander encore passage par
la Ferté, et quelques jours de séjour pour y faire des remèdes plus en
repos qu'il ne l'eût pu à Rouen. Tout était ruse, dessein et fausseté.
Il revint donc à la Ferté, où je ne lui envoyai personne pour le
recevoir, pour ne pas excéder dans ce qui ne devait être qu'hospitalité
à un exilé de sa sorte. Il y montra autant de faiblesse sur sa santé que
sur sa fortune. Il était charmé du parc, où il se promenait beaucoup,
mais il rentrait toujours avant l'heure du serein et couchait dans ma
chambre, mangeait avec deux ou trois de ses gens dans mon antichambre,
et ne sortait point de ces deux pièces, parce qu'elles ne donnaient
point sur l'eau comme toutes les autres. Il disait quelquefois la messe
à la chapelle, quelquefois à la paroisse. En sortant de l'église il lui
échappait souvent de dire à ce qui s'y trouvait\,: «\,Regardez et
remarquez bien ce que vous voyez ici, un cardinal-prince, doyen du sacré
collège, le premier après lé pape, qui dit la messe ici\,; voilà ce que
vous n'avez jamais vu et ce que vous ne reverrez plus après moi.\,»
Jusqu'au peuple riait à la fin de cette vanité si déplorable.

Il alla à la Trappe, où l'amertume extrême de son état, qu'il témoigna
sans cesse à l'abbé et à M. de Saint-Louis qui avait été fort connu,
aimé et estimé de M. de Turenne, et que lui-même connaissait fort, leur
fit grande pitié et ne les édifia pas. M. de Saint-Louis, qui, après
avoir mérité l'estime et les grâces du roi qui en parlait toujours avec
bonté et distinction, s'était retiré là, où depuis près de trente ans il
n'était occupé que de prière et de pénitence, essaya vainement de le
ramener un peu, et à la fin lui parla de la mort, de ce qu'on pense
lorsqu'on y arrive, et de l'utilité de se représenter ce terrible
moment. «\,Point de mort, point de mort\,! s'écria le cardinal, monsieur
de Saint-Louis, ne me parlez point de cela, je ne veux point mourir. »
Je m'arrête sur ces diverses bagatelles pour faire connaître quel était
ce personnage si rapidement élevé au plus haut, lui personnellement de
sa maison, par les grâces et la faveur de Louis XIV\,: un homme qui a
fait tant de bruit dans le monde par son orgueil, par son ambition, qui
a paru si grand tant qu'il a été porté par cette même faveur, qui a
donné le plus étonnant spectacle par ses fausses adresses, son
ingratitude et la lutte de désobéissance qu'il osa soutenir contre ce
même roi, son bienfaiteur, et par ses propres bienfaits, et qui depuis
sa disgrâce parut si petit, si vil, si méprisable jusque dans les
pointes qu'il hasarda encore, d'où il tomba dans le plus grand mépris
partout et jusque dans Rome, où nous le verrons languir pitoyablement et
y mourir enfin d'orgueil, comme toute sa vie il en avait vécu. De la
Ferté il dépêchait des courriers sans cesse\,; il lui est arrivé de s'y
trouver avec trois ou quatre valets, tous les autres étant en course. Il
y fut visité de quelques gens d'affaires. L'abbé de Choisy, si connu
dans le grand monde, le même qui s'alla faire prêtre à Siam, dont on a
une si agréable relation de ce voyage, et des lambeaux assez curieux de
Mémoires, était de ses amis de tous les temps. Il passa plusieurs jours
à la Ferté, d'où il fit un voyage à Chartres.

Ce séjour à la Ferté dura plus de six semaines. Il avait projeté de
faire entrer M. de Chartres dans ses affaires, malgré tout ce qui
s'était passé dans celle de M. de Cambrai. Il était de toute sa vie
vendu aux jésuites, qui de leur côté lui étaient livrés. Il crut donc
qu'en mettant M\textsuperscript{me} de Maintenon de son côté par M. de
Chartres, le roi ne pourrait tenir, attaqué, de ces deux côtés. Il fit
ce qu'il put pour s'attirer une visite de M. de Chartres qui était à
Chartres, à dix lieues de la Ferté. N'ayant pu l'obtenir, il se borna à
un rendez-vous quelque part comme fortuit, il n'y réussit point encore.
Il voulait engager ce prélat à faire revoir par le roi l'important
procès qu'il venait de perdre et qui l'avait si fort piqué, pour de là
l'embarquer. Ce fut l'objet du voyage de l'abbé de Choisy, qui y perdit
toute son insinuation, son esprit et son bien-dire. Il revint à la Ferté
avec force compliments, mais chargé de refus sur tout. On ne peut
exprimer quels furent les transports de rage avec lesquels ils furent
reçus, ni tout ce que vomit le cardinal de Bouillon contre un homme si
distant de lui, devant lequel il s'était humilié, et en avait
inutilement imploré la protection contre ses prétendus ennemis, contre
le roi, contre les ministres, contre ses amis. Ce dernier trait de
mépris acheva de lui tourner la tête. Il comprit son exil sans fin et
les dégoûts journaliers, inépuisables, sans secours, sans ressource,
sans espérance d'aucun moyen d'adoucir sa situation, beaucoup moins de
la changer. Je sus tout cela par le curé de la Ferté, qui était homme
d'esprit et savant, avec lequel il s'était familiarisé dans ses
promenades, qu'il avait même fait manger quelquefois avec lui, lui qui
n'avait pas voulu manger avec ce qu'il y avait de plus distingué à
Rouen, et devant lequel il ne se cachait pas. J'ai lieu de croire, mais
sans en être certain, que ce fut l'époque de la résolution qu'il exécuta
près de deux ans après, parce qu'il lui fallut tout ce temps pour
arranger dessus toutes ses affaires. Outre la consolation de se trouver
{[}dans{]} un lieu agréable, d'entière solitude et de parfaite liberté,
où choqué ni contraint sur rien, il faisait tout ce qu'il lui plaisait à
son aise, il attendait sans le dire le départ de la cour pour
Fontainebleau.

Ce long séjour que je n'avais pu prévoir ne laissait pas de me mettre en
peine, et je craignais que le roi, si justement piqué contre lui, ne le
trouvât mauvais. J'en parlai au chancelier et à M. de Beauvilliers\,; je
leur dis mon embarras, je leur fis aisément comprendre que je ne pouvais
chasser le cardinal de Bouillon de chez moi\,; que, comme il était vrai,
je n'avais jamais eu avec lui aucun commerce et n'en avais encore
actuellement aucun. Je me trouvai bien d'avoir pris cette précaution. À
fort peu de jours de là, il fut parlé, au conseil, du cardinal de
Bouillon à propos de ses procès perdus contre ces moines. Là-dessus le
roi dit qu'il était bien longtemps à la Ferté\,; que, si on voulait le
chicaner, on ne l'y laisserait pas\,; qu'il n'avait pas permission
d'approcher plus près de trente lieues, et qu'il n'y en a que vingt de
Versailles à la Ferté. Le chancelier saisit ce mot, et après lui le duc
de Beauvilliers pour me servir, et il parut que cela fut bien reçu.
Enfin, la cour arrivée à Fontainebleau, le cardinal de Bouillon partit
aussi de la Ferté, sans que pas un de ses gens sussent où il allait. Il
prit des chemins détournés, et il arriva enfin, toujours dans le même
secret réservé à lui seul, à Auny près de Pontoise, où il demanda à
coucher et où il fut reçu. C'était une maison de campagne du maréchal de
Chamilly, qui était alors à la Rochelle avec sa femme, où il commandait
et dans les provinces voisines, à qui il n'en avait ni écrit ni fait
parler. C'était s'approcher de Paris bien plus que de la Ferté\,; la
cause en fut pitoyable.

Il avait le prieuré de Saint-Martin de Pontoise, où il avait dépensé des
millions et fait une terrasse admirable sur l'Oise et des jardins
magnifiques. Il aima tant cette maison, et encore par vanité, car je lui
ai ouï dire que tout ce qui était des dehors était royal, que dans sa
faveur il obtint, moyennant un échange, de détacher cette maison et
quelques dépendances du prieuré et d'en faire un patrimoine, qui en
effet, est demeuré à M. de Bouillon. Il n'avait pu avoir permission d'y
aller, et voulut au moins la revoir encore une fois par la chatière\,;
et il donna le misérable spectacle de l'aller considérer tous les jours,
pendant les sept ou huit qu'il demeura à Auny, tantôt de dessus la
hauteur, tantôt tout autour par les ouvertures des murailles des bouts
des allées, et à travers des grilles, sans avoir osé mettre le pied en
dedans, soit qu'il voulût faire pitié au monde, par cette ridicule
montre d'un extrême désir dont la satisfaction lui était refusée, soit
qu'il espérât toucher par le respect de n'être pas entré dans sa maison
ni dans ses jardins. Cette bassesse fut méprisée, et ce fut tout. De là
il tira droit en Bourgogne, d'où il était venu, où il reçut enfin la
permission de s'en aller tout auprès de Lyon s'établir dans une maison
de campagne qui lui fut prêtée, pour n'être plus parmi des objets qui
l'octroient sans cesse de douleur.

Baluze dont j'ai parlé, et de son \emph{Histoire de la maison
d'Auvergne} fondée sur les faussetés du cartulaire de Brioude, dont j'ai
parlé (t. V, p.~322), avait presque toujours été avec le cardinal de
Bouillon à Rouen. Son livre, prêt à paraître en 1706, avait été remis
sous clef alors par l'étrange vacarme qu'excita l'imposture du
cartulaire de Brioude, et l'arrêt de mort de la chambre de l'Arsenal
contre le faussaire de Bar, convaincu de l'avoir fabriqué, et dont les
Bouillon eurent le crédit de faire commuer la peine en une prison
perpétuelle à la Bastille, où il avoua qu'ils le lui avaient fait faire.
Depuis quinze mois de cet événement, il ne s'en parlait plus. L'ouvrage
de Baluze, fait avec tout l'art possible, séparé de tout cet espace de
temps de son ruineux fondement, parut aux Bouillon pouvoir enfin se
montrer. Le chancelier leur ami, et sujet quelquefois à traiter les
choses un peu légèrement, leur en accorda le privilège. Il parut donc en
public, et y renouvela toute la scène du faussaire. Savants et
ignorants, le soulèvement fut général, et le mondé indigné ne se
contraignit ni sur les Bouillon ni sur le chancelier, qui leur avait
passé cette impression. Je ne pus m'empêcher de lui en dire mon avis. Il
en fut honteux à ne savoir où se mettre\,; et les Bouillon, avec toute
leur hardiesse, fort embarrassés. Ce fut à propos de ce nouvel éclat,
que Maréchal me conta que de Bar, désespéré de se voir confiné en prison
pour le reste de sa vie, malgré les assurances de protection infaillible
et des récompenses dont les Bouillon l'avaient repu pour lui faire
exécuter cette insigne fausseté, et lassé de ces imprécations contre eux
si inutiles, s'était cassé la tête contre les murailles\,; que lui,
Maréchal, avait été appelé pour le visiter dans cette furie et dans
cette blessure, de laquelle il était mort deux jours après.

Le roi, qui avait la faiblesse de ne partir jamais un vendredi, ne fut
pas si scrupuleux pour son petit-fils. Il fixa son départ au 14 mai. Il
semblerait néanmoins qu'à qui observerait les jours, celui de
l'assassinat d'Henri IV et de la mort de Louis XIII devrait être réputé
un jour malheureux pour la France, pour ses rois et pour ceux qui en
sont si récemment sortis. Mais le roi, qui n'a jamais compté que lui
pour roi de France, put s'apercevoir en cette occasion que sa cour ne le
comptait pas seul, malgré ses adorations. La messe du roi qui, selon la
coutume, fut de \emph{Requiem}, frappa tout le monde et l'attrista sur
le départ du jeune prince et {[}on{]} ne s'en put contenir. Je n'en fus
pas témoin\,; j'étais à Saint-Denis à l'anniversaire de celui, dont par
mon père, je tiens toute ma fortune\,; c'est à son exemple un devoir qui
l'emporte sur tout autre, et auquel je n'ai jamais manqué. Il est vrai
que je m'y suis toute ma vie trouvé tout seul, et que je n'ai jamais pu
m'accoutumer à un oubli si scandaleux de tant de races comblées par ce
grand monarque, dont plus d'une sans lui seraient inconnues et demeurées
dans le néant. À mon retour de Versailles, je trouvai qu'on y était
encore blessé du choix de ce jour funeste.

Mgr le duc de Bourgogne était parti à une heure après midi pour aller
coucher à Senlis, chez l'évêque, frère de Chamillart, dont toute la
famille était allée l'y recevoir. Il passa à Cambrai avec les mêmes
défenses de la première fois, mais il y dîna. À la vérité ce fut à la
poste même, où l'archevêque se trouva avec tout ce qui était à Cambrai.
On peut juger de la curiosité de cette entrevue, qui fut au milieu de
tout le monde. Le jeune prince embrassa tendrement son précepteur à
plusieurs reprises. Il lui dit tout haut qu'il n'oublierait jamais les
grandes obligations qu'il lui avait, et sans jamais se parler bas, ne
parla presque qu'à lui, et le feu de ses regards lancé dans les yeux de
l'archevêque, qui suppléèrent à tout ce que le roi avait interdit,
eurent une éloquence avec ces premières paroles à l'archevêque\,; qui
enleva tous les spectateurs, et qui, malgré la disgrâce, grossirent
alors et depuis la cour de l'archevêque de tout ce qui était de plus
distingué et qui\,; sous divers prétextes, de route et de séjour,
s'empressait à mériter d'avance ses bonnes grâces présentes et sa
protection future.

M. le duc de Berry partit le 15, dîna à Senlis chez l'évêque, ne passa
point par Cambrai, et joignit Mgr le duc de Bourgogne à Valenciennes le
soir même qu'il y était arrivé. C'était là qu'était M. de Vendôme depuis
son arrivée de la cour, et là qu'était le rendez-vous de tout le monde.
Le roi d'Angleterre ne tarda pas de s'y rendre dans un incognito si
précis toute la campagne, qu'il en devint scandaleux. Il mangea chez Mgr
le duc de Bourgogne jusqu'à l'arrivée de son équipage. Il eut après chez
lui une table de seize couverts où il invitait et où il fut très
gracieux, et mangea chez les officiers généraux qui l'en prièrent. Il
choisit son poste, bien que volontaire, à la tête des troupes de sa
nation, qui en furent comblées. Jusqu'aux Anglais de l'armée ennemie
s'en sentirent de la satisfaction, et la laissèrent échapper. Ce prince
vécut avec beaucoup de sagesse, mais fort parmi tout le monde, chercha à
plaire et y réussit. Il acquit même l'estime et l'affection des troupes
et des généraux par, son application et par toute la volonté qu'il
montra. Il ne figura pas assez pour s'y étendre davantage. L'électeur
gagna les bords du Rhin où le duc de Berwick l'était allé attendre.

Villars arriva avec sa femme presque à ses journées, fort lentement. Il
parut outré de changer de pays et d'armée. Il lui fâchait fort de
quitter de si abondantes sauvegardes, et n'était guère plus content de
ne pouvoir traîner sa femme après lui. Elle en était ravie. Il lui
échappa assez plaisamment qu'elle avait quitté le service. Villars
assura le roi publiquement que tous ses bataillons en Allemagne
excédaient le complet de cinq cents hommes chacun, et qu'ils étaient
tous beaux à merveilles\,; puis s'étant mis peu à peu sur la morale, et
toujours en public et parlant au roi, il dit tout haut que la meilleure
maxime pour les rois était de faire espérer beaucoup et de donner peu.
Je laisse à penser comment ce mot fut reçu d'un compagnon de sa sorte,
élevé et comblé au point où il se trouvait. L'électeur et Berwick ne
trouvèrent pas leur armée à beaucoup près telle que Villars la publiait,
mais ce dernier ne s'était pas contraint de dire publiquement, et plus
d'une fois, en parlant des puissances, que, s'il ne leur fallait que du
plat, de la langue, il leur en donnerait tout leur soûl. À cette fois,
il tint exactement parole.

Les Impériaux furent lents à s'assembler. Le duc d'Hanovre, depuis roi
d'Angleterre, commandait leur armée. Il comptait qu'elle serait
nombreuse et que le prince Eugène l'y suivrait bientôt. Ce dernier
partit fort tard de Vienne, s'amusa chez divers princes en chemin, forma
un puissant corps sur la Moselle, et sourd aux cris d'Hanovre, se fit
joindre par de gros détachements de son armée, par des ordres précis de
l'empereur, qui eut peine à apaiser M. d'Hanovre piqué et voulant s'en
retourner chez lui. Pour le dire de suite, dès que cette armée de la
Moselle ne put plus donner soupçons de torquets, l'électeur et Berwick
laissèrent à du Bourg la garde des lignes d'Haguenau, avec le nécessaire
pour les défendre contre les entreprises du duc d'Hanovre, et marchèrent
avec tout le reste sur la Moselle, où il se forma un gros orage dont on
ne put deviner la cause, tandis que Marlborough, à la tête de l'armée de
Flandre, se tenait dans une grande tranquillité. On prétendit qu'il
était convenu avec le prince Eugène d'attendre qu'il fût prêt, et de ne
rien entreprendre sans lui.

L'armée de Mgr le duc de Bourgogne était d'abord de deux cent six
escadrons et de cent trente et un bataillons en cinquante-six brigades.
Il avait la maison du roi, la gendarmerie, les carabiniers et le
régiment des gardes, dix-huit lieutenants généraux, et autant de
maréchaux de camp en ligne, sans les gens du détail. Dix sont devenus
depuis maréchaux de France, dont quatre n'étaient lors que brigadiers\,;
et nous en voyons aussi qui n'étaient pas de cette armée et qui
n'étaient alors que colonels. L'armée se trouva complète, belle, leste,
de la plus grande volonté. Jamais armée fournie avec plus d'abondance,
ni d'amas de toutes les sortes, avec un prodigieux équipage de vivres et
d'artillerie. Tout ce qui y servait se pressa d'arriver sur le départ
des princes. Il ne restait plus qu'à se mettre en mouvement. M. de
Vendôme, qui prenait aisément racine partout où il se trouvait à son
aise, montra peu de complaisance pour en sortir. Il fut seul de son
avis, mais il se fit croire avec Un air de supériorité dont Puységur
prévit les suites, et les écrivit au long à M. de Beauvilliers, qui ne
me cacha pas ses alarmes. Je le fis souvenir de notre conversation de
Marly, mais je le trouvai encore fort éloigné de penser que les choses
pussent aller jusqu'où je les lui avoir prédites. Profitons de
l'inaction de ce premier commencement de campagne pour raconter le peu
qui se passa jusqu'à sa véritable ouverture, qui {[}ne{]} nous permettra
guère après de la quitter.

Le roi nomma à la Pentecôte M. le duc d'Enghien chevalier de l'ordre
pour le premier jour de l'an. Il n'avait que seize ans, et M. le Duc n'y
songeait pas encore\,; mais il était fils de M\textsuperscript{me} la
Duchesse.

Le roi alla coucher le 18 juin à Petit-Bourg, et le 19 à Fontainebleau.
M\textsuperscript{me} de Pontchartrain était à Paris à l'extrémité. Ma
liaison intime avec cette famille, et plus encore l'union et l'intimité
plus que de soeurs qui était entre M\textsuperscript{me} de Saint-Simon
et elle, nous arrêta à Paris. Elle ne voyait presque plus personne, et
n'avait de consolation qu'avec M\textsuperscript{me} de Saint-Simon, qui
n'en trouvait aussi qu'auprès d'elle. Le caractère de cette femme
accomplie tiendrait trop de place ici\,; il la trouvera mieux parmi les
Pièces\footnote{On voit par ce passage que les morceaux renvoyés aux
  Pièces par Saint-Simon étaient quelquefois de sa composition\,; c'est
  un motif de plus pour regretter que le public ne puisse encore
  profiter de cette partie de"ses oeuvres. Voy. t. I°, p.~437, note.}.
Il est trop beau, trop singulier, trop instructif pour le laisser
ignorer. Il y avait longtemps qu'une si grande perte était prévue.
C'était une maladie de femme venue de trop de couches et trop près à
près, de trop peu de ménagement d'abord, qui rendit tous les divers
remèdes inutiles. Pontchartrain, qui avait là-dessus bien des reproches
à se faire, en pouvait combler la mesure par la contrainte continuelle
dans tout, et par son étrange humeur qu'il lui avait fait essuyer sans
cesse. La patience et la douceur dont elle ne s'était jamais lassée,
jusqu'à être outrée lorsqu'on pouvait s'apercevoir qu'elle en avait
besoin, avait infiniment pris sur elle, et fort aigri son sang, qu'on ne
put enfin calmer ni arrêter. Soit vérité, soit feinte, comme dans les
suites cela ne parut que trop, Pontchartrain sentit toute la grandeur de
sa perte, et plus d'un an avant qu'elle arrivât, il me confia que si ce
malheur, qu'il ne prévoyait que trop, lui arrivait, il avait pris le
dessein de se retirer\,; que, dès qu'il la verrait diminuer, il
tiendrait sa démission toute prête\,; que, dès que le malheur serait
arrivé, il l'enverrait au roi et se retirerait aussitôt dans un petit
appartement que son père avait à l'institution de l'Oratoire, où il
passait les bonnes fêtes\,; qu'il y demeurerait trois ou quatre mois
jusqu'à ce qu'il se fût déterminé à un lieu et à un genre de vie qui lui
convînt et qu'il pût continuer, sur quoi il exigea de moi un secret
inviolable.

Il serait inutile de rapporter ici ce que je lui dis pour détourner un
homme de son âge et chargé de famille d'une résolution si téméraire. Je
compris que je ne gagnerais rien que par degrés. Quoiqu'il n'eût rien
que de très rebutant, et que je le sentisse tel plus souvent que
personne, parce que je le voyais plus souvent et plus intimement,
j'avoue que je {[}fus{]} dupe, et qu'il me fit pitié. Je crus que la
confiance de son père, qui ne me cachait rien, ni des affaires, ni de sa
famille, et qui cent fois m'avait déposé ses douleurs sur son fils\,;
que celle de sa mère, qui n'était pas moindre\,; que cette intime
liaison de sa femme avec la mienne\,; que l'intérêt de ses enfants
demandaient également de moi tous les soins possibles pour détourner une
résolution qui serait un coup de mort pour le chancelier et la
chancelière, et qui serait la perte de leur famille. Bientôt après je
crus démêler qu'outre que ces sortes de résolutions sont souvent le
fruit des grandes douleurs, il imaginait en devoir une signalée à une si
grande perte, et que, privé de l'appui qu'il tirait de la considération
de sa femme, il désespérait de pouvoir se soutenir dans sa place. Ces
mélanges, qui venaient de la sensibilité du coeur et de l'orgueil de
l'esprit, me parurent former une résolution bien difficile à rompre. Je
ne crus donc pas faire une infidélité de communiquer ce secret à
M\textsuperscript{me} de Saint-Simon pour me servir de son sage conseil.
Elle en jugea comme moi. Lui-même bientôt après s'en ouvrit à elle.
Cette inquiétude me fit quitter bonne compagnie, et mes ouvrages de la
Ferté et mes plants que j'étais allé voir à Noël, sur un accident qu'on
crut qui emporterait M\textsuperscript{me} de Pontchartrain, pour
accourir à temps d'empêcher la démission. J'avais résolu de tâcher à la
faire passer par les mains du chancelier. Cela lui était dû par toutes
sortes de raisons, et c'était le meilleur moyen de l'arrêter.

La maladie, qui dura encore six mois, donna le temps à Pontchartrain de
s'ouvrir au P. de La Tour, général de l'Oratoire, qui confessait
M\textsuperscript{me} de Pontchartrain depuis son mariage, et à l'abbé
de Maulevrier, aumônier du roi, grand intrigant, avec de l'esprit et de
l'ambition, grand ami des jésuites et de M. de Cambrai, de qui j'ai
parlé quelquefois. Celui-ci le détourna de se retirer à l'institution
pour ne point faire cette peine aux jésuites, auxquels il était aussi
livré que son père était éloigné d'eux, et pour ne point donner de soi
des soupçons de jansénisme, qui pourraient attirer les affaires au P. de
La Tour, lequel aussi le détermina à s'en aller à Pontchartrain quand le
malheur serait arrivé, puis à différer sa démission de quelques
semaines, enfin de quelques mois. Il y en avait près de deux que nous ne
bougions presque point de cette funeste maison, lorsque
M\textsuperscript{me} de Pontchartrain mourut enfin sur les onze heures
du matin, le 25 juin. La cour était à Fontainebleau, le chancelier aussi
qui n'avait pu quitter, que sa femme désolée alla trouver aussitôt\,;
qui le trouva dans la plus amère affliction quoique prévue de si loin.
M\textsuperscript{me} de Saint-Simon, que j'avais eu soin de détourner
adroitement d'un si douloureux spectacle, avait, malgré sa vertu, besoin
de toutes sortes de secours. Je voulus demeurer auprès d'elle. Elle
savait où en était Pontchartrain et l'importance pour ses, enfants, ou
plutôt pour ceux de son amie, d'empêcher les folies qu'il voulait
exécuter, et me pressa tellement de ne le point abandonner, que je la
laissai avec lime la maréchale de Lorges, Rime de Lauzun et ma mère, et
m'en allai, sur un message pressant du P. de La Tour, le trouver chez
Pontchartrain, d'où, pour abréger beaucoup de choses, nous partîmes tous
trois en même carrosse, et Bignon, intendant des finances, en quatrième,
et nous en allâmes à Pontchartrain. Les trois belles-soeurs y vinrent le
jour même, et peu à peu la parenté et les liaisons y introduisirent plus
de monde.

Dans la situation où était toute cette famille, le chancelier et la
chancelière, qui n'aimaient point les belles-soeurs avec qui j'étais
fort bien, n'avaient de confiance qu'au P. de La Tour et en moi, et
Pontchartrain, qui voulait toujours parler de sa retraite qui n'était
sue là que de nous, laissait toute la compagnie pour être sans cesse
avec nous. Cela me força à demeurer pour arrêter toujours cette
résolution, jusqu'à ce que, Bignon, prêt à partir pour Fontainebleau,
cette résolution lui fût confiée pour la déclarer au chancelier, mais
sans porter de démission. Alors voyant l'affaire entre les mains du
chancelier, je m'en revins à Paris auprès de M\textsuperscript{me} de
Saint-Simon, et le P. de La Tour retourna à ses affaires. Ce ne fut pas
pour longtemps. Le chancelier, outré de plus d'une douleur, et de colère
contre son fils, sut le rapport de Bignon, m'écrivit la lettre du monde
la plus touchante pour me conjurer de n'abandonner pas ce fou dans ses
transports, et pour me témoigner qu'il n'avait de ressource qu'au P. de
La Tour et en moi, ni de repos qu'il ne me sût à Pontchartrain. Je
différai pourtant d'y retourner.

Phélypeaux cependant, frère du chancelier, arrivant de Bourbon, avait
été à Pontchartrain, où son neveu lui avait parlé comme à Bignon, et
l'avait aussi chargé de déterminer son père, qui lui avait écrit très
fortement et plusieurs fois, à le laisser faire. Phélypeaux, tout
apoplectique qu'il était revenu des eaux, ne put rien gagner sur son
neveu. Il se traîna à Fontainebleau où il acheva d'effaroucher son frère
par tous les détails qu'il lui rapporta, et de l'outrer contre son fils.
Il m'écrivit par son frère une lettre si forte et si pressante pour
retourner à Pontchartrain, que je ne pus m'en défendre, mais en même
temps si précise d'en chasser les belles-soeurs et toute la compagnie,
que je crus qu'elle excédait. Le fait était que, encore que le
chancelier travaillât avec le roi en la place de son fils, les affaires
périssaient faute de signatures et de manutention ordinaire\,; que le
roi, qui est l'homme du monde à qui les afflictions allaient le moins,
commençait à s'en lasser jusqu'à le trouver mauvais\,; que la cour en
parlait fort et blâmait en ridicule\,; que ce qui s'amassait de gens à
Pontchartrain, quoique parenté ou familiers, y donnait un air
d'assemblée et de fête tout à fait déplacé\,; d'appareil de spectacle,
et faisait une sorte d'amusement à son fils qui le retenait où il ne
devait pas être, et qui scandalisait par le contraste et le ridicule
éloigné de toute la bienséance de son état. Surtout le chancelier
insistait sur ce que son fils allât enfin à Fontainebleau, ce qu'il
s'éloignait entièrement de faire. Phélypeaux me fit une triste peinture
de l'état où il avait laissé son frère sur la ruine de sa famille et de
sa fortune\,; et, outre la lettre qu'il m'avait apportée, me conjura
encore de la part du chancelier de vouloir bien retourner à
Pontchartrain pour, tâcher d'en arracher son fils. À tant d'instances
M\textsuperscript{me} de Saint-Simon joignit ses représentations les
plus fortes de ne pas refuser un service si important qui m'était
demandé avec tant d'instance et de confiance. Je me résolus donc à y
retourner, mais avec le P. de La Tour, et en nous faisant précéder par
l'abbé de Maulevrier, à qui le chancelier avait parlé très fortement à
Fontainebleau, dès qu'il le sut instruit par son fils même.

Cet abbé qui aimait tant à se mêler de tout, et si principalement chez
les ministres, qui était sec, était chargé d'essayer de ramener l'esprit
de Pontchartrain aux volontés de son père, et d'insinuer à la compagnie
de s'en aller, belles-soeurs et autres. Nous le laissâmes partir et
n'allâmes que le lendemain, le P. de La Tour et moi. Nous trouvâmes que
l'abbé, armé des ordres du père et de la mère, ne les avait adoucis, ni
à la compagnie, ni aux belles-soeurs même, ni au fils. Ces trois femmes,
qui ignoraient pleinement le dessein de leur beau-frère, ne cherchaient
qu'à lui plaire, à profiter d'une douleur qui les réunissait, peut-être
à le soustraire tout à fait de père et de mère pour disposer de lui plus
à leur gré, et en tirer plus gros qu'elles ne faisaient, bien qu'elles
ne s'y fussent jamais épargnées. Elles lui firent des plaintes amères du
traitement scandaleux qu'elles recevaient pour l'amour de lui.
Pontchartrain, de longue main impatient des moindres apparences de joug,
frappé de l'idée de s'unir plus étroitement à ce qui était de plus
proche à sa femme, piqué d'honneur de plus, s'emporta d'une façon
étrange, s'opposa nettement au départ, et n'eut pas peine à arrêter des
personnes qui ne voulaient s'en aller que pour être retenues. L'abbé de
Caumartin nous vint conter l'histoire en descendant de carrosse, sur
quoi le P. de La Tour et moi jugeâmes qu'il n'était plus du tout
question d'exécuter ce que le chancelier m'avait si précisément demandé
par sa lettre et par son frère, mais d'adoucir l'irritation que l'abbé
de Maulevrier avait causée.

Le P. de La Tour aborda Pontchartrain, tandis que j'allai trouver les
dames. J'essuyai d'abord une sortie de la comtesse de Roucy\,; je
m'adressai à M\textsuperscript{me} de Blansac comme plus liante, mais
qui, avec infiniment d'esprit et une apparente douceur, était encore
bien plus fausse\,; et n'en allait que mieux à ses fins\,; je leur
abandonnai la sécheresse de l'abbé de Maulevrier tant qu'elles
voulurent\,; je leur dis que le chancelier, qui trouvait toujours son
fils si bien avec elles, espérait de sa solitude un retour nécessaire à
la cour, en un mot, je les apaisai et leurs maris. L'abbé de Maulevrier
s'en retournait à Fontainebleau. Je le chargeai d'une lettre pour le
chancelier en secret, qui m'en écrivit plusieurs avec la même
précaution. Les déclamations, les désespoirs, les égarements, les
raisonnements sans raison et sans fin de Pontchartrain\,; ses fureurs,
ses menaces, et parmi tout cela, ses emportements contre son père,
uniquement mais sans cesse partagés entre le P. de La Tour et moi, nous
mettaient sans cesse aussi à bout d'expédient, de patience et de
compassion. Je n'osais me laisser aller au soupçon de quelque feinte. Le
P. de La Tour, moins scrupuleux que moi, m'en parla. Nous nous y
confirmâmes. Les belles-soeurs crurent y voir clair à des vapeurs, à des
hurlements, à des transports qui leur parurent peu naturels. Elles s'en
ouvrirent même à nous.

Jusqu'aux valets l'écumèrent et ne s'en turent pas. Quoique nous
eussions obtenu enfin qu'il fit des signatures pressées, son père
s'impatientait cruellement. Il m'écrivit une lettre si vive, si touchée
de la perte commune, si éloquente sur ses malheurs, si offensée contre
son fils et contre ses belles-sœurs, si remplie de confiance et de
reconnaissance pour moi, que m'ayant prié en même temps de la brûler
après l'avoir montrée au P. de La Tour, je crus qu'il était de cette
même confiance de la lui renvoyer. Je lui mandai nos pensées au P. de La
Tour et à moi, et j'obtins qu'il m'écrivît une lettre que je pusse
montrer à son fils, qui, sur, une réponse qu'il en avait reçue, ne
voulait plus lui écrire. Enfin, comme le P. de La Tour et moi ne savions
plus que devenir, un valet de chambre de Phélypeaux m'apporta
secrètement une lettre de la chancelière, par laquelle elle
m'avertissait qu'elle avait pris le parti de venir elle-même, sans que
personne en sût rien que son mari, et qu'elle arriverait le lendemain.
Ce parti nous plut extrêmement, au P. de La Tour et à moi, qui fut
d'avis que je lui écrivisse pour l'instruire en chemin de la situation
où elle trouverait les choses, et de ce que nous croyions de la conduite
qu'elle devait tenir. Je l'envoyai attendre par un de mes gens fort sûr,
avec ma lettre, à deux lieues de Pontchartrain, qui l'arrêta et qui la
lui donna. Elle m'en a souvent bien remercié depuis comme de chose qui
lui avait été bien utile.

Peu après le dîner, il parut deux carrosses dans la montagne qui
surprirent fort tout le monde, parce qu'on ne venait plus guère à
Pontchartrain, mais qui étonnèrent bien plus lorsqu'à leur approche on
reconnut que c'était la chancelière. Une bombe eût moins effrayé les
belles-soeurs, qui furent sur le point de s'aller cacher. Le P. de La
Tour et moi, seuls dans la confidence, fîmes si bonne contenance que
personne ne s'en douta, ni ne soupçonna depuis que nous en sussions la
moindre chose. Le P. de La Tour gagna doucement sa chambre, et moi un
corridor pour voir la réception sans contrainte. Elle fut bonne, et à la
porte du cabinet qui donne dans la cour. La mère et le fils
s'enfermèrent d'abord seuls. Phélypeaux et les deux Bignon venus avec
elle vinrent à la compagnie. Le P. de La Tour tâcha de remettre la tête
fort étourdie aux belles-soeurs. La chancelière leur fit au mieux, et
dit qu'elle n'était point venue pour chasser personne, ni pour presser
son fils sur Fontainebleau, mais pour être avec lui tant qu'il
demeurerait à Pontchartrain, et en effet pour les importuner tous si
bien de sa présence et de ses compliments, qu'elle fît finir un séjour
si ridiculement poussé. Cela réussit bientôt. Je donnai encore une
journée à la chancelière, avec qui j'eus beaucoup d'entretiens, et je
m'en revins enfin à Paris pour ne plus retourner. Peu de jours se
passèrent dans l'embarras que j'avais laissé. Les belles-soeurs,
peut-être pour se raccommoder, ou pour abréger leur ennui, furent les
premières à porter leur beau-frère au départ. Il capitula sur la
réception que lui ferait son père, sur la vie particulière qu'il voulait
mener à la cour, où il ne voulait, disait-il, demeurer qu'une année. Qui
l'eût pris au mot l'aurait bien fâché. Enfin tout le monde partit à la
fois. La mère et le fils allèrent droit à Fontainebleau, où le
chancelier se contraignit à bien recevoir son fils, mais outré de tout
ce qui s'était passé, persuadé du jeu d'affliction, et que de
Pontchartrain il avait percé jusqu'à Fontainebleau où on en parlait
trop.

La conduite qu'il y tint, les personnages ridicules et différents qu'il
y fit, les affectations de parade et cent sortes de singularités en
public, achevèrent de l'y démasquer et de l'y faire mépriser, dont le
chancelier et sa femme étaient sans cesse désolés. M\textsuperscript{me}
de Saint-Simon plus simple, mais plus intimement touchée, eut
grand'peine à se résoudre à rentrer dans sa vie accoutumée et à
retourner à la cour. J'en étais d'autant plus pressé que le roi ne
s'accommodait ni des douleurs ni des absences, et que sur les derniers
temps de la vie de M\textsuperscript{me} de Pontchartrain,
M\textsuperscript{me} de Saint-Simon s'était excusée d'une fête dont le
roi l'avait nommée, qui l'avait trouvé mauvais. Nous logions à notre
ordinaire à Fontainebleau, chez Pontchartrain, au château. Nous y fûmes
presque continuellement occupés du chancelier et de la chancelière et de
leur fils, avec eux et avec le monde. Un détail si long et si peu
intéressant paraîtra sans doute étrange, aussi m'en serais je bien gardé
sans ce qui se verra en son temps et à quoi il était tout à fait
nécessaire.

\hypertarget{chapitre-xv.}{%
\chapter{CHAPITRE XV.}\label{chapitre-xv.}}

1708

~

{\textsc{Je vais me promener vers la Loire.}} {\textsc{- Mort de la
duchesse de Châtillon.}} {\textsc{- Mort de M\textsuperscript{me} de
Razilly.}} {\textsc{- Mariage du fils du duc d'Aumont et de la fille de
Guiscard.}} {\textsc{- Mariage du roi de Portugal avec une soeur de
l'empereur, et de l'archiduc avec une princesse de
Brunswick-Blankembourg-Wolfenbüttel.}} {\textsc{- Investiture de
Montferrat au duc de Savoie.}} {\textsc{- Mort et deuil du duc de
Mantoue.}} {\textsc{- Pensions à la duchesse de Mantoue.}} {\textsc{-
Indigence et négligence de l'Espagne.}} {\textsc{- Haine de M. le Duc et
de M\textsuperscript{me} la Duchesse pour M. le duc d'Orléans, et sa
cause.}} {\textsc{- Époque de la haine implacable de
M\textsuperscript{me} des Ursins et de M\textsuperscript{me} de
Maintenon pour M. le duc d'Orléans.}} {\textsc{- Petit succès en
Espagne.}} {\textsc{- Siège et prise de Tortose.}} {\textsc{- Perte de
la Sardaigne.}} {\textsc{- Perte de Minorque et du Port-Mahon.}}
{\textsc{- Prince Eugène en Flandre.}} {\textsc{- Projet sur Bruxelles
rejeté.}} {\textsc{- Conspiration dans Luxembourg découverte.}}
{\textsc{- Gand et Bruges surpris par les troupes du roi.}} {\textsc{-
L'électeur retourne sur le Rhin, et le duc de Berwick amène une partie
de l'armée en Flandre.}} {\textsc{- Paresse et funeste opiniâtreté du
duc de Vendôme.}} {\textsc{- Combat d'Audenarde.}} {\textsc{- Insolence
de Vendôme à Mgr le duc de Bourgogne.}} {\textsc{- Parole énorme de
Vendôme à Mgr le duc de Bourgogne.}} {\textsc{- Retraite derrière le
canal de Bruges.}} {\textsc{- Belle action du vidame d'Amiens, et autre
belle de Nangis.}}

~

Quelque occupé que j'eusse été et de cette perte et de ses suites, je ne
l'avais pas moins été d'être au fait de bien des choses considérables en
leur moment, mais dont la plupart se fondent après comme les morceaux de
glace, quoique bien des choses importantes dépendent souvent de celles
qui se fondent ainsi. J'étais dans l'intime confiance de M. le duc
d'Orléans\,; et ses amis, et sa position était telle qu'il n'y avait que
moi qui pusse y être pour tout ce qui concernait la cour. J'avais grand
soin de l'informer aussi de bien des choses qui le pouvaient guider ou
qui lui pouvaient servir, et je lui écrivais en chiffres, mais par ses
propres courriers quand ils s'en retournaient, et par-ci par-là,
quelques lettres de paille, et en clair, pour amuser, par la poste ou
par les courriers de la cour. J'étais demeuré un peu en arrière de
choses dont il fallait pourtant l'informer, et j'étais si excédé de la
vie dont je sortais que je fus bien aise aussi d'un peu de dissipation.
La Vrillière s'en allait presque seul à Châteauneuf, il me pressa de l'y
aller voir. J'y consentis. Je m'y enfermai une journée entière, matin et
soir, à faire à M. le duc d'Orléans un volume en chiffres, que j'envoyai
sûrement mettre à la poste d'Orléans, pour être à l'abri de l'ouverture.
De là, j'allai voir Cheverny et sa femme dans leur belle maison de
Cheverny, Chambord qui en est tout contre, dont j'entendais toujours
parler, et que je n'enviai pas. L'évêque de Blois, qui vint à Cheverny,
m'engagea aisément d'aller voir Blois, où j'avais grande curiosité de
voir la salle des derniers états, la prison du cardinal de Guise et de
l'archevêque de Lyon, et le lieu où mourut Catherine de Médicis. Je
trouvai que, pour bâtir le château neuf, Gaston avait détruit la salle
des états, et que le contrôleur, qui occupait l'appartement de cette
funeste reine, était sorti avec la clef. Je vis aussi Menars, et j'eus
lieu d'être content de ma curiosité par la singulière beauté des
terrasses de cette maison, de la situation de l'évêché à Blois, et du
grand parti que ce premier évêque a su en tirer pour le bâtiment qu'il y
a fait. Après huit ou douze jours d'éclipse, je retournai à
Fontainebleau.

La duchesse de Châtillon mourut. C'était M\textsuperscript{lle} de
Royan, fille d'une soeur de la princesse des Ursins, et La Trémoille
comme elle, qu'elle avait élevée et mariée chez elle à Paris, dont j'ai
parlé à propos de mon mariage. Elle était devenue extrêmement grasse, et
le roi l'avait fait prier de ne venir point à la cour quand
M\textsuperscript{me} la duchesse de Bourgogne aurait des soupçons de
grossesse\,; ni quand elle serait grosse. Elle avait acquis, en
contrefaisant une religieuse du couvent où elle avait été avant de venir
chez sa tante, un tic rare et peu perceptible jusqu'à quelque temps
après son mariage, et qui depuis s'était augmenté à un point qu'à toutes
minutes son visage se démontait à effrayer, sans qu'elle-même s'en
aperçût le plus souvent par la continuelle habitude.

La femme de Razilly mourut aussi, et ce fut une perte pour son mari et
pour sa famille, qui était fort nombreuse.

Le duc d'Aumont, qui avait beaucoup mangé et qui n'était pas d'humeur à
s'en contraindre, maria Villequier, son fils unique, à la fille unique
de Guiscard, à qui Langlée, frère de M\textsuperscript{me} de Guiscard,
avait laissé un grand bien. Guiscard, outre l'honneur de cette alliance,
s'accrocha volontiers à M. d'Aumont. Il était en disgrâce depuis
Ramillies, et celle du maréchal de Villeroy ne lui promettait pas sitôt
la fin de la sienne. Villequier, avec tout ce bien, trouvait des
assaisonnements fâcheux\,: un beau-père disgracié, et ses deux frères
roués ou pendus en effigie, passés aux ennemis, et qui faisaient parler
bien mal d'eux en attendant une fin qui fut encore plus triste.

L'empereur avait fait le mariage avec le roi de Portugal d'une de ses
soeurs, qu'un frère de M. de Lorraine éonduisait à Lisbonne\,; et de
l'archiduc son frère avec une princesse de
Brunswick-Blankenbourg-Wolfenbüttel, conduite par le prince Maximilien
d'Hanovre. Toutes deux étaient en voyage, et cette dernière avait passé
Milan, où on lui avait fait une magnifique entrée, pour passer ensuite à
Barcelone, où était l'archiduc, sur la flotte anglaise commandée par le
chevalier Leake. M. de Savoie ne se pressait point de mettre en
campagne. Il se plaignait d'avoir été trompé à la précédente guerre par
l'empereur Léopold, qui ne lui avait pas tenu ce qu'il lui avait promis.
Il tint donc si ferme à demeurer les bras croisés, jusqu'à ce qu'il eût
reçu la satisfaction qu'il demandait, que l'empereur se vit forcé de
finir avec lui. Il lui donna donc l'investiture de Montferrat, au grand
regret et préjudice du droit de M. de Lorraine, et des promesses
réitérées qu'il lui en avait faites.

M. le Prince ne le trouva pas meilleur, qui y prétendait aussi après la
mort du duc de Mantoue, qui arriva le 5 juillet à Padoue assez
promptement. Il laissa beaucoup d'argent comptant\,; de vaisselle, de
pierreries, de meubles magnifiques et de beaux tableaux, mais pas un
pouce de terre, depuis que l'empereur s'était emparé de ses États. En
lui finit la branche des souverains de Mantoue. Les Gonzague l'avaient
peu à peu usurpée, comme tous ces petits souverains d'Italie, et, comme
eux, en avaient fait un État héréditaire. Il y avait encore deux
branches de Gonzague, auxquelles l'empereur n'eut aucun égard. M. de
Mantoue ne fit point de testament. M\textsuperscript{me} de Mantoue fit
donner part au roi, par l'envoyé de Bantoue de sa part à elle, qui fut
traité pour cette fois en envoyé de souverain. Le roi en prit le deuil
en noir, et le quitta au bout de cinq jours. Il envoya un gentilhomme
ordinaire faire compliment à M\textsuperscript{me} de Mantoue, à qui il
donna quarante mille livres de pension, comme elle les touchait
auparavant, sur les quatre cent mille livres qu'il donnait à M. de
Mantoue, jusqu'à son rétablissement dans ses États, et qui se retenaient
dessus pour elle. Elle eut aussi les trente mille livres de pension du
roi d'Espagne qu'il donnait à son mari. Ainsi elle eut, outre son bien,
soixante-dix mille livres de pensions. M. de Lorraine prétendit hériter
de Charleville, et fit demander au roi de trouver bon qu'il en prît
possession. M. le Prince s'y opposa fortement pour les droits de
M\textsuperscript{me} la Princesse et l'emporta.

M. le duc d'Orléans s'était arrêté à Madrid plus longtemps qu'il n'avait
cru. Rien de prêt d'aucune sorte, indigence de tout, négligence encore
plus grande. Il fallut chercher des moyens d'y suppléer, et cela n'était
pas facile\,; c'est ce qui allongea son séjour. On en prit occasion à
Paris de faire courir le bruit qu'il était amoureux de la reine. M. le
Duc, enragé de son oisiveté et de la réputation que M. le duc d'Orléans
acquérait, M\textsuperscript{me} la Duchesse, qui le haïssait pour avoir
été trop bien ensemble, se rendirent les promoteurs de ce bruit à la
cour, à la ville, et qui gagna les provinces et les pays étrangers,
excepté l'Espagne, où il n'en fut pas mention parce qu'il n'y avait ni
vérité ni apparence. M. d'Orléans y était occupé à des choses plus,
sérieuses, et plût à Dieu eût-il été moins touché de trouver des
obstacles aux choses les plus urgentes, ou que sa douleur lui eût laissé
plus d'empire sur sa langue\,! Un soir qu'après avoir travaillé tout le
jour, comme il ne faisait autre chose depuis son arrivée, à chercher des
expédients pour subvenir à l'incurie extrême de tous préparatifs les
plus indispensables pour mettre en campagne et y faire quelque chose, il
se mit à table avec plusieurs seigneurs espagnols et des François de sa
suite, tout occupé de son dépit qui tombait sur M\textsuperscript{me}
des Ursins qui gouvernait tout, et qui n'avait pas songé à la moindre
des choses concernant la campagne. Le souper s'égaya et un peu trop. M.
le duc d'Orléans, un peu en pointe de vin et toujours plein de son
dépit, prit un verre, et regardant la compagnie (je fais excuse d'être
si littéral, mais le mot ne peut se masquer)\,: «\,Messieurs, leur
dit-il, je vous porte la santé du c..-capitaine et du c..-lieutenant.\,»
Le propos saisit l'imagination des conviés\,; personne pourtant, ni le
prince lui-même, n'osa faire de commentaire, mais le rire gagna chacun
et fut plus fort que la politique. On fit raison de la santé, sans
toutefois répéter lés mots, et le scandale fut étrange.

Une demi-heure après au plus, M\textsuperscript{me} des Ursins en fut
avertie\footnote{Voy. notes à la fin du volume.}. Elle sentit bien
qu'elle était le lieutenant et M\textsuperscript{me} de Maintenon le
capitaine\,; et, si on se souvient de ce que j'ai raconté là-dessus (t.
V, p.~9 et suiv.), on verra que cela ne pouvait s'entendre autrement. La
voilà transportée de colère-, qui mande le fait en propres termes à
M\textsuperscript{me} de Maintenon, laquelle, de son côté, entra en
furie. \emph{Inde irae}. Jamais elles ne l'ont pardonné à M. le duc
d'Orléans, et nous verrons combien peu il s'en est fallu qu'elles ne
l'aient fait périr. Jusqu'alors M\textsuperscript{me} de Maintenon
n'avait ni aimé ni haï M. le duc d'Orléans, et M\textsuperscript{me} des
Ursins n'avait rien oublié pour lui plaire. Ce fut aussi ce qui la piqua
le plus, de voir qu'avec ses soins les manquements pour le service
l'avaient porté à une plaisanterie si cruelle, et qui, en un seul mot,
révélait toute sa politique avec un ridicule qui ne se pouvait effacer.
De ce moment elles jurèrent la perte de ce prince. Il se peut dire qu'il
la frisa de bien près\,; mais, échappé de ce péril, il ne cessa
d'éprouver, tout le reste de la vie du roi, et jusque dans sa mort,
combien M\textsuperscript{me} de Maintenon lui fut une implacable et
cruelle ennemie, par toutes les sortes de persécutions qu'elle lui
suscita. Ce fut encore merveilles comment il n'y succomba pas\,; mais ce
n'en fut pas une moindre que l'étrange et triste état où elle sut,
réduire un prince de son rang, état qui a même influé sur le reste de sa
vie. Il ne tarda pas à s'apercevoir du changement de
M\textsuperscript{me} des Ursins à son égard, qui n'accommoda pas les
affaires qu'elle eût voulu depuis voir périr entre ses mains. Il est des
choses qui ne se peuvent raccommoder, et il faut convenir que ce
terrible mot était supérieurement de ce genre. Aussi M. le duc d'Orléans
n'y songea-t-il pas, et alla toujours son chemin à l'ordinaire. Je ne
sais même s'il a pu s'en repentir, quelque lieu qu'il en ait eu toute sa
vie, tant il le trouvait plaisant\,; et il m'a depuis impatienté plus
d'une fois en m'en parlant, riant de tout son coeur. J'en sentais tout
le poids et toutes les cruelles suites\,; et toutefois ce qui m'en
piquait le plus, tout en le lui reprochant, je ne pouvais m'empêcher
d'en rire aussi, tant ce grand et funeste ridicule de gouvernement deçà
et delà les Pyrénées était en deux mots clairement assené et plaisamment
exprimé.

À la fin M. le duc d'Orléans trouva moyen d'entrer en campagne, mais
sans voir jamais pour plus de quinze jours à la fois, et non pas même
toujours, de subsistances assurées. Il prit au commencement de juin le
camp de Ginestar, d'où il envoya Gaëtano, lieutenant général, avec trois
mille hommes de pied et huit cents chevaux, enlever à Falcete, à cinq
lieues de Ginestar, douze cents hommes de pied, quatre cents chevaux et
mille miquelets. Ils furent surpris et se voulurent sauver dans les
montagnes, mais ils furent suivis de si près, que leur cavalerie
s'enfuit à toutes jambes, qu'on leur tua près de cinq cents hommes, et
qu'on prit, outre cinq cents hommes prisonniers, beaucoup d'officiers,
tous leurs bagages et toutes leurs munitions. Don Joseph Vallejo,
détaché du même camp sur le chemin de Tortose à Tarragone, défit la
garde de tous les bestiaux du pays amassés en un lieu, battit les
miquelets qui s'opposèrent à sa retraite, et ramena mille boeufs et six
mille moutons que M. le duc d'Orléans fit distribuer à ses troupes. Il
fit enlever encore d'autres petits postes dont on lui amena beaucoup de
prisonniers. Il en fit aussi beaucoup auprès de Tortose, enleva cinq
barques qui y portaient des farines et des chairs salées, et l'investit
le 12 juin.

Il avait établi deux ponts sur l'Èbre, l'un au-dessus, l'autre
au-dessous de la place. Sa garnison était de neuf bataillons, deux
escadrons et deux mille miquelets. La tranchée fut ouverte la nuit du 21
au 22 à demi-portée de mousquet. Le terrain, presque tout roc, causa
bien de la difficulté, les vivres en causèrent, beaucoup davantage.
D'Asfeld, longtemps depuis maréchal de France, y fit de grands devoirs
d'homme de guerre, et de soins pour la subsistance. J'ai ouï dire à M.
le duc d'Orléans qu'il n'en serait jamais venu à bout sans lui, et qu'il
était le meilleur intendant d'armée qu'il fût possible. L'artillerie et
le génie servirent si mal que M. le duc d'Orléans se voulut charger
lui-même de ces deux parties si principales, qui lui causèrent beaucoup
de soins et de peine. Un de ses ponts se rompit\,; point de bateaux, de
planches, de cordages\,; tout manquait généralement. La réparation de ce
pont, outre le temps et l'inquiétude, coûta des peines infinies à ce
prince qui en vint enfin à bout. La nuit du 9 au 10 juillet, on se logea
dans le chemin couvert. Les assiégés le défendirent fort valeureusement,
et firent après une sortie pour en déloger les assiégeants qui les
repoussèrent. Le lendemain ils capitulèrent pour livrer leurs portes, et
partir quatre jours après, et être conduits à Barcelone. Ils firent
rendre en même temps le château d'Arcès au royaume de Valence, qui était
une retraite de miquelets qui incommodait beaucoup. Ils perdirent
environ la moitié de leur garnison, et M. le duc d'Orléans environ six
cents hommes, et personne de connu que Monchamp, son major général, un
des six aides de camp que le roi envoya au roi d'Espagne en Italie, pour
veiller sur sa personne, après la découverte de la conspiration dont
j'ai parlé alors. Ce fut une perte, que ce Monchamp, en tout genre.
Lambert, dépêché par M. le duc d'Orléans, vint apprendre cette bonne
nouvelle au roi\,; qui en fut d'autant plus aise que M. le duc d'Orléans
avait surmonté toutes les difficultés possibles. En Estramadure, ni
ailleurs en Espagne, il ne se passa rien de marqué. M. le duc d'Orléans
eut la gloire de resserrer, d'écarter et de pousser même Staremberg le
reste de la campagne, quoique plus faible que lui. Mais il était dit que
chaque année serait fatale à l'Espagne, et que, semblable à un puissant
arbre usé par les siècles, il lui en coûterait ses plus grosses branche
l'une après l'autre.

J'ai parlé en son temps du duc de Veragua qui, vice-roi de Sardaigne à
l'avènement de Philippe V, fut beaucoup plus qu'accusé d'avoir voulu,
pour de l'argent, livrer cette île à la maison d'Autriche, et en perdit
sa vice-royauté. C'était un homme de beaucoup d'esprit, d'adresse et de
souplesse, qui de retour à Madrid avait trouvé moyen de se mettre si
bien avec M\textsuperscript{me} des Ursins que non seulement tout fut
oublié, mais qu'il fut fait conseiller d'État, et de plus admis aux
affaires dans le cabinet. Il avait un fils qui n'avait pas moins
d'esprit, d'art et de capacité que lui, mais dont l'extérieur tortu,
grossier, sale et laid démentait toutes ces qualités. Il s'appelait le
marquis de La Jamaïque. Il vint, à je ne sais quelle occasion, chargé
d'un compliment au roi, et il parut à tout le monde un gros vilain
lourdaud, à qui le peu d'usage de notre langue augmentait encore les
désagréments naturels. Ils étaient embarrassés en Espagne à qui confier
la Sardaigne. Elle fut offerte à La Jamaïque, qui la refusa. On capitula
avec lui, on lui promit cent mille écus, mais il ne voulait point partir
sans les avoir touchés. Dans l'impossibilité de les lui compter on eut
recours aux expédients. La Sardaigne abondait en blés, on lui permit
d'en prendre jusqu'à concurrence du payement des cent mille écus\,;
moyennant cela il partit. Barcelone, et toute la Catalogne, en souffrait
une disette extrême, toute la côte en était dépourvue, Gènes se trouvait
hors de moyens de les secourir, et la défense d'y transporter des grains
était exactement observée\,; de manière qu'on se promettait tout en
Espagne du murmure des troupes de l'archiduc et des pays qu'il avait
occupés dans cette famine.

La Jamaïque profita de la conjoncture et leur fit passer des blés en
abondance. Non content de se payer ainsi des cent mille écus qui lui
avaient été accordés en blés de Sardaigne, il voulut profiter seul de
cet étrange commerce qui rendait la vie et les forces au parti de
l'archiduc. Cette tyrannie mit au désespoir la Sardaigne, qui ne peut
vivre que de la vente de ses blés, et qui, ne pouvant fléchir l'avarice
de son vice-roi, lui préféra l'archiduc, et traita secrètement, en sorte
que cette conquête ne lui coûta que d'envoyer quelques vaisseaux se
présenter devant Cagliari. Le vice-roi, abandonné en vingt-quatre
heures, remit l'île au commandant des vaisseaux pour l'archiduc, à une
condition qu'on lui tint\,: ce fut d'être porté libre, lui et tous ses
effets, en Espagne, avec tous ceux qui le voudraient suivre. Peu de
seigneurs s'embarquèrent avec lui, et nuls autres. Le merveilleux est
qu'il fut reçu à Madrid avec acclamations. Disons d'avance que ce ne fut
pas la plus considérable perte que fit l'Espagne cette année. Le
chevalier Leake se présenta au mois d'octobre à l'île de Minorque, qui
se soumit aussitôt à l'archiduc. Le port Mahon fit très peu de
résistance, tellement que, avec cette conquête et Gibraltar, les Anglais
se virent en état de dominer la Méditerranée, d'y hiverner avec des
flottes entières, et de bloquer tous les ports d'Espagne sur cette mer.
Il est temps de parler de la Flandre.

Le prince Eugène passa la Moselle le dernier juin, embarqua son
infanterie à Coblentz, et marcha sur Maestricht. On avait eu, dans notre
armée, quelque envie de surprendre Bruxelles, et il y avait quatre mille
échelles préparées pour ce dessein. Il fallut consulter le roi, qui n'en
fut pas d'avis, et ce projet demeura sans exécution. En même temps on
découvrit une conspiration à Luxembourg. Quelques ouvriers et des gens
du peuple crurent pouvoir profiter de la maladie du comte d'Hostel,
gouverneur de la place, qui était à l'extrémité, pour y faire entrer les
ennemis. Le prince Eugène s'en était mis à portée. Druy, lieutenant
général et lieutenant des gardes du corps, très bon officier et fort
galant homme, commandait là sous le comte d'Hostel. Il fit arrêter un
boulanger qui découvrit tous les complices, qui furent pendus.

Bergheyck, cependant, cherchait les moyens de tirer quelque reste de
parti de ce grand soulèvement qu'il avait si bien concerté, qui, selon
toutes les apparences, aurait réussi, si le succès d'Écosse avait
répondu à notre attente. Le grand bailli de Gand, fort accrédité dans la
ville, y avait continué ses pratiques, et mis les choses au point
d'exécution, tandis qu'à Bruges, Bergheyck procurait aussi les mêmes
menées pour réussir à la fois. Il n'y avait pas un bataillon entier dans
ces deux places, et les bourgeois y étaient fort bien intentionnés pour
l'Espagne. L'armée de Mgr le duc de Bourgogne semblait ne songer qu'à
subsister en attendant de voir ce que feraient les ennemis. Artagnan fut
détaché le 3 juillet, avec un gros corps, sous prétexte de
subsistance\,; et le soir du même jour, Chemerault partit du camp de
Braine-l'Alleu, avec deux mille chevaux et deux mille grenadiers, pour
faire un fourrage sur Tubise, mais en effet pour marcher diligemment à
Ninove. Il s'y arrêta quelque temps, et continua après sa marché sur
Gand. À six heures du matin, le 4, il s'en trouva à une lieue, où il
reçut nouvelles de La Faye, brigadier des troupes d'Espagne. Il lui
mandait qu'il était parti la veille de Pions avec soixante officiers ou
soldats de son régiment déguisés, et qu'il était maître de la porte de
la chaussée, dont il avait eu peu, de peine à s'emparer. Là-dessus,
Chemerault avec ses troupes passa à Gand le plus diligemment qu'il put,
mais non assez pour ne pas laisser La Faye en grand danger, et le grand
bailli et ses bourgeois en grande peine. Enfin il arriva et se rendit
maître de la ville sans essuyer un seul coup, et le peuple en témoignant
sa joie.

Chemerault trouva dans la ville quantité d'artillerie et de munitions.
Il dépêcha le chevalier de Nesle à Mgr le duc de Bourgogne., qu'il
trouva sur le midi faisant faire halte à son armée sur le ruisseau de
Pepingen, qui à cette nouvelle se remit aussitôt en marche. Comme la
tête arrivait au moulin de Goiche, l'armée ennemie parut sur les
hauteurs de Saint-Martin-Lennik. On crut qu'elle venait attaquer dans la
marche. La cavalerie se mit en bataille pour donner le temps à
l'infanterie d'arriver. Tout d'un coup on vit l'armée ennemie s'arrêter
et commencer à camper. Là-dessus notre armée fila vers la Dendre. Les
ennemis détendirent et marchèrent en arrière. L'arrière-garde de Mgr le
duc de Bourgogne passa la Dendre à Ninove, le 6, à sept heures du matin,
et toute l'armée vint camper, la droite sur Alost, la gauche à l'Escaut
et à Schelebel. Deux jours après, la citadelle de Gand capitula, dont
trois cents Anglais sortirent. Gacé, fils du maréchal de Matignon,
apporta la première nouvelle au roi. Scheldon, mestre de camp, réformé
anglais, aide de camp de M. de Vendôme, et qui avait fait la
capitulation avec la citadelle, apporta la seconde\,; et en même temps
Fretteville, dépêché par le comte de La Mothe, apprit au roi qu'il
s'était rendu maître de Bruges avec la même facilité. Il n'y avait dans
le secret de cette entreprise que Bergheyck qui la procura, les deux
fils de France, le chevalier de Saint-Georges, M. de Vendôme, Puységur,
et au moment de l'exécution les conducteurs de l'entreprise. Les deux
fils de France, avec le chevalier de Saint-Georges, suivis de la
principale généralité, entrèrent avec pompe à Gand, ou, pour marquer
leur confiance, ils descendirent à l'hôtel de ville, où ils furent
magnifiquement festoyés. Ce fut une joie à Fontainebleau qui se put dire
effrénée, et des raisonnements sur les fruits de ce succès qui passaient
de bien loin le but. Je fus fort sensible à un si agréable début, mais
j'en craignis l'ivresse, et je ne pus m'empêcher de mander à M. le duc
d'Orléans ce que j'en pensais.

La marche de l'armée du prince Eugène, de la Moselle en Flandre, fit
séparer en deux celle de l'électeur qui l'avait suivie quelque temps. Il
vint de sa personne passer quelques jours à Metz, retournant à
Strasbourg. Avec ce qu'il remenait, l'armée du Rhin était de
quarante-deux bataillons et de soixante-treize escadrons\,; le duc de
Berwick mena en Flandre trente-quatre bataillons et soixante-cinq
escadrons.

Il paraissait aisé de profiter de deux conquêtes si facilement faites en
passant l'Escaut, brûlant Audenarde, barrant le pays aux ennemis,
rendant toutes leurs subsistances très difficiles et les nôtres très
abondantes, venant par eau et par ordre dans un camp qui ne pouvait être
attaqué. M. de Vendôme convenait de tout cela et n'alléguait aucune
raison contraire\,; mais, pour exécuter ce projet si aisé, il fallait
remuer de sa place et aller occuper ce camp. Toute la difficulté se
renfermait à la paresse personnelle de M. de Vendôme, qui, à son aise
dans son logis, voulait en jouir tant qu'il pourrait, et soutenait que
ce mouvement dont on était maître serait tout aussi bon différé. Mgr le
duc de Bourgogne, soutenu de toute l'armée, et jusque par les plus
confidents de Vendôme, lui représenta vainement que, puisque de son
propre avis ce qui était proposé était le seul bon parti à prendre, il
valait mieux pris qu'à prendre\,; qu'il n'y avait aucun inconvénient à
le faire\,; qu'il s'en pouvait trouver à différer et à hasarder d'y être
prévenu, {[}ce{]} qui, de l'aveu même de Vendôme, serait un inconvénient
très fâcheux. Vendôme craignait la fatigue des marches et des
changements de logis, cela renversait le repos de ses journées que j'ai
décrit ailleurs. Il regrettait toujours les aises qu'il quittait\,; ces
considérations furent les plus fortes.

Marlborough voyait clairement que Vendôme n'avait du tout de bon et
d'important à faire que ce mouvement, ni lui que de tenter de
l'empêcher. Pour le faire, Vendôme suivait la corde qui était très
courte\,; pour l'empêcher, Marlborough avait à marcher sur l'arc fort
étendu et courbé, c'est-à-dire vingt-cinq lieues à faire, contre Vendôme
six au plus. Les ennemis se mirent en marche avec tant de diligence et
de secret, qu'ils en dérobèrent trois forcées, sans que Vendôme en eût
ni avis ni soupçon, quoique partis de fort proche de lui. Averti enfin
il méprisa l'avis, suivant sa coutume, puis s'assura qu'il les
devancerait en marchant le lendemain matin. Mgr le duc de Bourgogne le
pressa de marcher dès le soir\,; ceux qui l'osèrent lui en
représentèrent la nécessité et l'importance. Tout fut inutile, malgré
les avis redoublés à tous moments de la marche des ennemis. La
négligence se trouva telle qu'on n'avait pas seulement songé à jeter des
ponts sur un ruisseau qu'il fallait passer jusqu'à la tête du camp. On
dit qu'on y travaillerait toute la nuit.

Biron, maintenant duc et pair et doyen des maréchaux de France, avait
pensé être mis auprès de la personne de M. le duc de Berry cette
campagne. Il était lieutenant général, commandait une des deux réserves,
et il était à quelque distance du camp, d'où il communiquait d'un côté,
et de l'autre à un corps détaché plus loin. Ce même soir il reçut ordre
de se faire rejoindre par ce corps plus éloigné, et de le ramener avec
le sien à l'armée. En approchant du camp, il trouva un ordre de
s'avancer sur l'Escaut, vers où l'armée allait s'ébranler pour le
passer. Arrivé à ce ruisseau où on achevait les ponts et dont j'ai
parlé, Motet, capitaine des guides, fort entendu, lui apprit les
nouvelles qui avaient enfin fait prendre la résolution de marcher.
Alors, quelque accoutumé que fût Biron à M. de Vendôme par la campagne
précédente, il ne put s'empêcher d'être étrangement surpris de voir que
ces ponts non encore achevés ne le fussent pas dès longtemps, et de voir
encore tout tendu dans l'armée. Il se hâta de traverser ce ruisseau,
d'arriver à l'Escaut, où les ponts n'étaient pas faits encore, de le
passer comme il put, et de gagner lés hauteurs au delà. Il était environ
deux heures après midi du mercredi 11 juillet, lorsqu'il les eut
reconnues, et qu'il vit en même temps toute l'armée des ennemis, les
queues de leurs colonnes à Audenarde, où ils avaient passé l'Escaut, et
leur tête prenant un tour et faisant contenance de venir sur lui. Il
dépêcha un aide de camp aux princes et à M. de Vendôme, pour les en
informer et demander leurs ordres, qui les trouva pied à terre et
mangeant un morceau. Vendôme, piqué de l'avis si différent de ce qu'il
s'était si opiniâtrement promis, se mit à soutenir qu'il ne pouvait être
véritable. Comme il disputait là-dessus avec grande chaleur, arriva un
officier par qui Biron envoyait confirmer le fait, qui ne fit qu'irriter
et opiniâtrer Vendôme de plus en plus. Un troisième avis confirmatif de
Biron le fit emporter\,; et pourtant se lever de table, ou de ce qui en
servait, avec dépit, et monter à cheval, en maintenant toujours qu'il
faudrait donc que les diables les eussent portés là, et que cette
diligence était impossible. Il renvoya le premier aide de camp arrivé
dire à Biron qu'il chargeât les ennemis, et qu'il serait tout à l'heure
à lui pour le soutenir avec des troupes. Il dit aux princes de suivre
doucement avec le gros de l'armée, tandis qu'il allait prendre la tête
des colonnes et se porter vers Biron le plus légèrement qu'il pourrait.
Biron cependant posta ce qu'il avait de troupes le mieux qu'il put dans
un terrain fort inégal et fort coupé, occupant un village et des haies,
et bordant un ravin profond et escarpé, après quoi il se mit à visiter
sa droite, et vit la tête de l'armée ennemie très proche de lui. Il eut
envie d'exécuter l'ordre qu'il venait de recevoir de charger, mais dans
aucune espérance qu'il conçût d'un combat si étrangement disproportionné
que pour se mettre à couvert des propos d'un général sans mesure, et si
propre à rejeter sur lui, et sur n'avoir pas exécuté ses ordres, toutes
les mauvaises suites qui se prévoyaient déjà. Dans ces moments de
perplexité arriva Puységur avec le campement, qui, après avoir reconnu
de quoi il s'agissait, conseilla fort à Biron de se bien garder
d'engager un combat si fort à risquer. Quelques moments après survint le
maréchal de Matignon qui, sur l'inspection des choses et le compte que
Biron lui rendit de l'ordre qu'il avait reçu de charger, lui défendit
très expressément de l'exécuter, et le prit même sur lui.

Tandis que cela se passait, Biron entendit un grand feu sur sa gauche,
au delà du village. Il y courut et y trouva un combat d'infanterie
engagé. Il le soutint de son mieux avec ce qu'il avait de troupes,
pendant que plus encore sur la gauche les ennemis gagnaient du terrain.
Le ravin, qui était difficile, les arrêta et donna le temps d'arriver à
M. de Vendôme. Ce qu'il amenait de troupes était hors d'haleine. À
mesure qu'elles arrivèrent, elles se jetèrent dans les haies, presque
toutes en colonnes, comme elles venaient, et soutinrent ainsi l'effort
des ennemis et d'un combat qui s'échauffa, sans qu'il y eût moyen de les
ranger en aucun ordre\,; tellement que ce ne fut jamais que les têtes
des colonnes qui, chacune par son front et occupant ainsi chacune un
très petit terrain, combattirent les ennemis, lesquels étendus en lignes
et en ordre profitèrent du désordre de nos troupes essoufflées et de
l'espace vide laissé des deux côtés de ces têtes de colonnes, qui ne
remplissaient qu'à mesure que d'autres têtes arrivaient, aussi hors
d'haleine que les premières. Elles se trouvaient vivement chargées en
arrivant, et redoublant et s'étendant à côté des autres qu'elles
renversaient souvent, et les réduisaient, par le désordre de l'arrivée,
à se rallier derrière elles, c'est-à-dire derrière d'autres haies, parce
que la diligence avec laquelle nos troupes s'avançaient, jointe aux
coupures du terrain, causait une confusion dont elles ne se pouvaient
débarrasser. Il en naissait encore l'inconvénient de longs intervalles
entre elles, et, que les pelotons étaient repoussés bien loin avant
qu'ils pussent être soutenus par d'autres, qui survenant avec le même
désordre ne faisaient que l'augmenter, sans servir beaucoup aux premiers
arrivés à se rallier derrière eux à mesure qu'ils se présentaient au
combat. La cavalerie et la maison du roi se trouvèrent mêlés avec
l'infanterie, ce qui combla la confusion au point que nos troupes se
méconnurent les unes les autres. Cela donna loisir aux ennemis de
combler le ravin de fascines assez pour pouvoir le passer, et à la queue
de leur armée de faire un grand tour par notre droite pour en gagner la
tête, et prendre en flanc ce qui s'y était le plus étendu, et avait
essuyé moisis de feu et de confusion dans ce terrain moins coupé que
l'autre.

Vers cette même droite étaient les princes, qu'on avait longtemps
arrêtés au moulin de Royenghem-Capel pour voir cependant plus clair à ce
combat si bizarre et si désavantageusement enfourné. Dès que nos troupes
de cette droite en virent fondre sur elles de beaucoup plus nombreuses,
et qui les prenaient par leur flanc, elles ployèrent vers leur gauche
avec tant de promptitude, que les valets de la suite de tout ce qui
accompagnait les princes tombèrent sur eux, avec un effroi, une
rapidité, une confusion qui les entraînèrent avec une extrême vitesse,
et beaucoup d'indécence et de hasard, au gros de l'action à la gauche.
Ils s'y montrèrent partout, et aux endroits les plus exposés, y
montrèrent une grande et naturelle valeur, et beaucoup de sang-froid
parmi leur douleur de voir une situation si fâcheuse, encourageant les
troupes, louant les officiers, demandant aux principaux ce qu'ils
jugeaient qu'on dût faire, et disant à M. de Vendôme ce qu'eux-mêmes
pensaient. L'inégalité du terrain que les ennemis trouvèrent en
avançant, après avoir poussé notre droite, donna à cette droite le temps
de se reconnaître, de se rallier, et, malgré ce grand ébranlement, pour
n'en rien dire de plus, de leur résister. Mais cet effort fut de peu de
durée. Chacun avait rendu des combats particuliers de toutes parts,
chacun se trouvait épuisé de lassitude et du désespoir du succès parmi
une confusion si générale et si inouïe. La maison du roi dut son salut à
la méprise d'un officier des ennemis qui porta un ordre aux troupes
rouges, les prenant pour des leurs. Il fut pris, et voyant qu'il allait
partager le péril avec elles il les avertit qu'elles allaient être
enveloppées, et leur montra la disposition qui s'en faisait, ce qui lit
retirer la maison du roi un peu en désordre. Il augmentait de moment en
moment. Personne ne reconnaissait sa troupe. Toutes étoient pêle-mêle,
cavalerie, infanterie, dragons\,; pas un bataillon, pas un escadron
ensemble, et tous en confusion les uns sur les autres.

La nuit tombait, on avait perdu un terrain infini\,; la moitié de
l'armée n'avait pas achevé d'arriver. Dans une situation si triste, les
princes consultèrent avec M. de Vendôme ce qu'il y avait à faire, qui de
fureur de s'être si cruellement mécompté brusquait tout le monde. Mgr le
duc de Bourgogne voulut parler, mais Vendôme, enivré d'autorité et de
colère, lui ferma à l'instant la bouche en lui disant d'un ton impérieux
devant tout le monde\,: «\,Qu'il se souvînt qu'il n'était venu à l'armée
qu'à condition de lui obéir.\,» Ces paroles énormes et prononcées dans
les funestes moments où on sentait si horriblement le poids de
l'obéissance rendue à sa paresse et à son opiniâtreté, et qui par le
délai de décamper était cause de ce désastre, firent frémir
d'indignation tout ce qui l'entendit. Le jeune prince à qui elles furent
adressées y chercha une plus difficile victoire que celle qui se
remportait actuellement parles ennemis sur lui. Il sentit qu'il n'y
avait point de milieu entre les dernières extrémités et l'entier
silence, et fut assez maître de soi pour le garder. Vendôme se mit à
pérorer sur ce combat, à vouloir montrer qu'il n'était point perdu, à
soutenir que, la moitié de l'armée n'ayant pas combattu, il fallait
tourner toutes ses pensées à recommencer le lendemain matin, et pour
cela profiter de la nuit, rester dans les mêmes postes où on était, et
s'y avantager au mieux qu'on pourrait. Chacun écouta en silence un homme
qui ne voulait pas être contredit, et qui venait de faire un exemple
aussi coupable qu'incroyable, dans l'héritier nécessaire de la couronne,
de quiconque hasarderait autre chose que des applaudissements. Le
silence dura donc sans que personne osât proférer une parole, jusqu'à ce
que le comte d'Évreux le rompit pour louer M. de Vendôme, dont il était
cousin germain et fort protégé. On en fut un peu surpris, parce qu'il
n'était que maréchal de camp.

Il venait cependant des avis de tous côtés que le désordre était
extrême. Puységur, arrivant de vers la maison du roi, en fit un récit
qui ne laissa aucun raisonnement libre, et que le maréchal de Matignon
osa appuyer. Sousternon, venant d'un autre côté, rendit un compte
semblable. Enfin Cheladet et Puyguyon, survenant chacun d'ailleurs,
achevèrent de presser une résolution. Vendôme ne voyant plus nulle
apparence de résister davantage à tant de convictions, et poussé à bout
de rage\,: «\,Oh bien\,! s'écria-t-il, messieurs, je vois bien que vous
le voulez tous, il faut donc se retirer. Aussi bien, ajouta-t-il, en
regardant Mgr le duc de Bourgogne, il y a longtemps, monseigneur, que
vous en aviez envie.\,» Ces paroles, qui ne pouvaient manquer d'être
prises dans un double sens, et qui furent par la suite appesanties,
furent prononcées exactement telles que je les rapporte, et assenées de
plus, de façon que pas un des assistants ne se méprit à la signification
que le général leur voulut faire exprimer. Les faits sont simples, ils
parlent d'eux-mêmes\,; je m'abstiens de commentaires pour ne pas
interrompre le reste de l'action. Mgr le duc de Bourgogne demeura dans
le parfait silence, comme il avait fait la première fois, et tout le
monde, à son exemple, en diverses sortes d'admirations muettes. Puységur
le rompit à la fin pour demander comment on entendait de faire la
retraite. Chacun parla confusément. Vendôme, à son tour, garda le
silence, ou de dépit, ou d'embarras, puis il dit qu'il fallait marcher à
Gand, sans ajouter comment, ni aucune autre chose.

La journée avait été fort fatigante, la retraite était longue et
périlleuse\,; chacun mettait son espérance pour l'avenir dans l'armée
que le duc de Berwick amenait de la Moselle. On proposa de faire avancer
les chaises des princes, et de les mettre dedans pour les conduire plus
commodément vers Bruges, et au-devant de cette armée. Cette idée vint de
Puységur, d'O y applaudit fort, Gamaches ne s'y opposa pas. On les
demanda, et sur-le-champ on commanda cinq cents chevaux d'escorte.
Là-dessus Vendôme cria que cela seront honteux\,; les chaises furent
contremandées, et l'escorte déjà commandée servit depuis à ramasser les
fuyards. Alors ce petit conseil tumultueux se sépara. Les princes, avec
ce peu de suite qui, les avait accompagnés, prirent à cheval le chemin
de Gand. Vendôme, sans plus donner nul ordre, ni s'informer de rien, ne
parut plus en aucun lieu\,; ce qui s'était trouvé là d'officiers
généraux retournèrent à leurs postes, ou, pour mieux dire, où ils
purent, ainsi que le maréchal de Matignon, et firent passer en divers
endroits de l'armée l'ordre de se retirer. La nuit était tantôt close\,;
on entendait encore plusieurs combats particuliers en divers endroits\,;
enfin les premiers avertis s'ébranlèrent.

Cependant les officiers généraux de la droite et ceux de la maison du
roi tenaient leur petit conseil entre eux, et ne pouvaient comprendre
comment il ne leur venait point d'ordre, lorsque celui de la retraite
leur arriva. Mais tandis qu'ils demeuraient en cette attente et en
suspens, ils se trouvèrent environnés et coupés de toutes parts. Chacun
d'eux alors fut bien étonné. Ils recommençaient à raisonner sur les
moyens d'exécuter leur retraite, lorsque le vidame d'Amiens qui, comme
tout nouveau maréchal de camp, ne disait pas grand'chose, se mit à leur
remontrer que, tandis qu'ils délibéraient, ils allaient être enfermés\,;
puis, voyant qu'ils continuaient en leur incertitude, il les exhorta à
le suivre, et se tournant vers les chevau-légers de la garde dont il
était capitaine\,: «\,Marche à moi\,!» leur dit-il, en digne frère et
successeur du duc de Montfort\,; et, perçant à leur tête une ligne de
cavalerie ennemie, il en trouva derrière elle une autre d'infanterie
dont il essuya tout le feu, mais qui s'ouvrit pour lui donner passage. À
l'instant, le reste de la maison du roi, profitant d'un mouvement `si
hardi, suivit cette compagnie, puis les autres troupes qui se trouvèrent
là, et toutes firent leur retraite ensemble toute la nuit et en bon
ordre jusqu'à Gand, toujours menés par le vidame, qui, pour avoir su
prendre à temps et seul son parti avec sens et courage, sauva ainsi une
partie considérable de cette armée. Les autres débris se retirèrent
comme ils purent, avec tant de confusion, que le chevalier du Rosel,
lieutenant général, n'en eut aucun avis, et se trouva le lendemain
matin, avec cent escadrons qui avaient, été totalement oubliés. Sa
retraite ainsi esseulée, et en plein jour, devenait très difficile, mais
il n'était pas possible de soutenir le poste qu'il occupait jusqu'à la
nuit. Il se mit donc en marche.

Nangis, aussi tout nouveau maréchal de camp, aperçut des pelotons de
grenadiers épars, il en trouva de traîneurs, bref, de pure bonne
volonté, il en ramassa jusqu'à quinze compagnies, et par cette même
volonté, fit avec ces grenadiers l'arrière-garde de la colonne du
chevalier du Rosel, si étrangement abandonnée. Les ennemis passèrent les
haies et un petit ruisseau, l'attaquèrent souvent\,; il les soutint
toujours avec vigueur. Ils firent une marche de plusieurs heures qui fut
un véritable combat. À la fin, ils se retirèrent par des chemins
détournés que l'habitude d'aller à la guerre avait appris au chevalier
du Rosel, grand et excellent partisan. Ils arrivèrent au camp après y
avoir causé une cruelle inquiétude pendant quatorze ou quinze heures
qu'on ignora ce qu'ils étaient devenus.

Mgr le duc de Bourgogne ne fit que traverser Gand sans s'y arrêter, et
continua de marcher jusqu'à Lawendeghem avec la tête des troupes qui y
arrivait. Il y établit son quartier général et son camp le long et
derrière le canal de Bruges, pour y faire reposer ses troupes en sûreté,
avec l'abondance des derrières, en attendant qu'on prit un parti et la
jonction de Berwick. M. de Vendôme (je continue de rapporter simplement
les faits) arriva séparément à Gand entre sept et huit heures du matin,
trouva des troupes qui entraient dans la ville, s'arrêta avec le peu de
suite qui l'avait accompagné, mit pied à terre, défit ses chaussés, et
poussa sa selle tout auprès des troupes en les voyant défiler. Il entra
aussitôt après dans la ville sans s'informer de quoi que ce fût, se jeta
dans un lit, et y demeura plus de trente heures sans se lever, pour se
reposer de ses fatigues. Ensuite il apprit par ses gens que l'armée
était à Lawendeghem. Il l'y laissa, continuant à ne s'embarrasser de
rien, à bien souper et à se reposer de plus en plus dans Gand plusieurs
jours de suite, sans se mêler en aucune sorte de l'armée, dont il était
à trois lieues. Peu de jours après le comte de La Mothe prit le fort de
Plassendal, dont la garnison passa toute au fil de l'épée, qui fut un
poste important à la communication des canaux. Les ennemis allèrent
prendre le camp de Warwick, et se rendirent maîtres de nos lignes, où il
n'y avait que de petits détachements d'infanterie.

\hypertarget{chapitre-xvi.}{%
\chapter{CHAPITRE XVI.}\label{chapitre-xvi.}}

1708

~

{\textsc{Lettres au roi et autres.}} {\textsc{- Biron à Fontainebleau.}}
{\textsc{- Propos singulier de Marlborough à Biron sur le roi
d'Angleterre.}} {\textsc{- Audacieux mot à Biron du prince Eugène sur la
charge des Suisses qu'avait son père.}} {\textsc{- Situation de la cour
rappelée.}} {\textsc{- Conduite de la cabale de Vendôme.}} {\textsc{-
Lettre d'Albéroni.}} {\textsc{- Examen de la lettre d'Albéroni.}}

~

On cacha tant qu'on put la perte qu'on fit en ce combat, où il y eut
beaucoup de tués et de blessés. Biron, lieutenant général\,; Ruffé et
Fitzgérald, maréchaux de camp\,; Croï, brigadier d'infanterie\,; le duc
de Saint-Aignan, le marquis d'Ancenis, ces deux derniers blessés\,;
beaucoup d'officiers de gendarmerie, force officiers particuliers,
prisonniers\,; Ximénès, colonel du Royal-Roussillon infanterie, et La
Bretanche, brigadier de réputation, tués\,; quatre mille hommes et sept
cents officiers prisonniers à Audenarde, sans ce qu'on en sut depuis, et
la dispersion, qui fut prodigieuse.

Dès que Mgr le duc de Bourgogne fut à Lawendeghem, il écrivit au roi en
fort peu de mots, et se remit du détail au duc de Vendôme. En même
temps, il manda à M\textsuperscript{me} la duchesse de Bourgogne, en
termes formels, que l'ordinaire opiniâtreté et sécurité du duc de
Vendôme, qui l'avait empêché de marcher deux jours au moins plus tard
qu'il ne fallait, et que lui ne voulait, causait le triste événement qui
venait d'arriver\,; qu'un autre pareil lui ferait quitter le métier,
s'il n'en était empêché par des ordres précis auxquels il devait une
obéissance aveugle\,; qu'il ne comprenait ni l'attaque, ni le combat, ni
la retraite\,; qu'il en était si outré qu'il n'en pouvait dire
davantage. Le courrier qui portait ces lettres en prit, en passant à
Gand, une que Vendôme écrivit au roi, de cette ville, en se mettant au
lit, par laquelle il tâchait de persuader, en une page, que le combat
n'était pas désavantageux. Peu après il en dépêcha une autre par
laquelle il manda au roi, mais en peu de mots, qu'il aurait battu les
ennemis s'il avait été soutenu\,; et que si, contre son avis, on ne se
fût pas opiniâtré à la retraite, il les aurait certainement battus le
lendemain\,; pour le détail, il s'en remettait à Mgr le duc de
Bourgogne. Ainsi ce détail, renvoyé de l'un à l'autre, ne vint point,
aigrit la curiosité, et commença les ténèbres dans lesquelles Vendôme
avait intérêt de se sauver. Un troisième courrier apporta au roi une
fort longue dépêche, toute de la main de Mgr le duc de Bourgogne, une
fort courte de M. de Vendôme, qui s'excusait encore du détail sur divers
prétextes\,; et toutes les lettres que le courrier avait pour des
particuliers, le roi les prit, les lut toutes, une entre autres jusqu'à
trois fois de suite, n'en rendit que fort peu et toutes ouvertes. Ce
courrier arriva après le souper du roi, tellement que toutes les dames
qui suivent leurs princesses dans le cabinet le soir furent témoins de
ces lectures dont le roi ne dit presque rien, parce qu'à Fontainebleau,
où il n'y a qu'un cabinet, elles sont toutes dans le même.
M\textsuperscript{me} la duchesse de Bourgogne eut une lettre de Mgr le
duc de Bourgogne et une petite de M. le duc de Berry, qui lui mandait
que M. de Vendôme était bien malheureux, et que toute l'armée lui
tombait sur le corps. Dès que M\textsuperscript{me} la duchesse de
Bourgogne fut retournée chez elle, elle ne put se contenir de dire que
Mgr le duc de Bourgogne avait de bien sottes gens auprès de lui. Elle
n'en dit pas davantage.

Biron, relâché pour quelque temps sur sa parole à condition de ne passer
point par notre armée, arriva à Fontainebleau le 25 juillet. Sa sagesse
lui fut un bouclier utile à l'indiscrétion et à l'impétuosité des
questions. Le roi le vit plusieurs fois en particulier chez
M\textsuperscript{me} de Maintenon, où Chamillart ne fut pas toujours,
et le roi lui promit le secret, à quoi il était fort fidèle. Mais Biron,
encore plus politique, ne lui mentit point, mais se sauva tant qu'il put
de répondre sur le détachement qu'il avait avant l'action, et sur sa
prise, qui lui faisaient ignorer beaucoup de choses. Il était fort de
mes amis et je le vis tout à mon aise. Il m'instruisit beaucoup. Outre
ce qu'il me conta de l'armée et du combat, j'appris de lui deux faits
qui méritent de trouver place ici.

L'armée du prince Eugène n'avait pas joint lors du combat, mais sa
personne y était et il commandait partout où il se trouvait par
courtoisie de Marlborough, qui conservait l'autorité entière, mais qui
n'avait pas la même estime, la confiance, l'affection qu'Eugène s'était
acquise. Biron me dit que le lendemain du combat, étant à dîner avec
beaucoup d'officiers chez Marlborough, ce duc lui demanda tout à coup
des nouvelles du prince de Galles, qu'on savait être dans notre armée,
ajoutant des excuses de le nommer ainsi. Biron sourit dans sa surprise,
et lui dit qu'ils n'auraient point de difficulté là-dessus, parce que,
dans notre armée même, il ne portait point d'autre nom que celui de
chevalier de Saint-Georges, et s'étendit sur ses louanges assez
longtemps. Marlborough, qui l'écouta avec grande attention, lui répondit
qu'il lui faisait grand plaisir de lui en apprendre tant de bien, parce
qu'il ne pouvait s'empêcher de s'intéresser beaucoup en ce jeune prince,
et aussitôt se mit à parler d'autre chose. Biron remarqua en même temps
de l'épanouissement sur son visage et sur celui de la plupart de la
compagnie.

L'autre fait est du prince Eugène. Parlant avec lui du combat, ce prince
lui témoigna une grande estime de ce qu'il avait vu faire à nos troupes
suisses, qui en effet s'étaient fort distinguées. Biron les loua
beaucoup. Eugène en prit occasion d'en vanter la nation, et de dire à
Biron que c'était une belle charge en France que d'en être colonel
général. «\,Mon père l'avait, ajouta-t-il d'un air allumé, à sa mort
nous espérions que mon frère la pourrait obtenir\,; mais le roi jugea
plus à propos de la donner à un de ses enfants naturels, que de nous
faire cet honneur-là. Il est le maître, il n'y a rien à dire\,; mais
aussi n'est-on pas fâché quelquefois de se trouver en état de faire
repentir des mépris.\,» Biron ne répondit pas un mot, et le prince
Eugène, content d'un trait si piquant sur le roi, changea poliment de
conversation. Dans le peu que Biron fut parmi eux, il remarqua une
magnificence presque royale chez le prince Eugène, et une parcimonie
honteuse chez le duc de Marlborough, qui mangeait le plus souvent chez
les uns et les autres, un grand concert entre eux deux pour les
affaires, dont le détail roulait beaucoup plus sur Eugène, un respect
profond de tous les officiers généraux pour ces deux chefs, mais une
préférence tacite et en tout pour le prince Eugène, sans que le duc de
Marlborough en prît jalousie. Monseigneur entretint peu Biron, quoique
très familier avec lui\,; M\textsuperscript{me} la duchesse de Bourgogne
beaucoup et souvent. Il la mit en état de répondre à diverses choses
qu'on avait tâché d'embarrasser. On n'eut jamais un vrai détail. Ce ne
furent que morceaux détachés les uns après les autres. Mgr le duc de
Bourgogne ne fit pas assez de réflexion combien un détail
\emph{effectif} lui importait à donner, ce que Vendôme n'avait garde de
faire.

Maintenant il faut se souvenir de la situation de la cour et de ses
principaux personnages, de leurs vues, de leurs intérêts que j'ai
expliqués en divers endroits, et surtout de ma conversation avec le duc
de Beauvilliers, dans le bas des jardins de Marly, sur la destination de
Mgr le duc de Bourgogne pour la Flandre. On y a vu la liaison intime des
bâtards avec Vaudemont et ses puissantes nièces, et de Vendôme
principalement\,; celle des valets intérieurs principaux avec eux, et
Bloin surtout, le mieux de tous, et le plus dans la confiance libre du
roi, celui de tous aussi qui était le plus délié, le plus hardi, le plus
précautionné, qui avait le plus d'esprit et de monde, qui voyait le
plus, de bonne compagnie et de plus choisie, le plus initié dans tout
par ses galanteries, et qui, outre sa place de premier valet de chambre,
avait cent occasions de voir le roi à revers tous les jours, et de
prendre tous ses moments par ses détails continuels de Versailles et de
Marly dont il était le gouverneur et le tout, par une assiduité sans
quitter jamais, et par être sans cesse dans les cabinets à toutes les
heures de la journée. Il venait à Fontainebleau, y passait du temps, et,
là comme ailleurs, disposait des garçons bleus de tout le subalterne
intérieur, et de ces dangereux Suisses, espions et rapporteurs dont j'ai
parlé à propos de la scène terrible sur Courtenvaux. {[}Il faut se
rappeler{]} l'abandon de Chamillart, d'ailleurs si entêté, à M. de
Vendôme, à M. du Maine qu'il avait pris pour protecteur, surtout à M. de
Vaudemont qui était son oracle et qui lui faisait faire tout ce qu'il
voulait à l'instant, même les choses les plus contraires à son goût et à
son opinion, dont il s'est plu quelquefois à montrer des épreuves qui
jamais ne lui ont manqué\,: ce n'est point trop dire que ce ministre
était une cire molle entre ses mains, et Vaudemont en était si assuré,
qu'il en a fait jusqu'à des essais inutiles, sinon pour s'en vanter à
ses familiers.

Il faut surtout ne pas perdre de vue l'intérêt de tous ces personnages
de perdre et de déshonorer à fond Mgr le duc de Bourgogne, pour n'avoir
point à compter avec lui du vivant du roi, et à sa mort, s'en trouver
débarrassés pour gouverner Monseigneur sur le trône. C'était là
l'intérêt général qui les réunissait tous, quittes, comme je l'ai dit
ailleurs, à se manger après les uns les autres à qui le gouvernement
resterait. M\textsuperscript{lle} Choin et ses intimes en étaient
jusqu'au cou, et par même raison\,; et le pauvre Chamillart, qui n'en
voyait rien, dont l'intérêt était tout opposé par mille raisons, et trop
homme de bien et d'honneur pour tremper dans ce complot s'il avait pu le
connaître, était leur instrument aveugle sans pouvoir être, je ne dis
pas arrêté, mais enrayé le moins du monde par les ducs de Chevreuse et
de Beauvilliers d'ailleurs ses amis de confiance et de déférence, ni par
l'alliance si proche et si nouvelle qu'il venait de contracter avec eux
par le mariage de son fils. À plus forte raison j'y pouvais bien moins
encore, avec toute l'amitié et la confiance qu'il avait pour moi. Sa
femme et ses filles étaient dépourvues de tout sens, excepté la petite
Dreux, mais qui était entraînée, ses frères des stupides, et le reste de
l'intime familier des gens de peu, appliqués à leur fait, ineptes à la
cour à n'en entendre pas même le langage. M\textsuperscript{me} la
Duchesse s'unit intimement à ce redoutable groupe par les mêmes vues sur
Monseigneur, et par sa haine personnelle\,; mais cet arrière-recoin
s'expliquera mieux dans la suite. Il ne faut pas oublier l'intime
liaison de M\textsuperscript{me} de Soubise avec M\textsuperscript{lle}
de Lislebonne et M\textsuperscript{me} d'Espinoy, et les dangers de ses
conseils, dans la prudence de sa conduite particulière qu'elle mettait
aisément à part et à couvert, dans le triste état où pour lors sa santé
était réduite.

La cabale, d'abord étourdie du fâcheux événement, en attendait plus de
détail et de lumière, et, pour éviter les faux pas, s'arrêta dans les
premiers moments à écouter. Sentant bientôt le danger de son héros, elle
se rassura, jeta des propos à l'oreille pour sonder comment ils seraient
reçus, et prenant aussitôt plus d'audace s'échappa tout haut par
parcelles. Encouragés par cet essai, qui ne trouva pas de forte barrière
parmi le monde étonné et sans détail de rien, ils poussèrent la licence,
ils hasardèrent des louanges de Vendôme, des disputes vives contre
quiconque ne se livrait pas à leurs discours, et, s'encourageant par le
succès, osèrent passer au blâme de Mgr le duc de Bourgogne, et tôt après
aux invectives, parce que leurs premiers propos n'avaient pas été
réprimés. Il n'y avait que le roi ou Monseigneur qui l'eussent pu. Le
roi les ignorait encore, Monseigneur était investi, et n'était pas pour
oser imposer\,; le gros des courtisans, dans les ténèbres sur le détail
de l'affaire, et dans la crainte des personnages si accrédités et de si
haut parage, ne savaient et n'osaient répondre. Ils se contentaient de
demeurer dans l'attente et dans l'étonnement. Cela haussa de plus en
plus le courage de la cabale. Faute de détails que Vendôme n'avait garde
de fournir, on osa semer des manifestes dont l'artifice, le mensonge,
l'imposture ne gardèrent aucun ménagement, et furent poussés jusqu'à ce
qui ne peut avoir d'autre nom que celui d'attentat. Le premier qui parut
fut une lettre d'Albéroni, personnage duquel j'ai assez parlé pour
n'avoir pas besoin ici de le faire connaître. Elle est telle qu'elle ne
peut être renvoyée parmi les Pièces. La voici\,:

«\,Laissez, monsieur, votre désolation, et n'entrez pas dans le parti
général de votre nation, laquelle, au moindre malheur qui est arrivé,
croit que tout est perdu. Je commence par vous écrire que tous les
discours qui se tiennent contre M. de Vendôme sont faux, et il s'en
moque. À l'égard des trois marches qu'on dit qu'il s'est laissé dérober,
et qu'il n'avait qu'à défendre la Dendre, tout le monde sait ici que M.
de Vendôme voulait la défendre, et qu'après trois jours, il lui a fallu
se rendre au sentiment de ceux qui opinaient à passer l'Escaut pour
éviter de combattre, et c'est alors qu'ils y ont été obligés, comme Son
Altesse le leur avait prédit, leur disant que toutes les fois qu'ils
marqueront à M. le prince Eugène d'éviter d'en découdre, il les y
obligera malgré eux. Touchant que Son Altesse devait attaquer la tête
qui était à l'Escaut, il avait bien mieux pensé, car d'abord qu'il eut
avis par M. de Biron qu'une partie de l'armée ennemie avait passé, il
voulut l'attaquer pendant qu'il voyait la poussière des colonnes de
ladite armée qui était au delà de la rivière, à une demi-lieue
d'Audenarde, mais comme son avis fut seul il ne fut pas écouté. C'était
à dix heures du matin. À quatre heures après midi on donna ordre à M. de
Grimaldi, maréchal de camp de Sa Majesté Catholique, d'attaquer, à
l'insu de M. de Vendôme, qui pourtant, voyant l'attaque faite, dit qu'il
fallait la soutenir, et il donna ordre à M. Janet, son aide de camp, de
porter l'ordre à la gauche, afin qu'elle attaquât, qui en retournant fut
tué. Cet ordre ne fut pas exécuté par un mauvais conseil qui fut donné à
M. le duc de Bourgogne, disant qu'il y avait un ravin et un marais
impraticable. Cependant M. de Vendôme, accompagné de M. le comte
d'Évreux, y avait passé avec trente escadrons une heure auparavant. Pour
ce qui regarde la retraite, M. de Vendôme opina de ne la point faire la
nuit, mais, comme de ce sentiment il n'y avait que lui et M. le comte
d'Évreux, il fallut céder, et à peine eut-il dit à M. le duc de
Bourgogne que l'armée n'avait qu'à se retirer, que tout le monde à
cheval et avec une précipitation étonnante, chacun gagne Gand, jusqu'à
conseiller aux princes de prendre des chevaux de poste à Gand pour
gagner Ypres. M. de Vendôme, qui fut obligé de faire une grande partie
du temps l'arrière-garde avec ses aides de camp, arriva sur les neuf
heures du matin, prit sur-le-champ sa résolution ferme de vouloir mettre
l'armée derrière le canal qui est entre Gand et Bruges, malgré l'avis de
tous les officiers généraux qui l'ont persécuté trois jours durant de
l'abandonner, disant qu'il fallait tâcher de joindre M. de Berwick. Une
telle fermeté a sauvé l'armée du roi et le royaume, car l'épouvante qui
était dans l'armée aurait causé une esclandre bien pire que celle de
Ramillies, au lieu que M. de Vendôme se mettant derrière le canal, il a
soutenu Gand et Bruges qui est un point essentiel, rassuré les esprits,
et donné confiance aux troupes, a donné lieu aux officiers de se
reconnaître et de reconnaître leur terrain, enfin a mis les ennemis dans
l'inaction, et vous pouvez être sûr que, s'ils veulent faire un siège,
il faut qu'ils fassent celui d'Ypres, de Lille, de Mons ou de Tournai.
Or voyez quelles places et si jamais ils attaquent quelques-unes de
celles-là, M. de Vendôme prendra Audenarde, se rendra maître de tout
l'Escaut, et vous n'avez qu'à regarder la carte pour voir combien les
ennemis seraient embarrassés. Voilà la pure vérité, la même que M. de
Vendôme a mandée au roi, et que vous pouvez débiter sur mon compte. Je
suis Romain, c'est-à-dire d'une race à dire la vérité, \emph{in civitate
omnium gnara, et nihil reticente}, dit notre Tacite. Permettez-moi après
cela que je vous dise, avec tout le respect que je vous dois, que votre
nation est bien capable d'oublier toutes les merveilles que ce bon
prince a faites dans mon pays, qui rendront son nom immortel et toujours
révéré, \emph{injuriarum et beneficiorum aeque immemores\,;} mais le bon
prince est fort tranquille, sachant qu'il n'a rien à se reprocher et
que, pendant qu'il a suivi son sentiment, il a toujours bien fait.\,»

Voilà toute la lettre qui fut incontinent distribuée partout. Il s'agit
maintenant d'en faire l'analyse, quoique le mensonge et l'artifice en
sautent aux yeux.

Il faut avouer que, pour insinuer mieux ses faussetés, elle commence par
une vérité. Il n'est que trop vrai que, dès qu'il arrive un malheur aux
François, ils croient tout perdu et se conduisent de façon que tout
l'est en effet. C'est ce qu'a démontré Hochstedt, Barcelone, Ramillies,
Turin et toutes les actions malheureuses de cette guerre, au contraire
des ennemis qui se soutiennent et savent réparer leurs malheurs, comme
on l'a vu à Fleurus, à Neerwinden, et en toutes les affaires qui nous
ont réussi à la guerre précédente. Mais ce n'est pas le vice de la
nation, c'est {[}la faute{]} des généraux à qui la tête tourna à
Hochstedt et à Ramillies, et qui firent pis encore à Turin, où, de
complot formé, ils empêchèrent par deux fois M. le duc d'Orléans, outré
et fort blessé, de faire sa retraite en Italie, comme je l'ai expliqué
alors. Qu'il n'y ait mot de vrai dans les discours tenus contre M. de
Vendôme qui s'en moque, cela s'appelle une impudence tournée en lui en
habitude et aux siens, avec un succès qui ne suppose pas qu'on ose le
blâmer sans la plus grande évidence, à laquelle il faut venir.

On demeure si étonné de la hardiesse démesurée avec laquelle Albéroni
tâche de donner le change sur les trois marches des ennemis dérobées à
M. de Vendôme, qui ont causé tout le désastre, qu'on serait tenté de se
reposer de la réponse sur la notoriété publique qu'il ose lui-même
s'approprier. Jamais il ne fut question de deux partis à prendre, jamais
M. de Vendôme ne disconvint de celui seul qui était le bon et l'unique.
Il n'y eut de dispute que sur le temps. Mgr le duc de Bourgogne, tous
les officiers généraux en état de parler, jusqu'aux plus attachés et aux
plus familiers de M. de Vendôme, furent tellement persuadés du danger de
différer le mouvement à faire qu'ils l'en pressèrent trois jours durant,
et que leurs plaintes de n'être pas écoutés volèrent par toute l'armée.
Biron, qui dans son détachement en était instruit, ne put cacher sa
surprise à Motet de voir les ponts qui n'étaient pas encore faits sur ce
ruisseau de la tête du camp, et de le voir encore tendu lorsqu'il le
passa. Il ne s'en cacha point à Fontainebleau, et pas une lettre de
l'armée, quand à la fin on en reçut, qui ne rendît les mêmes
témoignages, et sur l'unanimité du parti unique, sans aucune dispute de
M. de Vendôme, et sur sa fatale opiniâtreté d'en avoir différé le
mouvement de trois jours, et sur les trois marches que les ennemis lui
dérobèrent, et sur son incrédulité à cet égard poussée jusqu'au moment
qu'il vit de ses yeux ce que Biron lui manda, qu'il méprisa avec
emportement les deux premières fois, et qu'il crut à demi, et à peine la
troisième, qui le fit monter à cheval.

Il est donc clair que ce parti de défendre la Dendre, que cette réponse
flatteuse sur le prince Eugène, est une histoire en l'air, controuvée
après coup pour donner à son maître un air de héros, et pour faire
malignement sentir que Mgr le duc de Bourgogne ne voulait point
combattre. Mais à qui Albéroni espère-t-il persuader que M. de Vendôme
fût assez peu compté dans son armée pour qu'elle ne se remuât qu'à la
pluralité des voix\,? Ces voix, qui étaient-elles\,? Ce n'est pas celle
de Mgr le duc de Bourgogne à qui Vendôme sut dire bientôt après devant
tout le monde qu'il se souvînt qu'il n'était venu à l'armée qu'à
condition de lui obéir. Était-ce le maréchal de Matignon, envoyé là
uniquement pour profaner son bâton à l'obéissance de Vendôme, et dont on
n'a jamais pensé que la capacité suppléât à la dignité\,? Étaient-ce des
lieutenants généraux\,? En quelle armée en a-t-on vu dont la voix fût
prépondérante à celle du général\,? et quelle comparaison de l'autorité
des maréchaux de France que nous avons vus à la tête des armées à celle
du duc de Vendôme\,? Enfin y avait-il là quelque mentor attaché par le
roi à son petit-fils, dont la sagesse, et la confiance du roi en elle,
suppléât au caractère et fût en droit de balancer Vendôme\,?
L'imagina-t-on de Gamaches, de d'O, de Razilly, ni d'eux, ni de pas un
des officiers généraux des plus distingués de l'armée\,? C'est ce qui
n'a été imaginé de personne, et que la cabale de Vendôme n'a aussi osé
avancer. Qui était donc en état, en droit, en moyen de le contredire\,?
Et quels que soient les conseils de guerre, en a-t-il tenu aucun\,? et
qui de ses partisans a osé l'avancer\,? Que veut donc dire Albéroni
quand il débite avec cette effronterie deux partis en dispute qui ne
furent jamais, et l'élection du plus mauvais, par lequel on se flattait
d'éviter un combat, contre le meilleur soutenu par Vendôme, mais qui ne
passa point, parce qu'il fut seul de son avis, tandis que ce fut, non
son avis, mais son opiniâtre et seule volonté qui, contre celle de Mgr
le duc de Bourgogne et les efforts de tout ce qui des généraux osa lui
parler, qui le retint trois jours sans s'ébranler, et sans pourvoir ni
aux ponts ni à la marche\,; dont le succès fut si malheureux, bien loin
qu'aucun avis ait prévalu sur le sien.

La même réponse servira au mensonge qui suit le premier, et qui se
répand sur toutes les parties de ce qu'il avance. Il dit que son héros,
qui avait bien mieux pensé (on ne voit pas en quoi), voulut attaquer les
ennemis sitôt qu'il eut avis d'eux par Biron, et qu'il vit la poussière
de leur armée au delà de la rivière à une demi-lieue d'Audenarde, à dix
heures du matin, mais qu'étant demeuré seul de son sentiment, il ne fut
point écouté. Sans rien répéter de ce qui vient d'être dit sur
l'autorité entière et sans partage de M. de Vendôme dans l'armée,
discutons le reste de ce court récit, court, dis-je, et serré pour jeter
de la poudre aux yeux, et cacher l'imposture par l'audace et l'air de
simplicité. Qui est plus croyable en ces faits, d'Albéroni ou de Biron,
de Puységur et du maréchal de Matignon, acteurs principaux dans le fait
dont il s'agit, et de tout ce qui se trouva avec et autour des princes
et de M, de Vendôme, qui mangeaient un morceau lorsqu'ils reçurent les
trois avis coup sur coup de la part de Biron\,? Mais démêlons les faits.

Biron, détaché de l'armée avec sa réserve, à portée d'un autre corps
plus éloigné, reçoit le soir précédant l'action ordre de se faire
joindre par ce corps et de marcher, etc. Il faut un temps pour envoyer à
ce corps le plus éloigné, un second pour qu'il se mette en marche et
qu'il joigne Biron, un troisième pour que Biron arrive au ruisseau de la
tête de l'armée où il trouve Motet qui travaillait aux ponts, et où
Biron s'étonne de voir le camp encore tout tendu. Quelle heure
pouvait-il donc être\,? De là il faut que l'armée détende, charge,
prenne les armes et monte à cheval, se forme, se mette en marche, passe
le ruisseau, en un mot, arrive au lieu où les princes et M. de Vendôme
mirent pied à terre pour manger. Aussi était-il deux heures après midi
lorsque Biron vit l'armée des ennemis, et par une conséquence sûre bien
plus de deux heures lorsque le premier avis de Biron arriva à la halte
des princes et de Vendôme, et non pas dix heures du matin comme Albéroni
le glisse adroitement. Or, qui ne sent de quelle conséquence sont en
pareilles circonstances quatre heures de plus ou de moins\,? Qui nous en
apprend l'heure\,? c'est Biron, c'est Puységur, c'est le maréchal de
Matignon qui le joignirent, ce sont les trois porteurs d'avis coup sur
coup, ce sont tous ceux qui étaient autour des princes et de M. de
Vendôme, lorsqu'ils les reçurent. De poussière, Albéroni pardonnera la
négative, Biron la vit de la hauteur qu'il avait gagnée\,; elle était
bien loin du lieu où Vendôme faisait sa halte, et la hauteur entre lui
et la poussière\,; quels yeux pouvait avoir Vendôme pour la découvrir\,!
Il la découvrit en effet si peu qu'il maintint faux le premier et le
second avis de Biron, qu'il ne cessa de manger qu'au troisième, qu'il
s'emporta et qu'il dit qu'il fallait donc que ce fussent tous les
diables qui eussent porté là les ennemis. Voilà donc une seconde
fausseté aussi avérée que la première. À l'égard de l'avis de Vendôme de
charger qui ne fut pas suivi, c'est un mensonge qui n'a pas même la
moindre couleur, puisque tout ce qui était là présent en si grand
nombre, d'officiers généraux et autres, furent témoins de ce qui s'y
passa, et l'ont tous dit, écrit et raconté.

Vendôme, après cet emportement qui le fit sortir de table, que lui causa
le troisième avis de Biron, lui renvoya le premier des trois hommes
qu'il lui avait envoyés, et fit ce que j'ai rapporté ci-devant, sans que
Mgr le duc de Bourgogne, {[}ni{]} qui que ce soit, lui dît un mot pour
lui rien représenter. Il n'y eut donc point de partage d'avis, ni abord,
puisque M. de Vendôme comptait les ennemis encore bien loin, par
conséquent hors de portée de pouvoir être chargés\,; ni depuis les avis,
puisque sur les deux premiers il se débattit tout seul pour soutenir que
les ennemis ne pouvaient être là, et que, sur le troisième, après sa
première fougue, il prit les partis qu'on a vus tout haut, et sans
réplique aucune, qui furent exécutés à l'instant, en présence de tout ce
qui les environnait de gens. Il ne put donc songer à faire charger qu'au
moment qu'il en donna l'ordre, et on s'y opposa si peu qu'on a vu que
Biron le reçut\,; qu'en peine de l'exécution, Puységur, survenu avec le
campement, l'en détourna, et qu'un instant après le maréchal de Matignon
arriva qui le lui défendit, et qui prit sur soi la défense. Voilà des
témoins qui valent mieux qu'Albéroni, et qui le démentent sur toutes ses
impostures. Celle qui suit, pour rendre les autres vraisemblables, est
une supposition manifeste. «\,C'est, à son dire, à dix heures du matin
que Vendôme reçoit avis de Biron que les ennemis paraissent, et que lui,
duc de Vendôme, voyant aussi la poussière de leurs colonnes, etc.,
voulut les faire charger et n'en fut pas cru\,; et tout de suite ajoute
qu'à quatre heures après midi on donna ordre à Grimaldi, maréchal de
camp de Sa Majesté Catholique, d'attaquer à l'insu de M. de Vendôme, qui
pourtant, voyant l'attaque faite, dit qu'il la fallait soutenir, et
envoya Janet porter ordre à la gauche d'attaquer, qui ne fut pas
exécuté, par un mauvais conseil donné à Mgr le duc de Bourgogne, disant
qu'il y avait un ravin et un marais impraticable, que cependant M. de
Vendôme avait passé accompagné de AI. le comte d'Évreux avec trente
escadrons.\,»

Disons d'abord que Grimaldi envoya aux ordres de ce qu'il ferait, que
celui qui y vint ne trouva plus M. de Vendôme, déjà parti pour aller à
Biron\,; que cet officier s'adressa à Mgr le duc de Bourgogne, qui,
ayant été témoin de l'ordre que M. de Vendôme avait envoyé à Biron
d'attaquer les ennemis, renvoya l'officier de Grimaldi avec le même
ordre d'attaquer, lequel en arrivant à lui le trouva déjà attaqué
lui-même, et en lieu où il ne put être soutenu à temps par l'obstacle du
ravin. Démêlons maintenant le petit roman d'Albéroni avec tout son
artifice.

Il vient d'être démontré qu'il était deux heures après midi quand Biron
aperçut l'armée des ennemis, et qu'il en envoya le premier avis, que
Vendôme n'en crut rien et ne s'ébranla de son repas qu'au troisième avis
du même Biron\,; on peut juger par là de l'heure qu'il pouvait être.
Cependant Albéroni veut qu'il ne fût que dix heures du matin. Mais que
fit donc son héros jusqu'à quatre heures après midi que sur l'attaque de
Grimaldi il commença à donner des ordres\,? Voilà six heures d'une
singulière patience depuis des nouvelles si intéressantes des ennemis,
et un prodigieux temps perdu que l'apologiste ne remplit de rien\,! Mais
il fallait gagner quatre heures après midi, parce qu'en effet M. de
Vendôme n'arriva guère plus tôt au lieu où on combattait. Est-ce en y
allant avec la tête des colonnes qu'il passa si aisément ce ravin\,?
Qu'est-ce que toute cette fable, sinon pour tomber sur Mgr le duc de
Bourgogne et pour montrer toujours Vendôme ardent à combattre et le
jeune prince toujours obstacle à l'empêcher\,? Il n'y a qu'à se souvenir
de ce qui vient d'être expliqué et démontré tout à l'heure de ce qui se
passa sur le troisième avis de Biron pour se convaincre que ce dernier
récit d'Albéroni est une imposture controuvée de point en point. À
l'égard du ravin, c'est Biron qui l'avait reconnu, c'est les ennemis qui
ne le passèrent qu'à force de fascines, ce sont des faits, mais qui
n'ont aucun trait à Mgr le duc de Bourgogne, qui n'imagina pas de
défendre ni d'ordonner quoi que ce soit qu'avec et de l'avis de M. de
Vendôme. Mais qui peut ignorer qu'un ravin, le plus creux et le plus
difficile, ne soit souvent à mille pas plus haut qu'un fossé ou un
enfoncement médiocre, et plus loin encore un rien qui se passe en
escadron\,? Pour Grimaldi, il ne reçut d'ordre que des ennemis qui
l'attaquèrent. C'est ce qui commença le combat. Pourvu que Mgr le duc de
Bourgogne soit en faute, tout est bon à Albéroni. «\,On ordonna, dit-il,
à Grimaldi d'attaquer à l'insu de M. de Vendôme, c'est-à-dire Mgr le duc
de Bourgogne, et, tout de suite, c'est ce prince qui, malgré l'ordre
envoyé par Vendôme à la gauche d'attaquer, défend de l'exécuter.\,» On
ne peut être moins d'accord avec soi-même, ni moins conséquent dans
l'appréhension de combattre qu'Albéroni prête si audacieusement à ce
jeune prince, ni se souvenir moins de n'être venu à l'armée qu'à
condition d'obéir à Vendôme, comme ce duc osa le lui dire en face et
tout haut devant tout le inonde, que ces contradictions si continuelles
et si hautement exécutées. C'est aussi faire trop peu de cas des hommes
de leur mentir si complètement et si grossièrement.

De ce joli petit conte, si bien inventé, Albéroni saute entièrement le
combat et vient tout d'un coup à la retraite. Il en a bien ses
raisons\,: disons-en un mot.

Aux fautes si funestes que la paresse, l'orgueil et l'opiniâtreté
avaient fait faire à M. de Vendôme, la rage de s'être si lourdement
trompé, et à la face de toute l'armée et de tant de gens qui avaient osé
l'avertir, mit le comble aux fautes précédentes, si des intentions plus
criminelles n'y eurent point de part. Au moins ce qui se passa dans la
suite de cette campagne en put autoriser les soupçons. Sans s'y arrêter,
on ne peut guère au moins disconvenir que la tête lui tourna, et qu'il
ne montra rien de capitaine en toute cette journée. Dans la pensée où il
était de l'éloignement des ennemis, rien ne le pressait d'envoyer si
fort à l'avance Biron et Grimaldi qui ne s'étaient pas portés là sans
son ordre, et il parut bien qu'il croyait les ennemis encore bien
éloignés, puisque le campement arriva avec Puységur aussitôt que Biron,
suite de son opiniâtre prévention. Si, au contraire, il avait cru les
ennemis si à portée, c'était une folie de leur exposer un aussi petit
nombre de troupes, qui de si longtemps ne pouvaient être soutenues.
L'engagement pris, c'est où la tête lui tourna comme au maréchal de
Villeroy à Ramillies, avec cette différence que le maréchal choisit
pernicieusement son terrain et que Vendôme ne fut pas le maître du
sien\,; que le maréchal, après cette première faute qui rendit toute sa
gauche inutile, fit avec le reste tout ce qu'il était possible à un
meilleur général que lui\,; que sa retraite se fit avec le plus grand
ordre, sans honte, sans dommage, et que la tête ne lui tourna qu'après,
par ne se croire en sûreté nulle part, et abandonner des places à l'abri
desquelles il eût pu réparer sa faute et son malheur, et qu'il céda aux
ennemis un pays immense qu'ils n'auraient pu espérer qu'après bien
d'autres succès et de dangereux sièges.

Ici M. de Vendôme, ivre de dépit et de colère, voit sa poignée de
troupes avancées exposée seule à toute l'armée des ennemis\,; et, sans
songer à ce qu'il veut entreprendre, enlève ce qu'il trouve sous sa
main, autre poignée de monde en comparaison de l'armée opposée\,; va à
perte d'haleine, les fait donner d'arrivée, de cul et de tête, sans
ordre et sans règle\,; redouble de la même sorte de tout ce qui suit à
mesure que chaque troupe arrive\,; les fait battre toutes en détail et
en confusion, n'a pas le tiers de son armée, puisque, de l'aveu de tous
et du sien même, la moitié n'en était pas arrivée à la nuit au lieu du
combat, et qu'une partie de l'autre arrivait encore à toute course,
chacun à part comme il se trouvait et pouvait, accourant au feu et
donnant tout de suite là où il le rencontrait. De là le pêle-mêle que
j'ai décrit, l'impossibilité de se remuer, de se reconnaître, de boucher
les intervalles trop étendus, de discerner les endroits propres, d'avoir
ni temps ni moyen de se remuer, de se démêler, de faire aucun mouvement
utile, en un mot un combat qui ne put être qu'un désordre, où il n'y eut
que les fuyards qui pussent gagner. Nul ordre cependant de M. de
Vendôme, nulle ressource de sa part que sa valeur, mais sans vue, sans
dessein, sinon de vaincre, mais vaincre le triple de soi à force de bras
sans aucun moyen de guerre, et dans ce chaos sans pouvoir en exécuter
aucun. M. de Vendôme commandait seul, toutes ses fautes ne se pouvaient
mettre sur le compte de personne\,; voilà pourquoi Albéroni saute le
combat à joints pieds. Suivons-le pendant la retraite.

«\,Pour ce qui regarde la retraite, dit-il, M. de Vendôme opina de ne la
point faire de nuit\,; mais comme de ce sentiment il n'y avait que lui
et le comte d'Évreux, il fallut céder.\,»

Voilà la première et seule vérité qui se trouve dans toute cette lettre,
mais frauduleusement estropiée. Non seulement Vendôme opina à ne se
point retirer de nuit, mais à ne se point retirer du tout, avec ses
\emph{sproposito} ordinaires, à disputer qu'il n'y avait rien de perdu,
qu'il se fallait tenir comme on pourrait chacun où il se trouvait, et
recommencer le combat dès qu'il serait jour. Au chaos qui était dans les
troupes, qui ne pouvait au moins diminuer pendant la nuit, sous le feu
des ennemis au triple d'elles, mêlées avec eux en des endroits,
enveloppées en d'autres, à portée de l'être encore plus par la
supériorité du nombre et l'audace du succès, sans qu'on pût y donner
aucun ordre, ni peut-être s'en apercevoir, comme avant la nuit il serait
arrivé à la maison du roi sans l'avis de l'officier ennemi pris par les
chevau-légers, à qui il porta un ordre les prenant pour des siens, on
laisse à penser ce que seraient devenues nos troupes pendant la nuit, et
de quel avantage on se pouvait flatter d'un combat si étrangement inégal
à recommencer avec le jour. La moitié de l'armée n'étant pas là, de
l'aveu de M. de Vendôme, contre toute celle des ennemis. Cette moitié,
battue partout, et partout en détail\,; combien de tués, de prisonniers,
de fuyards qui diminuaient encore ce petit nombre\,; peu de tués et de
blessés, point de fuyards parmi les victorieux, comme il arrive
toujours. L'autre moitié de l'armée serait arrivée, mais l'aurait-on su
placer à propos de nuit\,? Elle n'aurait donc approché que de jour, et
cependant le combat recommençait avec tous les désavantages que je viens
de remarquer. Malgré ce renfort, qui aurait démêlé la confusion de ce
renouvellement de combat, puisque la journée qui finissait n'avait cessé
de l'accroître\,? C'était donc achever de perdre cette première partie
de l'armée, sans nulle espérance raisonnable d'en tirer aucun succès, et
s'exposer ensuite avec l'autre moitié à la totalité de l'armée
victorieuse.

Voilà ce qui empêcha personne d'être de l'avis de M. de Vendôme, outre
qu'il n'y eut aucun de ce qui l'entendit qui ne fût indigné de
l'opiniâtreté avec laquelle il soutint qu'il n'était point battu,
excepté le peu de ceux qui, comme le comte d'Évreux, lui étaient vendus
sans réserve. M. de Vendôme parlait tellement contre sa pensée qu'il
céda contre son orgueil et sa coutume. Il voulait ou ce qu'il n'est pas
permis de penser, ou par une fanfaronnade si déplacée montrer qu'il
n'était point abattu, et faire accroire qu'il avait des ressources dans
sa capacité, quoique si éclipsée avant et pendant toute l'action. Il
devait bien sentir que qui que ce soit ne se laisserait persuader qu'il
n'y avait rien de perdu, qu'il fût raisonnable ni même possible de
demeurer toute la nuit comme on était, et de se commettre de nouveau,
dès qu'il serait jour, à recommencer un combat aussi désavantageux. Il
ne chercha donc qu'à imposer sur son courage de coeur et d'esprit, et à
se préparer pour la suite de quoi donner du spécieux aux ignorants et
aux sots, et à sa cabale de quoi dire, et rejeter toute la honte sur Mgr
le duc de Bourgogne, par l'énorme propos qu'il osa lui tenir, et
qu'Albéroni remet adroitement sous les yeux par ces paroles\,: «\,A
peine, continue sa lettre, eut-il (Vendôme) dit à M. le duc de
Bourgogne, que l'armée n'avait qu'à se retirer, que, tout le monde à
cheval, avec une précipitation étonnante, chacun gagne Gand, jusqu'à
conseiller aux princes de prendre des chevaux de poste à Gand, pour
gagner Ypres.\,»

Ce verbiage est bien artificieux, mais Albéroni s'y trahit lui-même du
premier mot. «\,A peine eut-il dit, etc.\,» Cela montre bien que celui à
qui il le dit n'était le maître de rien, puisqu'il fallut attendre cette
parole de M. de Vendôme pour que la retraite se fit. Par conséquent,
c'était à lui à la régler, à l'ordonner, à prescrire aux officiers
généraux qui étaient là, les dispositions de cette retraite, et en
envoyer les ordres à ceux qui n'y étaient pas. Attendait-il cela de la
capacité d'un prince de l'âge de Mgr le duc de Bourgogne, ou de son
autorité qu'il lui avait si nettement et si fraîchement déclarée être
nulle en sa présence\,? L'attendait-il du maréchal de Matignon qui, à
l'opprobre de son office, lui était subordonné en tout\,? L'attendait-il
des officiers généraux qui se trouvèrent là\,? En un mot, on voit un
homme qui ne sait plus depuis longtemps où il en est, qui ne conserve de
sens que pour jeter de la poudre aux yeux et rejeter ses fautes et sa
honte sur Mgr le duc de Bourgogne\,; qui dit que l'armée se peut retirer
et qu'il faut aller à Gand\,; qui n'ajoute pas un mot de plus, et qui en
laisse l'ordre et la manière à l'abandon et au hasard. Après cela,
Albéroni a bonne grâce de dire que chacun s'en alla avec
précipitation\,! Que peuvent devenir des gens qui n'ont point d'ordre,
qui n'osent en demander à un général qu'ils voient avoir perdu la
tramontane et ne savoir ce qu'il dit, être furieux jusqu'à insulter
l'héritier nécessaire de la couronne\,? Il est aisé de comprendre que
personne ne se hasarda à aucune question, que chacun se hâta de
s'éloigner d'un homme aussi dangereux, mais aussi roide à la repartie,
et que dans ce chaos nocturne, où personne ne reconnaissait ni sa
division, ni même sa troupe, chacun devint ce qu'il put, regardant
seulement Gand comme le lieu où se rassembler.

La proposition faite aux princes de gagner Ypres, de Gand, en poste et
celle de les mettre dans leurs chaises de poste avec une escorte, pour
gagner Gand, contre laquelle M. de Vendôme cria et qu'il empêcha, sont
des choses qui, n'ayant pas été goûtées d'eux ni exécutées, ne peuvent
aussi leur être imputées. La première était tout à fait ridicule, mais
elle n'était que cela, puisque, l'armée se retirant sur Gand, la crainte
du danger ne pouvait causer ce conseil. Celle des chaises de poste vint
d'un homme dont on n'accusera pas la valeur, ni le courage d'esprit, ni
l'ignorance en matière d'honneur. L'idée en vint à Puységur, qui fait
aujourd'hui l'honneur des maréchaux de France, trop frappé en ce moment
de la fatigue des princes qui, après avoir passé toute la journée à
cheval, avaient encore toute la nuit et la matinée à y être. Voyant
d'ailleurs la confusion inévitable avec laquelle cette retraite s'allait
faire, qui ne s'exécuterait que par parties séparées les unes des
autres, il n'imagina pas que les princes dussent suppléer à ce que M. de
Vendôme abandonnait à l'aventure, ni entreprendre de mettre en ordre un
si étrange chaos. Mais, sans pousser plus loin cette discussion, elle
devient inutile dès qu'il demeure sans contestation certain que les
princes n'adoptèrent ni n'exécutèrent ni l'une ni l'autre. Retournons à
la lettre, M. de Vendôme, continue-t-elle, qui fut obligé de faire une
grande partie du temps l'arrière-garde avec ses aides de camp, arriva
sur les neuf heures du matin (à Gand), prit sur-le-champ sa résolution
ferme de vouloir mettre l'armée derrière le canal qui est entre Gand et
Bruges, malgré l'avis de tous les officiers généraux qui l'ont persécuté
trois jours durant de l'abandonner, disant qu'il fallait tâcher de
joindre l'armée de M. de Berwick. Une telle fermeté a sauvé l'armée du
roi et le royaume, car l'épouvante qui était dans l'armée aurait causé
une esclandre bien pire que celle de Ramillies, au lieu que M. de
Vendôme se mettant derrière le canal, il a soutenu Gand et Bruges, qui
est un point essentiel, et, a rassuré les esprits et redonné la
confiance aux troupes, a donné lieu aux officiers de se reconnaître et
de connaître leur terrain, et enfin a mis les ennemis dans l'inaction,
et vous pouvez être sûr que, s'ils veulent faire un siège, il faut
qu'ils fassent celui d'Ypres ou de Lille, de Mons ou de Tournai.

La transition est admirable. M. de Vendôme fut obligé de faire une
grande partie du temps l'arrière-garde avec ses aides de camp. Mais qui
fait donc une arrière-garde en se retirant de devant les ennemis, si ce
n'est celui qui est chargé de l'armée\,? Mais où la fit-il M. de
Vendôme\,? que rassembla-t-il pour la faire\,? où parut-il\,? quels
ordres y donna-t-il\,? S'il n'eut que ses aides de camp avec lui,
qu'étaient devenues les troupes\,? Et pourquoi Albéroni omet-il de
marquer quelles furent celles que son héros honora de sa présence en
cette occasion\,? Voilà peut-être la première retraite où il n'ait été
mention nulle part du général, mais celle-ci tint du reste de la
journée. Chacun fit la sienne à part comme il put et voulut, et il ne se
peut une démonstration plus claire de cette vérité, outre le témoignage
de toute l'armée, que l'oubli des cent escadrons à la tête desquels le
chevalier du Rosel se trouva le lendemain en sa même place, sans avoir
reçu ni ordre ni avis de qui que ce fût, abandonné de toute l'armée
retirée pendant la nuit. Oublier cent escadrons, les laisser seuls à la
merci de l'armée victorieuse, il est bien difficile de trouver une
preuve plus évidente qu'un général a perdu absolument la tête, et qu'il
n'est occupé que de la retraite de sa personne, qu'il fait seul avec ses
aides de camp dans un oubli parfait de toutes ses troupes, et dans
l'incurie entière de ce que son armée devient. C'est un fait qui ne se
peut ni contester ni pallier, et qui prouve démonstrativement tout ce
que je viens de dire\,; aussi n'a-t-il été ni contesté ni pallié. M. de
Vendôme, avec son audace accoutumée, n'a pas fait le moindre semblant de
le savoir, ses défenseurs l'ont passé sous silence et se sont flattés
d'en étouffer la voix par le bruit et la hardiesse de leurs clameurs.

Albéroni a recours ici à la même ruse de la confusion des heures dont il
s'était servi sur celle de l'arrivée des avis de Biron au duc de
Vendôme. Il le fait arriver ici à Gand sur les neuf heures du matin.
C'est toujours près de deux heures de plus données à son arrière-garde
imaginaire. Mais il se donne bien garde de faire mention de ce qu'il
devint à Gand, ni de ce qu'il y trouva, ni combien il y resta. Trente
heures de lit sans s'informer ni des princes, ni de l'armée, ni de ce
que chacun était devenu ni devenait, tout cela est de même parure que
tout le reste, et que l'oubli total du chevalier du Rosel et de ses cent
escadrons. Albéroni, qui le sent, coule rapidement et se jette à la
résolution d'un poste admirable, malgré tous les officiers généraux.
Mais la vérité est que ce poste était déjà pris avant que le duc de
Vendôme y eût plus songé qu'à son armée, et qu'il ronflait
tranquillement dans son lit à Gand avant d'y avoir pensé, tandis que les
princes étaient venus dans ce même poste avec ce qui avait pu y arriver
de troupes qui s'y rendirent successivement. Puységur, si longuement et
si savamment maréchal des logis de l'armée de Flandre, et sur lequel M.
de Luxembourg s'est toujours si utilement reposé de ses marches, de ses
campements, de ses fourrages et de tous les terrains, était bien l'homme
à donner ce conseil à Mgr le duc de Bourgogne, et Vendôme et les siens à
se l'approprier après. Il est vrai qu'après que Vendôme fut arrivé à
Lawendeghem, il y eut des raisonnements sur ce que dit Albéroni, et
qu'il fut résolu de s'arrêter dans ce camp. Mais le choix et la fermeté
à y rester sont des louanges gratuites, dont le bruit n'est bon qu'à
couvrir tout ce qui vient d'être remarqué, et qui a été trop public pour
oser être contesté.

Albéroni prétend que ce camp si savamment choisi a rendu la confiance
aux troupes et réduit les ennemis à l'inaction. Il vit bientôt l'Artois
sans contribution, M. de Berwick tout occupé à le protéger, de gros
détachements de la grande armée y marcher encore, et néanmoins n'y
pouvoir empêcher le désordre. Ce n'est pas là une inaction et dans un
pays jusqu'alors si fort éloigné de ces ravages. À l'égard de la
confiance, pas un officier supérieur n'en eut en M. de Vendôme. La
licence, le peu de subordination, la tolérance de tout, la familiarité
affectée avec le menu avaient gagné le soldat, le cavalier, le dragon,
le menu officier, et la jeunesse débauchée, inappliquée, licencieuse.
Tout cela adorait M. de Vendôme, tout cela faisait la multitude et le
cri public, tout cela se répandait dans les garnisons, dans les
provinces, dans Paris, où la cabale savait bien en tirer toutes sortes
d'avantages. «\,Or vous voyez, continue la lettre, quelles places\,! et
si jamais ils attaquent quelques-unes de ces places, M. de Vendôme
prendra Audenarde, et se rendra maître de tout l'Escaut, et vous n'avez
qu'à regarder la carte pour voir combien les ennemis seraient
embarrassés.\,»

Cela s'appelle payer bien hardiment d'effronterie. L'impossibilité de la
négative force Albéroni de laisser glisser un aveu tacite que le succès
de ce combat met les ennemis en moyen de faire le siège de celle de ces
quatre grandes places qu'ils voudront\,; et il tâche d'éblouir
là-dessus, en promettant les prouesses de son héros sur Audenarde en ce
cas, et sur l'Escaut. Il sent bien ce que c'est qu'Audenarde pour être
le juste équivalent d'une de ces places si importantes, dont les unes
ferment toute entrée dans le pays ennemi, et les autres l'ouvrent
entièrement dans le nôtre\,; il renvoie donc à la carte par une habile
réticence, comptant bien que le très grand nombre qui ne connaît rien
par rapport aux mouvements des armées, l'en croira sur sa parole, en les
étourdissant de ce grand mot de devenir maîtres de l'Escaut. La suite de
cette campagne infortunée a montré les avantages que M. de Vendôme sut
tirer de sa défaite et de vanteries prématurées de son valet. Je n'aurai
que trop lieu de m'y étendre lorsqu'il en sera temps. Achevons la
lettre. «\,Voilà, dit-elle, la pure vérité, la même que M. de Vendôme a
mandée au roi, et que vous pouvez débiter sur mon compte. Je suis
Romain, c'est-à-dire d'une race à dire la vérité. \emph{In civitate
omnium gnara et nihil reticente}, dit notre Tacite.\,»

Après avoir suivi mot à mot Albéroni, comme je viens de faire, et
montré, avec une évidence à laquelle on ne se peut refuser, que sa
lettre n'est qu'un tissu d'artifices et de mensonges, les uns adroits,
les autres hardis, sans mélange d'aucune trace de vérité, il n'y a plus
à répondre à cette forfanterie. Jusqu'à son origine qu'il ose débiter en
preuve est fausse, outre qu'il y a bien loin de Rome du temps de Tacite
et de son histoire à Rome d'aujourd'hui, et des personnages peints dans
cette histoire à un homme de la lie du peuple, tel qu'Albéroni. Avec un
peu de jugement, il eût évité de citer celui qui nous a montré Séjan
dans tous ses vices, ses desseins pernicieux, sa superbe, l'abus si
dangereux de sa faveur, et qui en opposite nous a laissé la vie
d'Agricola, également bon citoyen, et véritablement grand dans la paix
et dans la guerre. On n'a pas peine à voir auquel des deux M. de Vendôme
ressemble le plus. Mais Albéroni Romain\,! Il était d'un petit village
auprès de Bayonne, où ses parents, vinrent d'Italie s'habituer. Pourquoi
une transplantation si éloignée\,? Elle sent bien le crime et la fuite
de la punition, mais je l'ignore, parce qu'on ne s'est pas avisé encore
de donner l'histoire des Albéroni. Son père y vivait de son métier de
jardinier et vendait tous les jours des fruits, et plus encore des
légumes, à Bayonne, où mille gens l'ont ouï dire à leur père, et où
quelques-uns encore l'ont vu. Celui-ci s'en retourna dans son village
originaire, près de Parme. J'ai raconté ailleurs comment il fut connu du
duc de Parme, qui lui fit prendre le petit collet pour qu'il pût
approcher de ses antichambres, à l'occasion de quoi il s'en servit
auprès de M. de Vendôme, et par quelles bassesses et quelles infamies il
le gagna, combien il fut le rebut des bas valets et de leur table, et
les coups de bâton qu'il en reçut en pleine marche d'armée, sans que M.
de Vendôme fût ému de ses plaintes et de ses pleurs. Le voici maintenant
devenu son principal confident et son apologiste. Il continue\,:
«\,Permettez-moi après cela que je vous dise avec tout le respect que je
vous dois (c'était une lettre faite pour courir, et qui n'était écrite à
personne), que votre nation est bien capable d'oublier toutes les
merveilles que ce bon prince a faites dans mon pays, qui rendront son
nom immortel et toujours révéré, \emph{injuriarumn et beneficiorum aeque
immemores}. Mais le bon prince est fort tranquille, sachant qu'il n'a
rien à se reprocher, et que pendant qu'il a suivi son sentiment, il a
toujours fort bien fait.\,»

Albéroni ne pouvait mieux terminer sa lettre. Il y dit enfin au moins
une vérité\,: c'est que, de tout ce qui se disait, M. de Vendôme n'en
était pas moins tranquille. Son audace le soutenait contre la clarté du
jour\,; de plus il connaissait ses forces. Il les avait tant de fois si
heureusement essayées qu'il ne craignit pas de les éprouver contre
l'héritier nécessaire de la couronne. Il avait de forts croupiers,
l'intérêt était grand et commun, les mesures bien prises\,; pour cette
fois Albéroni a dit une vérité. Mais de nous parler de l'Italie et des
merveilles de son héros, qu'en dirent le prince Eugène et Staremberg,
qu'en dirent tous les officiers principaux, quand par son retour le
bâillon leur tomba de la bouche\,? Il y laissa tout perdu, et il le
sentit si bien que sa plus grande joie fut de quitter l'Italie. J'ai
raconté tous ces faits en leur temps, et avec quelle précipitation il en
partit sans avoir voulu donner quelques jours de plus à la nécessité la
plus urgente, ni instruire et rendre raison de rien à M. le duc
d'Orléans qui lui succédait, parce qu'il ne sut que lui dire.

\hypertarget{chapitre-xvii.}{%
\chapter{CHAPITRE XVII.}\label{chapitre-xvii.}}

1708

~

{\textsc{Campistron et sa lettre.}} {\textsc{- Lettre du comte d'Évreux
à Crosat\,; son caractère.}} {\textsc{- Grand sens de la duchesse de
Bouillon et son adresse.}} {\textsc{- Succès de ces lettres.}}
{\textsc{- Mesures pour Mgr le duc de Bourgogne.}} {\textsc{- Duchesse
de Bourgogne.}} {\textsc{- Le roi impose à demi sur les lettres.}}
{\textsc{- Adresse des Bouillon.}} {\textsc{- Vigueur de la cabale de
Vendôme.}} {\textsc{- Chamillart conseille mal Mgr le duc de Bourgogne
pour tous deux.}} {\textsc{- Époque de la haine pour Chamillart de
M\textsuperscript{me} la duchesse de Bourgogne.}} {\textsc{- Singulière
adresse du duc de Vendôme auprès de M\textsuperscript{me} la duchesse de
Bourgogne.}}

~

Cette lettre d'Albéroni inonda en peu de jours la cour, la ville, les
provinces. Deux jours après qu'elle eut commencé à se débiter et à
étonner par sa hardiesse, il s'en distribua une autre, mais avec grande
mesure. J'en vis une entre les mains du duc de Villeroy. Il ne l'avait
que pour quelques heures avec promesse de n'en point laisser tirer de
copies, et je jugeai qu'elle lui venait de Bloin, son grand ami de table
et de plaisir. Elle était de Campistron, qui ne s'en cachait pas, et qui
en était donné pour auteur par ceux qui la montraient. Campistron était
de ces poètes crottés qui meurent de faim et qui font tout pour vivre.
L'abbé de Chaulieu l'avait ramassé je ne sais où, et l'avait mis chez le
grand prieur, d'où, sentant que la maison croulait, il en était sorti
comme les rats et s'était fourré chez M. de Vendôme. Quoique son
écriture ne fût pas lisible, il était devenu son secrétaire,
inconvénient qui dans la suite valut toute la confiance de M. de Vendôme
à Albéroni, auquel il dictait les lettres qu'il ne voulait pas exposer
aux copistes de Campistron. Sa lettre était bien écrite pour le style,
écrite même en homme de guerre à faire juger qu'un autre que lui y avait
mis la main. Elle était, comme celle d'Albéroni, un tissu de mensonges
sans un seul mot de vérité, mais dont le profond artifice, adroitement
conduit, se présentait avec toute la délicatesse et le spécieux le plus
propre à lui donner un air de vérité, en couvrant en même temps tout le
vrai de ténèbres et à rebuter de les vouloir percer. Tout l'art possible
y est principalement employé, et on voit que c'est tout le but de la
pièce, au dessein de tomber à plomb sur Mgr le duc de Bourgogne, de
l'attaquer personnellement sur tout ce qui est le plus sensible, et de
lui arracher ce que les hommes ont de plus précieux. Il ne se peut une
pièce mieux faite dans cette vue, ni plus cruellement assenée. Ses
moindres traits sont d'appeler Gamaches et d'O les gouverneurs des
princes\,; de les nommer des marauds\,; de dire que le maréchal de
Matignon méritait d'être mis au conseil de guerre, malgré sa dignité,
pour avoir été de leur avis sur la retraite\,; que M. de Vendôme les
avait publiquement traités ainsi, et en face, et parlant à eux, et qu'il
en avait écrit au roi en mêmes termes.

L'énormité de cette lettre, en comparaison de laquelle celle d'Albéroni
n'était que fleurs et mesure, en fit faire les différents usages. Celle
d'Albéroni fut répandue à pleines mains pour préparer, soulever,
exciter\,; l'autre ne se confia qu'en mains sûres pour la montrer
partout, mais avec un air de mystère et de confiance qui ajoutât à
séduction, et qui fît valoir, aux dépens de Mgr le duc de Bourgogne, le
malheur de l'État que M. de Vendôme n'eût pas été cru, et le sien
d'avoir affaire à un prince, contre qui, avec de si bonnes raisons, il
ne lui était pas permis de se défendre en révélant tout ce qui s'était
passé. Avec cette adresse, la pièce ne laissa pas d'être vue jusque dans
les cafés, les spectacles, et les autres lieux publics de jeux, de
débauche, et même de promenades publiques, et parmi les nouvellistes. On
eut soin qu'elle ne fût pas ignorée dans les provinces, et jusque dans
les pays étrangers, mais toujours avec tant de précautions qu'on
demeurât les maîtres de toutes les copies, également actifs à la
répandre partout, et précautionnés à n'en laisser échapper aucune dont
ils auraient trop craint l'usage contre eux.

Le comte d'Évreux fut le seul de son état qui se mit de niveau avec ces
deux valets. Né quatrième cadet de M. de Bouillon, avec une figure fort
ordinaire et un esprit au-dessous, le jargon du monde et surtout celui
des femmes, et tout ce qu'il avait en lui tourné à l'ambition, suppléa
aux autres qualités, avec des vues et une certaine adresse. J'ai raconté
dans le temps par quelles routes il parvint à la charge de la cavalerie,
et le triste mariage qu'il fit, qui fut un nouveau lien pour lui au duc
de Vendôme. Ils étaient enfants des deux soeurs, et son beau-père
s'était chargé des affaires de Vendôme. Il s'attacha de plus en plus à
lui, et il compta par son secours sur une rapide fortune. Il s'y livra
d'autant plus entièrement que Vendôme lui donna tous les agréments qu'il
put dans l'armée, et par charge et personnellement, et qu'il l'avait
fort aidé l'hiver précédent aux décisions que le roi fit en faveur de sa
charge contre celle de colonel général des dragons qu'avait Coigny. Le
comte d'Évreux, qui voyait ses frères dans la disgrâce, et hors de toute
espérance du côté du roi, et fort peu de celui de leur père, ne visait
pas à moins qu'à sa charge de grand chambellan, et comptait que, pour
l'emporter, il ne lui fallait rien moins que toute la protection du duc
de Vendôme. Telle fut la cause de son abandon à lui, du personnage qu'il
crut faire en cette journée d'Audenarde, et qu'il voulut couronner en se
faisant son champion par un raffinement de politique.

Il écrivit donc à Crosat une apologie de M. de Vendôme dans le même
esprit des deux dont je viens de parler, et qui ne cédait guère à
Campistron sur le compte de Mgr le duc de Bourgogne, duquel il avait
toujours été traité avec une bonté marquée, mais de qui il n'espérait
pas comme de M. de Vendôme, auquel il jugea qu'il ne pouvait faire un
sacrifice plus agréable, ni qui l'engageât plus puissamment à un grand
retour. Cette lettre était faite pour être montrée, et Crosat n'avait
garde de la retenir captive. Touché de l'honneur du maître auquel il
s'était donné, plus encore de se parer d'une lettre que lui écrivait un
gendre dont il se faisait un si grand honneur, il la montra quatre jours
durant à qui la voulut voir, et en laissa échapper quelques copies. Le
bruit qu'elle fit réveilla M\textsuperscript{me} de Bouillon, qui, avait
infiniment d'esprit et qui frémit des suites. Elle courut chez Crosat,
lui chanta pouille d'avoir ainsi commis son fils, avec cette hauteur et
cet air imposant dont elle savait faire un si grand usage, n'eut point
de repos qu'elle n'eût retiré le peu de copies que Crosat en avait
laissé glisser, et dépêcha à son fils pour lui faire honte et peur de sa
folie, et lui demander une autre lettre à Crosat qu'on pût faire passer
pour la première et l'unique, puisqu'il n'y avait pas moyen de nier
qu'il lui en avait écrit une, et qui fût tournée de manière à pouvoir
être montrée sans danger et néanmoins passer pour la première. Je ne
sais si elle lui en envoya le modèle, mais son courrier la rapporta
telle qu'elle la désirait. On verra bientôt le grand parti qu'elle en
sut tirer.

En même temps que la lettre d'Albéroni et les extraits retenus des deux
autres devinrent publics, la cabale se déchaînait par degrés en cadence.
Leurs émissaires paraphrasaient les lettres dans les cafés, dans les
lieux publics, parmi la nation des nouvellistes, dans les assemblées de
jeu, dans les maisons particulières. Les halles mêmes, dont Beaufort fut
roi si longtemps dans la minorité de Louis XIV, en furent remplies\,;
les mauvais lieux, le pont Neuf, en retentirent\,; les provinces les
plus éloignées en furent soigneusement remplies. Les vaudevilles, les
pièces de vers, les chansons atroces sur l'héritier de la couronne, et
qui érigeaient sur ses ruines Vendôme en héros, coururent par Paris et
par tout le royaume avec une licence et une rapidité qu'on ne se mit en
aucun soin d'arrêter\,; tandis qu'à la cour et dans le grand monde, les
libertins et le bel air applaudit, et que les politiques raffinés, qui
connaissaient mieux le terrain, s'y joignirent et entraînèrent si bien
la multitude qu'en six jours il devint honteux de parler avec quelque
mesure du fils de la maison dans sa maison paternelle. En huit cela
devint dangereux, parce que les chefs de meute, encouragés par le succès
de leur cabale si bien organisée, commencèrent à se montrer, à prendre
fait et cause, et à laisser sentir qu'ils la regardaient tellement comme
la leur que quiconque oserait contredire aurait tôt ou tard affaire à
eux.

Dès avant ce fracas, le duc de Beauvilliers, rempli de tout ce que je
lui avais dit dans les jardins de Marly sur la destination de Mgr le duc
de Bourgogne, et informé par ses lettres de Flandre était venu dans ma
chambre nie faire comme une amende honorable, le coeur pénétré de
douleur. Je me contentai de le prier de comprendre qu'on ne gagnait rien
en place à ignorer tout ce qui se passait à la cour, les intérêts, les
liaisons, les vues, les motifs, et de se persuader enfin que mon
éloignement du rang, des prétentions, des vices, des personnes, ne me
faisait point bâtir des chimères. Je convins avec lui, lors du fracas,
qu'il était hors du vraisemblable\,; mais je le priai de s'avouer aussi
que les choses les moins croyables arrivaient plus souvent qu'on ne
pensait, et n'étaient pas au-dessus de la prévoyance, quand, au temple
de l'ambition, on ne captive pas son esprit jusqu'à méconnaître les
ambitieux, et à se faire un scrupule de croire des gens capables de tout
ce qu'elle leur inspire, dans des places, dans une faveur et dans des
apparences favorables à y réussir. Nous raisonnâmes beaucoup, et à bien
des reprises, lui, le duc de Chevreuse et moi, sur les moyens d'ouvrir
les yeux au roi et d'arrêter cette furie. Ce n'était pas que tout fût
corrompu à la cour en faveur du duc de Vendôme\,; mais la crainte
arrêtait, et la plus qu'apparente inutilité de s'opposer au torrent
persuadait le silence et l'inaction. Boufflers et bien d'autres étaient
de ceux-là.

Nous convînmes, les deux ducs et moi, de ce qu'il fallait faire passer à
Mgr le duc de Bourgogne sur sa conduite à tenir tant là qu'ici, pour ses
lettres, et cependant je faisais avertir M\textsuperscript{me} la
duchesse de Bourgogne, par M\textsuperscript{me} de Nogaret, de tout ce
que je jugeais qu'elle devait savoir et faire. Elle-même m'envoyait
cette dame consulter avec moi, et me dire franchement où elle en était
avec le roi et M\textsuperscript{me} de Maintenon, ce qu'elle y pouvait,
et ce qu'elle n'y pouvait pas. Je ne crois pas qu'elle eût de goût pour
la personne de Mgr le duc de Bourgogne, ni qu'elle ne se trouvât
importunée de celui qu'il avait pour elle. Je pense aussi qu'elle
trouvait sa piété pesante, et d'un avenir qui le serait encore plus.
Mais parmi tout cela elle sentait le prix et l'utile de son amitié, et
de quel poids serait un jour sa confiance. Elle n'était pas moins
touchée de sa réputation, d'où dépendait tout son poids pendant bien des
années, jusqu'à ce qu'il en pût avoir par lui-même devenu roi, et que,
jusque-là, succombant à cet orage, déshonoré, et par conséquent l'objet
de la honte et de la peine du roi et de Monseigneur, il n'en pouvait
résulter que les plus grands malheurs, au moins la plus triste vie, dont
il était impossible qu'elle-même ne portât sa part. Je lui fis
comprendre par la même dame à qui elle avait affaire. Elle était fort
douce, et encore plus timide\,; mais la grandeur de l'intérêt l'excita
par-dessus son naturel. Elle se trouva de plus cruellement piquée et
offensée des insultes de Vendôme à son époux, parlant publiquement à
lui, et de tout ce que ses émissaires publiaient d'atroce et de faux.
Quelque mesuré, quelque en garde que la conscience de Mgr le duc de
Bourgogne le retînt contre lui-même, il n'avait pu s'empêcher de
répandre son coeur dans ses lettres à son épouse, qui, avec ce qui lui
revint d'ailleurs, furent pour elle de vifs aiguillons. Elle fit donc
tant et si bien qu'elle l'emporta auprès de M\textsuperscript{me} de
Maintenon sur les artifices voilés, et les charmes enchanteurs pour elle
de M. du Maine. Elle la gagna, elle l'émut, elle l'engagea de parler au
roi assiégé de toutes parts, et auprès duquel il n'y avait qu'elle qui
pût percer en faveur de la vérité et de son petit-fils. La princesse y
réussit jusqu'à opérer un miracle.

Depuis l'éclat de l'affaire de l'archevêque de Cambrai,
M\textsuperscript{me} de Maintenon, qui avait échoué à culbuter M. de
Beauvilliers, ne l'avait vu que par des hasards rares, et encore plus
rarement lui avait dit quelques paroles générales. Mais jamais un
particulier d'un instant, elle l'avait toujours regardée en ennemie. En
cette occasion, le désir de servir la princesse et le prince lui fit
vouloir un entretien particulier avec le duc pour se concerter avec lui
et se bien instruire des faits. Elle en eut plusieurs, et, lui confia ce
qui se passait d'elle au roi là-dessus à mesure, et raisonnait avec lui
sur ce qu'il y aurait à dire et à faire. Ce n'était pas qu'elle lui eût
pardonné d'être demeuré en place malgré elle\,: on le verra en son lieu.
Mais tant qu'elle eut besoin de ses lumières et de son concert pendant
toute cette campagne, elle se livra à lui de bonne foi sur tout ce qui
en concerna les événement et les suites, et lui aussi en profita dans
les mêmes vues, et se concerta avec elle en tout avec la même confiance.
Dans tout cela je ne fus pas seulement nommé à M\textsuperscript{me} de
Maintenon, ni d'elle, mais je savais tout ce qui se passait d'elle par
M. de Beauvilliers et par M\textsuperscript{me} de Nogaret.
M\textsuperscript{me} de, Maintenon ébranla le roi et le piqua ensuite
en lui apprenant les lettres et tout ce qui était répandu. Il en parla
en plein conseil d'État et demanda avec quelque chaleur si on n'en avait
pas ouï parler. On répondit un peu en tâtonnant qu'on n'avait vu que
celle d'Albéroni\,; et comme le roi témoigna curiosité de la voir,
Torcy, qui, timidement mais de tout son coeur, était indigné de tout ce
qui se publiait, et qui peut-être, averti par Beauvilliers, s'en était
nanti à tout hasard, la tira de sa poche, et, par ordre du roi, en fit
la lecture.

Le roi se récria, mais toutefois ménageant un peu M. de Vendôme, et
demanda assez sévèrement à Chamillart pourquoi il ne lui avait point
parlé de ces lettres. Il s'en tira en niant qu'il les eût vues\,; mais
sur-le-champ il reçut ordre du roi d'écrire de sa part à Vendôme, à son
Albéroni (ce fut son terme), à Crosat et à son gendre (ce fut encore son
expression), des lettres fortes, et aux trois derniers qu'ils
mériteraient punition, et ordre de demeurer dans le silence. À Crosat en
particulier, défense de laisser voir à qui que ce fût la lettre du comte
d'Évreux, et cela fut exécuté aussitôt. Je ne comprends pas comment
Campistron fut oublié. Le roi sentit peut-être que la gravité de son
crime demandait plus que des paroles, et voulut éviter à Vendôme un
châtiment qui retombait sur lui. Les ministres, de leur côté, timides,
se contentèrent de répondre et n'osèrent rien dire de leur chef. Telle
était la terreur de Vendôme et de sa cabale jusque dans le conseil du
roi, et telle la réduction de la vérité et de Mgr le duc de Bourgogne
dans l'intimité du cabinet du roi, son grand-père.

Crosat sortit mieux d'affaire par la prévoyance que j'ai remarqué
qu'avait eue M\textsuperscript{me} de Bouillon. M. de Bouillon arrivait
de Turenne où il avait fait un voyage, dans lequel il s'était donné la
plate satisfaction de brûler le maréchal de Noailles en effigie de
paille et de carton à califourchon sur son petit château d'Ayen, comme
les Anglais brûlent un pape de paille tous les ans à Londres. Ils
étaient alors dans la plus grande animosité de leur éternel procès sur
la mouvance et les droits de Turenne. Il trouva tout ce vacarme.
Instruit par sa femme de ce qu'elle avait fait, ils distribuèrent la
seconde lettre du comte d'Évreux, qu'ils assurèrent fermement être
l'unique que leur fils eût écrite, et la véritable, qui, sans parler des
généraux, disait seulement qu'il n'y avait rien de gâté, et que l'armée
était de quatre-vingt mille hommes, pleine de courage, et s'en tenait
sur ces généralités sans entrer en rien. Ils blâmèrent l'imprudence du
comte d'Évreux, et M. de Bouillon alla porter cette lettre au roi, et
lui faire une apologie, dont le besoin et le fréquent usage de sa race
leur ont donné à tous une grande expérience. Mais cette seconde lettre
en disait trop peu pour pouvoir passer pour la première. Il se trouva
des gens charitables qui le firent sentir au roi et à
M\textsuperscript{me} de Maintenon, et qui leur contèrent le tour de
politique et de sagesse de M\textsuperscript{me} de Bouillon, de sorte
qu'ils n'en furent pas les dupes. Pour Mgr le duc de Bourgogne, {[}il{]}
le fut ou le voulut bien être tout du long. Il reçut les apologies et
les protestations du comte d'Évreux, et chercha à lui faire oublier le
dégoût de la réprimande que le roi lui avait fait faire, par lui marquer
des bontés et des distinctions qui scandalisèrent étrangement contre
lui, et qui refroidirent à son égard l'armée, et beaucoup de ceux qui
tenaient pour lui à la cour.

La cabale fut étourdie de voir M\textsuperscript{me} de Maintenon
échapper à M. du Maine, et se dévouer à M\textsuperscript{me} la
duchesse de Bourgogne, de ce que le roi avait dit au conseil qui, avec
raison, en était regardé comme le fruit, et des lettres que Chamillart
avait eu ordre d'écrire. Mais, réflexion faite, ils trouvèrent que le
peu que le roi avait dit et fait répondait peu à ce qu'il devait à son
petit-fils, et à ce qu'il donnait toujours à l'empire qu'il avait laissé
prendre à M\textsuperscript{me} de Maintenon sur lui. Ils en conclurent
que le roi avait été entraîné plutôt qu'aigri, et qu'en tenant ferme,
ils l'embarrasseraient entre son goût si décidé pour M. du Maine, pour
M. de Vendôme, pour la bâtardise en général, pour ses valets principaux
en particulier, et sa déférence d'habitude pour M\textsuperscript{me} de
Maintenon, et son amitié d'amusement pour M\textsuperscript{me} la
duchesse de Bourgogne\,; et que, s'ils pouvaient tenir bon comme ils
avaient commencé, le roi se laisserait moins aller à l'une et à l'autre
qu'il ne s'en trouverait importuné et fatigué, et assez peut-être pour
leur fermer la bouche. Au pis aller, ils virent aller leurs desseins en
fumée par toute autre conduite\,; ils y sacrifièrent donc tout, et
redoublèrent de jambes à répandre ces lettres et tout ce qu'ils purent
inventer de plus atroce sous l'artifice le plus captieux. Ils étaient
trop bien conduits pour se méprendre. Bloin et M. du Maine connaissaient
bien le roi\,; ils l'obsédaient\,; il se plaisait à l'être par eux\,; le
goût et l'habitude y était. Les cris de M\textsuperscript{me} la
duchesse de Bourgogne redoublèrent à mesure que la cabale redoubla ses
coups\,; M\textsuperscript{me} de Maintenon l'appuya, et le roi s'en
rebuta au point qu'il gronda durement plus d'une fois la princesse, et
lui reprocha qu'on ne pouvait plus tenir à son humeur et à son aigreur.
Ce coup porta jusqu'en Flandre. Chamillart, régenté par Vaudemont et ses
nièces, et si enivré du duc du Maine et de M. de Vendôme, dont l'intérêt
le plus vif était d'achever la perte radicale du jeune prince, d'autant
plus nécessaire à achever qu'elle était si publiquement commencée,
Chamillart, dis-je, se laissa induire à écrire à Mgr le duc de Bourgogne
{[}une lettre{]} par laquelle, oubliant ce qu'ils étaient l'un et
l'autre, il lui conseillait de bien vivre avec M. de Vendôme.

Cette lettre fit tout l'effet qu'en avaient espéré ceux qui l'avaient
ménagée. Mgr le duc de Bourgogne, si brillant à Nimègue avec le maréchal
de Boufflers, et à Brisach entre Tallard et Marsin, avait été abattu dès
l'ouverture de la campagne par les contrariétés et les procédés
audacieux que Vendôme avait affectés avec lui. Élevé dans la frayeur du
roi, ce serait trop peu dire la crainte, elle s'étendait jusqu'à ceux
qui avaient son affection et sa confiance au point qu'il ne pouvait
douter que Vendôme les possédait. Sa sagesse le rendait défiant de
soi-même, et sa dévotion extrême, mais encore peu éclairée jusqu'aux
discernements nécessaires, le rapetissait et l'étrécissait. Sensible au
point où il était, la conduite de Vendôme à son égard et les deux propos
qu'il avait eu l'insolence de lui adresser en public, le tenaient de
court par religion à proportion de la colère et de l'indignation qu'il
en avait conçues. Gamaches et d'O n'étaient pas ses confidents, et ne
l'auraient pas même été bons, et il n'avait personne dans l'armée à qui
ouvrir son coeur et par qui s'éclairer.

Les lettres de M. de Beauvilliers étaient, comme lui, remplies de piété,
de modération, de mesure\,; celles de M\textsuperscript{me} la Duchesse,
il n'en avait pas la même opinion. Il n'en recevait point d'autres, et
il était abandonné à son chagrin et à ses réflexions. L'embarras où il
se trouva changea l'extérieur qui jusqu'alors avait tant plu à l'armée.
Il se renferma dans son cabinet à écrire de longues lettres, il se
rendit peu visible. Le sérieux et un air d'embarras succédèrent à l'air
gai et ouvert qu'il avait eu auparavant. Cette lettre de Chamillart,
venue en cadence de cette aigreur du roi à. M\textsuperscript{me} la
duchesse de Bourgogne, qu'elle ne lui laissa pas ignorer pour qu'il ne
lui imputât pas de faire pour lui moins qu'elle ne pouvait, le resserra
de plus en plus, et le plongea dans une amertume qui fut visible. Il se
rapprocha de Vendôme peu à peu, qui, à son ordinaire, allait chez lui
tête haute, et qui, profitant de sa douceur, avait l'audace d'y mener
Albéroni à sa suite. Le jeune prince, affecta de parler davantage à
Vendôme, et même à Albéroni quand l'occasion s'en présentait. Ce
changement solitaire d'une part, et de l'autre cette faiblesse, fit un
fâcheux effet dans l'armée. Ceux qui s'étaient le plus élevés en faveur
de la vérité et de Mgr le duc de Bourgogne commencèrent à craindre tout
de bon et à se taire, à se présenter moins chez lui, et à se rapprocher
de M. de Vendôme, et le gros de l'armée qui ne voit que l'écorce, à
blâmer le jeune prince, pour ne pas dire pis. Ce qui en avait toujours
été contre lui à s'applaudir et à insulter\,; et la cabale à triompher
de sa fermeté, à profiter plus insolemment que jamais de la conjoncture,
à répandre doucement le conseil de Chamillart à Mgr le duc de Bourgogne,
et la rebuffade du roi à M\textsuperscript{me} la duchesse de Bourgogne,
malgré l'appui de M\textsuperscript{me} de Maintenon, à qui ils osèrent
espérer d'imposer par leur audace, et la forcer de se ménager avec eux.

Mgr le duc de Bourgogne, qui sentit bien que son changement de conduite
avec M. de Vendôme ne plairait pas à M\textsuperscript{me} la duchesse
de Bourgogne, ni à ceux qui s'intéressaient en lui, s'en excusa à elle
sur le conseil de Chamillart qui, selon lui, ne pouvait être hasardé de
sa tête, et qui lui avait fait craindre, s'il n'y déférait pas, d'être
rappelé honteusement. À ce coup je mis si bien le doigt sur la lettre
aux ducs de Beauvilliers et de Chevreuse que\,; avec tous leurs
scrupules et leur charité, ils ne purent ne se pas rendre à l'évidence
des vues et du but des chefs mâles et femelles de la cabale.
M\textsuperscript{me} la duchesse de Bourgogne fut outrée contre
Chamillart, et ne lui pardonna jamais sa lettre à son époux, et les
funestes effets qu'elle causa.

J'étais instruit à mesure, et de tout, comme j'instruisais de même le
côté où je tenais, et je me gouvernai de façon à l'être aussi de l'autre
par des conversations avec Chamillart, à qui toutefois je me montrais à
découvert, et par des gens assez neutres qui ne laissaient pas d'en
savoir beaucoup, et qui ne se cachaient pas de moi, quoique je me
montrasse tout publiquement tel que j'étais, jusqu'à disputer souvent
avec beaucoup de chaleur. Parmi tout cela, j'étais fort peiné de
Chamillart. Son aveuglement me piquait, je craignis pour lui qui, bien
que partie importante, ne laissait pas en comparaison des bâtards, des
Lorrains et des valets, d'être la partie faible, et déjà mal avec
M\textsuperscript{me} de Maintenon, d'avec qui cette conduite
l'éloignait encore. La colère de M\textsuperscript{me} la duchesse de
Bourgogne me fit peur pour lui. J'avertis ses filles de sa sottise et de
la colère de la princesse. L'ivresse leur offusquait l'entendement\,;
elles me soutinrent que j'étais mal informé. À la fin
M\textsuperscript{me} Dreux s'aperçut de quelque chose\,; elle parla à
M\textsuperscript{me} la duchesse de Bourgogne qui dissimula, et la
petite Dreux crut tout en sûreté. Vendôme, qui en fut averti, ne
raisonna pas de même, tout superbe qu'il fût. La piété et la timidité du
prince le rassuraient, mais il était inquiet de ce qu'il lui était
revenu de M\textsuperscript{me} la duchesse de Bourgogne et de
M\textsuperscript{me} de Maintenon, de nouveau outrées de cette lettre,
et qui ne s'en prenaient pas à Chamillart seul. Il craignit une
Italienne offensée, qui trouvait tant d'honneur et d'applaudissement à
l'être, qui avait mis M\textsuperscript{me} de Maintenon dans ses
intérêts\,; qui partageait avec elle l'injure et le dépit d'avoir été
surmontées en crédit, et qui, avec elle et sous sa conduite, était si
libre avec le roi, et si, à portée de lui à toutes les heures. Ces
réflexions eurent assez de pouvoir sur le duc de Vendôme pour l'abaisser
à témoigner à Mgr le duc de Bourgogne son déplaisir de ce que
M\textsuperscript{me} la duchesse de Bourgogne gardait si peu de mesure
sur son compte, et, sans descendre dans aucune excuse ni justification
sur quoi que ce fût, le prier de lui en écrire parce qu'il n'osait le
faire lui-même. L'audace de ce trait fait voir ce que la timidité et la
piété mal entendue attire de mépris, même aux dieux de ce monde. En même
temps, il fut adroit et hardi\,: hardi en ce que, ne se mettant en
aucune sorte de devoir, il employait celui à qui il en devait tant, et
en tant de sortes, celui par qui il avait offensé la princesse, à lui
conserver la porte d'une excuse marquée ou d'un respect vague, comme il
le voudrait\,; adroit en ce qu'après avoir subjugué le prince dans sa
propre armée avec un scandale si éclatant, mis la ville, la cour, les
provinces presque en entier de son côté à visage découvert, vaincu la
princesse en crédit au milieu de la cour et dans l'intrinsèque du roi,
il lui présentait une voie de réconciliation, au moins apparente, qu'il
se flattait d'autant plus qu'elle pourrait ne pas rejeter qu'il
n'ignorait pas les reproches qu'elle avait déjà essuyés\,; et que le
refus de le recevoir par ce témoignage de respect lui en devait faire
craindre d'autres, tandis que le roi lui saurait gré de rendre à sa
petite-fille cette soumission pleine de modestie apparente. C'était, à
vrai dire, un grand effort de politique. Le plus surprenant est que Mgr
le duc de Bourgogne ne fit aucune difficulté de se charger du
compliment. Il fut reçu comme il méritait de l'être. Elle répondit à son
époux qu'elle le priait de se persuader que jamais elle n'aimerait ni
n'estimerait Vendôme, et de lui dire de sa part qu'elle ne parlait
point, et qu'elle ne savait pourquoi on l'avait entretenu d'elle. Elle
ajouta ensuite à M. le duc de Bourgogne que rien ne lui ferait oublier
tout ce que Vendôme avait fait contre lui, et que c'était l'homme du
monde pour qui elle aurait toujours le plus d'aversion et de mépris.
Nous verrons avec quel courage elle sut lui tenir parole. Vendôme
comprit de la sécheresse de la réponse à quoi il devait s'en tenir.
Aussi n'alla-t-il pas plus loin. Son orgueil put se repentir d'avoir été
même jusque-là.

\hypertarget{chapitre-xviii.}{%
\chapter{CHAPITRE XVIII.}\label{chapitre-xviii.}}

1708

~

{\textsc{Intrigue d'Harcourt pour le ministère.}} {\textsc{- Mouvements
sourds du maréchal de Villeroy.}} {\textsc{- Situation, vues et manéges
de d'Antin.}} {\textsc{- Caractère, vues, manéges de
M\textsuperscript{me} la Duchesse, et son éloignement de
M\textsuperscript{me} la duchesse de Bourgogne et de
M\textsuperscript{me} la duchesse d'Orléans.}} {\textsc{- Duchesse de
Villeroy intime de M\textsuperscript{me} la duchesse d'Orléans et fort
en faveur de M\textsuperscript{me} la duchesse de Bourgogne.}}
{\textsc{- Caractère de la duchesse de Villeroy et ses chemins.}}
{\textsc{- Convenances de liaison entre M\textsuperscript{me} la
duchesse de Bourgogne et M\textsuperscript{me} la duchesse d'Orléans.}}
{\textsc{- Conduite de M\textsuperscript{me} la Duchesse à l'égard de
M\textsuperscript{me} la duchesse de Bourgogne.}} {\textsc{- Embarras de
d'Antin avec M\textsuperscript{me} la Duchesse sur M\textsuperscript{me}
la duchesse de Bourgogne.}} {\textsc{- Il se conserve bien enfin avec
toutes deux.}}

~

Ces capitales, intrigues en enfantèrent de petites\,: Harcourt était en
Normandie refroidi avec M\textsuperscript{me} de Maintenon, dont
l'humeur volage était de prendre en gré, puis en confiance, sans raison,
et de laisser là sans cause ceux qu'elle y avait pris. Je n'ai point su
s'il y avait eu d'autres raisons, mais l'ambition d'Harcourt en était
fort affligée. Il crut l'occasion bonne à saisir de ces étranges
aventures, et s'en vint à Fontainebleau sans y être attendu. Entrer dans
la cabale dominante n'était pas un moyen de rentrer en privance avec
M\textsuperscript{me} de Maintenon\,; de s'y déclarer contraire\,; les
ducs de Beauvilliers et de Chevreuse l'y auraient trop incommodé. Il
était au fait de tout et de la situation présente de Chamillart. Son but
fut toujours le ministère\,; il se flatta d'y parvenir à ses dépens.
Mais, pour y arriver, il ne fallait pas se rendre M. du Maine contraire,
dont il avait toujours été le client, et qui était l'âme et le grand
ressort de la cabale de Vendôme. Il résolut donc à faire le bon citoyen
qui cède à ses alarmes et qui accourt. Il trouva à Fontainebleau
Catinat, qui y avait été mandé, et avec qui le roi eut plusieurs
conférences, moins sur la Flandre que sur la Savoie, où le maréchal de
Villars fut souvent embarrassé. Harcourt, avec adresse, tâcha de laisser
croire qu'il avait été mandé aussi, et fut peiné au dernier point de n'y
avoir pas réussi, et de n'avoir pu parvenir à voir le roi en
particulier. M\textsuperscript{me} de Caylus, sa bonne amie et cousine
germaine, n'était point venue à Fontainebleau, et lui manqua beaucoup. À
son défaut, il s'abaissa à courtiser M\textsuperscript{me} d'Heudicourt,
et même M\textsuperscript{me} de Dangeau, avec qui il lui fut aisé de
faire le capitaine et le politique. Avec ses raisonnements, il les
persuada si bien, et leur donna des alarmes si chaudes, qu'elles ne
donnèrent point de repos à M\textsuperscript{me} de Maintenon qu'elle ne
l'eût entretenu. De cette sorte, il ne perdit pas son voyage, et se
remit comme il put à {[}se{]} rapprocher {[}de{]} ce sanctuaire.

D'autre part marchait sourdement un autre homme qui, las de s'enfoncer
dans le désespoir, reprenait haleine jusqu'à la joie et à l'orgueil, à
mesure du danger de la Flandre et des fautes du réparateur des siennes.
De sa maison de Villeroy, où il s'était établi pendant Fontainebleau, il
y faisait de courts et de rares voyages, et il n'y en faisait aucun sans
que M\textsuperscript{me} de Maintenon l'entretînt chez elle, à la
ville, avec le plus grand mystère. Elle avait toujours conservé du goût
et de l'estime pour lui, et elle était épouvantée sur la Flandre,
jusqu'à se prendre à tout. Elle lui demanda des mémoires sur cette
guerre, qu'il lui faisait donner par Desmarets, son ami de tout temps.
Le maréchal, qui n'ignorait pas où Vendôme et Chamillart en étaient avec
elle, tombait rudement sur tous les deux\,; ainsi Harcourt et lui
confirmaient, sans le savoir, ce qu'ils faisaient l'un et l'autre. Il
fit beaucoup de mal à Chamillart et plut plus qu'Harcourt, parce qu'il
ne garda aucune mesure sur le duc de Vendôme. Ce commerce secret se
soutint pendant toute la campagne de Flandre, et flatta Villeroy des
plus agréables espérances, quoiqu'il n'aperçût aucun changement
favorable dans le roi. Il avait encore pour lui M\textsuperscript{me} la
duchesse de Bourgogne, liés par la haine commune des deux hommes qui
leur étaient odieux. Il était appuyé de sa belle-fille intimement, comme
je le dirai bientôt, avec M\textsuperscript{me} la duchesse de
Bourgogne, et il était instruit de tout par son fils, qui servait alors
de capitaine des cardes. Ainsi, ce maréchal, si profondément abîmé,
commençait à voir de loin la clarté du jour, et ne renonçait pas aux
plus grands retours de la fortune.

D'Antin n'était pas celui qui formait les moins hautes pensées. Ancré
par les facilités que lui donnait sa charge, il ne bougeait de
l'intérieur des cabinets, et hors les heures du lever et du coucher du
roi, ses premiers valets de chambre n'étaient pas plus privilégiés, ni
guère plus assidus que lui. Dans ces temps si particuliers, le roi,
souvent pressé par le silence qu'il s'imposait ailleurs, se soulageait
par quelques mots sur les nouvelles que d'Antin saisissait, et comme
très bon homme de guerre qu'il était, dans l'éloignement de ses périls,
il n'avait pas de peine à briller parmi les valets ni même avec les deux
bâtards, à s'emparer de la conversation et à la prolonger, d'autant que
le roi, souvent inquiet, se plaisait à l'entendre discourir pertinemment
sur les mouvements et les discussions de la Flandre. Lors même que
Chamillart apportait des nouvelles à ces heures-là, d'Antin s'approchait
hardiment, et si on déployait une carte, il s'en saisissait à l'instant,
et y montrait ce qu'on cherchait et souvent ce qu'on voulait dire\,; et
il n'en manquait pas l'occasion de faire valoir ses talents, toujours au
poids de la flatterie.

Une situation si brillante le rendit bientôt considérable aux deux
partis pour savoir de lui les choses plus particulières, mais infiniment
plus à M\textsuperscript{me} la duchesse de Bourgogne qu'aux partisans
de M. de Vendôme, qui savaient aisément tout par les valets et par M. du
Maine, à qui la faiblesse que le roi avait pour lui cachait peu de
choses. M\textsuperscript{me} la duchesse de Bourgogne voyait le roi en
garde contre elle sur la Flandre, et qu'à cause d'elle, il ne s'ouvrait
pas là-dessus à M\textsuperscript{me} de Maintenon comme sur presque
toutes les autres choses. Les valets étaient à M. du Maine, à Bloin,
plusieurs directement à M. de Vendôme, presque aucun à
M\textsuperscript{me} de Maintenon, qui ne les voyait presque jamais,
excepté Fagon qui, en homme d'honneur, déplorait ce qu'il voyait, mais
qui, en politique, se renfermait dans ce qui ne le commettait point. La
jeune princesse eut donc recours à d'Antin. Elle le traita avec plus de
distinction. Il le sentit, et, en habile homme, il comprit qu'elle
devait être ménagée\,; qu'il le pouvait sans choquer les chefs de
l'autre parti avec qui tous il était si anciennement ou si naturellement
lié\,; que la princesse pourrait dans les suites le porter aux choses
les plus hautes s'il savait se servir à propos de la passion qui
l'occupait alors tout entière, et qui méritait d'autant plus toute son
attention à lui, que M\textsuperscript{me} de Maintenon partageait cette
même passion avec elle. Il se mit donc à lui rendre compte de ce qu'elle
désira, et, en un moment, se mit sur le pied de l'avertir et d'entrer
dans sa confidence. Ce manége lui réussit au point que la princesse, qui
avec raison faisait cas de son esprit et de sa capacité, s'ouvrit à lui
des lettres de son époux, lui en montra même et lui consulta ses plus
importantes réponses.

Je savais tout cela par M\textsuperscript{me} de Nogaret, qui, par ordre
de M\textsuperscript{me} la duchesse de Bourgogne, me disait souvent les
avis de d'Antin, et me demandait ce que j'en pensais. Il poussa sa
pointe et ses louanges mêlées avec ses conseils jusqu'à hasarder de
marcher, mais légèrement, sur les traces de l'abbé de Polignac. Cette
double conduite ne la toucha point, mais n'était pas aussi pour
l'offenser. Il s'introduisit chez elle aux heures de privante, se rendit
assidu à son jeu, et il essaya par cette voie de pénétrer jusque chez
M\textsuperscript{me} de Maintenon, à quoi néanmoins il réussit peu par
l'extrême clôture de ce sanctuaire. Assuré des bâtards et des valets,
sûr aussi que M\textsuperscript{me} la duchesse de Bourgogne et
M\textsuperscript{me} de Maintenon par elle ne lui seraient point
contraires, il ne pensa à rien moins qu'à la place de Chamillart, à
portée, comme il était, d'entrer avec le roi dans tout ce qui regardait
la guerre de plus inquiétant et de plus délicat, et peu à peu de s'y
mettre de plus en plus et culbuter un ministre malheureux en succès,
déjà dépouillé des finances, tombé dans la disgrâce de
M\textsuperscript{me} de Maintenon, et sans retour auprès de
M\textsuperscript{me} la duchesse de Bourgogne. Harcourt et lui étaient
ainsi rivaux sans le savoir\,; mais d'Antin avait bien plus beau jeu par
ce commerce direct et continuel avec le roi, où l'autre ne pouvait
atteindre, même par audiences rares. Quand je dis qu'ils en voulaient
tous deux à la place de Chamillart, je m'explique. Ce n'était pas à sa
charge. Le roi, accoutumé à les remplir de gens de peu pour les chasser
comme des valets s'il lui en prenait envie, et pour empêcher que leur
autorité ne les portât à des fortunes trop hautes et embarrassantes,
n'aurait jamais fait un seigneur secrétaire d'État. Ils ri imaginaient
pas aussi sortir le roi de cette politique, et Harcourt était trop
glorieux pour vouloir être le premier secrétaire d'État de l'ordre de la
noblesse qu'il y eût jamais eu en France. Mais ils visaient tous deux à
entrer dans le conseil, avec une inspection sur la guerre immédiate et
supérieure à celui qui succéderait à Chamillart.

Plein de ces espérances, d'Antin courait légèrement sa carrière, lorsque
M\textsuperscript{me} la Duchesse s'aperçut que sa liaison avec
M\textsuperscript{me} la duchesse de Bourgogne passait le jeu et le
frivole, et s'en piqua extrêmement. Dans une taille contrefaite, mais
qui s'apercevait peu, sa figure, était formée par les plus tendres
amours, et son, esprit était fait pour se jouer d'eux à son gré sans en
être dominé. Tout amusement semblait le sien\,; aisée avec tout le
monde, elle avait l'art de mettre chacun à son aise\,; rien en elle qui
n'allât naturellement à plaire avec une grâce nonpareille jusque dans
ses moindres actions, avec un esprit tout aussi naturel, qui, avait
mille charmes. N'aimant personne, connue pour telle, on ne se pouvait
défendre de la rechercher ni de se persuader jusqu'aux personnes qui lui
étaient les plus étrangères, d'avoir réussi auprès d'elle. Les gens même
qui avaient le plus lieu de la craindre, elle les enchaînait, et ceux
qui avaient le plus de raisons de la haïr avaient besoin de se les
rappeler souvent, pour résister à ses charmes. Jamais la moindre humeur,
en aucun temps, enjouée, gaie, plaisante avec le sel le plus fin,
invulnérable aux surprises et aux contretemps, libre dans les moments
les plus inquiets et les plus contraints, elle avait passé sa jeunesse
dans le frivole et dans les plaisirs qui, en tout genre et toutes les
fois qu'elle le put allèrent à la débauche. Avec ces qualités, beaucoup
d'esprit, de sens pour la cabale et les affaires, avec une souplesse qui
ne lui coûtait rien\,; mais peu de conduite pour les choses de long
cours, méprisante, moqueuse piquante, incapable d'amitié et fort capable
de haine, et alors, méchante, fière, implacable, féconde en artifices
noirs et en chansons les plus cruelles dont elle affublait gaiement les
personnes qu'elle semblait aimer et qui passaient leur vie avec elle.
C'était la sirène des poètes, qui en avait tous les charmes et les
périls\,; avec l'âge, l'ambition était venue, mais sans quitter le goût
des plaisirs, et ce frivole lui servit longtemps à masquer le solide.

Les assiduités et l'attachement si marqué de Monseigneur pour elle,
qu'elle avait enlevé au peu d'esprit, aux humeurs et à l'aigreur de
M\textsuperscript{me} la princesse de Conti, la rendaient considérable.
On a vu ailleurs sa liaison intime avec la Choin et les nièces de
Vaudemont, en attendant qu'elles se mangeassent les unes les autres à
qui demeurerait l'entière autorité sur Monseigneur lorsqu'il serait
devenu le maître. Elle ne pouvait donc pas avoir en attendant des vues
différentes des leurs, surtout à l'égard de Mgr le duc de Bourgogne\,;
d'ailleurs elle se voyait en état de figurer grandement par là dans tous
les temps. Elle en sentait aussi le besoin par rapport à M. le Duc,
jaloux, brutal, farouche, d'une humeur insupportable et féroce, que,
pendant longtemps, le désir de commander des armées pendant longtemps,
et toujours la crainte du roi avait retenu à son égard, et qu'elle avait
un si pressant intérêt de retenir toujours dans la même mesure.
M\textsuperscript{me} la princesse de Conti était devenue tout à fait
nulle, et M\textsuperscript{me} la duchesse d'Orléans à peu près de
même, ayant néanmoins tout ce qui peut donner beaucoup à compter\,; mais
il n'est pas temps de s'étendre sur elle. Il ne s'agissait jamais pour
rien de l'autre princesse de Conti, de M\textsuperscript{me} la
Princesse, ni de Madame\,: aucune d'elles n'avaient jamais existé pour
rien. C'était donc M\textsuperscript{me} la duchesse de Bourgogne qui
seule offusquait M\textsuperscript{me} la Duchesse. Aimable et bien plus
jeune qu'elle, il ne se put qu'elle ne fût regardée, et par des esclaves
que M\textsuperscript{me} la Duchesse comptait, parmi les siens. Nangis,
entre autres, devint quelquefois un spectacle pour qui avait d'assez
bons yeux pour profiter de ce plaisir, qui n'était pas médiocre, et dont
Marly fut le théâtre le plus commode et le plus ordinaire.

Un rang dans les nues rabaissait bien proche de terre une divinité si
fort accoutumée à l'être\,; et quoiqu'elle eût négligé des privantes
gênantes, inalliables avec la liberté et les plaisirs, celles que
M\textsuperscript{me} la duchesse de Bourgogne s'était personnellement
acquises avec le roi et M\textsuperscript{me} de Maintenon nettoient
sans cesse M\textsuperscript{me} la Duchesse au désespoir. Ses projets
sur Monseigneur lui en étaient une autre source. Elle craignait tout de
ce côté-là d'une jeune princesse tout occupée à lui plaire, qui y
réussissait, et qu'elle avait lieu de craindre qui n'eût trouvé le
chemin de son coeur. Maîtresse d'elle, il n'y parut pas. Elle ajouta aux
recherches de devoir et de respect toutes celles qu'elle crut propres à
la bien mettre avec M\textsuperscript{me} la duchesse de Bourgogne. Le
grand défaut de celle-ci était la timidité. On s'étendra ailleurs
davantage sur elle. On lui avait fait peur de ce qui était caché sous
les charmes de M\textsuperscript{me} la Duchesse. Elle ne répondit donc
à ses avances qu'en tremblant, avec beaucoup de politesse, mais sans
passer au delà, et cette retenue fut un autre aiguillon à la vaincre.
Une autre intrigue déconcerta ce projet.

La duchesse de Villeroy avait passé les premières années de son mariage
dans une sorte de retraite, et à la cour presque comme n'y étant pas,
par des raisons qui ne méritent pas de trouver place ici.
M\textsuperscript{me} la duchesse d'Orléans menait une vie fort
régulière et fort éloignée de la dissipation et des plaisirs. Les dames,
avant l'arrivée de M\textsuperscript{me} la duchesse de Bourgogne, se
partageaient volontiers entre les trois filles du roi, et s'adonnaient
plus à une qu'aux deux autres. La maréchale de Rochefort, dame d'honneur
de M\textsuperscript{me} la duchesse d'Orléans, avait le grappin sur la
duchesse de Villeroy, l'amie si intime de son père, de son frère et de
toute sa famille\,; et la liberté de sa maison plaisait bien plus à
cette jeune mariée, que la contrainte où elle croyait être chez sa
belle-mère, qui n'était pas même toujours à la cour. Cette liaison la
mit naturellement dans celle de M\textsuperscript{me} la duchesse
d'Orléans. Elles se convinrent toutes deux, et lièrent une amitié
étroite qui dura toujours intime. Enfin le maréchal de Villeroy, comme
s'il eût eu un pressentiment de sa disgrâce, mais en effet ennuyé de
voir sa belle-fille renfermée chez M\textsuperscript{me} la duchesse
d'Orléans, et jaloux de voir quelques jeunes femmes, et peut-être
M\textsuperscript{me} de Saint-Simon et M\textsuperscript{me} de Lauzun
approchées de M\textsuperscript{me} la duchesse de Bourgogne à qui on en
laissait voir très peu en cette familiarité, demanda la même faveur pour
sa belle-fille à M\textsuperscript{me} de Maintenon, qui la lui accorda
aussitôt. La maréchale d'Estrées, qui toujours s'en l'était de quelqu'un
comme un amant d'une maîtresse, se prit là d'une telle amitié pour la
duchesse de Villeroy, qu'elle ne la pouvait quitter. Les plus légères
absences étaient réparées par des lettres et par des présents. Cette
intimité lia la duchesse de Villeroy avec toutes les Noailles et avec
M\textsuperscript{me} d'O, et bientôt par elles avec
M\textsuperscript{me} la duchesse de Bourgogne, si fortement, que le
goût de la maréchale d'Estrées ayant changé bientôt après, comme cela
lui arrivait toujours, la duchesse de Villeroy demeura de son chef une
espèce de favorite, et la demeura toujours depuis.

Elle se ménagea avec soin, avec sagesse et prudence, et même avec
dignité. C'était une personne de fort peu d'esprit, mais de sens, de
vues, de conduite, haute, courageuse, franche et vraie, fort altière,
fort inégale, fort pleine d'humeur, même volontiers brutale, qui aimait
fort peu de personnes, mais qui n'en était que plus attachée à ce
qu'elle aimait, et qui, à l'exemple de son oncle l'archevêque de Reims,
se rendait si nettement et si publiquement justice sur sa naissance,
qu'elle en embarrassait très souvent. Elle était grande, un peu haute
d'épaules, de vilaines dents et un rire désagréable avec le plus grand
air, le plus noble, le plus imposant, et un visage très singulier et
fort beau. Personne ne paraît tant une cour et un spectacle, et elle
dansait fort bien. Le roi, qui, avec des sentiments fort opposés à ceux
de sa jeunesse, conservait toujours un goût et un penchant pour les
femmes aimables, mit la duchesse de Villeroy des fêtes et des voyages de
Marly, d'abord par complaisance pour le maréchal de Villeroy, et, après
sa disgrâce, pour elle-même.

M\textsuperscript{me} la Duchesse n'avait jamais pu pardonner à
M\textsuperscript{me} la duchesse d'Orléans le rang et les honneurs qui
la distinguaient si fort des princesses du sang. Quoi que celle-ci eût
pu faire vers cette soeur, l'autre s'en était toujours éloignée. Leur
rapprochement à la mort de M\textsuperscript{me} de Montespan n'avait
pas duré. Ce même éloignement s'était bassement communiqué à leurs
favorites. La duchesse de Villeroy ne s'était pas contrainte sur
M\textsuperscript{me} la Duchesse, qui à son tour ne l'avait pas
ménagée. Sa faveur auprès de M\textsuperscript{me} la duchesse de
Bourgogne ne lui inspira rien de favorable pour M\textsuperscript{me} la
Duchesse. M\textsuperscript{me} d'O désirait depuis longtemps de former
une liaison entre M\textsuperscript{me} la duchesse de Bourgogne et
M\textsuperscript{me} la duchesse d'Orléans\,; mais sa politique, qui
lui faisait tout craindre et ménager, l'avait ralentie dans les progrès.
La duchesse de Villeroy, plus hardie, se mit en tête d'y réussir, et en
eut tout l'honneur. Les deux princesses ne se convenaient guère, et
néanmoins leur liaison très véritable dura toujours.

La paresse, l'empesé, les mesures toujours compassées de l'une, la
vivacité, la liberté de l'autre, l'extrême timidité de toutes deux,
avaient besoin de tiers qui soutinssent cette liaison dont nous verrons
les progrès et les fruits. Toutes deux y avaient déjà intérêt. {[}La
liaison{]} que l'attachement de Monseigneur pour M\textsuperscript{me}
la princesse de Conti lui avait fait désirer avec elle, s'était bientôt
changée en simples bienséances par le changement de Monseigneur. Elle
sentait le faible du roi pour ses filles, elle n'osait s'éloigner de
toutes à la fois. Elle n'ignorait pas que M\textsuperscript{me} la
Duchesse cherchait à lui faire une affaire avec le roi et avec
Monseigneur de n'avoir pas répondu aux avances qu'elle en avait reçues,
et à la faire passer dans leur esprit pour dédaigner les princesses. Il
ne restait donc plus que M\textsuperscript{me} la duchesse d'Orléans,
dont l'amitié un peu particulière pût démentir ces plaintes\,; elle se
trouvait d'autant mieux placée que sa conduite avait été sans reproche,
et que M. le duc d'Orléans était frère de M\textsuperscript{me} sa mère.
M\textsuperscript{me} la duchesse d'Orléans en avait des raisons plus
pressantes isolée au milieu de la cour\,; épouse par force d'un prince
si au-dessus d'elle qui se piquait d'indifférence pour elle, et d'être
toujours amoureux ailleurs avec éclat, chargée de trois filles dont
l'aînée commençait à peser par son âge, auxquelles sa naissance fermait
tout établissement en Allemagne, tout la pressait de faire l'impossible
pour la marier à M. le duc de Berry, et c'est à quoi l'amitié de
M\textsuperscript{me} la duchesse de Bourgogne la pouvait conduire.
M\textsuperscript{me} la Duchesse, qui se trouvait dans le même cas, et
qui possédait Monseigneur, osait aussi lever les yeux jusqu'à cette
alliance\,; elle ne pouvait se dissimuler que la situation où elle se
trouvait avec M\textsuperscript{me} la duchesse de Bourgogne ne l'en
approchait pas. Ce qui acheva de la piquer fut le personnage qu'elle lui
vit soutenir sur le combat d'Audenarde. Toute la cour jusque-là peu
attentive à une jeune princesse dont toutes les faveurs ne pouvaient
consister qu'à donner quelques légers agréments, entrevit d'abord de
quoi elle était capable, et quelque temps après par la suite et le
succès de sa conduite, comprit qu'elle pourrait bien vouloir et se
mettre en état de devenir la maîtresse roue de la machine de la cour et
peut-être encore de l'État. Ce fut le poignard qui perça le sein de
M\textsuperscript{me} la Duchesse. Dès lors sa politique changea à
l'égard de M\textsuperscript{me} la duchesse de Bourgogne. Ce ne fut
plus des soins et des empressements, mais une indifférence insolemment
marquée. Elle espéra lui donner de la crainte du côté de Monseigneur, et
l'amener ainsi à ce que, ses avances n'avaient pu en obtenir. Elle ne
s'en tint pas là\,: elle hasarda de se moquer d'elle, d'en parler
licencieusement, de mêler des menaces sur Monseigneur, et cela devant
des personnes qu'elle savait liées avec d'autres par qui ces propos
pourraient être rendus à M\textsuperscript{me} la duchesse de Bourgogne
et lui faire peur. Elle les sut, en effet\,; mais ils rie réussirent pas
mieux qu'avaient fait ses souplesses, sinon à exciter une haine dont il
ne lui serait pas aisé d'éviter les coups. La cour intérieure disposée
de la sorte, il n'est pas étrange que M\textsuperscript{me} la Duchesse,
fort unie avec d'Antin par les plaisirs, par ce qu'ils s'étaient, par la
cour et les vues sur Monseigneur, peut-être encore plus par la sympathie
des mêmes voies et des mêmes vertus, par l'habitation continuelle des
mêmes lieux, se sentit offensée des ménagements si assidus de d'Antin
pour M\textsuperscript{me} la duchesse de Bourgogne.
M\textsuperscript{me} la Duchesse les reprocha à d'Antin comme une
liaison prise avec son ennemie. D'Antin glissa, badina, mais rie se
détourna point. Sa soeur s'en irrita davantage. Elle éclata, et se porta
jusqu'à vouloir donner des ridicules à son frère et à
M\textsuperscript{me} la duchesse de Bourgogne. Cela fit peur à d'Antin.
Il craignit de reculer tout d'un coup pour avoir voulu marcher trop
vite. Il tâcha d'apaiser M\textsuperscript{me} la Duchesse par moins
d'empressement pour lime la duchesse de Bourgogne. Il fut peut-être
assez adroit pour le faire valoir à toutes les deux.

Quoi qu'il en soit, ceux qui pénètrent le fond de cette bizarre intrigue
se divertissent souvent des embarras de d'Antin, des hauteurs de
M\textsuperscript{me} la Duchesse avec lui, et de le voir enrager plus à
découvert qu'il n'eût voulu de ne pouvoir être en deux lieux à la fois.
Cela dura pendant tout Fontainebleau et après encore. À la fin l'heureux
gascon fut assez habile pour en sortir sans avoir aliéné
M\textsuperscript{me} la duchesse de Bourgogne et sans s'être gâté avec
M\textsuperscript{me} la Duchesse. Je ne voyais tout cela que par
ricochet, mais les filles, de Chamillart, qui le voyaient en plein chez
M\textsuperscript{me} la Duchesse qui ne se cachait pas d'elles, surtout
de ma belle-soeur, et qui y passaient presque toutes leurs soirées
jusque bien avant dans la nuit où d'Antin était souvent à ces heures-là,
me contaient tout, et-me nettoient, par ce que nous rassemblions, en
état de tout savoir et à mesure.

\hypertarget{chapitre-xix.}{%
\chapter{CHAPITRE XIX.}\label{chapitre-xix.}}

1708

~

{\textsc{Décret violent de l'empereur contre l'Italie.}} {\textsc{-
Projets de la réunir en ligue contre lui.}} {\textsc{- Prince de Conti
désiré pour la Flandre, demandé pour l'Italie.}} {\textsc{- Ruse de
Vaudémont au secours de Vendôme.}} {\textsc{- Tessé plénipotentiaire à
Rome et en Italie\,; sa commission\,; son départ.}} {\textsc{- L'Artois
sous contribution.}} {\textsc{- Faute de Mgr le duc de Bourgogne.}}
{\textsc{- Conduite de Vendôme.}} {\textsc{- Boufflers entre dans Lille,
et remet à flot Surville et La Freselière.}} {\textsc{- Cause de la
disgrâce du dernier.}} {\textsc{- Troupes, etc., dans Lille.}}
{\textsc{- Le Rhin tranquille.}} {\textsc{- Troupes mal choisies dans
Lille et autres fâcheux manquements.}} {\textsc{- Dispositions de
Boufflers.}} {\textsc{- Sécurité de Vendôme.}} {\textsc{- Lille investi
(12 août).}} {\textsc{- Misérables flatteries.}} {\textsc{- Tranchée
ouverte (22 août).}} {\textsc{- Albéroni à Fontainebleau.}} {\textsc{-
Retour par Petit-Bourg à Versailles.}} {\textsc{- Opiniâtre lenteur de
Vendôme à s'ébranler.}} {\textsc{- Jonction de l'armée du duc de Berwick
avec celle de Mgr le duc de Bourgogne.}} {\textsc{- Berwick prend une
seule fois l'ordre du duc de Vendôme\,; se déporte de tout
commandement.}} {\textsc{- Maréchal de Matignon s'en va malade et ne
revient plus.}} {\textsc{- Force de l'armée après la jonction.}}
{\textsc{- L'armée à Tournai.}} {\textsc{- Dévotions mal interprétées.}}
{\textsc{- Divisions.}} {\textsc{- Chemins pris par l'armée.}}
{\textsc{- Camps des deux armées opposées.}} {\textsc{- Inquiétude de la
cour.}} {\textsc{- Flatteries misérables.}} {\textsc{- Je parie contre
Cani que Lille sera pris sans combat et sans secours.}} {\textsc{- Bruit
étrange sur ce pari, et sa suite.}} {\textsc{- Position des deux
armées.}} {\textsc{- Fatale et artificieuse opiniâtreté de Vendôme.}}
{\textsc{- Mensonge en plein de Pont-à-Mark.}} {\textsc{- Mensonge en
plein de Mons-en-Puelle.}}

~

L'empereur avait fait passer, dès le mois de juin, à la diète de
Ratisbonne un décret qu'il fit incontinent après afficher dans Rome et
par toute l'Italie. Il y déclarait abusif l'hommage du royaume de Naples
au Saint-siège\,; que Naples et Sicile n'en relevaient point, que le
pape n'avait aucun droit à la nomination des évêchés et des autres
bénéfices de ces royaumes. L'empereur y déclarait qu'il voulait rentrer
dans tous les droits de l'empire en Italie, réunir les fiefs usurpés,
examiner l'aliénation des autres, et qu'il prétendait que le pape fît
raison au duc de Modène des usurpations que la chambre apostolique avait
faites sur lui. La vérité est que les droits de l'empire en Italie
étaient la plupart fort clairs, qu'ils s'étendaient beaucoup, que les
usurpations étaient grandes, et peu ou point fondées. Cet édit ou décret
fit grand'peur à Rome et à toute l'Italie\,; la puissance de l'empereur
y parut très redoutable. On s'y repentit de l'y avoir moins crainte que
celle des Français, et de l'avoir tant aidé à les en chasser. Venise,
qui y avait le plus contribué, fut la première à exciter le pape sur le
danger commun, à lui proposer une ligue de toute l'Italie avec la
France, où on ne désespérait pas de faire entrer M. de Savoie, qui se
pourrait laisser toucher du danger commun, et d'y attirer la France,
pressée comme elle se trouvait, qui par cette puissante diversion ne
serait plus seule et se reverrait comme avant la bataille de Turin.

Venise, qui, la première, avait mis cette affaire sur le tapis, et qui
ne cessait d'en presser la conclusion, craignait trop l'empereur dans sa
terre ferme d'Italie et du Frioul pour oser se montrer, mais voulait
paraître être entraînée. Ce fut donc Rome qui en fit au roi les
premières ouvertures. Il les reçut avec froideur parce qu'il ne voyait
pas grande apparence que le duc de Savoie y voulût entrer, qu'il ne
voyait rien de la part de Venise, et qu'il n'a jamais bien goûté
l'importance des diversions. On fut donc longtemps à se résoudre de
permettre au pape d'acheter des armes, de lever des troupes dans son
propre comtat d'Avignon, enfin de lui donner des officiers de nos
troupes ses sujets. On en était alors aux suites du combat d'Audenarde.
L'Artois sous contribution, Arras, Dourlens, la Picardie menacés, les
troupes que Berwick avait amenées du Rhin répandues pour couvrir ces
pays, Cheladet, avec un gros détachement de la grande armée, occupé au
même secours, et le roi fort touché de ces ravages si proches dont il
n'avait pas ouï parler depuis sa minorité. Le contrecoup de la mauvaise
humeur en retomba naturellement sur l'affaire d'Audenarde.

M\textsuperscript{me} de Maintenon, piquée au vif d'avoir vu son crédit
faiblir sous celui de Vendôme, tira sur le temps, hasarda de le faire
rappeler, et de lui substituer le prince de Conti qui s'était toujours
déclaré pour Mgr le duc de Bourgogne dans tout ce qui s'était passé en
Flandre, dont la naissance et la réputation imposerait et calmerait
tout. La ligue d'Italie le demandait pour chef, pour ôter toute dispute
entre les divers généraux par la supériorité de son rang, et donner par
son nom plus de poids aux affaires. Le roi fut fort en balance. Le
maréchal d'Estrées, qui voulait toujours figurer, poussé de plus par son
frère, qui soupirait ardemment après un chapeau, se proposait pour
l'ambassade de Rome comme un homme également propre aux négociations et
au commandement des troupes. Je sus par Caillières, à qui Torcy l'avait
dit, que j'étais aussi sur les rangs. Cet avis m'engagea à renouveler
les raisons que j'avais eues d'éviter cette ambassade la première fois
que j'y avais été destiné, mais dont je ne fus délivré que par la
promotion du cardinal de La Trémoille. J'en parlai fortement au duc de
Beauvilliers, au chancelier, à Chamillart. J'y ajoutai les raisons du
commandement des troupes que je leur fis valoir en faveur du maréchal
d'Estrées, parce que peu m'importait qui allât à Rome pourvu que ce ne
fût pas moi, et je fis dire les mêmes choses à Torcy par Caillières. Peu
de jours après ces mesures, j'appris par ce dernier qu'on avait changé
de dessein sur un ambassadeur que le pape ne serait pas en puissance de
protéger dans Rome, même contre les insultes de l'empereur, et celles
que le cardinal Grimani, qui était par intérim vice-roi de Naples, lui
voudrait faire faire, et qui commettraient trop la dignité du roi.

M. du Maine écuma ce qui se passait. Il prit l'alarme sur la froideur du
roi à l'égard de la ligue d'Italie, et sur l'envoi très possible du
prince de Conti en Flandre, qui était l'unique chose à faire pour y
prévenir tous les inconvénients d'une division devenue sans remède et la
moindre satisfaction raisonnablement due à Mgr le duc de Bourgogne. Les
chefs de la cabale, avertis par celui-ci qui en était l'âme, n'en furent
pas moins effarouchés que lui. Après tant de grands pas faits et si
éclatants pour réussir dans leur dessein, c'eût été pour eux le dernier
désespoir de se voir privés de la massue qui avait déjà si bien joué sur
le jeune prince, et de laquelle ils se proposaient bien de l'atterrer
sans ressources avant la fin de la campagne. Vaudemont vint au secours.
Il fit un mémoire sur la ligue d'Italie qui ne laissa rien à désirer sur
son utilité, sa possibilité et son exécution prompte. Soit que Tessé,
dans une fortune qui ne pouvant plus croître ne demandait plus que le
bon esprit d'en savoir jouir en repos, eût encore le désir de faire,
soit que Vaudemont l'eût entêté de l'emploi d'Italie, il lui donna comme
par amitié son mémoire, à condition, pour se mieux cacher et {[}le{]}
produire plus efficacement, que Tessé le donnerait comme sien. Torcy, à
qui il le remit, avait toujours été d'avis de cette ligue. Il trouva le
mémoire frappant. Il en fut d'autant plus surpris qu'il connaissait la
portée de Tessé. Il le lut au conseil, et y fut applaudi, et il
détermina le roi. Presque aussitôt après, le roi donna audience
particulière au nonce, après à l'ambassadeur de Venise, enfin à M. le
prince de Conti, qu'il fit entrer dans son cabinet. Le tête-à-tête y fut
court. Le prince alla de là chez lui, où le nonce vint et y fut
longtemps enfermé avec lui. Dans le haut de l'après-dînée il fut chez
M\textsuperscript{me} de Maintenon à la ville fort longtemps. C'était le
lieu où, à Fontainebleau, elle faisait venir ceux qu'elle voulait
entretenir à loisir sans être interrompue. Je ne crois pas qu'elle eût
jamais entretenu M. le prince de Conti de la sorte, ni même guère reçu
chez elle que des moments. Cette audience fit beaucoup parler.

Sept ou huit jours après, Tessé fut déclaré plénipotentiaire du roi à
Rome, et pour toute l'Italie, avec pouvoir de prendre le caractère
d'ambassadeur si et quand il le jugerait à propos, et de général des
troupes s'il y en allait. Sa mission fut de traiter et de convenir des
contingents de chacun en troupes, artillerie, munitions, vivres,
fourrages, argent\,; des choses à faire, des temps à être prêts et de
ceux à exécuter\,; de presser et veiller à tout, de commander partout en
attendant le prince de Conti promis, mais non encore déclaré, de lui
préparer les voies, à servir sous lui, ou à part à ses ordres, d'aller
et venir par l'Italie comme plénipotentiaire où besoin serait, ou de
demeurer à Rome ambassadeur comme il serait jugé le plus à propos. Il
obtint une grosse somme pour son équipage, partit le 11, septembre avec
pouvoir d'offrir vingt mille hommes de pied et quatre mille chevaux. Il
s'embarqua à Antibes, d'où le marquis de Roye le passa à Gênes sur les
galères du roi. Là il s'associa pour tout le reste du voyage de
Monteléon. C'était un homme de beaucoup d'esprit, et surtout d'intrigue,
dévoué à Vaudemont jusqu'à l'abandon, et que nous avons vu l'acteur
principal du mariage du duc de Mantoue. C'était de quoi soulager et
éclairer Tessé, et tenir Vaudemont bien averti, et en état d'influer. De
Gênes ils allèrent chez le grand-duc, ensuite à Venise, enfin à Rome,
furent reçus partout avec de grands honneurs et de grandes
démonstrations de joie, et s'arrêtèrent assez longtemps en chacun de ces
lieux.

Par cette ligue mieux concertée, l'empereur se fût trouvé une puissance
sur les bras en Italie formidable par comparaison à ses autres besoins
qui lui auraient rendu la défensive fort embarrassante, et nous un
soulagement présent dont les suites pouvaient être les plus importantes
pour une heureuse continuation de guerre ou pour une paix avantageuse,
et cela par l'impétuosité de la cour de Vienne. Mais il avait fallu trop
de machines et de temps pour nous mettre et nous arranger cette ligue
dans la tête. Le roi ne fit qu'accepter tard et avec peine un projet
qu'il eût dû former, proposer et presser. Il perdit un temps le plus
précieux à employer qu'il eût peut-être eu de tout son règne. La
démarche éclatante qu'il en fit enfin, au lieu de ne l'avoir apprise que
par les effets, alarma les alliés. Ils sentirent tout le poids d'une
diversion si puissante. Hormis la Flandre, où ils s'étaient trop engagés
pour pouvoir reculer, ils cessèrent de songer à rien faire d'aucun auge
côté, jusqu'à ce qu'ils se fussent mis en sûreté de celui de l'Italie.

Cependant le pape, encouragé et fatigué de la lenteur de ses alliés
d'Italie, leur voulut donner un exemple qui les pressât de l'imiter. Il
leva des troupes de tous côtés, munit ses places, fortifia divers
postes, prit à son service des officiers généraux partout où il put. Il
tâcha de suspendre le luxe et de tirer de l'argent des cardinaux riches.
Il obtint, quoique avec peine, les suffrages et les signatures du sacré
collège pour tirer du château Saint-Ange le trésor que Sixte V y avait
amassé et laissé pour les plus grands besoins de l'Église. Il y avait
cinq millions d'or, il se servit de cinq cent mille écus à payer ses
troupes et aux préparatifs de guerre qu'il commença et fit assez
heureusement contre ce peu d'Impériaux épars par Italie. Leur gros était
dans l'armée du duc de Savoie. N'allons pas maintenant plus loin de ce
côté-là, et revenons à Fontainebleau et en Flandre.

Le duc de Berwick, établi dans Douai, était arrivé trop tard pour sauver
l'Artois des courses et des contributions. Sa présence servit seulement
à les en faire retirer avec plus d'ordre, sans leur faire rien perdre de
leur butin. Leur gros s'était établi à la Bassée, d'où ils avaient pensé
surprendre Dourlens, et s'étendre alors en Picardie. Ils s'étaient aussi
rendus maîtres d'un faubourg d'Arras et avaient manqué heureusement
cette place. Ils eurent trois millions cinq cent mille livres de ce
malheureux pays. Ils l'exigèrent la plupart en provisions de toutes les
sortes, ce qui montra leur dessein de faire un grand siège. Le prince
Eugène, retourné au-devant de son armée, s'était longtemps arrêté à
Bruxelles, et y avait fait préparer un convoi immense qui fut de plus de
cinq mille chariots, outre ceux des gros bagages de leur armée qu'ils
envoyèrent à vide pour revenir pleins avec ce convoi. Lorsqu'il fut en
état, le prince Eugène l'escorta lui-même avec son armée jusqu'à celle
du duc de Marlborough avec une peine et des précautions infinies. On ne
pouvait ignorer dans la nôtre de si grands préparatifs et une marche si
pesante et si embarrassée. Le duc de Vendôme en voulut profiter et la
faire `attaquer par la moitié de ses troupes. Le projet en était beau,
et le succès semblait y devoir être favorable. En ce cas l'action était
également glorieuse et utile\,: elle ôtait aux ennemis le fruit de leur
victoire, leur causait une perte infinie par celle de ce prodigieux amas
dont nous aurions profité en partie\,; leur siège était avorté, et ils
ne pouvaient plus rien entreprendre que très difficilement du reste de
la campagne. Ypres, Mons, Lille ou Tournai, une de ces places était leur
objet, et rien de si important que d'en empêcher le siège. Néanmoins,
Mgr le duc de Bourgogne s'opposa à l'attaque du convoi. Il fut soutenu
dans cet avis par quelques-uns, contredit par un bien plus grand nombre.
Pour moi, j'avoue franchement que je ne compris jamais quelles pouvaient
être les raisons de ne le pas attaquer, et que je ne pus me satisfaire
de ce peu qui en furent alléguées, encore moins par rapport à Mgr le duc
de Bourgogne, sitôt après la désastreuse affaire d'Audenarde, et tout ce
qui s'en était suivi sur son compte.

M. de Vendôme, si opiniâtre jusqu'alors, et si rempli de cette
obéissance à ses vues, sous la condition de laquelle Mgr le duc de
Bourgogne avait le commandement honoraire de son armée, ne s'en souvint
plus dans cette occasion décisive. Il céda tout court en protestant de
son avis, et laissa tranquillement passer le convoi. Il suivait son
projet qui n'était pas de faire une belle et utile campagne, mais d'en
faire faire une à ce prince qui le perdît sans retour. L'opiniâtreté et
l'audace y avaient servi à Audenarde\,; il n'espéra pas ici un moindre
succès de sa déférence\,; par tous les deux, il alla également à son
but. Tel fut l'étrange malheur qu'il n'y eut personne que d'O et
Gamaches auprès de Mgr le duc de Bourgogne. Il écrivit ses raisons au
roi et à son épouse dans la crainte d'être désapprouvé, laquelle eut le
bon esprit d'en être très affligée, et de ne le laisser apercevoir qu'à
ce qu'elle avait de plus confidentes. Le roi, voyant la chose manquée,
fit semblant d'être satisfait des raisons de son petit-fils. Ce qui me
surprit fort fut que, traitant cela avec Chamillart tête à tête, il me
soutint que Mgr le duc de Bourgogne avait raison. Je le pressai de m'en
dire les siennes. Il me les promit dans un autre temps qui n'est jamais
venu. La conjecture est qu'il n'en avait aucune, que l'affaire était
manquée, qu'il était fort éloigné du projet de Vendôme, quoique entraîné
par parties sans s'en douter, et que, fâché d'avoir eu à blâmer le jeune
prince à Audenarde, quoique fort mal à propos, il voulut tout aussi mal
à propos le défendre ici, pour ne pas paraître lui en être toujours
contraire.

Boufflers n'était rien moins que content dans sa grande fortune. Il ne
se consolait point du panneau qui lui avait coûté son changement de
charge. Il ne s'accoutumait point à ne plus commander d'armées, tout
aussi peu à se trouver naturellement suspendu de ses fonctions de
gouverneur de Flandre, depuis que le théâtre de la guerre y était
établi. Il était aussi gouverneur particulier de Lille. C'était un homme
fort court, mais pétri d'honneur et de valeur, de probité, de
reconnaissance et d'attachement pour le roi, d'amour pour la patrie. Il
crut que les ennemis préféreraient Lille aux autres places qu'ils
étaient en état d'assiéger. Il en dit ses raisons au roi, et sans en
avoir parlé à personne, il lui demanda la permission de s'y aller jeter,
et de défendre la place qui serait assiégée, puisque toutes étaient de
son gouvernement général. Il fut loué et remercié, mais éconduit.
Boufflers, qui s'était préparé en secret pour avoir de l'argent et ce
qui lui était nécessaire, n'avait pas fait cette proposition pour en
demeurer à l'honneur de l'avoir faite. Il revint à la charge dans une
audience qu'il eut au sortir du lever du roi, dans son cabinet, qu'il
lui avait demandée. Le roi fut après à la messe, et de là chez
M\textsuperscript{me} de Maintenon où il fit entrer le maréchal avec
lequel il fut assez longtemps. Tout au sortir de cette seconde audience
(c'était le jeudi 26 juillet), il partit. En cette dernière audience il
fit deux actions d'un aussi galant homme qu'il était. Il demanda au roi
et obtint avec peine que Surville et La Freselière allassent à Lille
servir sous lui. Il n'avait avec eux ni parenté ni liaison
particulière\,; ils étaient perdus sans retour. Il saisit cette occasion
de les remettre à flot, sans qu'eux ni personne pour eux eussent pu le
deviner.

On a vu en son lieu l'étrange affaire qui perdit Surville. La
Freselière, fils d'un père aimé et révéré de tout le monde et des
troupes, mort fort vieux, lieutenant général, et lieutenant général de
l'artillerie, lui avait succédé en cette dernière charge qu'il faisait
avec capacité et valeur. Devenu maréchal de camp, il ne pouvait prendre
jour qu'une seule fois dans l'armée par campagne, seulement pour y être
reconnu. Il prétendit le prendre à son tour comme tous les autres, et il
y avait été favorisé la campagne avant celle-ci parle maréchal de
Villars, dans l'armée duquel il commandait l'artillerie. Celle-ci, il se
mit dans la tête d'établir en droit ce qu'il n'avait eu que par
tolérance. Il fut refusé. Il insista et le fut encore. Le toupet lui
monta, il envoya la démission de sa charge, sans que tout ce que M. du
Maine put lui dire et faire fût capable de l'arrêter.

C'était vers la mi-mai, au moment du départ. La réponse à cette folie
fut un ordre de se rendre à la Bastille. Avant partir, Boufflers était
allé de chez M\textsuperscript{me} de Maintenon chez Chamillart
s'informer de ce qu'il trouverait à Lille, et travailler courtement
là-dessus avec lui, de chez qui il partit. Ce fut de dessus son bureau
qu'il écrivit à La Freselière en lui envoyant l'ordre que Chamillart
expédia sur-le-champ. Boufflers prit celui qu'il fit expédier en même
temps pour Surville, passa en Picardie à une terre d'Hautefort qui était
sur son chemin, où Surville s'était retiré pour vivre, et l'emmena à
Lille avec lui. Nous devions aller, M\textsuperscript{me} de Saint-Simon
et moi, avec le maréchal et la maréchale de Boufflers le lendemain de ce
départ à Villeroy voir la maréchale. Toute la cour, qui ne le sut que
fort tard, applaudit fort à une si belle action et décorée de tant de
générosité. La défense de Namur répondait de celle que Boufflers ferait
ailleurs. Il eut à Lille toutes sortes de munitions de guerre et de
bouche, force artillerie, trois ingénieurs principaux, dix-neuf
bataillons, deux autres bataillons d'invalides, quelque cavalerie, deux
régiments de dragons, et il enrégimenta trois mille hommes de la
jeunesse de la ville et des environs qui voulut de bon gré servir au
siège. Les ennemis y amenèrent d'abord cent dix pièces de batterie et
cinquante mortiers.

L'électeur de Bavière était cependant à Langendel avec un pont sur le
Rhin, couvert d'une redoute, et le duc d'Hanovre dans ses lignes
d'Etlingen, delà le Rhin, avec un détachement commandé par Mercy
derrière la forêt Noire, tous ces côtés-là fort tranquilles.

Il était pourtant vrai que la plupart des bataillons qui étaient dans
Lille se trouvèrent de nouveaux dont la plupart n'avait jamais entendu
tirer un coup de mousquet, et qu'il n'y avait que médiocrement de
poudre. Il s'y trouva quantité d'autres manquements. Boufflers mit à
profit le peu de temps qu'il eut libre depuis son arrivée à Lille. Il y
avait apporté cent mille écus du sien qu'il avait empruntés, répondit
pour le roi de tout ce qu'il prit ou emprunta en Flandre, ce qui alla à
plus d'un million, et enrégimenta quatre mille fuyards d'Audenarde,
qu'il trouva encore relaissés dans la ville et dans les environs.
L'armée du roi était toujours à Lawendeghem, tranquille derrière le
canal de Bruges. M. de Vendôme s'y moquait de l'opinion du siège de
Lille, comme d'une imagination folle et ridicule, et sa cabale faisait
l'écho à Paris et à la cour qui en furent les dupes. On aurait pu dans
l'intervalle jeter bien des choses très nécessaires qui manquaient dans
Lille, si on l'avait voulu croire l'objet des ennemis. M. de Vendôme
avait eu l'imprudence ou la malice de déclarer tout haut que Mgr le duc
de Bourgogne avait ordre de secourir à quelque prix que ce fût la place
que les, ennemis assiégeraient, mais que, pour Lille, il la prenait sous
sa protection, et répondrait bien que les ennemis ne se hasarderaient
pas à une entreprise d'un si grand engagement dans notre pays. Lille
était investi le 12 août, à ce que le roi apprit le 14 par plusieurs
courriers de Flandre\,; que le même jour il en arriva un de l'armée,
d'où on mandait qu'on croyait les ennemis déterminés à faire le siège de
Tournai, et que là-dessus l'armée allait marcher. On en voulut douter
encore quelques jours\,; à la fin les visages allongèrent, mais la
flatterie prit d'autres langages. Les uns ne craignirent point de dire
d'un ton indifférent qu'on s'était passé de Lille si longtemps qu'on
s'en passerait bien encore. Vaudemont et la cabale le prirent d'un autre
ton. Ils répondirent qu'une entreprise si folle était le plus grand
bonheur qui pût arriver, et qu'il fallait que les prospérités eussent
aveuglé les ennemis, pour s'être engagés si avant dans notre pays pour y
échouer devant une place de cette importance, et avec une armée moins
nombreuse que la nôtre. Ces misérables contes ne déplurent pas au roi,
mais infiniment à lime la duchesse de Bourgogne, qui le fit sentir à
quelques daines qui osèrent les lui tenir.

Le roi Auguste, qui n'avait point de troupes en Flandre, vint incognito
à l'armée des ennemis. Le prince Eugène fit le siège, et ouvrit la
tranchée le 23 août. Le duc de Marlborough commandait l'armée
d'observation. Il passa l'Escaut pour se mettre en situation d'empêcher
la jonction du duc de Berwick avec Mgr le duc de Bourgogne, dont l'armée
était toujours en son même camp de Lawendeghem. Tandis qu'on était tout
occupé de ces intéressantes nouvelles à Fontainebleau, Albéroni y arriva
sans y être attendu et mit pied à terre chez Chamillart. Il y passa
vingt-quatre heures, ne vit ni le roi ni le monde, et s'en retourna tout
court. On peut juger de la curiosité qu'il donna à tout le monde, et de
tous les raisonnements qui se firent. Était-il secrètement mandé\,?
était-ce réprimande\,? était-ce envoi, excuses personnelles ou
éclaircissements des faits passés\,? Mais rien de tout cela, pas même
raisonnements sur les affaires de Flandre. Le duc de Parme tremblant,
mais fort désireux de la ligue d'Italie, avait pris cette voie pour la
presser, pour offrir tout ce peu qu'il pouvait faire, pour entrer dans
des détails bientôt discutés quand on parle, mais qui sont sans fin
quand on écrit. Ce fut là le vrai sujet du voyage d'Albéroni. Mais de
croire qu'entre lui et Chamillart, il n'y eut point quelque, épisode de
Flandre, et qu'il ne vit point en secret M. du Maine, M. de Vaudemont,
et les plus importants de la cabale, je pense que ce serait fort se
tromper. Quatre ou cinq jours après, le roi partit de Fontainebleau le
lundi 27 août, pour aller coucher à Petit-Bourg et le lendemain à
Versailles.

Le roi témoigna ne vouloir rien épargner pour se conserver une place
aussi importante que Lille, et qui était personnellement une de ses
premières conquêtes. Il parut surpris de la tranquillité de son armée
toujours derrière le canal de Bruges, dans ce même camp où elle était
venue d'Audenarde. Il y dépêcha un courrier avec un ordre positif de
marcher au secours. M. de Vendôme le renvoya avec des représentations et
des délais, qui lui en attirèrent un second avec les mêmes ordres encore
plus pressants. Personne dans l'armée n'en comprenait l'inaction. Mgr le
duc de Bourgogne pressait et faisait d'autant plus presser M. de Vendôme
par ce peu de gens d'assez de poids pour l'oser faire, que ce prince se
souvenait des propos d'Audenarde et de ceux qu'avait réveillés
l'opposition qu'il avait montrée à attaquer le grand convoi du prince
Eugène. Les efforts furent vains au premier courrier. Ils ne réussirent
pas mieux au second, par le retour duquel Mgr le duc de Bourgogne ne
laissa pas ignorer au roi qu'il ne tenait pas à lui ni aux généraux
qu'il ne fût obéi. Vendôme demeurait ferme en ses remises et ne voulait
point s'ébranler.

À cette dernière désobéissance le roi se fâcha autant qu'il put se
fâcher contré M. de Vendôme, et dépêcha un troisième courrier avec le
même ordre à ce duc et un autre ordre particulier à son petit-fils de
marcher avec l'armée, malgré M. de Vendôme, s'il continuait à vouloir
différer. Alors il n'y eut plus moyen de s'en défendre, mais {[}il
marcha{]} avec lenteur, sous prétexte de rassembler ce qui était séparé
et de faire les dispositions nécessaires. Plus de prévoyance, ou plutôt
de volonté, eût prévenu ce dernier délai dans un temps où {[}on{]} en
avait perdu un si précieux, et où tous les instants n'en étaient que
plus chers. Lorsqu'il fallut se déterminer sur le choix de la route à
prendre pour joindre le duc de Berwick qui avait reçu les ordres pour
s'avancer de son côté, M de Vendôme maître absolu, ou complaisant sans
réplique, comme il lui convenait pour ses vues, et comme il l'avait bien
montré à Audenarde, sur l'attaque du convoi, et en dernier lieu pour se
mettre en marche de Lawendeghem, ne voulut admettre aucun
raisonnement\,; il décida avec autorité pour le chemin de Tournai, et
dit en même temps que, lorsqu'on s'approcherait de Lille, il permettrait
les délibérations, parce que les divers partis qu'on pourrait prendre le
mériteraient bien.

Le détail de ce qui se passa jusqu'à la jonction serait ici inutile. Il
suffit de dire que Mgr le duc de Bourgogne arriva avec son armée le
mardi 28 août à Ninove sur le minuit. Le lendemain jeudi 29, le duc de
Berwick le vint saluer sur les neuf heures du matin. Il était accompagné
d'un très petit nombre de gens principaux de son armée qu'il avait
laissée à Garrfarache, et qui joignit le 30 la grande, armée dans sa
marche à Lessines.

Berwick, avec ses dignités et son bâton de maréchal de France, orné des
lauriers d'Almanza, et plus que tout cela aux yeux du roi, bâtard encore
plus que Vendôme puisqu'il l'était lui-même, passa comme ses confrères
sous les Fourches claudiennes\footnote{Saint-Simon veut parler des
  \emph{Fourches caudines}, si connues par le désastre des Romains.} le
jour même de la jonction de son armée, pour laquelle il prit l'ordre du
duc de Vendôme avec une indignation dont il ne se cacha pas. Il ne mit
pas le pied chez M. de Vendôme\,; il déclara publiquement qu'il
remettait son armée à Mgr le duc de Bourgogne, pour être incorporée dans
la sienne par un nouvel ordre de bataille et de campement\,; qu'il
n'avait plus rien à y faire, qu'il ne prétendait à aucun commandement,
ni à aucune fonction, et qu'il ne se mêlerait de quoi que ce soit, sinon
de se tenir auprès de la personne de Mgr le duc de Bourgogne.

Razilly s'en était allé pour ne plus revenir à cause de la mort de sa
femme, et d'O avait été mis en sa place auprès de M. le duc de Berry. Le
maréchal de Matignon était allé malade à Tournai, avec un passeport des
ennemis. Il y fut assez mal, et de là, sous prétexte de sa santé, gagna
Paris d'où il eût mieux fait de n'avoir bougé. Berwick avait proposé cet
expédient pour s'épargner le calice de prendre l'ordre. Il fut accepté
pour le lui éviter chaque jour, mais le roi se roidit à le lui faire
avaler une fois en arrivant, pour qu'il ne manquât rien au triomphe de
Vendôme sur tous les maréchaux de France. On peut juger de l'effet que
produisit cette suspension et cette séparation dans l'armée\,; quelle
aigreur\,! quelle division\,! Jamais armée si formidable qu'après cette
espèce d'incorporation\,: cent quatre-vingt-dix-huit escadrons,
quarante-deux en outre de dragons, cent trente bataillons outre, ce qui
en fut dispersé dans les places et dans les postes, et ce qui n'avait
pas rejoint depuis Audenarde\,; tous les corps distingués, la plupart
des vieux et de ceux d'élite, celle de la cour en militaire\,; double
équipage de vivres et d'artillerie, abondance d'argent et de toutes
choses, commodités à souhait du pays et du voisinage de nos places\,;
vingt-trois lieutenants généraux, vingt-cinq maréchaux de camp en ligne,
soixante-dix-sept brigadiers, en un mot, ce qui de mémoire d'homme ne
s'était jamais vu, et une ardeur de combattre qui ne pouvait être plus
vive, plus naturelle, plus générale.

Dans cet état, on marcha à Tournai\,; on y séjourna pour faire passer la
rivière plus commodément, et on comptait sur un combat décisif. Beauvau,
évêque de Tournai, publia des dévotions pour implorer la bénédiction de
Dieu sur nos armes. Mgr le duc de Bourgogne y assista entre autres à une
procession générale. La cabale et les libertins ne le lui pardonnèrent
pas\,; les interprétations furent les plus malignes, et fort
publiques\,; on trouva d'ailleurs que son temps eût été plus
nécessairement employé à des délibérations sur les partis à prendre au
sortir de Tournai, et que c'était prier que de s'acquitter d'un devoir
si urgent et si principal. Il y avait en effet beaucoup à s'aviser sur
les différents partis à prendre, mais il n'y eut presque point de
consultations. Ce peu même fut aigre et tumultueux. Vendôme saisit toute
l'autorité\,; le jeune prince, trop battu, trop mal soutenu, le laissa
faire. Chacun de ce qui était là de principal trembla et mesura ses
paroles. Berwick, uniquement attaché à suivre Mgr le duc de Bourgogne,
se renfermait à lui dire en particulier ce qu'il pensait, et affectait
assez de témoigner son mécontentement et son inutilité. Il s'en ouvrit
en particulier à d'O, et continua à ne voir Vendôme que chez le prince,
improuvant en effet la plupart de ce qui se faisait.

Vendôme se prenait à lui aigrement de sa réserve, de son inutilité, de
son air de censeur dans son silence, surtout des douces oppositions que
le jeune prince montrait quelquefois à ses sentiments, quoique
inutilement\,: Berwick ne fut pas ménagé par la cabale, mais elle
ménagea incomparablement moins l'héritier nécessaire de la couronne, et
acheminait contre lui ses desseins à grands pas. Enfin, parmi toutes ces
agitations, on envoya les bagages à Valenciennes, on acheva de passer
l'Escaut à Tournai. On en partit le 2 septembre, et on se mit à longer
la Marck par des pays coupés et fâcheux, doublant presque le chemin à
cause de la tortuosité du ruisseau. Jusqu'au capitaine des guides
trouvait ce parti-là le moins bon de tous à prendre, soit que l'armée se
fût éloignée du cours du ruisseau pour le doubler après à sa source,
comme on fit, soit qu'elle l'eût passé près de Tournai où il n'y avait
rien de plus facile. Après beaucoup de peine et de fatigue, elle arriva
le 4 septembre à Mons-en-Puelle, vers la source de la Marck, où elle
séjourna cinq jours. Elle s'était approchée ainsi du grand chemin de
Douai à Lille. Elle attendait Saint-Hilaire, avec beaucoup d'artillerie
de Douai pour en être joint à Orchies. Marlborough campait cependant au
dedans de la Marck, sa droite à Pont-à-Marck, sa gauche à
Pont-à-Tressin. Pendant ce petit séjour de notre armée il faut voir ce
qui se passait à la cour, d'où elle attendait des ordres sur le choix
des partis à prendre.

L'agitation y était extrême, jusqu'à l'indécence. On n'y était occupé
que de l'attente d'une bataille décisive\,; chacun était entraîné à la
désirer dans la réduction où en étaient les choses\,; il ne semblait
même plus permis d'en douter.

L'heureuse jonction des deux armées avait été regardée comme un présage
certain du succès. Chaque retardement aigrissait l'impatience\,; depuis
le départ de Tournai jusqu'au courrier dépêché de Mons-en-Puelle, il
n'en était point venu. Chacun était dans l'inquiétude, le roi même
demandait des nouvelles aux courtisans, et ne pouvait comprendre ce qui
retardait les courriers. Les princes et tout ce qui servait de seigneurs
et de gens de la cour étaient dans cette armée. On voyait à Versailles
le danger de ses plus proches, de ses amis, et, les fortunes en l'air
des maisons les plus établies. Les prières de quarante heures étaient
partout\,; M\textsuperscript{me} la duchesse de Bourgogne passait les
nuits à la chapelle, tandis qu'on la croyait au lit, et mettait ses
dames à bout par ses veilles. À son exemple, les femmes qui avaient
leurs maris à l'armée ne bougeaient des églises. Le jeu, les
conversations même avaient cessé. La frayeur était peinte sur les
visages et dans les discours d'une manière honteuse. Passait-il un
cheval un peu vite, tout courait sans savoir où. L'appartement de
Chamillart était investi de laquais, jusque dans la rue\,; chacun
voulait être averti du moment qu'il arriverait un courrier\,; et cette
horreur dura près d'un mois jusqu'à la fin des incertitudes d'une
bataille. Paris, comme plus loin de la source des nouvelles, était
encore plus troublé, les provinces à proportion davantage. Le roi avait
écrit aux évêques pour qu'ils fissent faire des prières publiques, et en
des termes qui convenaient au danger\,; on peut juger quelle en fut
l'impression et l'alarme générale.

La flatterie parmi tout cela ne laissait pas de se présenter de front,
et de se transformer en mille différentes manières\,; jusque-là que
M\textsuperscript{me} d'O s'en allait plaignant le sort de ce pauvre
prince Eugène, dont les grandes actions et la réputation allaient périr
avec lui dans une si folle entreprise, et que, tout ennemi qu'il était,
elle ne pouvait s'empêcher de regretter un capitaine d'un si rare
mérite. La cabale, plus bruyante que jamais, répondait d'une victoire
assurée et de la certitude que le secours de Lille ne pouvait échapper à
M. de Vendôme. J'écoutais ces propos avec indignation\,; j'avais très
présent tout ce qui s'était passé avant et après Audenarde\,; qu'il
n'avait fallu rien moins pour ébranler M. de Vendôme de derrière le
canal de Bruges que trois ordres exprès par trois courriers consécutifs,
et le dernier chargé d'un ordre précis à Mgr le duc de Bourgogne de
faire marcher l'armée malgré lui, s'il s'y opposait encore\,; les délais
que sous divers prétextes il y avait apportés\,; le choix d'autorité
d'un chemin le plus long\,; treize jours de marche, de son aveu, pour
arriver sur Lille, encore s'il n'arrivait point d'embarras, sans compter
les séjours imprévus et nécessaires. Il fallait, disait-il après, le
temps de délibérer le par où on s'y prendrait pour le secours. Je voyais
un si grand temps perdu, et si précieux, tant de loisir au prince Eugène
de bien assurer toutes ses avenues et cependant de presser le siège, et
à Marlborough de bien choisir ses postes, de les reconnaître, de prévoir
tout, pour, de quelque côté qu'on voulût percer, se présenter au-devant
avec tous ses avantages, que le projet de Vendôme et de sa cabale, qui
m'avait saisi en gros dès le choix de Mgr le duc de Bourgogne pour
commander cette armée, me devint évident. Je ne crus jamais que M. de
Vendôme voulût secourir Lille, mais qu'après avoir osé attaquer le
prince aussi hardiment et aussi cruellement qu'il avait fait de dessein
manifestement formé, pendant toute la campagne, sa résolution était bien
prise de lui faire avorter ce secours si important entre les mains, de
l'accabler de tout le blâme, et de l'écraser de la sorte sans retour.

Un soir que, dans l'impatience de ce courrier qu'on attendait toujours
de Mons-en-Puelle, je causais chez Chamillart avec cinq ou six personnes
de sa famille après souper, et où était La Feuillade, pénétré de ma
conviction et du dépit de toutes les vanteries de bataille\,; de
victoires et de secours que j'entendais là sans mot dire de colère,
jusqu'à en désigner le jour et le moment, la patience m'échappa tout
d'un coup, et je proposai à Cani, que j'interrompis, de parier quatre
pistoles qu'il n'y aurait point de combat, et que Lille serait pris et
point secouru. Grand bruit parmi ce peu que nous étions d'une
proposition si étrange, et force questions des raisons qui m'y pouvaient
porter. Je n'avais garde de leur dire la véritable\,; je répondis
froidement que c'était mon opinion. Cani et son père, à l'envi, me
protestèrent que, outre le désir ardent de Vendôme et de toute l'armée,
les ordres les plus précis et les plus réitérés étaient partis pour le
secours\,; que c'était jeter mes quatre pistoles dans la rivière que de
les parier\,; et qu'ils m'en avertissaient parce que Cani parierait à
jeu sûr. Je leur dis avec le même flegme, mais qui couvrait tout ce qui
bouillait en moi, que j'étais persuadé de tout ce qu'ils avançaient,
mais qu'en deux mots je ne changeais point d'avis, et que je le
soutenais à l'anglaise. Je fus encore exhorté, je tins bon, et toujours
avec ce peu de paroles. À la fin, ils consentirent en se moquant de moi,
et Cani me remerciant du petit présent que je lui voulais bien faire.
Nous tirâmes quatre pistoles lui et moi de notre poche, et nous les
mîmes entre les mains de Chamillart. Jamais homme ne fut plus étonné. En
serrant ces huit pistoles, il m'emmena tout à l'autre bout de la
chambre. «\,Au nom de Dieu, me dit-il, faites-moi la grâce de me dire
sur quoi vous fondez votre persuasion, car je vous répète, en foi
d'homme d'honneur, que j'ai dépêché les ordres les plus positifs, et
qu'il n'y a plus aucun moyen de s'en dédire.\,» Je me tirai d'avec lui
par le temps perdu que les ennemis auraient bien employé, et par
l'impossibilité qui se trouverait à l'exécution des ordres et des
désirs. Je n'avais garde, quelque intimes que nous fussions, d'en dire
davantage à un pupille de Vaudemont et de ses nièces, et aussi entêté de
Vendôme, et trop homme d'honneur, mais trop incapable cri même temps
d'ouvrir les yeux pour espérer de lui faire rien voir d'un projet qu'ils
n'avaient eu garde de lui laisser apercevoir, et pour lequel, sans s'en
douter, il les avait jusqu'alors si utilement servis.

Rien de plus simple que ce pari et que la manière dont il s'était fait,
dans un particulier où je passais une partie de presque toutes mes
soirées. Je n'avais pas même voulu m'expliquer sur rien, sinon tête à
tête avec Chamillart, de l'amitié et de la discrétion duquel j'étais
assuré, lorsqu'il me pressa dans ce bout de la chambre où il me promit
même le secret de ce que je lui dirais, et où je ne lui dis rien que de
vague, de mesuré, de public. Une très prompte expérience, et très
fâcheuse dans la suite, m'apprit qu'il n'y avait rien de plus imprudent.
Dès le lendemain, ce pari fut la nouvelle de la cour\,; on ne parla
d'autre chose. On ne vit point à la cour sans ennemis. Je n'y devais
donner d'envie à personne\,; mais les amis considérables que j'y avais
me faisaient regarder comme quelqu'un et quelque chose à mon âge. Les
Lorrains ne me pouvaient pardonner diverses choses que j'ai racontées,
et beaucoup d'autres qui n'ont pas valu la peine d'être écrites. M. du
Maine, dont j'avais esquivé les prodigieuses avances et qui ne pouvait
ignorer ce que je sentais sur son rang, ne m'aimait pas, par conséquent
M\textsuperscript{me} de Maintenon. Je m'étais trop vivement déclaré
lors du combat d'Audenarde pour que la cabale de Vendôme me le
pardonnât. Ils ne laissèrent donc pas tomber mon pari. M. le Duc et
M\textsuperscript{me} la Duchesse s'y joignirent pour l'affaire de
M\textsuperscript{me} de Lussan que j'ai racontée, et ma cessation de
les voir\,; d'Antin, outré fort mal à propos d'une préférence pour
l'ambassade de Rome, qui même n'avait pas eu lieu, et grandement
dédommagé par la fortune qu'il avait saisie depuis, s'y épargna
peut-être moins que personne. Mon laconisme fit peut-être sentir aux
coupables à qui et à quoi j'imputais la perte prochaine de Lille\,;
bref, ce fut dès le lendemain un vacarme épouvantable. La noirceur alla
jusqu'à m'accuser d'improuver tout, d'être mécontent et de me délecter
de tous les mauvais succès. Ces propos furent soigneusement portés
jusqu'au roi\,; ils lui furent adroitement persuadés\,; cette réputation
de tant d'esprit et d'instruction, dont ils s'étaient si bien trouvés
après mon choix pour Rome, fut renouvelée et rafraîchie dans son esprit
avec art, et je me trouvai entièrement perdu auprès de lui sans le
savoir que plus de deux mois après, et sans même me douter de rien à son
égard de fort longtemps. Tout ce que je pus alors fut de laisser tomber
ce grand bruit, et me taire pour ne pas donner lieu à pis.

Enfin ce courrier de Mons-en-Puelle tant attendu arriva, et ne fit que
renouveler les transes et l'aigreur des esprits. Il rapporta que l'armée
était enfin à Mons-en-Puelle campée sur quatre lignes, la droite vers
Blouïs, la gauche sur Tumières, la réserve et les dragons à
Alligny-sur-la-Marck, dans laquelle il n'y avait pas une goutte d'eau\,;
qu'on attendait Saint-Hilaire et sa nombreuse artillerie venant de
Douai\,; que les ennemis avaient leur droite appuyée vers Hennequin à un
marais, leur gauche à Frettin et un autre marais, plusieurs chemins
creux devant eux, surtout à leur droite\,; qu'ils occupaient le village
d'Entiers devant leur gauche\,; qu'ils se retranchaient partout, et
Entiers même, et qu'ils travaillaient à établir quantité de batteries\,;
que notre armée se disposait à déboucher devant eux dans la plaine pour
se mettre en bataille et tâcher à les chasser de là\,; que nous
occupions les châteaux de Plouy-de-l'Assessoy et du Roseau, et la cense
d'Ainville\,; que ce débouché n'avait qu'un quart de lieue de large
entre les bois du Roi à gauche et le château du Roseau à droite, où
commence un pays inaccessible\,; qu'on y travaillait à faire huit
chemins\,; que notre grosse artillerie devait aller par Falempin, parce
qu'on comptait de porter notre gauche par Seclin, vis-à-vis la droite
des ennemis. En cette disposition, il y avait deux partis à choisir,
l'un de déposter les ennemis de vive force, l'autre de jeter du secours
dans Lille qui le pouvait aisément recevoir par le côté de la citadelle,
tandis qu'on tenait les ennemis de si près. Ce dernier parti était
l'avis de tous les généraux, celui de laisser consumer aux ennemis leurs
munitions et leurs vivres, de les jeter dans la nécessité des convois,
et d'attendre de leur impuissance ce qui ne s'en pouvait espérer par la
force.

M. de Vendôme, qui avait tant hésité et retardé pour s'ébranler, qui,
ferme pour le chemin de Tournai, ensuite pour longer la Marck, avait si
nettement déclaré qu'il serait d'avis de mûres délibérations lorsqu'il
serait question des moyens et de la manière du secours, ne s'en souvint
plus dès qu'on en fut là. Il maintint fort et ferme qu'il fallait
attaquer\,; ses dépêches ne chantaient que bataille et victoire, ses
chiens de meute ne publiaient autre chose, tandis qu'ayant pu si
commodément passer la Marck près de Tournai, il avait constamment refusé
d'abréger huit journées, et beaucoup de peine et de fatigues, se porter
de plain-pied dans un pays ouvert et tout proche de Lille, préféré les
inconvénients dont il se trouvait maintenant enveloppé sur la seule
crainte de trouver les ennemis au-devant de lui avant d'être
suffisamment déployé devant eux, sur la seule confiance de les écraser à
force d'artillerie qui lui en fit aller chercher le renfort de
Saint-Hilaire par le long détour qu'il voulut prendre. Mais parlons ici
franchement. Rien de tout cela\,; mais le second tome d'Audenarde, mais
plus pourpensé. La même lenteur et la même opiniâtreté à s'ébranler, la
même ruine par la perte d'un temps précieux, ne rien faire quand il
pouvait tout faire, vouloir tout quand il ne pouvait plus rien, et qu'il
le sentait mieux que personne. Ainsi voulut-il passer la nuit comme on
était après le combat d'Audenarde, et le recommencer le lendemain,
quoiqu'il vît ce dessein insensé et impraticable\,; ainsi publia-t-il
qu'il eût battu les ennemis si on l'eût voulu croire, pour affubler Mgr
le duc de Bourgogne du dommage et de la honte de toute cette action, et
s'en attirer gloire et honneur, tandis que, complaisant une seule fois à
l'opposition de l'attaque du convoi, pour l'insulter mieux, il s'était
rendu si absolu toutes les autres, et l'avait si audacieusement montré
au jeune prince parlant publiquement à lui. On voit la même conduite, la
même cadence en ce secours\,; et quand par ses lenteurs et ses détours,
en fermant la bouche à tout le monde, il a tant fait que de laisser
prendre et accommoder en plein loisir à Marlborough un poste
inattaquable, et qu'il juge très bien qui ne s'attaquera pas, il ferme
la bouche à tous après avoir promis la liberté de délibérer, crie,
écrit, corne bataille et victoire, et prépare à Mgr le duc de Bourgogne
tout l'affront d'avoir manqué le secours.

Ce prince, qui n'avait pas oublié les propos d'Audenarde, tint aussi
pour attaquer les ennemis. Ce courrier tant attendu fut dépêché pour
recevoir les ordres du roi sur le parti auquel on devait s'arrêter,
tandis que les dispositions s'achevaient, et que Saint-Hilaire se hâtait
de joindre. Mais ce ne fut pas tout ce qu'il rapporta. On apprit que le
jour qu'on était arrivé à Orchies, M. de Vendôme avait fait passer à
Pont-à-Marck quelques troupes de l'autre côté de ce ruisseau pour
reconnaître les ennemis qui, le ruisseau entre eux et notre armée,
l'avaient côtoyé le plus près qu'ils avaient pu, et que ce détachement
les ayant trouvés éloignés, parce que ce jour-là ils s'étaient mis dans
le poste que je viens d'expliquer, M. de Vendôme envoya prier Mgr le duc
de Bourgogne de pousser à Pont-à-Marck où il était, et où il lui avait
proposé de faire passer l'armée\,; que tous les officiers généraux
trouvèrent dangereux de se commettre à une action demi-passés, ce qui
pouvait arriver si le duc de Marlborough était averti à temps et se
reployait sur nous\,; que Mgr le duc de Bourgogne ne se déclara pas
assez nettement, quoique Cheladet, lieutenant général, criât qu'il
fallait rompre son épée et n'en porter jamais si on ne passait point
dans un moment si favorable\,; que le duc de Berwick, outré de tout ce
que j'ai raconté, garda un silence opiniâtre\,; qu'enfin le temps
s'étant écoulé en délibérations, la marche s'était continuée sur
Orchies. Il n'est pas croyable le bruit qu'en fit la cabale, et les
avantages qu'elle en prit sur le fils de la maison dans sa maison même,
et partout. Il retentit dans les provinces et dans Paris par, le soin de
ses émissaires, et cela s'établit et pénétra partout. Comme il venait
peu de lettres de Flandre, et toutes laconiques et vaines, chacun
s'étant fait sage par son expérience, il n'est pas possible de
représenter l'excès de l'étonnement, lorsqu'au retour de tout le monde
de l'armée, on sut que tout ce qu'il y avait de véritable de ce grand
débat de Pont-à-Marck, c'était qu'Artagnan, lieutenant général, y avait
passé en effet à la tête d'un gros détachement, avec ordre de longer la
Mark de l'autre côté jusqu'à sa source, qui en était fort proche, afin
de reconnaître le pays et d'y faire faire trois chemins pour faciliter
l'armée à reployer sur les ennemis après qu'elle aurait doublé la source
de la Marck\,; le tout sans que M. de Vendôme, ni autre quel qu'il fût,
eût imaginé de faire passer l'armée à Pont-à-Marck, de l'autre côté de
ce ruisseau, ni de changer quoi que ce fût au premier projet.

La nouvelle consultation faite au roi par les dépêches de ce courrier si
l'on combattrait ou non, le fâcha à tel point, après les ordres positifs
qu'il en avait donnés tant de fois qu'il ne put s'empêcher, contré sa
coutume, d'en laisser voir sa colère. Il dit avec émotion que,
puisqu'ils voulaient encore des ordres, ils en auraient trois heures
après, et trois heures après son arrivée ce même courrier repartit avec
des ordres plus pressants que jamais. Mais on n'en fut pas quitte pour
ce mensonge de dispute de Pont-à-Marck. Il fut répandu avec une
assurance et un déchaînement qui ferma la bouche jusqu'au retour des
officiers principaux de l'armée de Flandre, qu'il s'était tenu un
conseil de guerre à Mons-en-Puelle pour discuter le pour et le contre de
l'attaque des ennemis, et si le pour l'emportait, les moyens et la
manière de la faire\,; que d'O et Gamaches bonnetèrent\footnote{Opinèrent
  du bonnet, sans parler.}\,: les officiers généraux leur représentèrent
avec autorité qu'il s'agissait beaucoup moins de la conservation de
Lille que de celle des princes\,; qu'intimidés de la sorte, M. de
Vendôme fut le seul pour l'attaque\,; que Mgr le duc de Bourgogne, qui
était d'abord de cet avis, se rendit à l'opinion uniforme des officiers
généraux\,; que M. le duc de Berry maltraita un peu le duc de Guiche en
ce conseil\,; que le duc de Berwick se déclara aussi pour la négative\,;
que ce fut en conséquence de ce qui s'était passé en ce conseil que le
courrier avait été dépêché pour consulter encore une fois le roi et
recevoir ses derniers ordres\,; que Vendôme y avait parlé aigrement et
fortement, mais en général, et qu'en sortant de l'assemblée il avait
traité d'O et Gamaches durement. Il est inconcevable avec quelle
célérité cette nouvelle fut répandue, fut reçue, pénétra tout, révolta
tout le monde, et fit de bruit et de désordre. La cour, Paris, les
provinces en retentirent. D'O et Gamaches y passèrent pour avoir agi
dans l'esprit et le désir de Mgr le duc de Bourgogne, sans lequel ils
n'eussent osé d'eux-mêmes se charger d'une commission si dangereuse, si
honteuse, si importante, d'où résultèrent des cris et des clameurs sans
retenue aussi tristes contre Mgr le duc de Bourgogne, que flatteurs pour
le duc de Vendôme. Toutefois ce qu'il y eut de véritable est qu'il ne
fut non seulement pas la moindre question de conseil de guerre, mais pas
même mention de consulter personne. Bien est-il vrai que la cabale que
Vendôme avait dans l'armée fit si bien qu'elle persuada généralement
toutes les troupes, mais sans dire un mot de ce conte imaginaire de
conseil de guerre, que le duc de Vendôme et les siens seuls voulaient
combattre, que Mgr le duc de Bourgogne s'y opposait\,; que cela fit un
fracas étrange dans l'ardeur où elles étaient d'en venir aux mains, et
l'impatience extrême des retardements, d'où la licence s'y glissa au
point qu'elles se mirent à crier au Vendômiste ou au Bourguignon sur
ceux qui passaient à la tête des camps ou des postes, suivant
l'attachement qu'elles leur croyaient, et plus encore suivant l'opinion
bonne ou mauvaise qu'elles avaient de leur courage. Cela dura, entretenu
sous main, après avoir été excité de même. Le contrecoup en fut porté
avec la dernière promptitude à la cour, à Paris, dans les provinces, à
nos autres armées, enfin jusque chez les étrangers et chez les ennemis,
et fit l'effet le plus sinistre. Je me contente ici d'un récit nu dans
la plus exacte vérité. Il est tellement au-dessus de toute réflexion que
je n'y en ferai aucune.

\hypertarget{chapitre-xx.}{%
\chapter{CHAPITRE XX.}\label{chapitre-xx.}}

1708

~

{\textsc{Chamillart à l'armée.}} {\textsc{- Aigreur hardie de M. le
Duc.}} {\textsc{- Vendôme et Berwick replâtrés par Chamillart.}}
{\textsc{- Canonnade d'Entiers.}} {\textsc{- L'armée repasse l'Escaut.}}
{\textsc{- Chamillart de retour à Versailles.}} {\textsc{- Divers
mouvements du roi.}} {\textsc{- Indifférence de Monseigneur.}}
{\textsc{- Monseigneur entraîné pour toujours contre Mgr le duc de
Bourgogne.}} {\textsc{- Audacieux et calomnieux fracas contre Mgr le duc
de Bourgogne.}} {\textsc{- Mensonge en plein sur le P. Martineau.}}
{\textsc{- Mensonges en plein sur Nimègue et Landau.}} {\textsc{-
Prévention du roi.}} {\textsc{- Déchaînement incroyable contre Mgr le
duc de Bourgogne.}} {\textsc{- Fautes sur fautes de Vendôme.}}
{\textsc{- Mort et deuil d'un fils de quatre ans et demi de M. du
Maine.}} {\textsc{- Misère de M. le Prince.}} {\textsc{- Ducasse arrive
avec les galions.}} {\textsc{- Exilles et Fenestrelle pris par le duc de
Savoie.}} {\textsc{- Éloge du maréchal de Boufflers et ses soins à
Lille.}} {\textsc{- Grande défense à Lille.}} {\textsc{- Le chevalier de
Luxembourg se jette avec secours dans Lille\,; est fait lieutenant
général.}} {\textsc{- L'électeur de Bavière à Compiègne, où Chamillart
le va trouver.}} {\textsc{- Bruxelles tristement manqué par l'électeur
de Bavière.}} {\textsc{- Inondations et mouvements contre les convois.}}
{\textsc{- La Motte chargé de s'opposer au convoi.}} {\textsc{- Sa
protection\,; son caractère.}} {\textsc{- Battu par le convoi de
Winendal.}}

~

Parmi tout cela, Vendôme, presque toujours au lit ou à table à
Mons-en-Puelle, déchargé, suivant sa coutume, de tous les détails sur
les uns et sur les autres, ne pensa jamais qu'à multiplier ses chemins
et son artillerie, et ne compta de venir à bout des ennemis qu'en les
écrasant par un feu d'enfer. Au retour du courrier, et Saint-Hilaire
prêt à joindre, la surprise fut extrême à la cour d'y voir disparaître
Chamillart, et à l'armée de l'y voir arriver presque aussitôt que le
courrier. En effet, le 7 septembre, un vendredi matin, ce courrier si
souvent nommé arriva à Versailles, et en fut redépêché trois heures
après. Quelques heures ensuite il en fut envoyé un autre pour faire
avancer l'escorte au-devant de Chamillart, et le soir de, ce même jour,
ce ministre partit à huit heures et demie de Versailles, allant coucher
comme on crut à l'Étang, mais pour l'armée de Flandre. Il arriva à
Mons-en-Puelle le lendemain samedi à six heures du soir.

La cabale triompha de ce voyage avec cette audace, vrai ou faux, de
tirer avantage de tout. Elle publia que le seul objet de ce voyage était
d'arrêter M. de Vendôme dans l'importance de ses fonctions, qu'il
voulait tout quitter, que ce contretemps avait paru si fâcheux que le
roi avait mieux aimé se priver pour quelques jours de son ministre,
quoique si nécessaire dans les circonstances présentes, et l'envoyer au
duc de Vendôme pour l'empêcher, comme que ce pût être, d'abandonner
l'armée et les affaires de la guerre, comme il le voulait. D'autres plus
simples débitèrent que le roi, embarrassé de tant d'avis divers sur un
point si critique, avait envoyé Chamillart, instruit à fond de ses
intentions, pour écouter chacun sur les lieux\,; décider ensuite, et
gagner ainsi le temps qui se perdait en courriers. Mais la vérité est
que le roi, qui, sur les ordres si exprès et si positifs qu'il venait de
donner par ce dernier courrier, ne doutait pas d'une bataille à son
arrivée, désira que Chamillart fût sur les lieux pour être en état,
après le combat, d'ordonner de toutes choses pour que rien ne manquât et
en bien profiter s'il était heureux, ou s'il bastait mal, mettre ordre à
tout, et empêcher les suites de têtes tournées comme à Ramillies,
veiller à la conservation de tout ce qui se pourrait en surintendant
dont les ordres s'étendent dans tous les départements, en homme
d'autorité et de confiance à la main des généraux, capable de, consulter
avec eux et de les décharger de tous autres soins que des purement
militaires. Quelque sage que fût cette mission, la plupart la trouvèrent
ridicule. M. le Duc, toujours enragé de ne rien faire, dit tout haut
qu'il n'était pas douteux que ce voyage n'eût fait plaisir à tout le
monde, parce que, dès qu'on l'avait su, chacun en avait pensé mourir de
rire. Cani demeura auprès du roi pendant l'absence de son père, lui
porta les dépêches, écrivit plusieurs fois sous lui les réponses ou les
ordres qu'il dictait, et pourvut au courant des affaires, ce qui parut
d'une confiance bien singulière, pour son âge.

Le duc de Berwick donna un lit à Chamillart. Il travailla sur-le-champ à
raccommoder le duc de Vendôme avec lui. Que ne peut point un ministre et
un ministre favori\,? Les deux ducs se visitèrent réciproquement\,;
Berwick consentit à parler et à traiter affaires avec Vendôme, mais
toujours sans vouloir de commandement. Mgr le duc de Bourgogne se
rapprocha aussi de Vendôme, qui, éloigné de nouveau, daigna, de son
côté, faire quelques pas. Tout cela fut brusque, mais sincère aussi,
comme on le peut imaginer. Ils passèrent en délibérations la plupart de
la nuit. M. le duc de Berry y fut admis à tout, et y montra du sens et
beaucoup d'envie de faire. Aussi, pour le dire en passant, Vendôme le
fit-il fort valoir, et sa cabale ne perdit point d'occasion de l'exalter
de toute la campagne. C'était le fils favori de Monseigneur, à qui ils
n'avaient garde de déplaire\,; c'était exciter la jalousie de Mgr le duc
de Bourgogne, s'ils l'avaient pu, et c'était se servir de l'un pour
perdre et plus sûrement anéantir l'autre.

Le 9, lendemain de l'arrivée de Chamillart, il passa les défilés avec
les princes, les ducs de Vendôme et de Berwick, et une très courte élite
d'officiers généraux, et furent reconnaître les retranchements des
ennemis. Ils les longèrent de très près d'un bout à l'autre, y
essuyèrent même assez de feu, et dès lors il résulta de cet examen une
impossibilité réelle de forcer un poste si bon de soi, auquel l'art
avait ajouté tout ce qui s'en pouvait attendre. Ils occupaient le même
terrain que j'ai expliqué de la Marck à la Deule, ayant Temple-Mars au
centre. Malgré ce qui sautait aux yeux de tous, Vendôme tint toujours
fort et ferme pour attaquer. C'était un parti pris qui convenait trop à
ses vues pour l'abandonner, un parti conforme aux ordres tant de fois
réitérés, aux désirs si marqués du public, à l'ardeur si manifestée des
troupes, un parti de valeur et d'audace, qui le ferait briller de gloire
à bon marché, parce qu'il en voyait bien l'exécution impossible, et
qu'il n'était pas assez fou pour l'entreprendre contre sa propre
conviction, et contre l'avis sans exception de tout ce qui avait été
admis à cette importante promenade. Cette artificieuse rodomontade
n'empêcha pas Chamillart, libre en Flandre de la tutelle de Vaudemont et
de ses nièces, de mander au roi la vérité telle qu'il l'avait trouvée et
que l'avaient vue comme lui tous ceux qui avaient visité les lignes de
Marlborough avec lui, et nettement que les choses étaient en tel état,
qu'on avait eu raison de lui demander encore une fois ses ordres. Il en
fallait croire ce ministre si peu prévenu pour Mgr le duc de Bourgogne,
si admirateur du duc de Vendôme, et qui sortait d'être témoin de la
colère du roi sur ce dernier courrier, et des ordres que lui-même avait
dépêchés par les siens trois heures après son arrivée.

Le 10, l'armée marcha, passa sans aucun obstacle partie dans la source
de la Marck, partie au-dessus, et se mit la droite à Ennevelin, le
centre à Avelin, la gauche à l'hôpital près d'Houpelin. Mais les ennemis
ayant retiré la même nuit quatre brigades d'infanterie et quelques
dragons qu'ils avaient dans Seclin, nous y portâmes notre gauche. M. de
Vendôme fit canonner le village d'Entiers, auquel leurs retranchements
étaient attachés, et qu'ils avaient aussi très bien retranché. Ils
canonnèrent aussi notre camp, surtout ce qui se trouva le plus vis-à-vis
d'Entiers. M. de Vendôme, qui, avec sa présomption accoutumée, ne
doutait pas de trouver Entiers abandonné, trouva fort étrange que rien
n'y eût branlé, et qu'il ne parût pas au bout de dix-huit heures de
canonnade que rien y fût endommagé. Les choses se trouvant au même état,
le 12, sans apparence de pouvoir attaquer le village d'Entiers tandis
que tant d'artillerie y réussissait si peu, et sans espérance qu'elle y
fît plus d'effet, sans moyen d'attaquer les retranchements, même sans
nous être rendus maîtres d'Entiers\,; ou au moins l'avoir détruit, les
visages commencèrent à s'allonger, et M. de Vendôme à s'apercevoir que
ce feu d'enfer, par lequel il avait compté de les écraser, ne leur
nuirait guère et les embarrasserait encore moins. Enfin, après avoir
occupé quatre jours ce camp d'où M. de Vendôme prétendait tout
foudroyer, il fallut le quitter, lui-même avouant enfin qu'il ne s'y
pouvait rien entreprendre. Il fut donc résolu de faire un grand tour
pour les aller prendre par leurs derrières. On ne fut pas sans
inquiétude qu'ils n'ouvrissent leurs retranchements pour faire à l'armée
du roi la civilité de la reconduire, mais tout se passa tranquillement.
Ils ne songeaient qu'à avancer leur siège, le mettre à couvert, prendre
la place, et point à voler le papillon, ni à se commettre. L'armée alla
donc camper à Bersé, puis à Templeuve où on voulait demeurer quelques
jours\,; mais par le défaut de subsistance, il fallut passer l'Escaut
pour en trouver. Elle le passa donc le 17, et campa la droite à Erinnes,
et la gauche au Saussoy près de Tournai. On fit en même temps quelques
détachements à portée de rejoindre au moment qu'on le voudrait.

Chamillart arriva de l'armée à Versailles pendant le souper du roi, le
mardi 18 septembre. Le roi travailla avec lui au sortir de table jusqu'à
son coucher, et ne fut qu'un moment avec les princesses. Chamillart
rendit compte de tout ce qu'il avait vu, et de la pleine espérance dans
laquelle il avait laissé M. de Vendôme de couper tous les convois des
ennemis, et de leur ôter toute subsistance, c'est-à-dire de les réduire
enfin à abandonner leur siège.

Le roi avait besoin de ces intervalles de consolation et d'espérances.
Quelque maître qu'il fût de ses paroles et de son visage, il sentait
profondément l'impuissance où il tombait de jour en jour de résister à
ses ennemis. Ce que j'en ai raconté sur Samuel Bernard, à qui il fit
presque les honneurs de ses jardins à Marly, d'intelligence avec
Desmarets, pour en tirer un secours qu'il refusait, et qui ne se pouvait
trouver ailleurs, en est une grande preuve. On remarqua beaucoup à
Fontainebleau que la ville de Paris y étant venue le haranguer à
l'occasion du serment de Bignon nouveau prévôt des marchands, comme
Lille venait d'être investie, il répondit non seulement avec bonté, mais
qu'il se servit du terme a de reconnaissance pour sa bonne ville, n et
qu'en le prononçant son visage s'altéra, deux choses qui de tout son
règne ne lui étaient point échappées. D'un autre côté, il avait
quelquefois des distractions de fermeté qui édifiaient moins qu'elles ne
surprenaient. Lors de la jonction du duc de Berwick avec la grande
armée, il remarqua un soir, chez M\textsuperscript{me} de Maintenon,
beaucoup de tristesse et d'inquiétude en M\textsuperscript{me} la
duchesse de Bourgogne. Il s'en étonna et lui en demanda la causa. Il
chercha à la rassurer par le repos et la satisfaction qu'il se sentait
de la jonction de ses armées.

«\,Et les princes, vos petits-fils\,? reprit-elle vivement. J'en suis en
peine, lui répondit-il, mais j'espère que tout ira bien. Et moi,
répliqua-t-elle, c'est de cela aussi que je suis triste et en peine.\,»
Le roi, lors de ce frémissement de la cour que j'ai raconté sur
l'attente à tous moments d'une bataille, désolait la cour par ses
sorties de tous les jours de Versailles pour la chasse ou pour la
promenade, parce qu'on ne pouvait savoir qu'après son retour les
nouvelles qui arrivaient pendant qu'il était dehors\,: soit que ce fût
une habitude qu'il ne voulût pas montrer dépendante de son inquiétude,
soit qu'il n'en eût pas assez pour que ces amusements lui cédassent.

Pour Monseigneur il en paraissait tout à fait exempt, jusque-là que le
jour qu'on attendait Chamillart de retour de Flandre, après Ramillies,
où le roi l'avait envoyé voir et chercher lui-même des nouvelles dont
lui ni personne ne recevait aucune, Monseigneur s'en alla dîner à
Meudon, et dit qu'à son retour il saurait toujours bien les nouvelles.
Il en fit autant plus d'une fois, tandis que cette attente d'une
bataille en Flandre, pour le secours de Lille, collait tout le monde aux
fenêtres pour voir arriver les courriers. Il se trouva présent lorsque
Chamillart vint apporter au roi la nouvelle de l'investiture de cette
place, et qu'il en lut la lettre. À la moitié Monseigneur s'en alla. Le
roi le rappela pour entendre le reste. Il revint et l'entendit. La
lecture achevée, il s'en alla encore, et sans avoir dit un seul mot.
Entrant chez M\textsuperscript{me} la princesse de Conti, il y trouva
M\textsuperscript{me} d'Espinoy, qui avait des grands biens de ses
enfants en Flandre, et qui avant ceci comptait d'aller faire un tour à
Lille. «\,Madame, lui dit-il en arrivant et en riant, comment
feriez-vous à cette heure pour aller à Lille\,?» Et tout de suite leur
en apprit l'investiture. Ces choses-là blessaient véritablement
M\textsuperscript{me} la princesse de Conti. Arrivés à Fontainebleau
pendant tous les mouvements de cette armée, Monseigneur se mit un jour à
réciter, par amusement, une longue enfilade de noms bizarres d'endroits
de la forêt. «\,Mon Dieu, Monseigneur, s'écria-t-elle, la belle mémoire
que vous avez là\,! C'est bien dommage qu'elle ne soit chargée que de
pareilles choses\,!» Il ne tint qu'à lui de sentir le reproche, mais il
ne songea pas qu'il en pût profiter.

Malgré cette insensibilité, la cabale de Vendôme, dont il était
environné et possédé, réussit auprès de lui dans toutes ses vues. Il
loua fort un soir à son coucher M. le duc de Berry devant tout le
monde\,; il le fit encore d'autres fois, et jamais il ne fit mention en
bien de Mgr le duc de Bourgogne. Il dit même une autre fois à son
coucher qu'il ne le comprenait point, qu'il s'était trouvé plusieurs
fois à la tête des armées, mais qu'il n'y avait jamais contredit MM. de
Duras, de Lorges et de Luxembourg, avec qui il était, parce qu'il les
croyait plus capables que lui. Il oubliait apparemment Heilbronn, où il
ne voulut jamais attaquer le prince Louis de Bade, quoi que pût faire et
lui dire M. le maréchal de Lorges, lui en remontrer l'importance et la
facilité, qui l'a eu sur le coeur toute sa vie. La crédulité de
Monseigneur pour ceux qui l'obsédaient allait à un point incroyable à
qui n'en a pas eu l'expérience, comme j'aurai occasion dans la suite de
le montrer. Il avala donc contre son propre fils tout le poison qui lui
fut présenté\,; il laissa voir qu'il en était plein, et il n'en revint
de sa vie. Son goût n'était pas pour lui ni pour ceux qui avaient eu
soin de son éducation. Une piété trop exacte le contraignait et
l'importunait\,; son coeur était pour le roi d'Espagne, et ne s'est
jamais démenti pour lui. Il aimait aussi M. le duc de Berry, qui
l'égayait par son goût pour la liberté et les plaisirs. La cabale en sut
bien profiter. Elle avait un trop puissant intérêt à écarter
foncièrement Mgr le duc de Bourgogne de l'estime, de l'affection, de la
confiance de Monseigneur, qu'ils voulaient gouverner, quand il serait le
maître, et n'avoir point à lutter contre le fils et l'héritier de la
maison, pour ne pas entretenir soigneusement l'éloignement qu'ils
avaient formé.

Ils se mirent donc, au retour de Chamillart, à publier hardiment que
Vendôme seul avait voulu combattre dans tous les temps, qu'il eût fait
lever lé siège honteusement aux ennemis, qu'il les aurait battus,
écrasés, sauvé la France, si à dix fois différentes on eût voulu le
croire. L'éponge était passée sur Audenarde, les délais du départ de
derrière le canal de Bruges effacés, l'oisiveté réelle de Mons-en-Puelle
ignorée. Tout retentit des mensonges grossiers du dessein proposé à
Pont-à-Marck, et du conseil de guerre de Mons-en-Puelle. La carte
blanche avait, ajoutaient-ils faussement, été envoyé depuis à leur
héros, mais trop tard, et ces éloges redoublés retombaient à plomb
contre Mgr le duc de Bourgogne. On rappela tout ce qui avait été inventé
de pis sur Audenarde, on lui disputa les choses précédentes les plus
notoires qui lui avaient fait le plus d'honneur, qui jusqu'alors étaient
demeurées certaines sans contredit aucun. On lui reprochait ce qui
s'était passé à Nimègue, dont j'ai parlé. M. du Maine, sur qui tout
porta à la double douleur du roi, qui ne l'a pas fait servir depuis,
trouvait trop bien son compte à la confusion du fait passé, que la
cabale n'avait garde de l'oublier, et de n'y pas insister. Elle
obscurcissait le jeune prince à Brisach, et semait avec adresse que, las
de tant d'efforts qu'il y avait faits, et prévoyant qu'il lui en
coûterait de plus grands encore devant Landau, il était revenu avec tant
de promptitude qu'il n'en avait reçu la permission qu'en chemin.

Les plus modérés en apparence prirent un autre tour, et d'une adresse
bien plus dangereuse. Ils n'accusaient point sa valeur et ne disaient
rien qui eût un air odieux\,; ils s'en prirent à sa dévotion. Ils
disaient que la réflexion sur tant de sang répandu, sur la perte de tant
d'âmes, sur la mort de tant de gens tués sans confession, s'il donnait
la bataille, l'avait épouvanté\,; qu'il n'avait pu se résoudre d'en être
responsable à Dieu\,; que par cette raison il avait voulu s'en décharger
sur le roi, et avoir encore une fois ses ordres précis\,; que c'est ce
qui lui avait fait dépêcher ce courrier de Mons-en-Puelle. De là ils
passaient aux raisonnements politiques, discutaient le peu d'aptitude
d'un prince si scrupuleux pour commander des armées et gouverner un
royaume\,; rendirent autant qu'ils purent sensibles leurs craintes et
leur opinion. De là tombant sur quelques amusements véritablement trop
petits, et sur d'autres déplacés de ce prince, ils exagérèrent quelques
tenues de table trop longues, et quelques parties de volant, et
tournèrent en ridicule des mouches guêpes crevées, un fruit dans de
l'huile, des grains de raisin écrasés en rêvant, et des propos
d'anatomie, de mécanique et d'autres sciences abstraites, surtout un
particulier trop long et trop fréquent avec le P. Martineau, son
confesseur. Pour rendre le prince plus petit et plus incapable, voici
l'histoire qu'ils inventèrent sur du vrai qu'ils firent courir partout.

Le P. Martineau eut la curiosité de visiter les retranchements du duc de
Marlborough à la suite des princes, lorsque avec les ducs de Vendôme et
de Berwick, Puységur et fort peu d'autres officiers généraux et
Chamillart, ils les longèrent de près, comme je l'ai raconté, pour
examiner si et par où ils pouvaient être attaqués. À ce fait véritable,
voici ce qu'ils y ajoutèrent de parfaitement faux. C'est que le P.
Martineau était si affligé de ce que Mgr le duc de Bourgogne s'était
opposé à cette attaque, qu'il l'avait mandé à ses amis, dans la crainte
même d'être accusé d'avoir pu donner un avis si éloigné de son
sentiment. Non contents d'un si noir artifice, et qui mettait en valeur
et en fait de guerre un prince si fort au-dessous de son confesseur, ils
osèrent répandre que Martineau avait eu peur qu'on ne se prît à lui dans
l'armée d'un parti qui la désespérait, et qu'il n'avait pu s'empêcher de
s'y laisser entendre que s'il en avait été cru, les retranchements
auraient été attaqués. La calomnie devint publique. Le P. de La Chaise
qui en fut averti, et qu'il se disait de plus que le P. Martineau lui en
avait mandé sa pensée, se crut obligé de montrer au roi ce que le P.
Martineau lui avait écrit de la curiosité qu'il avait eue, sans qu'il y
eût dans toute la lettre un seul mot qui pût donner lieu à ce qui se
publiait. Le P. de La Chaise la fit voir à bien des gens pour laver
cette calomnie, qui ne laissa pas de porter tout entière sur Mgr le duc
de Bourgogne et en ridicule et en sérieux, comme les inventeurs se
l'étaient bien proposé.

Voilà donc les trois mensonges les plus impudents, les trois histoires
les plus complètement composées qu'il soit possible d'imaginer,
celle-ci, l'affaire de Pont-à-Marck, et le conseil de guerre de
Mons-en-Puelle, ignorés parfaitement dans l'armée, démentis par tout ce
qui en arriva officiers généraux et particuliers, dont l'étonnement fut
extrême d'apprendre à leur retour ce dont ils n'avaient jamais ouï
parler, et qui néanmoins coururent les provinces, les autres armées, et
les pays étrangers, avec des circonstances à n'en pouvoir douter.
Répondre au fait de Nimègue, qui l'eût osé\,? C'eût été rouvrir les
plaies de M. du Maine, et celle du roi par conséquent. À l'égard de
Brisach et de Landau la chose fut agitée en plein conseil du roi.
Tallard, qui prévoyait ce qui pouvait arriver du projet de Landau, et
qui, en effet, causa la bataille de Spire, ne proposa ce siège qu'à
condition expresse du retour de Mgr le duc de Bourgogne, Brisach pris.
Ce prince écrivit au roi pour demeurer et faire ce siège\,; il contesta
et n'oublia rien de tout ce qu'il put représenter de plus fort. Tallard
et Marsin en furent témoins, et enfin il ne partit que sur la dernière
réponse du roi qui, après plusieurs refus et ordres de revenir, lui
manda positivement que le siège de Landau ne s'entreprendrait résolument
point, tant qu'il serait à l'armée.

Quoi de plus clair que ces réponses et que ces faits\,? Mais toute
évidence fut ici inutile. Le complot était trop bien fait, et la cabale
trop habile et trop organisée. Ses émissaires de tous états étaient
infinis. Ils pénétraient partout, ils persuadaient partout les louanges
de leur héros et leurs plus cruels artifices contre un prince qu'ils
avaient bien résolu de perdre, et contre qui, après en avoir tant fait,
ils ne se crurent pas en sûreté de reculer, mais dont ils n'eurent
jamais la moindre envie. Maîtres déjà de la maison paternelle, comment
ne l'être pas du public\,? On a vu à quel point ils avaient persuadé et
aliéné Monseigneur et tous les avantages qu'ils avaient pris sur le roi,
malgré M\textsuperscript{me} la duchesse de Bourgogne, et
M\textsuperscript{me} de Maintenon même. Outre ce qu'il lui échappait à
ses bâtards et à ses valets de trop conforme aux impressions qu'il
recevait d'eux, toujours à l'affût de lui en donner des plus sinistres,
il s'étonna aigrement plus d'une fois en public, parmi ces crises, de ce
que la bataille ne se donnait point, et après, de ce que les
retranchements n'étaient pas encore attaqués. Le rare est que, dans
toute sa cour, ce n'était presque jamais qu'à Vaudemont qu'il adressait
la parole sur la Flandre, et que, si quelqu'un à ces portées-là, même
des princes du sang, hasardait de mêler quelque mot dans la
conversation, cela tombait aussitôt, le roi le plus ordinairement, n'y
répondant point, et Vaudemont toujours tenant le dé et le sachant manier
à merveilles. La cabale triompha donc si pleinement partout, qu'il fut
vrai que ce qu'elle osa à Audenarde ne fut que des coups d'essai et que
c'en fut ici de maîtres. Non seulement le public de tous états était
enlevé, non seulement la mode et le bon air étaient gagnés, mais le
rapide progrès fut tel qu'il emporta les politiques, et qu'il est vrai
exactement de dire qu'il n'y avait pas sûreté à paraître le moins du
monde pour Mgr le duc de Bourgogne dans sa maison paternelle, et que
tout ce qui y exaltait à ses dépens le duc de Vendôme était sûr de
plaire au roi et à Monseigneur. De là on peut juger quel put être le
déchaînement et la licence, jusque-là que le roi, n'osant aussi trouver
publiquement mauvais que quelqu'un osât parler en faveur de son
petit-fils, réprimanda publiquement, le prince de Conti qui le faisait
en toute occasion, et qui haïssait Vendôme, d'avoir parlé et raisonné
des affaires de Flandre chez la princesse de Conti, sa belle-soeur,
tandis qu'on ne parlait et qu'on ne s'entretenait d'autre chose à
Versailles. Pour d'écriture, il n'en était point. Personne n'osait rien
mander à l'armée de ce qu'il se passait et se disait à Paris et à la
cour, ni de l'armée rien qui pût éclaircir ni apprendre quoi que ce fût,
tant la terreur de Vendôme y était répandue.

Mgr le duc de Bourgogne vivait à l'armée en de cruelles brassières. Sa
douceur, sa timidité, sa piété avaient augmenté l'audace, et l'audace
portée à l'excès avait achevé de l'abattre. M. de `Beauvilliers, plus
timide qu'il ne devait l'être, M. de Chevreuse, enchaîné de
raisonnements et de mesure, se désolaient avec moi, et m'avouaient
souvent que je ne leur avais prédit que trop vrai, et vu que trop clair.
Mais de remède, ils n'en voyaient que dans la patience, dans le retour
de l'armée qui éclaircirait bien des choses, et dans le temps\,; et
quand je les pressais pour des partis plus prompts et plus décents, ils
me fermaient la bouche, ils s'affligeaient de ce qu'il n'était plus
temps, ils m'opposaient la volonté impuissante de M\textsuperscript{me}
de Maintenon qui se laissait voir entière sur cet article au duc de
Beauvilliers, comme je l'ai déjà dit\,; et à cette réponse majeure je
n'avais rien à répliquer. Je n'ignorais pas où on en était de ce côté-là
par M\textsuperscript{me} la duchesse de Bourgogne avec qui mon commerce
allait toujours sur la Flandre par M\textsuperscript{me} de Nogaret. Le
peu de temps que cette princesse pouvait avoir à elle, elle le donnait à
ses larmes et à écrire, et dans la vérité, elle parut infatigable, et
pleine de force et de bons conseils. M\textsuperscript{me} de Maintenon
était touchée au dernier point de sa douleur, et piquée au vif de
sentir, pour la première fois de sa vie, qu'il y avait des gens qui, par
rapport à eux, avaient pris sur elle le dessus auprès du roi.

Tandis que le roi reprenait un peu haleine, ses généraux s'occupaient
toujours des moyens de secourir Lille. Vendôme, fécond en projets
spécieux et hardis, voulait faire un grand tour pour prendre Marlborough
par ses derrières, tantôt le tromper par de fausses marches, l'engager à
dégarnir ses retranchements, et revenir tout court sur soi les attaquer.
Mais lent en effet à toute exécution facile, comme on ne l'avait que
trop éprouvé, pouvait-on se flatter de tromper des chefs si attentifs et
si actifs, et de quelques succès par de longs détours qui marqueraient
le projet assez tôt à des ennemis bien postés et qui, pour ainsi dire,
n'auraient qu'à se retourner dans leur cerceau pour faire à temps face
partout et opposer les mêmes obstacles\,? Berwick et tout ce qu'il y
avait là de meilleur parmi les principaux officiers généraux
s'opposèrent à ces entreprises vaines et ruineuses. Ce maréchal, si
légèrement réconcilié avec le duc de Vendôme, avait déjà recommencé à
déplaire à un homme qui n'était pas plus sincèrement revenu à lui. On
commença aussi à s'apercevoir que si, après avoir tant perdu de temps
précieux à s'ébranler et à arriver, au lieu de s'enivrer de l'espérance
d'une bataille, on eût tourné toutes ses pensées à jeter des secours
dans Lille durant qu'on le pouvait, comme je l'ai remarqué, à donner à
la place les moyens de durer, à fatiguer cependant les ennemis, à les
jeter dans la nécessité des convois, et à leur en ôter les moyens par
les postes qu'on pouvait prendre, on serait venu à bout de leur arracher
cette conquête et de les précipiter, de plus, dans des embarras les plus
fâcheux pour leur retraite. Ce fut donc à cette ressource, mais trop
tard, qu'on se résolut de s'attacher désormais, et l'armée fit les
mouvements et les détachements nécessaires pour y réussir.

Parmi des événements si intéressants, il en arriva un à la cour qui le
fut fort peu, mais qui toucha fort le roi. M. du Maine perdit son
troisième fils, qui avait quatre ans et demi.

Le roi continua de faire pour lui ce qu'il n'avait point fait pour les
enfants de la reine, dont il a perdu beaucoup, et dont on n'a jamais
pris le deuil quand ils n'avaient pas sept ans faits. Il ordonna que
Monseigneur et la cour le prendraient pour huit jours, et il envoya
Souvré, maître de sa garde-robe, faire compliment de sa part à M. le
Prince et à M\textsuperscript{me} la Princesse à Écouen, où ils étaient.
M. le Prince ne manqua pas de se donner le plaisir de venir à Versailles
jouir de la distinction de croire y figurer avec le roi, parce qu'il n'y
eut que le roi et lui qui ne prirent pas le deuil.

Incontinent après, il vint une consolation plus solide que n'avait été
cette affliction. Ducasse, qui était allé chercher les galions dont on
avait si grand besoin, les ramena riches de cinquante millions en or et
argent, et de dix millions de fruits. Il arriva au port du Passage et y
entra le 27 août. Bientôt après aussi on sut que M. de Savoie avait pris
Fénestrelle. Il avait aussi pris Exilles quelque temps auparavant,
malgré les forfanteries du maréchal de Villars qui, libéral en
courriers, parce qu'il ne les payait point, promettait toujours des
merveilles, et se donnait souvent pour être sur le point d'attaquer et
battre ce prince. Il prit deux ou trois méchants petits postes
retranchés dans les montagnes qu'il fit fort valoir et fut réduit toute
la campagne à prendre l'ordre des ennemis. Heureusement pour lui,
quelque important que fût un côté si jaloux, ce fut un point dans la
carte, en comparaison des choses qui se passaient en Flandre, qui
absorbaient toute l'attention.

Le prince Eugène n'avait pas dissimulé sa joie, lorsqu'il sut qu'il
aurait affaire au maréchal de Boufflers, et qu'il craignait moins un
homme comblé d'honneurs et de récompenses qu'il n'eût fait un officier
général dont toutes les espérances de fortune auraient été fondées sur
sa défense. Il éprouva qu'il s'était trompé, et je ne comprends pas
comment le souvenir de la défense de Namur ne lui avait pas donné une
autre, opinion de Boufflers qui, à la vérité en fut fait duc, mais qui,
à cette exception, grande à la vérité, était déjà tout ce qu'il était à
Lille. L'ordre, l'exactitude, la vigilance, c'était où il excellait. Sa
valeur était nette, modeste, naturelle, franche, froide. Il voyait tout
et donnait ordre à tout sous le plus grand feu, comme s'il eût été dans
sa chambre\,; égal dans le péril, dans l'action rien ne lui échauffait
la tête, pas même les plus fâcheux contretemps. Sa prévoyance s'étendait
à tout, et dans l'exécution il n'oubliait rien. Sa bonté et sa
politesse, qui ne se démentait en aucun temps, lui gagnait tout le
monde\,; son équité, sa droiture, son attention à se communiquer et à
prendre conseil, sa patience à laisser débattre avec liberté, sa
délicatesse à faire toujours honneur de leurs conseils, quand ils
avaient réussi, à ceux qui les lui avaient donnés, et des actions à ceux
qui les avaient faites, lui dévouèrent les coeurs. Les soins qu'il prit
en arrivant pour faire durer les munitions de guerres et les vivres,
l'égale proportion qu'il fit garder en tous les temps du siège, en la
distribution du pain, du vin, de la viande et de tout ce qui sert à la
nourriture où il présida lui-même, et les soins infinis qu'il fit
prendre et qu'il prit lui-même des hôpitaux, le firent adorer des
troupes et des bourgeois. Il les aguerrit, je dis les troupes de salade,
qui faisaient la plus nombreuse partie de sa garnison, les fuyards
d'Audenarde et les bourgeois qu'il avait enrégimentés, et en fit des
soldats qui ne furent pas inférieurs à ceux des vieux corps.

Accessible à toute heure, prévenant pour tous, attentif à éviter, autant
qu'il le pouvait, la fatigue aux autres et les périls inutiles, il
fatiguait pour tous, se trouvait partout, et sans cesse voyait et
disposait par lui-même, et s'exposait continuellement. Il couchait tout
habillé aux attaques, et il ne se mit pas trois fois dans son lit depuis
l'ouverture de la tranchée jusqu'à la chamade. On ne peut comprendre
comment un homme de son âge, et usé à la guerre, put soutenir un pareil
travail de corps et d'esprit, et sans sortir jamais de son sang-froid et
de son égalité. On lui reprocha qu'il s'exposait trop\,; il le faisait
pour tout voir par ses yeux et pourvoir à tout à mesure\,; il le faisait
aussi pour l'exemple et pour sa propre inquiétude que tout allât et
s'exécutât bien. Il fut légèrement blessé plusieurs fois, s'en cachait
tant qu'il pouvait, et n'en changeait rien à sa conduite journalière\,;
mais un coup à la tête l'ayant renversé, il fut porté chez lui malgré
lui. On le voulut saigner, il s'y opposa de peur que cela lui ôtât des
forces, et voulut sortir. Sa maison était investie, il fut menacé par
les cris des soldats qu'ils quitteraient leurs postes s'ils le
revoyaient de plus de vingt-quatre heures de là\,; il les passa assiégé
chez lui, forcé à se faire saigner et à se reposer. Quand il reparut, on
ne vit jamais tant de joie. Abondance à sa table, sans aucune
délicatesse, il se traita toujours à proportion comme les autres pour
les vivres, et outre ce qu'il avait porté d'argent pour soi, il en
emprunta encore en arrivant tout ce qu'il put, et s'en servit
libéralement pour le service, pour donner aux soldats et secourir des
officiers, avec une simplicité admirable dans toutes ses actions, et
voilà comme il arrive quelquefois que la bonté et la droiture de l'âme
étend l'esprit et l'éclaire dans de grandes occasions.

Il faudrait un journal de ce grand siège pour raconter les merveilles de
la capacité et de la valeur de cette défense. Les sorties furent
fréquentes, et tout fut disputé pied à pied tant que chaque pouce de
terre le put être. Ils repoussèrent jusqu'à trois fois de suite les
ennemis d'un moulin, le reprirent et à la troisième fois le brûlèrent.
Ils soutinrent l'attaque de leur chemin couvert par trois endroits à la
fois, et par dix mille hommes, depuis neuf heures du soir jusqu'à trois
heures du matin, et le conservèrent. Ils en reprirent quelques jours
après la seule traverse dont les ennemis étaient demeurés maîtres,
qu'ils leur enlevèrent par une sortie. Dans une autre, ils rechassèrent
les assiégeants des angles saillants de la contrescarpe dont ils étaient
maîtres depuis huit jours. Ils repoussèrent par deux fois sept mille
hommes qui attaquèrent leur chemin couvert et un tenaillon\footnote{Partie
  des fortifications construite vis-à-vis l'une des faces de la
  demi-lune.}\,; à la troisième ils perdirent un angle du tenaillon,
mais ils demeurèrent maîtres des traverses, du chemin couvert et d'un
retranchement fait derrière ce tenaillon, et le prince Eugène fut blessé
à cette attaque. Quelques jours, après, le chemin couvert des ouvrages à
corne fut encore attaqué et conservé, mais l'autre angle de ce même
tenaillon demeura aux ennemis. Tant d'actions et si grosses affaiblirent
fort la garnison. La poudre commençait à manquer. Le maréchal de
Boufflers trouvait moyen de donner souvent de ses nouvelles. On songea à
y faire entrer quelques secours, s'il était possible. Le chevalier de
Luxembourg, maréchal de camp, et aujourd'hui maréchal de France, fut
chargé de le tenter. Il y marcha de Douai et l'exécuta bravement la nuit
du 28 au 29 septembre, et y jeta avec lui deux mille cavaliers, ayant
chacun un fusil au lieu de mousqueton, et soixante livres de poudre en
croupe, ce qui donna à la place deux mille fusils et plus de cent mille
livres de poudre. Deux régiments d'infanterie qui s'y devaient jeter
avec lui ne purent y réussir\,; il y eut peu de perte. Le chevalier de
Luxembourg fut fort applaudi d'une si vigoureuse action, et fut fait
sur-le-champ lieutenant général.

Le 5 octobre, le chemin couvert et le tenaillon furent attaqués par
seize mille hommes. L'action fut longue et bien disputée. Ils
emportèrent enfin le tenaillon et une demi-lune derrière, mais les
assiégés conservèrent encore quelques coupures du chemin couvert. Cette
demi-lune ne fut prise que par la faute d'un lieutenant-colonel qui
s'était endormi, et qui fut surpris tout au commencement de l'action.
Boufflers fut assez bon pour n'avoir pas voulu le nommer. L'action du 9
au 10 octobre fut encore plus vive. Ils attaquèrent par trois fois le
chemin couvert, et furent repoussés autant de fois\,; à la quatrième,
ils l'emportèrent, arrachèrent les palissades des traverses et mirent
quantité de gabions. Quatre cents dragons firent une sortie sur eux, les
rechassèrent pas un long combat, ôtèrent les gabions, rétablirent les
palissades, tellement que les ennemis n'en furent de rien plus avancés.
Ce fut le quinzième grand combat depuis le commencement du siège. Le 13
octobre, le chemin couvert fut attaqué en plein jour, trois fois à
heures différentes, et les assiégeants toujours repoussés. Ils y
revinrent une quatrième avec plus de troupes, et se rendirent maîtres
d'une traverse du chemin couvert. La brèche du bastion gauche était de
cinquante toises, que le maréchal avait fort fait escarper et accommoder
avec des arbres et tout ce qu'il avait pu trouver de grilles de fer. Le
chevalier de Luxembourg fit le 16 une grande sortie, renversa quelques
travaux, tua assez de monde, mais il ne put les chasser du chemin
couvert. Ils travaillaient fort alors à saigner le fossé et à faire de
nouvelles brèches avec leur artillerie. On ne finirait point à coter
simplement tous les beaux faits d'armes qui s'y exécutèrent.

On était cependant fort occupé de toutes les mesures qu'on pouvait
prendre pour empêcher les convois aux ennemis, qui en avaient déjà amené
un fort considérable devant la place, et en même temps de profiter de
l'occupation de toutes leurs troupes pour faire quelque diversion, et se
dédommager par quelque chose. L'électeur de Bavière avait remis à du
Bourg le commandement de l'armée du Rhin qui n'avait qu'à subsister
tranquillement, séparée des Impériaux par ce fleuve, lesquels ne
pensaient aussi qu'à vivre. Le duc d'Hanovre hors d'état de rien
entreprendre, et lassé d'une campagne si insipide, était retourné chez
lui, et l'électeur était à Compiègne, où le roi lui fit trouver toutes
sortes d'équipages de chasse, et où il lui envoya le duc d'Humières qui
en était gouverneur et capitaine, pour lui en faire les honneurs. Il y
vivait dans ces amusements, lorsque sa petite cour fut tout d'un coup
surprise d'y voir arriver Chamillart. Ce qui l'y conduisit éclata peu de
jours après. L'électeur s'en alla en poste à Mons avec peu de suite\,;
Bergheyck dont les soins infatigables pour la subsistance de nos
troupes, le détail et l'ordre de toutes choses, furent sans cesse d'une
utilité infinie, Puyguyon, lieutenant général, Saint-Nectaire, Ourches,
maréchaux de camp, et l'électeur sur le tout, s'approchèrent de
Bruxelles par divers côtés avec trois mille chevaux et vingt-quatre
bataillons. Ils avaient un train d'artillerie et des vivres avec eux.
Tout cela arriva sur Notre-Dame de Hall, et tout aussitôt après à
Bruxelles, qu'on crut insultable et dégarni de troupes. C'était vers le
20 septembre. Les ennemis, tard avertis, mais qui excellèrent toujours à
mettre tous les instants à profit, y jetèrent tout ce que le temps leur
permit de troupes, et par là réduisirent l'électeur à une attaque dans
les formes. Cela leur donna le temps d'assembler un assez gros corps
pour marcher à Bruxelles. Nous n'en avions aucun pour pouvoir soutenir
l'électeur, qui, trouvant tout autre chose que des bourgeois sans
défense, et sur l'affection desquels il comptait toujours, se vit en
péril d'être battu et pris par ses derrières. Il leva donc si
brusquement cette manière informe de siège qu'il y laissa toute son
artillerie, et toutes les marques d'une retraite plus que précipitée, et
rentra dans Mons peu de jours après en être sorti.

La Connelaye, capitaine aux gardes qui commandait à Nieuport, eut ordre
alors d'en lâcher les écluses. On espérait par là mettre assez d'eau
dans le pays pour empêcher les convois que les ennemis ne pouvaient
tirer que d'Ostende, ou les obliger à un détour qui donnerait le temps
d'arriver aux troupes qu'on envoyait au comte de La Motte, chargé de les
couper. Le duc de Berwick alla à Bruges, où quarante bataillons et
cinquante escadrons se rassemblèrent en même temps. Les chariots que les
ennemis envoyaient à Ostende pour charger le convoi ne purent passer
l'inondation. Ils prirent le parti d'aller s'ouvrir le chemin par
Plassendal où était le comte de La Mothe et où Puyguyon marcha en même
temps avec quarante bataillons. Cependant les chariots vides arrêtés par
l'inondation trouvèrent le moyen de passer, et arrivèrent à Ostende. La
question fut du retour. Ils le firent comme par degrés, et avec les plus
grandes précautions pour s'approcher au plus près, et passer ensuite à
force ouverte.

Berwick tout porté sur les lieux fut pressé par les officiers principaux
de faire lui-même l'attaque de ce convoi\,; mais il répondit qu'il ne
fallait pas ôter à un gentilhomme qui servait depuis bien des années
l'occasion d'acquérir le bâton de maréchal de France, puis leur ferma la
bouche, en leur montrant l'ordre précis de la cour qui commettait cette
expédition à La Mothe. Lui et la duchesse de Ventadour, qui l'avait
obtenu de Chamillart son ami, étaient enfants des deux frères.
M\textsuperscript{me} de Ventadour le regardait comme le sien, c'était
un homme désintéressé, plein de valeur, d'honneur et d'ambition, qui
servait toute sa vie, été et hiver, qui a voit toujours eu des corps
séparés depuis longtemps, et qui touchait au but\,; mais en temps
l'homme le plus court, le plus opiniâtre et le plus incapable qui fût
peut-être parmi les lieutenants généraux. Berwick se retira de sa
personne, et La Mothe se mit en marche. Les ennemis avaient retranché le
poste de Winendal pour couvrir la marche de leur convoi, qui était
immense. La Mothe crut faire merveilles d'attaquer ce poste. Les
dispositions en furent longues et peut-être médiocres. Elles donnèrent
le temps aux ennemis d'y être renforcés et au convoi de s'avancer. La
Mothe ne pensa pas même à débander un gros corps de dragons qu'il avait
pour en embarrasser du moins la tête et l'arrêter, tandis qu'il serait
occupé à l'attaque de Winendal. Bref, il l'attaqua\,; Cadogan le
défendit mieux, ébranla La Mothe, sortit sur lui, le poussa, le battit,
le dissipa avec la moitié moins de forces que n'en avait La Mothe, et
cependant le convoi arriva au camp du prince Eugène qui manquait
absolument de tout, et y rendit l'abondance et la joie.

\hypertarget{chapitre-xxi.}{%
\chapter{CHAPITRE XXI.}\label{chapitre-xxi.}}

1708

~

{\textsc{Menin et Ath manqués par les Albergotti oncle et neveu.}}
{\textsc{- Vendôme, pour fermer les convois, assiège Leffinghem\,; où le
chevalier de Croissy est près pour la troisième fois de la guerre.}}
{\textsc{- État de Lille.}} {\textsc{- Capitulation de Lille.}}
{\textsc{- Boufflers en rien subordonné à Vendôme.}} {\textsc{-
Boufflers entre dans la citadelle de Lille.}} {\textsc{- Leffinghem pris
l'épée à la main par les troupes de Vendôme.}} {\textsc{- Le duc de
Beauvilliers m'arrête à la cour.}} {\textsc{- Calomnies grossières
contre moi.}} {\textsc{- Mort de Tréville\,; abrégé de lui.}} {\textsc{-
Mort et caractère de Lyonne.}} {\textsc{- Enfants de ministres emblent
toutes les charges de la cour.}} {\textsc{- Jarzé remercié de
l'ambassade de Suisse, le comte du Luc y est nommé.}} {\textsc{- Duc
d'Enghien chevalier de l'ordre.}} {\textsc{- Mort en spectacle du
maréchal de Noailles\,; son caractère et celui de sa femme.}} {\textsc{-
Retour du duc de Noailles à la cour.}} {\textsc{- Mort de Saint-Mars,
gouverneur de la Bastille\,; de Bernaville lui succède.}} {\textsc{-
Mort et caractère de la maréchal de Villeroy.}} {\textsc{- Mort et
caractère de la comtesse de Beuvron.}} {\textsc{- Mort et caractère du
comte de Marsan.}}

~

Le dépit de ce triste succès fut extrême dans l'armée, et la douleur à
la cour où on triomphait des assiégeants assiégés eux-mêmes, également
hors d'état de continuer le siège par le manquement général de toutes
choses, et de savoir par où se retirer à travers tous les différents
postes de notre armée. La Mothe y fut un peu pillé, mais la même
protection qui lui avait valu la commission dont il s'était si mal tiré
sut bien le protéger encore assez pour le faire paraître au roi plus
malheureux qu'ignorant. Albemarle menait le convoi. Vendôme s'en alla à
Bruges prendre le commandement des troupes qu'avait La Mothe. On ne
laissa pas d'être surpris et de raisonner sur la prière que le duc de
Marlborough envoya faire presque aussitôt après à Mgr le duc de
Bourgogne de lui vouloir accorder un passeport pour ses équipages, et
qui lui fut envoyé, mais uniquement pour les siens. On jugea qu'il
voulait mettre à couvert beaucoup d'argent qu'il avait tiré des
sauvegardes\,; mais ne pouvait-on pas soupçonner, après l'arrivée du
convoi, ou qu'il se moquait, ou qu'il avait envie de découvrir quelque
chose par un envoi qui parut avec raison fort déplacé\,?

M. de Vendôme, qui avait quarante-trois bataillons et soixante-trois
escadrons, mit sa droite au Moordick et sa gauche au canal qui va de
Bruges à Plassendal, pour empêcher les convois d'Ostende et de l'Écluse.
Marlborough s'alla camper à Rousselaer, faisant mine de l'attaquer pour
faire passer les convois, contre lesquels les inondations furent fort
grossies. Les ennemis y jetèrent des barques pour y décharger leurs
chariots, qui amenèrent au prince Eugène tout ce qu'elles purent.

Parmi tous ces mouvements si vifs on songeait toujours à des
entreprises\,; on avait des intelligences dans Menin, on en crut la
surprise facile, on la résolut. La commission était agréable, son succès
promettait un avancement certain à celui qui en serait chargé.
Albergotti était ami intime de M. de Vendôme pour lui avoir sacrifié
dans les derniers temps M. de Luxembourg à qui il devait tout\,; il
l'était de M\textsuperscript{lle} Choin, par conséquent fort bien avec
Monseigneur et par là même considéré de Mgr le duc de Bourgogne. Il fit
donner cette commission à son neveu, qui était brigadier et qui
s'appelait Albergotti comme lui. Le luxe et la bonne chère avaient
corrompu nos armées, surtout en Flandre\,; des haltes froides n'y
étaient plus que pour des drilles\footnote{Vieux mot qui d'abord
  signifiait haillons, et qui fut employé par extension pour désigner
  les misérables et surtout les mauvais soldats.}\,; on y était servi
avec la même délicatesse et le même appareil que dans les villes et aux
meilleures tables. Les apprêts retardèrent, le détachement attendit
longtemps\,; il arriva sur Menin quatre heures plus tard que l'heure
concertée\,; les ennemis eurent le temps d'être avertis et de couvrir la
place. Albergotti n'eut d'autre parti à prendre que de revenir. Un autre
en aurait été perdu, mais avec de si bons appuis il n'y parut seulement
pas.

À peu de temps de là, son oncle voulut réparer cette faute\,; il partit
de l'armée avec un gros détachement pour aller surprendre Ath, où il
avait une intelligence. Il fit comme son neveu, il arriva trop tard, et
les gens qui y étaient déjà entrés furent obligés d'en sortir et de se
sauver au plus vite. L'extrême sang-froid d'Albergotti n'en fut pas
ému\,; il revint au camp et n'essuya aucuns reproches, ni de ceux qui là
commandaient, ni de la cour. Le gros des troupes et de Paris le ménagea
beaucoup moins. On volait ainsi le papillon de tous côtés. L'armée
subsistait tranquillement près de Tournai, tandis que M. de Vendôme
assiégeait Leffinghem, et promettait que, dès qu'il l'aurait pris, il ne
pourrait plus rien passer au prince Eugène, qui recevait en attendant
tous ses besoins par dés barques. Le chevalier de Croissy fut pris dans
une sortie et mené à Leffinghem. Il avait déjà été pris deux autres fois
de cette guerre. Les ennemis avaient trois mille hommes dans Leffinghem,
à ce que M. de Vendôme mandait au roi\,; il se trouvera bientôt qu'il
n'y en avait que la moitié\,; mais ces suppositions du double étaient
marché donné pour Vendôme. Le roi et le public s'étaient accoutumés à
lui en passer bien d'autres.

Avec toutes ses prouesses Lille succombait. Les ennemis y avaient fait
le 20 et le 21 trois brèches nouvelles, saigné le fossé et achevé une
galerie qui allait jusqu'au pied d'une des brèches. La place devenait
insultable\,; la poudre et les munitions manquaient, les vivres diminués
jusqu'à une extrême incommodité, et presque plus de viande. Tant
d'insurmontables nécessités résolurent enfin le maréchal de Boufflers,
de l'avis de toute sa brave garnison, de battre la chamade. Il ne lui
fut rien refusé de tout ce qu'il demanda. Les principaux articles furent
que les malades et blessés qui sont dans la ville pourront être
transportés dans nos places\,; que les mille huit cents chevaux entrés
avec le chevalier de Luxembourg seront conduits à Douai par le plus
court chemin, les privilèges des habitants conservés, et quatre jours
accordés à M. de Boufflers pour se retirer dans la citadelle avec tout
ce qu'il y voudra faire entrer en tout genre. Cette capitulation fut
signée le 23 octobre, après deux mois de tranchée ouverte, et avoir
combattu sans cesse à disputer le terrain jusqu'à un pouce.

Ce qu'il y eut de singulier en cette capitulation fut la liberté de
l'envoyer à Mgr le duc de Bourgogne pour être tenue, s'il l'approuvait,
sinon demeurer nulle et comme non avenue. Je dis exprès Mgr le duc de
Bourgogne. Boufflers avait expressément obtenu du roi, et en partant,
qu'il ne prendrait et ne recevrait jamais l'ordre, ni aucuns ordres du
duc de Vendôme, qu'il ne lui serait subordonné en aucun cas possible, et
qu'il ne reconnaîtrait que Mgr le duc de Bourgogne. Coetquen fut chargé
de la lui porter à son camp sous Tournai. Il le trouva jouant au volant,
et sachant déjà la triste nouvelle. La vérité est que la partie n'en fut
pas interrompue, et que, tandis qu'elle s'acheva, Coetquen alla voir qui
il lui plut. Cette réception fut étrangement blâmée, et scandalisa fort
l'armée avec raison, dont la cabale ennemie tira de nouvelles armes
contre le prince. Coetquen retourna vers lui avec l'approbation de la
capitulation, et chargé de louanges pour le maréchal et pour sa
garnison, mais avec point ou fort peu d'argent. Boufflers envoya au roi
Tournefort, entré avec le chevalier de Luxembourg, et lieutenant des
gardes du corps, rendre compte de sa défense, qui reçut de la cour, de
Paris, et de toute l'Europe, les plus grands applaudissements. Par sa
lettre, il pressa fort le roi de faire payer l'argent qu'il avait été
obligé d'emprunter des bourgeois pour les travaux et pour faire
subsister la garnison. Il comptait d'avoir six mille hommes y compris
quelques dragons dans la citadelle. Il offrit à tous les soldats qui y
étaient destinés de donner congé à ceux qui n'y voudraient pas entrer.
Pas un seul ne l'accepta. Comme il y entra le dernier pour achever de
donner quelques ordres, pendant quelques heures, elles parurent si
longues aux soldats que l'inquiétude leur en prit, et si fort qu'elle
alla jusqu'au murmure. Dès qu'il parut leur joie éclata en louanges les
plus flatteuses, et tous promirent de faire des merveilles sous un chef
qui leur en montrait si bien l'exemple et qui prenait tant de soin
d'eux. Ce fut donc le 26 octobre au soir qu'ils furent tous renfermés
dans la citadelle, qui était un vendredi.

Le jeudi, veille de ce jour, M. de Vendôme fit attaquer Leffinghem
l'épée à la main. Puyguyon avait là un camp qui l'assiégeait sous ses
ordres depuis trop de temps pour un poste comme celui-là, que les
ennemis avaient accommodé, et où ils avaient mis quinze cents hommes
avec un colonel anglais. Ils venaient de débarquer quatorze bataillons
sur les dunes près de Leffinghem pour le secourir. Forbin et le
chevalier de Langeron les en empêchèrent avec les troupes qu'ils avaient
à Nieuport, sur les vaisseaux et sur les galères, à qui ils firent
mettre pied à terre. La présence de ce secours imminent et la prise de
Lille excitèrent M. de Vendôme à emporter enfin ce poste. Il le fut en
effet, et si aisément qu'il n'en coûta pas une douzaine de soldats. On
leur en tua une centaine, et on eut tous les autres prisonniers, presque
tous Anglais. Le pauvre comte de La Mothe, qui était venu se promener au
camp de Puyguyon, se trouva à l'action. Vendôme, à son ordinaire, en fit
un trophée. Il envoya le chevalier de Roye en porter la nouvelle au roi,
qui, infatigablement le même pour Vendôme, le régala d'un brevet de
mestre de camp au chevalier de Roye pour la bonne nouvelle.

J'avais compté d'aller à la Ferté assez tôt après le retour de
Fontainebleau pour y profiter encore un peu de la belle saison.
Plusieurs amis considérables me voulurent arrêter par rapport aux
grandes attentes où on était sur la Flandre. J'étais pleinement
convaincu qu'il ne s'y passerait rien et que Lille ne servit point
secouru. D'ailleurs je commençais à me sentir à bout de l'audace et du
triomphe de la cabale ennemie de Mgr le duc de Bourgogne, et je ne
respirais que l'éloignement de la cour, lorsque le duc de Beauvilliers,
épuisé de raisons pour me retenir, s'avisa de me demander si je ne
voudrais pas au moins, pour l'amour de Mgr le duc de Bourgogne, faire
l'effort de demeurer encore quelques jours à la cour. Il désarma ainsi
mon impatience. Je lui promis de rester jusqu'à ce que lui-même me
rendît la liberté, mais je le priai de ne pas excéder le peu de forces
que je pouvais conserver parmi ces criminelles menées auxquelles on ne
pouvait rien opposer. Il me le promit, et de plus, de mander à Mgr le
duc de Bourgogne la violence que je me faisais en sa seule
considération. Ce délai ne me réussit pas et ne servit de rien à ceux
qui l'avaient désiré. J'étais odieux à toute cette cabale. Elle avait
emmuselé les plus convaincus de ses crimes. J'ose dire à peine que
j'étais peut-être le seul à qui il restât assez de courage pour le
conseil et pour ne pas tenir la vérité captive\,; qu'ils ne laissaient
pas de craindre le premier\,; que l'autre leur était d'autant plus
odieux qu'ils avaient tout subjugué. Non contents des clameurs qu'ils
firent retentir partout sur le pari dont j'ai parlé et dont ils firent
un si pernicieux usage, ils eurent recours à un autre artifice, de la
grossièreté duquel ils n'eurent pas honte, parce qu'ils l'avaient perdue
sur tout il y avait longtemps. Ils se mirent donc à semer que je tombais
sur Mgr le duc de Bourgogne plus rudement que personne. Le monde, témoin
de ma vivacité pour lui, et contre eux, en rit. Je méprisai aussi une
imposture si manifeste, mais à la fin elle réussit à mettre le comble à
mon dépit, et à mon impatience d'aller respirer chez moi un air plus
sain et plus tranquille, et M. de Beauvilliers me le permit. Reprenons
durant cet intervalle diverses choses que la suite des événements de
Flandre a fait laisser en arrière.

Tréville mourut à Paris dans le temps que les ennemis investirent Lille.
J'ai assez fait connaître ce personnage peu guerrier, fort du grand et
du meilleur monde, quelque temps courtisan, puis dévot et retiré, revenu
peu à peu dans un monde choisi, toujours recherché, toujours galant,
toujours brillant d'esprit et de goût, pour n'avoir plus à en rien dire.
Ses vrais amis l'avaient fait rentrer un peu en lui-même. Depuis
plusieurs années il vivait plus retiré et plus particulièrement occupé
de son salut. Il était fort à son aise et point marié. Son père, comme
je l'ai dit, était mort commandant une des deux compagnies des
mousquetaires.

Lyonne, fils aîné de ce grand ministre des affaires étrangères, mourut
bientôt après dans une obscurité aussi profonde que le lustre de son
père avait été éclatant. C'est très ordinairement le sort des enfants
des ministres. Mais de ce règne seulement, ils ont trouvé, avec tant
d'autres moyens de s'élever, celui de faire à leur famille des charges
de la maison du roi une planche après le naufrage. Ainsi la noblesse en
demeure exclue et le demeurera apparemment toujours\,; tellement
qu'excepté les grandes charges, toujours de ce règne, possédées par des
ducs et des maréchaux de France, on voit aujourd'hui les Cent-Suisses et
les deux charges de maître de la garde-robe, celles de grand maréchal
des logis et de capitaine de la porte aux enfants des ministres morts ou
congédiés. À l'égard de celles de premier écuyer et de premier maître
d'hôtel, je ne pense pas qu'on les trouve plus hautement possédées, non
plus que celle de grand maître des cérémonies encore du ministère. Reste
celle de grand prévôt demeurée à un gentilhomme\,; car pour les
bâtiments qui de mains viles avaient passé à un seigneur, ils sont
bientôt retombés à peu près d'où ils avaient été tirés. Lyonne, qui en
fut un des premiers exemples, eut la charge de maître de la garde-robe,
de Montglat, père de Cheverny, que le mauvais état de ses affaires lui
fit vendre. Une assiduité exacte d'une année entière, et de deux années
l'une, fut plus forte que Lyonne. Il servit peu sa première année,
encore moins sa seconde, après quoi il ne prit plus la peine de paraître
à la cour. La Salle, qui était l'autre {[}maître de la garde-robe{]},
servit continuellement pour tous deux, et c'est ce qui le rendit si
agréable au roi. Lyonne passa sa vie à Paris avec des nouvellistes. Il
avait son banc fixe aux Tuileries avec eux, dont pas un n'était connu de
personne. Il avait été riche, s'était brouillé avec sa femme, Lyonne
aussi et héritière, qu'il avait perdue, et ne vit jamais un homme qui
eût un nom ni un état. Il ne laissa qu'un fils très bien fait, brave,
bon officier, qui fit la folie d'épouser la servante d'un cabaret de
Phalsbourg, qui s'est trouvée une femme de vertu et de mérite. Il n'en a
point eu d'enfants. Il a voulu longtemps faire casser ce mariage, sans
avoir pu y réussir, et n'a presque point vécu avec sa femme. Il était un
des favoris de M. le Duc dans sa toute-puissance, pendant laquelle il
mourut assez brusquement, et fort regretté. Sa femme a toujours vécu
dans la piété et dans la retraite, où elle est encore aujourd'hui à
Paris.

Jarzé, nommé avec la surprise de tout le monde, comme je l'ai dit, à
l'ambassade de Suisse, s'en repentit. C'était un homme fort avare,
quoique sans enfants. Il était allé chez lui en Anjou. Il y fit une
grande chute qui l'incommoda d'autant plus qu'il n'avait qu'un bras. Il
manda qu'il était hors d'état de faire son ambassade. Elle fut donnée au
comte du Lue qui, comme Jarzé, avait perdu un bras, et tous deux à la
bataille de Cassel.

Le roi donna, à un chapitre extraordinaire tenu pour le duc d'Enghien,
permission de porter l'ordre au cardinal de La Trémoille, en attendant
qu'il fût reçu. Il avait été nommé à la Pentecôte.

Bientôt après, le maréchal de Noailles donna à toute la cour le
spectacle d'une mort qui put lui fournir de grandes réflexions. C'était
un homme d'une grosseur prodigieuse et entassé, qui, précisément comme
un cheval, mourut aussi de gras fondu. Aussi était-il grand mangeur, et
faisait chez lui grande et délicate chère, mais pour sa famille et pour
un très petit nombre d'autres gens. Né dans l'intérieur de la cour d'un
père et d'une mère en charge, et qui tenaient intimement au cardinal
Mazarin et à la reine mère, il en avait pris tout l'esprit et conformé
en tout le sien, tout pesant, grossier et moins que médiocre qu'il
était. Jamais homme plus renfermé, plus particulier, plus mystérieux, ni
plus profondément occupé de la cour\,; point d'homme si bas pour tous
les gens en place\,; point d'homme si haut, dès qu'il le pouvait, et
avec cela fort brutal. On l'a vu sans cesse, et en public, duc et
capitaine des gardes, porter comme un page la queue de
M\textsuperscript{me} de Montespan, tandis que celle de la reine ne
l'était, et ne l'est encore, que par l'exempt des gardes en service
auprès d'elle\,; et ce même homme, commandant en Languedoc, avait ses
gardes le long de son drap de pied à la messe, et ses aumôniers tournés
vers son prie-Dieu, avec la même pompe et toutes les mêmes cérémonies de
la messe du roi, et tout le reste de même. Le roi, qui était l'idole à
qui il offrait tout son encens, étant devenu dévot, le jeta dans la
dévotion la plus affichée. Il communiait tous les huit jours, et
quelquefois plus souvent. Les grandes messes, vêpres, le salut, il n'y
manquait que pour des temps de cour ou des moments de fortune. Avec tout
cela, il était fort accusé de n'avoir pas renoncé à la grisette, et d'en
faire des parties secrètes avec Rouillé du Coudray, son ami intime, et
grand et très public débauché, à la fortune duquel il contribua fort, et
son fils encore plus dans la régence de M. le duc d'Orléans.

Louville m'en a conté une aventure que je ne certifie pas, mais qu'il
m'a assurée, et, quoique sujet quelquefois à se frapper et à s'engouer,
il était homme fort vrai. L'histoire est telle\,: M. de Noailles était
amoureux d'une fille de la musique du roi, fort jolie\,; et cet amour
qui fit du bruit, j'en ai fort ouï parler dans le temps. Il était en
quartier, et alors il logeait dans l'appartement de quartier sous le
cabinet du roi. M. de Noailles et la fille convinrent de leurs faits\,;
elle vint passer la nuit chez lui. Malheureusement le cardinal de
Noailles arriva trop matin, et à son ordinaire alla descendre chez son
frère. Les valets lui dirent qu'il n'était pas éveillé\,; cela ne
l'arrêta point, il se fait ouvrir et entre. On peut juger de ce que put
devenir le couple fortuné. La fille se fourre la tête dans le lit, et le
chevet par-dessus. Le maréchal s'écrie dolemment qu'il a une migraine à
mourir, qu'il ne peut ni parler, ni entendre parler, qu'il ne sait s'il
pourra se lever pour aller chez le roi, et qu'il veut se reposer en
attendant. Le bon cardinal prend cela pour argent comptant, plaint son
frère, lui conseille de se donner la matinée, et sort pour le laisser en
repos. Voilà les amants bien soulagés. La fille, qui étouffait de
l'issue de l'aventure, et de ce qu'elle s'était mise sus, n'eut rien de
plus pressé que de sortir de sa cache, de prendre ses cottes et de
s'enfuir. Le maréchal voulait tuer le valet confident. Il continua de
faire le malade, mais il fallut pourtant aller chez le roi, où il fit
accroire à. son frère qu'il faisait un grand effort. On prit grand soin
d'étouffer l'aventure\,; mais tout se sait à la fin. Il faisait sa cour
jusqu'aux basses maîtresses de Monseigneur. Ce prince aima quelque peu
de temps la Raisin, qui était fort belle et comédienne excellente. Elle
se trouva un peu incommodée à Fontainebleau. M. de Noailles y envoyait
sans cesse savoir de ses nouvelles, lui faisait toutes sortes de
présents, et l'allait voir avec les plus grands respects du monde. Avec
tout cela, ce n'était ni un méchant homme ni un malhonnête homme\,; et
quoique très avare de crédit, il n'a pas laissé de faire des plaisirs et
de rendre des services. Il plaisait au roi par son extrême servitude et
par un esprit fort au-dessous du sien, à M\textsuperscript{me} de
Maintenon aussi, au contraire de sa femme qu'ils n'aimaient point, et
dont ils craignaient l'esprit, les menées, la hardiesse.

C'était elle qui gouvernait mari, enfants, famille, affaires\,; manège
de cour, avec une gaieté, une liberté d'esprit, comme si elle n'eût
jamais rien eu à faire, et qui, à force d'esprit et d'adresse, sans
s'étonner ni se rebuter de rien, fit toujours du roi et de
M\textsuperscript{me} de Maintenon tout ce qu'elle voulut, pareillement
de M\textsuperscript{me} la duchesse de Bourgogne, et gouverna à son gré
toutes les princesses, tous les ministres et tous les gens en place, et
tout cela sans bassesses\,; une femme noble, magnifique, libérale,
pleine d'entrailles pour ses enfants, pour sa famille, pour son nom,
extrêmement capable d'amitié, qui eût toujours des amis en nombre, et
qui en mérita encore davantage\,; une femme qui ne disait pas tout ce
qu'elle pensait, mais jamais ce qu'elle ne pensait pas\,; naturellement
bonne, douce, sans humeur, franche autant que la cour le peut permettre
avec prudence, à qui aussi il ne fallait pas marcher sur le pied, qui
disait alors à qui que ce pût être son fait, mais qui n'était point
haineuse. Elle vit encore pleine de sens, d'esprit et de santé à
quatre-vingt-sept ans, en patriarche de sa nombreuse famille, fort riche
et fort donnante, dévote tant qu'elle peut, toujours allante, et faisant
les délices de ses amis dont elle a encore beaucoup, et conserve ce
badinage avec lequel elle a toujours réussi aux choses même les plus
sérieuses.

M. de Noailles ne se consola point d'avoir donné sa charge à son fils.
Ce vide lui fut insupportable, quoique toujours à la cour et dans la
même considération. Dans les premiers temps les gardes continuèrent à
prendre les armes pour lui dans leurs salles. Le roi le sut et le trouva
mauvais, ils ne les prirent plus. Cela fut insupportable au maréchal à
tel point qu'il cessa d'y passer, et qu'il fit toujours depuis le tour
par les cours pour aller chez sa fille de Guiche, et partout où il avait
affaire. Sa maladie fut très brusque et courte. Il mourut le 2 octobre,
sur les cinq heures du soir, dans son fauteuil, au milieu de sa famille
et de toute la cour qu'il avait tant aimée, en présence de
M\textsuperscript{me} la duchesse de Bourgogne, à qui tous spectacles
étaient bons, et des trois filles du roi qui accoururent et le virent
passer. Le cardinal son frère eut la douleur que le Saint-Sacrement fut
longtemps dans l'appartement du malade, qui mourut sans avoir pu le
recevoir. Le deuil fut nombreux, l'affliction peu étendue\,; la
maréchale de Noailles a eu le bon esprit de n'avoir presque pas remis le
pied à la cour depuis, et encore des moments de devoir, et jamais depuis
la mort du roi. Le duc de Noailles, qui commandait en Roussillon, où il
n'y avait rien à faire, revint à la cour fort tôt après.

Saint-Mars, gouverneur de la Bastille, mourut en même temps fort vieux.
Bernaville, lieutenant du roi sous lui, lui succéda dans cet emploi de
première confiance.

La maréchale de Villeroy mourut le 20 octobre, à Paris, d'une maladie
fort courte, et qui n'avait point paru dangereuse. Elle était soeur du
duc de Brissac, mari de la mienne. Leur mère était soeur du duc de Retz,
père de l'héritière qui épousa le duc de Lesdiguières, duquel l'autre
maréchale de Villeroy était tante paternelle, en sorte que par la mort
du duc de Lesdiguières, gendre de M. de Duras, les Villeroy ont eu les
deux immenses successions de Lesdiguières et de Retz. La maréchale de
Villeroy était sans cela fort riche par la prédilection entière de sa
mère. Le maréchal de Villeroy et elle, dans les commencements, n'avaient
pas toujours été fort contents l'un de l'autre. Le vieux maréchal, plus
sage que son fils, et qui avait éprouvé le même sort avec sa femme, les
empêcha de se brouiller. Il y eut toujours entre eux plus de
considération réciproque que de tendresse. La maréchale était
extrêmement petite, la gorge nulle, d'ailleurs d'une grossesse tellement
démesurée, qu'à peine pouvait-elle se remuer. Ses bras étaient plus gros
qu'une cuisse ordinaire, avec un petit poignet et une petite main
mignonne au bout, la plus jolie du monde. Le visage exactement comme un
gros perroquet, et deux gros yeux sortants qui ne voyaient goutte. Elle
marchait aussi tout comme un perroquet. Avec une figure si peu
imposante, jamais femme n'imposa tant. Avec une grande hauteur, elle
avait une grande politesse, noble, discernée, qui est devenue si rare et
qui touche si fort. Personne aussi n'avait plus d'esprit, ni plus de
sens et de justesse, avec un tour unique et très salé et plaisant, quand
elle voulait, mais toujours avec dignité. Elle était d'un excellent
conseil, et la meilleure et la plus sûre amie du monde, et, avec toute
sa gloire, d'un commerce le plus aisé et le plus délicieux. Tout le
monde ne lui convenait pas, mais un choix délicat.

C'était la personne du monde qui se respectait le plus et qui se faisait
le plus naturellement respecter par les autres. Le roi et
M\textsuperscript{me} de Maintenon la craignaient, et jamais elle ne fit
un pas pour s'en approcher, quoique passant sa vie à Versailles, où elle
avait toujours chez elle une cour, indépendamment de son mari, et en ses
absences. Elle souffrait du ridicule de ses grands airs. Souvent il
ôtait en particulier sa perruque chez elle\,; elle ne disait mot, mais
elle ne s'y accoutumait point. Elle eut le bon sens de n'être rien moins
qu'éblouie de l'envoi de son mari en Italie\,; elle en craignit les
revers et m'en parla franchement, quoiqu'elle me reprochât quelquefois,
comme en badinant, que je n'aimais point le maréchal. À sa prison elle
fut outrée de douleur. Je la vis dès les premiers jours, que sa porte
était fermée, excepté à ses plus intimes amis. Son bon esprit ne put
être consolé par toutes les marques de bonté que le roi prodigua au
maréchal, et par tout ce qu'il lui manda à elle. À son retour elle fut
vivement touchée de son inflexibilité à rejeter le salutaire conseil du
chevalier de Lorraine, que j'ai expliqué en son temps. Mais elle fut
abîmée de douleur à la bataille de Ramillies et de tout ce qui la
suivit. Il y avait déjà longtemps qu'elle était fort dans la piété, qui
augmenta toujours depuis. Elle tomba entre des mains qui en abusèrent.
Le P. Poulinier, qui a été abbé de Sainte-Geneviève, était un saint,
mais de ces saints grossiers et durs, et sans aucune connaissance du
monde. C'était la femme du monde la plus sensible et d'une conversation
qu'on ne pouvait quitter. Il la condamna au silence le plus exact sur le
malheur de son mari, et sur Chamillart qu'elle accusait de les avoir
fort aggravés, et elle y fut si fidèle que non seulement il ne lui en
échappa jamais rien, mais si quelque ami particulier se licenciait un
peu là-dessus devant elle, elle changeait aussitôt de discours, et s'il
y revenait, elle le faisait agréablement taire\,; elle était occupée en
des réparations continuelles.

Elle avait la folie des Cossé sur leur naissance, et l'avait fait
souvent sentir à ses enfants, et quelquefois à son mari. Depuis elle me
disait quelquefois en riant, mais tête à tête, que les Villeroy
n'étaient pas si mauvais que je le pensais, et je riais aussi. L'époque
de Ramillies fut celle de sa retraite qu'elle fit insensiblement, et
bientôt après elle se retira entièrement de tout. Cette femme,
accoutumée à la plus excellente compagnie, qui ne pouvait se remuer ni
lire, se mit à passer sept ou huit mois à Villeroy toute seule, et à
Paris à fermer sa porte à tout le monde. Ses meilleurs amis n'y étaient
reçus que mandés, et peu souvent. Sa charmante conversation, à force de
se retrancher tout, était devenue pesante\,; elle exigeait {[}ces
retranchements{]} des autres avec tant de rigueur qu'on ne savait de
quoi l'entretenir. Sa vue l'empêchait de travailler\,; le jeu, qu'elle
avait fort aimé, elle se l'était retranché depuis longtemps sous ce
prétexte de sa vue. Ainsi sa vie se passait dans son fauteuil en prière,
et en lectures de piété que lui faisaient ses domestiques. Je lui disais
souvent qu'elle se ferait mourir\,; elle glissait et badinait là-dessus,
et avec son agrément ordinaire me jetait quelques mots fort à propos de
morale et de pénitence. Je ne lui dis que trop vrai. Une vie si opposée
à celle qu'elle avait toujours, menée et si contraire à la nature, à
laquelle rien n'était accordé, la tua en deux ou trois ans. Son P.
Poulinier, qui ne la voulut jamais croire mal, ne prit pas la peine de
la voir en sa dernière maladie\,; elle reçut tous ses sacrements sans
lui. Peu avant de mourir elle me demanda\,; elle oublia que j'étais à la
Ferté\,; j'eus une douleur extrême de sa perte et de m'être trouvé
absent. Sa mort fut celle des justes, et avec toute sa connaissance et
les plus grands sentiments. Ses amis, en très grand nombre, en furent
amèrement touchés\,; elle n'avait que soixante ans.

La comtesse de Beuvron ne tarda pas à la suivre. Son nom était
Rochefort, d'une bonne noblesse de Guyenne, et on voyait bien encore
qu'elle avait été belle, à soixante-dix ans qu'elle mourut. Elle avait
été fille de la reine\,; on l'appelait M\textsuperscript{lle} de
Théobon. Le comte de Beuvron l'épousa, celui dont j'ai parlé à
l'occasion de la mort de la première femme de Monsieur, dont le
chevalier, depuis comte de Beuvron, était capitaine des gardes. Elle
était veuve depuis longtemps, et sans enfants, avec fort peu de bien.
C'était une femme de beaucoup d'esprit et de monde, de fort bonne
compagnie, pour qui Madame prit la plus grande et la plus constante
amitié. Elle lui écrivait tous les jours sans y jamais manquer,
lorsqu'elle n'était pas auprès d'elle. Les intrigues du Palais-Royal
l'avaient éloignée plusieurs années de Madame, comme je l'ai raconté à
l'occasion de ce qu'elle la prit auprès d'elle, avec la maréchale de
Clérembault, à la mort de Monsieur qui lui avait défendu de les voir. La
comtesse de. Beuvron était toujours demeurée dans la plus grande union
avec la famille de son mari, et était comptée dans le monde. Elle était
extrêmement de mes amies. Elle en avait, et en méritait, qui la
regrettèrent fort. D'ailleurs c'était une femme qui avait bec et ongles,
très éloignée d'aucune bassesse, assez informée, mais qui aimait fort le
jeu.

Fort tôt après mourut le comte de Marsan, frère cadet de M. le Grand et
du feu chevalier de Lorraine, qui n'avait ni leur dignité, ni leur
maintien, ni rien de l'esprit du chevalier, qui, non plus que le grand
écuyer, n'en faisait aucun cas. C'était un extrêmement petit homme,
trapu, qui n'avait que de la valeur, du monde, beaucoup de politesse et
du jargon de femmes, aux dépens desquelles il vécut tant qu'il put. Ce
qu'il tira de la maréchale d'Aumont est incroyable. Elle voulut
l'épouser et lui donner tout son bien en le dénaturant. Son fils la fit
mettre dans un couvent, par ordre du roi, et bien garder. De rage, elle
enterra beaucoup d'argent qu'elle avait en lieu où elle dit qu'on ne le
trouverait pas, et, en effet, quelques recherches que le duc d'Aumont
ait pu faire, il ne l'a jamais pu trouver. M. de Marsan était l'homme de
la cour le plus bassement prostitué à la faveur et aux places,
ministres, maîtresses, valets, et le plus lâchement avide à tirer de
l'argent à toutes mains. Il avait eu tout le bien de la marquise
d'Albret, héritière, qui le lui avait donné en l'épousant, et avec
laquelle il avait fort mal vécu. Il en tira aussi beaucoup de
M\textsuperscript{me} de Seignelay, soeur des Matignon, qu'il épousa
ensuite\,; et quoique deux fois veuf, et de deux veuves, il conserva
toujours une pension de dix mille francs sur Cahors, que l'évêque La
Luzerne lui disputa, et que M. de Marsan gagna contre lui au grand
conseil. Il tira infiniment des gens d'affaires, et tant qu'il put des
contrôleurs généraux. Ce riche Thévenin, dont j'ai parlé à l'occasion du
legs qu'il fit au chancelier de Pontchartrain, qu'il refusa, Marsan le
servit dans sa maladie, qui fut longue, comme un de ses valets, et fut
la dupe de cette infamie qui ne lui valut rien. {[}A l'égard de{]}
Bourvalais, autre fameux financier, auprès duquel il fut plus heureux,
il disait qu'il était le soutien de l'État, dont quelqu'un impatienté
lui répondit qu'il l'était en effet, comme la corde l'est des pendus.
Lui surtout et Matignon, son beau-frère, tirèrent des trésors des
affaires qui se firent du temps de Chamillart, à tous les environs
duquel il faisait une cour rampante. M. le Grand, qui en était blessé,
l'appelait le chevalier de La Proustière, et disait qu'il avait pris le
perruquier de l'abbé de La Proustière pour lui faire mieux sa cour.
C'était un très bon homme, assez imbécile, cousin germain de Chamillart
et de sa femme, qui gouvernait toute la dépense et le domestique de leur
maison, honnête homme et désintéressé, mais fort incapable.

Jamais fadeur ne fut pareille à celle de M. de Marsan, avec toutes ses
manières d'un vieux galant auprès des dames, et ses bassesses avec les
gens qu'il ménageait. Il n'avait pas honte d'appeler
M\textsuperscript{me} de La Feuillade \emph{ma grosse toute belle}, qui
était une très bonne femme, mais beaucoup plus maritorne que celle de
don Quichotte. Elle-même en était embarrassée, et la compagnie en riait.
Enfin un homme si bas et si avide, qui toute sa vie avait vécu des
dépouilles de l'Église, des femmes, de la veuve et de l'orphelin,
surtout du sang du peuple, mourut enragé de malefaim par une paralysie
sur le gosier, qui, lui laissant la tête dans toute sa liberté et toutes
les parties du corps parfaitement saines, l'empêcha d'avaler. Il fut
plus de deux mois dans ce tourment, jusqu'à ce qu'enfin une seule goutte
d'eau ne put plus passer sans que cela l'empêchât de parler. Il faisait
manger devant lui ses gens, et sentait tout ce qu'on leur donnait avec
une faim désespérée, et mourut en cet état, qui frappa tout le monde si
fort instruit des, rapines dont il avait toute sa vie vécu. Il avait
vingt mille livres de pension du roi, qui en donna douze mille aux deux
fils qu'il laissa de sa seconde femme, huit mille à l'aîné, quatre mille
au second. Il n'en avait point eu de la première. Il avait soixante-deux
ans.

\hypertarget{chapitre-xxii.}{%
\chapter{CHAPITRE XXII.}\label{chapitre-xxii.}}

1708

~

{\textsc{Victoires du roi de Suède sur les Moscovites.}} {\textsc{-
Lewenhaupt défait par le czar.}} {\textsc{- Divers succès des
mécontents, qui perdent les montagnes de Hongrie.}} {\textsc{- Estaing
défait les miquelets en Catalogne.}} {\textsc{- Succès en Espagne qui
terminent la campagne.}} {\textsc{- Retour du maréchal de Villars à la
cour.}} {\textsc{- Le pape sans secours, fort malmené par les troupes
impériales, est forcé à recevoir à Rome Prié, plénipotentiaire de
l'empereur.}} {\textsc{- Intrigue de chapeaux à Rome.}} {\textsc{-
L'abbé de Polignac obtient la nomination du roi d'Angleterre.}}
{\textsc{- Démêlé de Fériol, ambassadeur de France à Constantinople.}}
{\textsc{- Mort, naissance et caractère du comte de Fiesque.}}
{\textsc{- Mort, naissance et caractère de Bréauté.}} {\textsc{- Mort et
caractère de l'abbé de La Rochefoucauld.}} {\textsc{- Mort de l'abbé de
Châteauneuf.}} {\textsc{- Mort et abrégé de la comtesse de Soissons.}}
{\textsc{- Époque et suite de la charge de surintendante.}} {\textsc{-
Mort d'Overkerke, général en chef des Hollandais.}} {\textsc{- Desmarets
fait ministre d'État\,; marie sa fille au marquis de Béthune-Orval.}}
{\textsc{- Mariage d'Armentières avec la fille de M\textsuperscript{me}
de Jussac.}} {\textsc{- Fortune de lui et de ses frères.}} {\textsc{-
Retour de M. le duc d'Orléans à la cour.}} {\textsc{- Mariage de
Tonnerre avec la fille de Blansac.}} {\textsc{- Je suis averti à la
Ferté, par l'évêque de Chartres, qu'on m'a mis fort mal auprès du roi.}}
{\textsc{- Je retourne bientôt après à la cour.}}

~

Le roi de Suède eut divers événements avec les Moscovites. Il les battit
dans la fin d'août, leur tua beaucoup de monde et trois de leurs
généraux, passa le Borysthène, se proposant toujours de percer jusqu'à
Moscou et de détrôner le czar, qui deux mois après eut sa revanche sur
le général Lewenhaupt, qu'il défit entièrement, allant joindre le roi de
Suède avec un fort gros corps des recrues, de l'argent et force
provisions de guerre et de bouche, dont ce prince commençait fort à
manquer dans des pays assez déserts que les Moscovites avaient eux-mêmes
dévastés pour lui, ôter toute subsistance. À son tour, le roi de Suède
gagna une autre bataille, força les retranchements que les Moscovites
avaient faits devant eux, en tua beaucoup et en prit quantité, et
s'ouvrit ainsi le passage pour continuer sa route vers Moscou, succès
qui lui devint funeste.

Ragotzi se soutint en Hongrie. Son parti se maintint dans la haine de la
cour de Vienne, quoique quelques-uns de ses généraux se fussent
accommodés avec elle, et les mécontents battirent un fort gros corps des
troupes impériales. Néanmoins ils perdirent bientôt après toutes leurs
places des montagnes.

En Catalogne, d'Estaing battit, tua, prit et dissipa un grand nombre de
miquelets et quelques troupes réglées qui étaient avec eux, ce qui donna
un grand pays de subsistance. Asfeld emporta la ville de Denia et son
château, avec mille Portugais ou Anglais prisonniers de guerre, et prit
ensuite celle d'Alicante, dont il bloqua aussi le château. Cela termina
la campagne en Espagne, et M. le duc d'Orléans s'en alla à Madrid pour
les ordres nécessaires et les mesures à prendre pendant l'hiver et pour
la campagne suivante. Le comte de Staremberg, qui commandait l'armée de
l'archiduc, essaya, après la séparation de l'armée, une entreprise sur
Tortose qui fut bien près de réussir. Le détachement qu'il y envoya
s'était saisi d'un ouvrage et d'un faubourg que cet ouvrage couvrait. Le
gouverneur, qui était Espagnol, enferma d'abord dans une église les
bourgeois qui lui étaient suspects, attaqua les ennemis, reprit
vaillamment le faubourg et l'ouvrage, et les chassa entièrement. Ce fut
grand dommage qu'il y fut tué.

La campagne était finie en Savoie, où nous perdîmes quelques places,
comme je l'ai rapporté. Le maréchal de Villars y aurait fait une plus
triste campagne encore sans les progrès du pape sur cette poignée
d'Impériaux laissée en Italie, dont tout lé corps était à l'armée du duc
de Savoie, et qui le voulut quitter pour aller imposer au pape. Tôt
après, les armées du roi et de Savoie entrèrent en quartier d'hiver, et
le maréchal de Villars arriva à la cour avec les airs avantageux qui ne
le quittaient jamais, et qui lui réussirent toujours auprès du roi, qui
fut le seul qui crut qu'il avait fait une belle campagne.

Il parut divers manifestes de l'empereur qui fit arrêter le nonce à
Vienne, le relégua ensuite tellement, qu'il fut rappelé. Tant qu'il ne
fut question que de paroles et de cette poignée d'Impériaux en Italie,
le pape se conduisit fort vigoureusement\,; mais, après la séparation
des armées en Savoie, et quand toutes les troupes qu'y avait l'empereur
furent entrées dans l'État ecclésiastique, le pape eut lieu de se
repentir de s'être trop hâté, et {[}d'avoir{]} trop compté sur une ligue
aussi lentement tissue et aussi mal exécutée que le fut celle qui avait
enfin été résolue, et la réclama en vain. Il demanda Feuquières pour
commander les troupes de cette ligue, qui lui fut accordé, mais ce fut
tout. Il souffrit tant d'insolences du cardinal Grimani, vice-roi de
Naples par intérim, qu'il l'eût privé de la pourpre, comme il l'en
menaça plus d'une fois, si les plus sages cardinaux en avaient été crus.
Les. Impériaux cependant vivaient à discrétion dans l'État
ecclésiastique. Les troupes du pape, destituées d'alliés, n'osaient se
présenter nulle part devant eux. Cette oppression força le pape à
recevoir enfin dans Rome le marquis de Prié en qualité de
`plénipotentiaire de l'empereur\,; au grand regret du maréchal de Tessé,
à qui des raisons de cérémonial avaient fait prendre le caractère
d'ambassadeur extraordinaire. Il les faut maintenant laisser dans ces
embarras, dont on ne verra la fin que dans les commencements de l'année
prochaine.

Il s'était passé depuis six ou sept mois une intrigue à Rome dont en ce
temps-ci l'abbé de Polignac sut profiter. La mort de l'évêque de Munster
avait mis sur les rangs pour lui succéder l'évêque d'Osnabrück et
d'Olmütz, frère du duc de Lorraine, et le baron de Metternich aussi
ardemment soutenu par les Hollandais, qui craignaient un prince appuyé
et dangereux dans leur voisinage, que le prince de Lorraine l'était par
l'empereur, dont l'amitié et l'intérêt étaient également pour ce prince.
Metternich, très canoniquement élu, craignit les voies de fait, et porta
l'affaire à Rome, qui, après un examen d'autant plus exact que le pape
craignait d'irriter l'empereur, ne laissa pas de décider en faveur de
Metternich. L'empereur se fâcha, menaça et obtint un examen nouveau,
contre toutes les règles et tout exemple. Ce coup d'autorité ne lui
réussit pas mieux\,; Metternich gagna une seconde fois sa cause. Après
ce double succès, les Hollandais menacèrent à leur tour, malgré les
liens de la ligue commune contre la France, et finalement l'empereur
céda, et Metternich prit possession.

Vienne, piquée d'avoir succombé, en voulut tirer une réparation tout à
fait en la disposition du pape, et lui demanda un chapeau pour le prince
de Lorraine. Le pape, qui en était avare, et qui craignait d'accoutumer
l'empereur à prescrire, différa tant qu'il put, et l'habile abbé de
Polignac saisit la conjoncture pour se faire d'un asile peu honorable,
et d'une planche après tant de naufrages, une route pour arriver à la
pourpre, que nous lui avons vu manquer une fois par la préférence du roi
pour l'archevêque de Bourges, pour la nomination de Pologne, comme je
l'ai raconté en son temps. J'ai dit qu'il était fort connu du pape dès
son premier voyage à Rome, et lié d'amitié avec lui parle commerce des
belles-lettres, desquelles ce pape s'était toujours piqué. On peut juger
que l'insinuant et ambitieux abbé, depuis son retour à Rome, n'avait
rien laissé à faire pour s'avancer de plus en plus en ses bonnes grâces.
Il y avait si bien réussi que Sa Sainteté ne cherchait qu'un prétexte de
le promouvoir, et de rougir, ainsi notre rote, qui, à l'exception de la
plus que singulière fortune du cardinal de La Trémoille, ne l'avait pas
été depuis Henri IV, en la personne de M. Sérafin, bâtard inconnu du
chancelier Olivier, et si estimé du cardinal d'Ossat.

Le pape désirait fort, sur l'exemple de La Trémoille, faire passer
Polignac aux deux couronnes ensemble, pour compensation du prince de
Lorraine. Mais la dextérité de l'abbé, ni le crédit de ses amis, ne
purent faire goûter cet expédient au roi\,; et l'empereur, enflé des
prospérités de sa grande alliance, déclara nettement que, si le pape
faisait un sujet pour les deux couronnes avec le prince de Lorraine, il
prétendait avoir en même temps un autre chapeau au nom de l'archiduc,
comme roi d'Espagne. Cette prétention était absurde. L'archiduc n'était
point roi d'Espagne, à Rome moins que partout ailleurs, où Philippe V
était seul reconnu, avait reçu un légat à Naples, tenait actuellement un
ambassadeur à Rome, qui était le duc d'Uzeda, et avait un nonce à
Madrid. L'empereur d'ailleurs ne pouvait contester au roi un droit égal
au sien, et il n'avait pas le moindre prétexte de plainte que l'abbé de
Polignac passât pour la France avec le prince de Lorraine pour lui,
c'était le roi d'Espagne seul qui en aurait été lésé. À cette
difficulté, il s'en joignit une autre dans notre cour.

M\textsuperscript{me} de Soubise, qui, pour être depuis longtemps
mourante et alors fort près de sa fin, n'en était pas moins attentive à
l'élévation des siens et à l'établissement de ses enfants, fut bientôt
informée de ce qui se passait là-dessus. Elle sentit combien une
promotion de traverse éloignerait celle des couronnes. Elle écrivit donc
au roi, et lui demanda d'insister à ce que le prince de Lorraine passât
comme couronne pour l'empereur. Le roi n'eut garde de lui refuser cette
complaisance, mais elle ne fit qu'augmenter la difficulté. L'empereur,
qui sentait ses forces et qui voulait engager à une reconnaissance
indirecte de son frère, comme roi d'Espagne, déclara que dans une
promotion, même pour les couronnes, il prétendait un chapeau sur le
compte particulier de l'archiduc. Cette fermeté éloigna encore plus la
promotion des couronnes, sans débarrasser le pape de la prétention de
l'empereur pour le prince de Lorraine. Là-dessus M\textsuperscript{me}
de Soubise demanda au roi de faire passer son fils avec le prince de
Lorraine, en reprenant sa nomination comme de couronne, qui alors
pourrait servir à l'abbé de Polignac. Mais la difficulté d'un chapeau
pour l'archiduc demeura en l'un et l'autre cas `si entière, qu'elle
devint obstacle à toute promotion. L'empereur s'en irrita, il n'en
sentit pas moins la faiblesse du pape, qui n'avait pas eu le courage de
rejeter avec hauteur une si étrange proposition. Mais cependant l'abbé
de Polignac prit un autre tour. Il avait toujours fort ménagé la cour de
Saint-Germain en France et à Rome\,; il se tourna vers elle pour avoir
sa nomination. Cette marque de royauté était comme la seule qui restât
au malheureux roi d'Angleterre, et Rome n'en pouvait pas faire de
difficulté à un prince qui perdait tout pour la religion, qui n'avait
d'asile que Rome, et qui y était traité en roi. Avec toutes ces raisons,
ce prince crut en avoir de bonnes d'introduire l'exercice de son droit
par un sujet agréable au pape et protégé par la France. Torcy, qui, dans
l'affaire de la nomination de Pologne, n'avait pas voulu décider entre
ses deux amis, et avait remis le choix au roi, sans porter l'un plus que
l'autre, fut ravi d'une occasion de revenir sur l'abbé de Polignac, et
le servit de toutes ses forces. Il obtint donc en ce temps-ci la
nomination du roi d'Angleterre pour la promotion des couronnes, et le
pape, qui ne demandait qu'un prétexte de le faire cardinal, l'agréa avec
plaisir.

Fériol, ambassadeur du roi à Constantinople, s'y brouilla fort sur la
fin de cette année. Le grand vizir, mécontent du ministre de Hollande,
lui fit plusieurs menaces suivies de mauvais traitements faits à ses
domestiques, qui lui firent craindre de n'être pas en sûreté chez lui,
dans un pays où tant d'expériences ont appris même aux ambassadeurs des
premières têtes couronnées que leur caractère et le droit des gens est
peu respecté. Ce ministre de Hollande voulut se réfugier chez
l'ambassadeur d'Angleterre. Sa surprise fut grande du refus absolu qu'il
fit de le recevoir, malgré l'union si étroite des deux nations, et si
conjointement alliées dans la guerre contre la France. Le Hollandais, ne
sachant que devenir, espéra trouver plus de générosité dans l'ennemi que
dans l'allié. Il s'adressa à Fériol, qui le reçut chez lui et prit sa
protection, en quoi il mérita louange et approbation, mais avec une
hauteur sur les plaintes du grand vizir qu'il aurait dû éviter, et qui
lui attira beaucoup de dégoûts dont il se tira avec la même hauteur. Il
arriva en ce temps-ci un aga pour s'en plaindre de la part de la Porte.
Le fait et le contraste m'ont paru d'une singularité à mériter de n'être
pas oubliés.

Je devais avoir parlé de la mort du comte de Fiesque avant celle du
maréchal de Noailles, qui la suivit de peu de jours. Ce comte était
d'une branche aînée de cette illustre maison, qui a donné des papes, des
souveraines, et une foule de cardinaux, de prélats et de personnes
considérables, l'une des quatre premières de Gênes. Après le malheur de
celui qui périt en tombant dans la mer, au moment de sa conjuration si
secrètement concertée pour le faire souverain de sa république, toute sa
maison fut proscrite. Une branche aînée vint s'établir en France, dont
celui-ci fut le dernier. Scipion, comte de Fiesque, son bisaïeul, fut
chevalier d'honneur d'Élisabeth d'Autriche, femme de Charles IX, et de
Louise de Lorraine, épouse d'Henri III, qui le fit chevalier du
Saint-Esprit le dernier jour de 1578. Il n'abandonna point la reine
Louise dans sa retraite, et mourut à soixante-dix ans à Moulins, en
1598. Alphonsine Strozzi, sa femme, fut dame d'honneur de la reine. Leur
fils unique fut tué jeune au siège de Montauban, à la tête de son
régiment. Sa veuve, qui était Le Veneur, fille et petite-fille des cieux
comtes de Tillières, chevaliers du Saint-Esprit, fut dame d'atours de la
seconde femme de Gaston, et gouvernante de Mademoiselle. Elle eut une
fille, mère de Bréauté, dont je parlerai tout à l'heure, et trois fils.
L'un demeura abbé, un autre chevalier de Malte, tué devant Mardick en
1646, et l'aîné, qui épousa la tante paternelle de la duchesse d'Arpajon
et du marquis de Beuvron, père du maréchal duc d'Harcourt, qui fut mère
du comte de Fiesque, de la mort duquel je parle. Elle était veuve, sans
enfants, de Louis de Brouilly, marquis de Piennes, de laquelle j'ai
suffisamment parlé (t. II, p.~321). Elle n'eut qu'une fille, mère de
Guerchy, fait chevalier de l'ordre en 1639, et le comte de Fiesque dont
il s'agit ici.

C'était un homme de fort bonne compagnie, d'esprit et orné, un fort
honnête homme qui avait été galant, avec une belle voix, qui chantait
bien, et qui faisait rarement des vers, mais aisément, jolis, et d'un
tour fort naturel. Il fit une chanson sur Bechameil et son entrée en sa
terre de Nointel si plaisante, si ridicule, si fort dans le caractère de
Bechameil, qu'on s'en est toujours souvenu. Le roi, qui le sut, la lui
fit chanter un jour à une chasse, et en pensa mourir de rire. Il était
singulier, brusque, particulier, avait peu servi, et fait quelques
campagnes aide de camp du roi, qui, bien aise de l'obliger sans qu'il
lui en coûtât rien, et aux dépens des Génois qu'il voulait mortifier,
lui fit payer cent mille écus par eux, pour de vieilles prétentions,
lorsque le doge de Gènes vint en France. Ce fut M. de Seignelay, son
ami, qui les lui valut, sans que lui-même y eût pensé. C'était un homme
né fort libre, ennemi de toutes sortes de contraintes, et qui fit
toujours peu de cas du bien et de la fortune. Il fut toujours considéré
et recherché par la meilleure compagnie. On a vu en son lieu son étrange
aventure avec M. le Duc, qui tacha de la réparer depuis, et qui le
servit dans cette dernière maladie comme un de ses domestiques. On a vu
aussi son intime liaison avec M. de Noirmoutiers, à qui il donna le peu
qu'il avait par son testament. Il n'avait jamais été marié, et n'avait
que soixante et un anis. Sa soeur est morte depuis fort peu d'années,
abbesse de Notre-Dame de Soissons pendant près de cinquante ans, et une
très digne et bonne abbesse. Le comte de Fiesque avait beaucoup d'amis
considérables dont il fut fort regretté.

Bréauté, son cousin germain, le suivit deux mois après. C'était un fort
gros et grand homme, petit-neveu paternel du Bréauté célèbre par son
duel, ou plutôt son combat de vingt-deux Français contre vingt-deux
Espagnols. Ces Bréauté étaient d'une, fort ancienne maison de Normandie,
illustrée par les alliances et les emplois, et dont plusieurs étaient
pour aller loin qui furent tués jeunes. Le père de celui-ci fut de
ceux-là, que le maréchal de Bassompierre loue fort en ses Mémoires. Son
fils aîné, élevé enfant d'honneur de Louis XIII, fut tué à dix-huit ans,
aux lignes d'Arras, en 1654, sans avoir été marié. Le cadet, est celui
dont je parle, qui avait très peu servi, et qui, avec fort peu d'esprit,
n'avait pas laissé d'être mêlé à la cour autrefois. Il se maria
médiocrement et se ruina en plein. On prétendit que ce fut à souffler.
Il perdit son fils unique à dix-neuf ans, qui avait un régiment, et sa
femme ensuite. La dévotion suivit la misère, il se retira à
Saint-Magloire, d'où il fallut sortir quelque temps après, faute d'y
pouvoir payer sa pension. Le duc de Foix, dont il était parent, le
retira généreusement chez lui. Mais lui et M\textsuperscript{me} de Foix
étaient fort répandus dans le monde, dînaient rarement chez eux, et n'y
soupaient jamais. Bréauté, qui était de grand appétit et gourmand, ne
s'accommodait pas de la nourriture du domestique. Il allait chercher à
vivre aux tables du voisinage, où il ennuyait souvent par ses sermons.
Il était tout occupé de piété et de bonnes oeuvres. Ce fut lui qui
entreprit la fameuse affaire de Langlade, condamné aux galères, et mort
à la Tournelle, pour un vol commis chez le comte de Montgommery où il
logeait. Bréauté fit reconnaître son innocence, rétablir sa mémoire, et
marier bien la fille unique qu'il avait laissée, des dommages et
intérêts qu'il lui fit obtenir. Il lui était resté de sa soufflerie des
remèdes qu'il faisait lui-même. Apparemment qu'il les fit mal à la fin,
car il mourut très brusquement pour en avoir pris pour une légère
incommodité avec une santé très robuste. Je l'ai fort vu à l'hôtel de
Lorges, qui lui était fort commode parce que M. de Foix logeait
vis-à-vis.

Deux abbés fort différents l'un de l'autre moururent incontinent après,
l'abbé de La Rochefoucauld et l'abbé de Châteauneuf. Le premier était
oncle paternel de M. de La Rochefoucauld. Il avait un mois moins que lui
et soixante-quatorze ans. Le peu qu'il avait il le partagea toujours
avec lui, tarit qu'il fut pauvre\,; leur amitié fut la plus intime et
dura toute leur vie. Ils logeaient ensemble et ne se quittèrent jamais,
tellement que l'abbé de La Rochefoucauld passa sa vie à la cour sans en
être, et sans sortir presque jamais de chez M. de La Rochefoucauld, où
il était absolument le maître. Cela lui donnait quelque considération,
même du roi. D'ailleurs, c'était le meilleur gentilhomme du monde, le
plus noble et le plus droit, mais aussi le plus imbécile, et qui
ressemblait le mieux à un vicaire de village. Il était passionné de la
chasse, et n'en manquait jamais\,; cela l'avait fait appeler l'abbé
Tayaut. Il n'eut jamais d'ordres, mais force abbayes, et grosses, que M.
de La Rochefoucauld lui fit donner, et qu'il eut toutes à sa mort pour
son petit-fils, dont nous verrons qu'il se repentit bien.

L'abbé de Châteauneuf est celui qui fut envoyé en Pologne redresser la
conduite de l'abbé de Polignac, dont j'ai parlé à cette occasion, homme
de beaucoup d'esprit, de savoir et de bonne compagnie, désiré dans les
meilleures, et frère de Châteauneuf ambassadeur à Constantinople, en
Portugal et en Hollande, mort conseiller d'État, et ancien prévôt des
marchands longtemps depuis.

Quelque temps auparavant la comtesse de Soissons était morte à Bruxelles
dans le plus grand délaissement, pauvre et méprisée de tout le monde,
même fort peu considérée du prince Eugène, son célèbre fils. Ce fut en
sa faveur que le cardinal Mazarin, son oncle, inventa au mariage du roi
la nouvelle charge de surintendante, à cause de quoi il en fallut une en
même temps à la reine mère, qui fut la princesse de Conti, son autre
nièce, et comme tout va toujours en se multipliant et en
s'affaiblissant, Madame, parce qu'elle était fille d'Angleterre, en eut
une aussi, qui fut M\textsuperscript{me} de Monaco. C'est l'unique
exemple pour les filles de France.

Rien n'est pareil à la splendeur de la comtesse de Soissons, de chez qui
le roi ne bougeait avant et après son mariage, et qui était la maîtresse
de la cour, des fêtes, et des grâces, jusqu'à ce que la crainte d'en
partager l'empire avec les maîtresses la jeta dans une folie qui la fit
chasser avec Vardes et le comte de Guiche, dont l'histoire est trop
connue et trop ancienne pour la rapporter ici. Elle fit sa paix et
obtint son retour par la démission de sa charge, qui fut donnée à
M\textsuperscript{me} de Montespan, dont le mari ne voulut recevoir
aucune chose du roi, qui, ne sachant comment la faire asseoir, ne
pouvant la faire duchesse, supposa que la charge de surintendante
emportait le tabouret. La comtesse de Soissons, de retour, se trouva
dans un état, bien différent de celui d'où elle était tombée. Elle se
trouva si mêlée dans l'affaire de la Voisin, brûlée en Grève pour ses
poisons et ses maléfices, qu'elle s'enfuit en Flandre. Son mari était
mort fort brusquement à l'armée, il y avait longtemps, et dès lors on en
avait mal parlé, mais fort bas dans la faveur où elle était. De Flandre
elle passa en Espagne, où les princes étrangers n'ont ni rang ni
distinction. Elle ne put donc paraître en aucun lieu publiquement, et
moins au palais qu'ailleurs.

La reine, fille de Monsieur, n'avait point d'enfants, et avait tellement
gagné l'estime et le coeur du roi son mari, que la cour de Vienne
craignit tout de son crédit pour détacher l'Espagne de la grande
alliance faite contre la France. Le comte de Mansfeld était ambassadeur
de l'empereur à Madrid, avec qui la comtesse de Soissons lia un commerce
intime dès en arrivant. La reine, qui ne respirait que France, eut une
grande passion de voir la comtesse de Soissons. Le roi d'Espagne, qui
avait fort ouï parler d'elle, et à qui les avis pleuvaient depuis
quelque temps qu'on voulait empoisonner la reine, eut toutes les peines
du monde à y consentir. Il permit à la fin que la comtesse de Soissons
vînt quelquefois les après-dînées chez la reine par un escalier dérobé,
et elle la voyait seule et avec le roi. Les visites redoublèrent et
toujours avec répugnance de la part du roi. Il avait demandé en grâce à
la reine de ne jamais goûter en rien qu'il n'en eût bu ou mangé le
premier, parce qu'il savait bien qu'on ne le voulait pas empoisonner. Il
faisait chaud le lait est rare à Madrid, la reine en désira, et la
comtesse, qui avait peu à peu usurpé les moments de tête à tête avec
elle, lui en vanta d'excellent qu'elle promit de lui apporter à la
glace. On prétend qu'il fut préparé chez le comte de Mansfeld. La
comtesse de Soissons l'apporta à la reine qui l'avala, et qui mourut peu
de temps après, comme M\textsuperscript{me} sa mère. La comtesse de
Soissons n'en attendait pas l'issue et avait donné ordre à sa fuite.
Elle ne s'amusa guère au palais, après avoir vu avaler ce lait à la
reine\,; elle revint chez elle où ses paquets étaient faits et s'enfuit
en Allemagne, n'osant pas plus demeurer en Flandre qu'en Espagne. Dès
que la reine se trouva mal, on sut ce qu'elle avait pris et de quelle
main\,; le roi d'Espagne envoya chez la comtesse de Soissons qui ne se
trouva plus\,; il fit courir après de tous les côtés, mais elle avait si
bien pris ses mesures qu'elle échappa. Elle vécut obscurément quelques
années en Allemagne, tantôt dans un lieu, tantôt dans un autre. Mansfeld
fut rappelé à Vienne, où il eut à son retour le premier emploi de cette
cour, qui est la présidence du conseil de guerre. À la fin la comtesse
de Soissons retourna en Flandre, puis à Bruxelles, où je crus avoir dit
que, tandis que Philippe V en fut maître, lés maréchaux de Boufflers, de
Villeroy, et tous les Français distingués, eurent défense de la voir. Il
se peut dire qu'elle y passa le reste de sa vie et qu'elle y mourut en
opprobre. M\textsuperscript{me} la duchesse de Bourgogne en prit le
deuil pour six jours, que le roi ne porta point ni la cour, quoique la
princesse de Carignan, mère du comte de Soissons, fût princesse du sang,
la dernière de sa branche.

En ce même temps mourut aussi, au camp devant Lille, M. d'Overkerke,
général en chef des Hollandais et de leur armée, qui était des bâtards
de Nassau-Orange, et qui avait été dans l'intime confiance du roi
Guillaume, dont il était grand écuyer.

Desmarets, revenu de si loin au contrôle général des finances, très bien
avec Chamillart, et appuyé des ducs de Chevreuse et de Beauvilliers, qui
tous trois l'y avaient porté avec tant de soeurs, fit entendre par eux
la grandeur et la capacité de son travail, la nécessité pour le bien des
affaires de l'accréditer dans le public, et la convenance de le faire
ministre d'État, comme l'avaient été ceux qui l'avaient précédé dans son
emploi. Le roi, qui comptait alors avoir besoin de lui, et qui
commençait à s'y accoutumer, se laissa prendre à cette amorce et le fit
ministre. Il avait déjà deux filles mariées, l'une à Goesbriant, l'autre
à Bercy, intendant des finances, qui faisait tout sous lui. Incontinent
après cette grâce, il maria bien autrement la troisième, ce fut au
marquis de Béthune-Orval, qui avait la perspective du duché de Sully
après le duc de Sully qui n'avait point d'enfants, et après le chevalier
de Sully qu'on croyait marié secrètement, de façon à n'en avoir point
non plus. Ce M. de Béthune était un homme qui n'avait point paru à la
cour et comme point à la guerre, riche, mais noyé dans une mer de procès
qu'on l'accusait d'aimer beaucoup, et à la poursuite desquels il
occupait toute sa vie. Le roi voulut donner deux cent mille livres à la
fille de Desmarets, comme il avait accoutumé aux mariages des filles de
ses ministres, mais celui-ci ne le voulut pas dans la presse où étaient
les finances. Au lieu de cette somme, le roi voulut donner une pension
de douze mille livres\,; Desmarets ne la voulait que de huit mille,
enfin elle fut de dix mille livres.

Il se fit quelques jours auparavant un autre mariage, par des
circonstances singulières qui le rendirent heureux. Depuis les deux
Eustache de Conflans, père et fils, tous deux capitaines des gardes du
corps de Charles IX et d'Henri III, et le dernier chevalier du
Saint-Esprit et chevalier d'honneur de la reine Marie de Médicis, cette
maison était entièrement tombée. Le dernier Eustache avait vendu presque
toutes ses terres. Il perdit un second fils fort jeune, de la plus
grande espérance\,; ce que l'aîné fit de mieux fut de se raccrocher par
les biens de sa mère, qui était Jouvenel, dont il eut Armentières, et
par un riche mariage avec une Pinart. Il en fit un second fort plat. Du
premier un fils unique qui mena une vie honteuse et obscure, et mourut
sans enfants d'un indigne mariage qu'il avait fait. Sa soeur du second
lit ne se maria point, elle retira tout ce qu'elle put de ces débris\,;
la duchesse d'Orval se retira chez elle où elle a passé presque toute sa
vie, ayant de la considération et des amis. On l'appelait
M\textsuperscript{lle} d'Armentières. Elle vécut fort vieille. Étant
devenue riche par ses soins et par la mort de son frère, elle assista à
son tour son amie qui était devenue pauvre, substitua son bien à ses
cousins, et en laissa l'usufruit à la duchesse du Lude, son amie intime
de tous les temps. Ses cousins étaient dans la dernière pauvreté. Ils
sortaient du frère puîné du premier Eustache, capitaine des gardes de
Charles IX, dont ils étaient la quatrième génération, et divisés en deux
branches. Ils n'avaient pu faire aucune alliance, et ils vivaient à leur
campagne de leurs choux et de leur fusil. L'aînée de ces deux, branches
finissait à un seul mâle qui se fit prêtre pour avoir du pain, et que le
succès de ce mariage fit dans la suite évêque du Puy. Le chef de la
branche cadette, devenu celui de toute cette maison, vécut de même, et
se trouva heureux d'épouser en 1667 une fille de d'Aguesseau, maître des
comptes, dont le fils a été si estimé et si considéré, intendant de
Languedoc, puis conseiller d'État, et du conseil royal des finances, et
le petit-fils est depuis devenu chancelier de France, avec diverses
fortunes. De ce mariage sortirent trois fils appelés à la substitution
de M\textsuperscript{lle} d'Armentières.

L'aîné, brave homme et honnête homme, mais sans la moindre trace
d'esprit que l'éducation n'avait pu réparer, se battit contre Pertuis
dans leur première jeunesse, et {[}ils{]} furent tous deux enfermés
quinze ou seize ans durant dans une citadelle. Les deux cadets se
trouvèrent avoir beaucoup d'esprit, et de désir de se relever, malgré
leur pauvreté et l'obscurité où ils se trouvaient. L'aîné des deux fut
envoyé enfant, et sans pain, page du grand maître de Malte, le cadet
s'intrigua comme il put et servit de même. Tous deux, à force de
vouloir, firent des connaissances, et s'ornèrent l'esprit à force, de
lecture, dans laquelle ils acquirent beaucoup. La maréchale de Chamilly,
qui les connut à La Rochelle, où ils servaient, les prit en amitié, les
attira chez elle à Paris, où ils virent la bonne compagnie, dont ils
surent profiter. Ils firent une autre connaissance que cette maréchale
ne leur procura pas, mais qui devint le fondement de leur fortune\,: ce
fut {[}celle{]} de M\textsuperscript{me} d'Argenton. Elle les trouva de
si bonne compagnie qu'elle les présenta à M. le duc d'Orléans, avec qui
elle les fit souper chez elle, et leur acquit sa familiarité. Il vaqua
chez lui une place de chambellan qu'il procura à Conflans, et bientôt
après une autre à d'Armentières qui sortait de sa prison. Ils se firent
des amis au. Palais-Royal\,; Armentières, par le même crédit, devint
maître de la garde-robe.

M\textsuperscript{me} de Jussac, dont j'ai parlé lorsqu'on la mit sans
titre auprès de M\textsuperscript{me} la duchesse d'Orléans qu'elle a
voit élevée, et qui l'aimait passionnément, avait une fille mariée à M.
de Chaumont, du nom d'Ambly, qui avait un régiment. Elle en avait une
autre fort jolie, dont elle voulut aussi se défaire, mais son bien était
fort court. Son bonheur fit que Sassenage, premier gentilhomme de la
chambre de M. le duc d'Orléans, revenu malade d'Espagne, fort dégoûté de
son emploi, s'en voulut défaire. Il fallut attendre le retour de ce
prince, qui, pour la première fois, pressé pour la même grâce par
M\textsuperscript{me} d'Argenton d'une part et par M\textsuperscript{me}
la duchesse d'Orléans de l'autre, donna l'agrément de la charge de
Sassenage à d'Armentières, en faisant son mariage avec la fille de
M\textsuperscript{me} de Jussac, qui y trouva encore d'autres facilités
de grâces, et qui, toujours avec l'appui de M\textsuperscript{me}
d'Argenton, fit passer à Conflans la charge de maître de la garde-robe
qu'avait son frère devenu premier gentilhomme de la chambre.

M. le duc d'Orléans arriva le 6 décembre, et fut aussi bien reçu que le
méritait sa glorieuse et pénible campagne, qui ne le raccommoda pourtant
pas avec M\textsuperscript{me} des Ursins, ni avec M\textsuperscript{me}
de Maintenon.

Ce fut en ce temps-ci que le comte de Tonnerre épousa la fille de
Blansac, dont j'ai assez parlé (t. IV, p.~308) pour n'avoir rien à y
ajouter. Ce mariage le fit sortir de la Bastille immédiatement avant de
le célébrer.

J'ai avancé le récit de quelques menus événements de la fin de cette
année, comme j'en ai retardé quelques-uns auparavant, pour ne pas
interrompre celui des choses de Flandre, où il est temps de retourner.
Mais auparavant il faut dire que je ne fus pas longtemps à la Ferté sans
y recevoir une lettre de l'évêque de Chartres, datée de Saint-Cyr, qui
m'avertissait qu'on m'avait rendu les plus mauvais offices du monde
auprès du roi et de M\textsuperscript{me} de Maintenon, et qui avaient
pris. Je lui écrivis à l'instant par un exprès pour avoir plus
d'éclaircissement qu'un avis si vague, et pour lui fournir, sur ce que
je savais qu'on avait répandu contre moi sur Lille et sur mon pari, de
quoi me défendre en attendant qu'il m'eût instruit et que je pusse avec
plus de précision parer aux coups, qu'on m'avait portés. Je ne fus pas
surpris, mais embarrassé d'être instruit, parce que M. de Chartres était
retourné à Chartres lorsque mon exprès arriva à Saint-Cyr, et qu'il ne
voulut pas depuis m'en apprendre davantage. De cette affaire-là, j'en
fus noyé plus d'un an, et la façon dont j'en sortis se verra en son
temps. Je ne demeurai pas longtemps à la Ferté, et je voulus être à la
cour pour le retour de M. le duc d'Orléans, et surtout pour celui de Mgr
le duc de Bourgogne.

\hypertarget{note-i.-la-grande-duchesse-de-toscane.}{%
\chapter{NOTE I. LA GRANDE-DUCHESSE DE
TOSCANE.}\label{note-i.-la-grande-duchesse-de-toscane.}}

La grande-duchesse de Toscane, dont parle Saint-Simon (p.~2), était
Marguerite-Louise d'Orléans, fille de Gaston et de Marguerite de
Lorraine, laquelle avait épousé Cosme III de Médicis, grand-duc de
Toscane. L'exclamation du grand écuyer, prince de la maison de Lorraine,
s'explique par la longue rivalité des maisons de Bourbon et de Lorraine.
On sait que les Guise étaient de cette dernière maison.

\hypertarget{note-ii.-bartet-son-aventure-avec-le-duc-de-candale-ses-lettres-uxe0-mazarin.}{%
\chapter{NOTE II. BARTET, SON AVENTURE AVEC LE DUC DE CANDALE, SES
LETTRES À
MAZARIN.}\label{note-ii.-bartet-son-aventure-avec-le-duc-de-candale-ses-lettres-uxe0-mazarin.}}

Saint-Simon parle (p.~120 de ce volume) de l'aventure de Bartet avec le
duc de Candale, mais sans entrer dans aucun détail. Comme ses assertions
ne sont pas toutes exactes, il ne sera pas inutile de faire connaître
Bartet et l'aventure à laquelle Saint-Simon fait allusion. Bartet était
Béarnais, et fils d'un paysan. Son esprit, au-dessus de sa condition,
fit sa fortune\,: il alla à Rome, s'attacha à Casimir Vasa, qui devint
roi de Pologne, et se fit nommer son résident en France\footnote{Voy.
  dans les \emph{Mémoires de Conrart} l'article intitulé \emph{Bartet
  secrétaire du cabinet}. Voy. aussi les \emph{Mémoires de Mademoiselle}
  à l'année 1655.}. Plus tard il devint un des principaux agents de
Mazarin. Pendant l'exil du cardinal, il lui portait les dépêches de la
reine Anne d'Autriche et rapportait les réponses de Mazarin. La faveur
dont Bartet jouit à la cour, lorsque le cardinal eut triomphé de ses
ennemis et affermi sa puissance, lui inspira une vanité qui le rendit
ridicule et odieux. Il ne craignit pas d'entrer en lutte avec de grands
seigneurs, et entre autres avec le duc de Caudale, fils du duc
d'Épernon.

Le duc de Candale était un des seigneurs de cette époque les plus
renommés pour sa beauté, sa magnificence et l'éclat de ses aventures.
Bartet, son rival en amour, dit devant plusieurs témoins que, si l'on
ôtait au duc de Candale ses grands cheveux, ses grands
canons\footnote{Les canons étaient des ornements de toile ronds, fort
  larges, souvent ornés de dentelles, qu'on attachait au-dessous du
  genou et qui tombaient jusqu'à la moitié de la jambe. Molière s'est
  moquéDe ces larges canons, où comme en des entravesOn met tous les
  matins ses deux jambes esclaves.}, ses grandes manchettes et ses
grosses touffes de galants\footnote{Noeuds de rubans qui servaient à
  orner les vêtements. Voy. p.~453, la note sur le mot \emph{petite oie}
  qui avait la même signification.}, il serait moins que rien, et ne
paraîtrait plus qu'un squelette et un atome\footnote{\emph{Mémoires de
  Conrart}, article \emph{Bartet}.}. Le duc de Candale, instruit de
cette insolence, s'en vengea avec une audace qui peint l'époque, et
montre combien les grands seigneurs se croyaient au-dessus des lois. Il
envoya un de ses écuyers, à la tête de onze hommes, arrêter en plein
jour la voiture de Bartet, dans la rue Saint-Thomas du Louvre. On ne lui
donna pas la bastonnade, comme dit. Saint-Simon, mais pendant qu'une
partie des gens du duc de Candale arrêtaient les chevaux de Bartet, et
menaçaient son cocher de leurs pistolets, d'autres entrèrent dans le
carrosse, et, armés de ciseaux, lui coupèrent la moitié des cheveux et
de la moustache, et lui arrachèrent son rabat, ses canons et ses
manchettes. Le jour même de cette aventure (28 juin 1655), Bartet envoya
son frère à Mazarin avec la lettre suivante\footnote{Archives des
  affaires étrangères, France, t. CLIV, pièce 95 autographe.}\,:

«\,Je dépêche mon frère à Votre Éminence pour lui rendre compte d'une
malheureuse affaire qui m'est survenue à ce matin. Je sortais à dix
heures de chez M. Ondedei\footnote{L'abbé Ondedei, parent de Mazarin,
  devint évêque de Fréjus.}, à qui je n'avais point parlé, parce qu'il
était avec M. l'évêque d'Amiens, et m'en allais dans mon carrosse avec
deux petits laquais derrière. À l'entrée de la rue Saint-Thomas du
Louvre, du côté du quai, j'ai vu venir à moi quatorze hommes à cheval,
avec quelques valets à pied, tous armés d'épées, et de pistolets, et de
poignards, qui ont crié à mon cocher qu'il arrêtât. J'ai titré la tête à
la portière, et ai cru d'abord qu'ils me prenaient pour un autre, ne me
sachant aucune méchante affaire\,; mais les ayant reconnus pour être des
valets de chambre et des parents d'un conseiller de la province dont je
suis\footnote{Ce conseiller du parlement de Pau auquel Bartet imputa
  d'abord l'attentat contre sa personne se nommait Casaux. Voy.
  \emph{Mémoires de Conrart}. art. \emph{Bartet}.}, avec qui j'ai une
querelle de famille il y a plus de dix ou douze ans, je n'ai plus douté
qu'ils ne fussent là pour m'assassiner. Je leur ai donc demandé, comme
ils sont venus à moi le pistolet et le poignard à la main, s'ils
voulaient me tuer, et leur ai dit même qu'ils me trouvaient en fort
méchante condition\,; mais deux d'entre eux sont montés dans mon
carrosse, et ayant tiré des ciseaux, m'ont coupé le côté droit de mes
cheveux, et m'ont arraché un canon, et s'en sont allés sans ajouter
aucune voie de fait à cet outrage.

«\,Comme mes laquais, mon cocher, un de mes amis familiers qui était
dans mon carrosse, et moi, les avons reconnus pour être des gens de mon
pays, amis, parents et serviteurs de celui avec qui j'ai cette vieille
querelle dont je viens de parler à Votre Éminence, je me suis retiré
chez moi, et d'abord me suis pourvu par les voies de la justice, comme
plus propres à ma profession, et plus conformes même à mon naturel. Je
supplie donc Votre Éminence, Monseigneur, que je demeure encore ici
peut-être quinze jours, qu'il faudra que j'emploie à faire achever les
informations, qui sont déjà commencées, et mettre ma poursuite en état
qu'elle puisse aller son chemin par les formes de la justice en mon
absence. Ainsi, je supplie encore Votre Éminence qu'il lui plaise
d'ordonner à M. de Langlade qu'il serve ce commencement de mon quartier
jusqu'à mon arrivée.

«\,Je demanderais à Votre Éminence la puissance de sa protection, si
celle de la justice ordinaire ne suffisait pas, et si je ne croyais
trouver au moins autant d'amis et de considération dans Paris qu'un
homme de province qui est réduit à des assassins et à un assassinat. Il
ne me reste donc qu'à demander en grâce à Votre Éminence qu'elle croie
que je ne puis pas rien oublier au monde, de quelque nature que puissent
être, des moyens honnêtes et légitimes pour la réparation de mon
honneur, et pour venger un outrage dont l'impunité me rendrait
méprisable dans le monde, et bien indigne de l'honneur que j'ai d'être
au roi par la libéralité de la reine et celle de Votre Éminence qui l'a
produite, de celui que j'ai encore d'être ministre du roi de Pologne, et
d'être cru au point que je le suis serviteur de Votre Éminence, et sous
votre protection particulière en cette qualité-là.

«\,Je ne suis pas si embarrassé de mon affaire que je ne pense encore
rendre compte à Votre Éminence des siennes dont j'ai connaissance\,;
mais je sais que M. Ondedei est à la source des choses et des personnes,
et qu'il n'oublie rien pour les faire et les dire à Votre Éminence.
Ainsi, Monseigneur, j'en demeurerai là présentement, et n'ajouterai plus
rien à celte présente importunité que les protestations les plus fidèles
du monde que je lui fais de vivre et de mourir,

«\,Monseigneur,

«\,De Votre Éminence,

«\,Le très humble, très obéissant, très fidèle

«\,et très obligé serviteur,

«\,Bartet\,»

Bartet ne tarda pas à connaître l'auteur véritable de cet attentat,
comme le prouve la lettre qu'il écrivait à Mazarin le 1er juillet
1655\footnote{Archiv. des aff. étrangères\,; France, t. CLIV, pièce 107
  autographe.}\,:

«\,Monseigneur,

«\,Il m'est arrivé un bien plus grand malheur que celui dont je rendis
compte à Votre Éminence, avant-hier, par mon frère, puisque c'est M. de
Candale qui dit avoir commandé l'assassinat que je croyais avoir été
fait en moi par ce conseiller de ma province avec qui j'ai une querelle
de famille. Il faut bien, Monseigneur, que mes ennemis l'aient emporté
sur son esprit d'un artifice bien terrible, et qu'ils l'aient circonvenu
bien cruellement pour moi, puisqu'ils lui ont persuadé divers discours
qu'ils m'attribuent avec une si injuste précipitation, qu'ils ne lui ont
pas seulement laissé le temps de les examiner, de les vérifier et de les
tenir pour établis dans le monde. Ç'a donc été par ses propres
domestiques et par d'autres gens de mon pays que je fus assassiné
avant-hier, en la manière que j'ai pris la liberté de l'écrire à Votre
Éminence.

«\,Dans la première interprétation de mes assassins et de mon
assassinat, je ne demandais point à Votre Éminence une protection
particulière, parce que la qualité de l'action même, celle de mon ennemi
prétendu, et la justice ordinaire, m'en donnaient une assez puissante\,;
mais aujourd'hui qu'un homme de la puissance, pour ainsi dire, et de la
qualité de M. de Candale, se vante publiquement de m'avoir fait
assassiner, je n'ai presque point de protection à espérer après celle
des lois, si le roi ne m'en donne une particulière par la faveur de
Votre Éminence, par laquelle Sa Majesté laisse faire la justice
ordinaire de sou royaume, et comme son sujet, et comme ayant l'honneur
d'être son domestique, et encore résident à sa cour d'un roi étranger,
qui me couvre du droit des gens, si inviolable en toutes les cours du
monde.

«\,M. de Caudale se plaint de trois choses présentement, dont il ne m'a
jamais fait faire de plaintes par aucun homme du monde. La première, et
qui est celle sur laquelle il a réglé l'assassinat commis par ses gens,
est que j'ai dit, parlant de lui, que, si on lui ôtait ses canons, sa
petite oie\footnote{On appelait ainsi les rubans, plumes, noeud de
  l'épée, garniture des bas, des souliers, etc. Dans les
  \emph{Précieuses ridicules} le marquis de Mascarille dit aux
  Précieuses (scène X)\,: «\,Que vous semble de ma petite oie\,? La
  trouvez-vous congruente à l'habit\,?»} et ses cheveux, il serait comme
un autre homme. Je réponds à cela qu'il n'y a homme au monde qui me le
puisse maintenir, parce que la vérité est, comme devant Dieu, que je ne
l'ai jamais dit. J'ajoute encore que faire assassiner les gens sur un on
dit qu'on n'établit point, et dont il ne pourra jamais donner de preuve,
est une manière de se faire justice à soi-même qui n'est pratiquée en
aucun lieu de la terre\,; et personne ne trouve que, quand la chose
serait comme il l'a bien voulu croire, il en peut être si implacablement
offensé que de se résoudre à me faire assassiner en plein jour, dans
Paris, par des gens reconnus à lui, à la face des lois et des
magistrats, dans les rues.

«\,Il se plaint encore que je lui ai parlé chez M. de
Nouveau\footnote{M. de Nouveau était directeur des postes.}, il y a un
mois, avec irrévérence (c'est le mot dont il se sert). Cela est si vague
et si général qu'il n'y a point d'irrévérence qu'on ne se puisse forger
tous les jours\,: mais celui-là en fut un auquel, sur la définition d'un
mot français, vingt personnes de la cour, et M. de Nouveau même, qui y
étaient, savent qu'on ne peut pas parler avec plus de révérence que je
fis\footnote{Conrart, à l'article cité, parle de cette aventure dans les
  termes suivants\,: {[}Bartet{]} dit que M. de Candale étant dans une
  chambre avec **\emph{, et lui ayant rencontré M\textsuperscript{me}
  Cornuel dans une autre, elle était venue au-devant de lui et lui avait
  demandé s'il trouvait que ce fût bien parler que de dire un }esprit
  fretté\,? A* quoi il répondit qu'elle s'adressait bien mal de choisir
  un pauvre Gascon pour juge d'une phrase française\,; mais que, si elle
  voulait qu'il en dît son sentiment, il trouvait que cette façon de
  parler ne valait rien\,; qu'il fallait être sans jugement pour parler
  ainsi, et cent autres exagérations semblables, qui sont de son style
  ordinaire\,; qu'elle avait ajouté que M. de Candale disait pourtant
  que c'était lui qui s'en était servi\,; et que, sur cela, M. de
  Candale étant sorti de l'autre chambre, elle lui avait crié tout haut
  que M. Bartet soutenait qu'il n'avait jamais dit un \emph{esprit
  fretté\,;} ce que Bartet lui-même confirma avec les mêmes
  amplifications dont il avait déjà usé. Ce qui fâcha, à ce qu'il dit,
  M. de Caudale, lequel ayant eu ensuite d'autres dégoûts que j'ai
  touchés, il lui avait fait jouer cette pièce à la vue de tout
  Paris.\,»}.

«\,Il ajoute que j'ai fait depuis quelque temps à Votre Éminence des
discours fort désavantageux de lui\,; sur quoi je n'ai rien à alléguer
pour ma justification que les témoignages propres de Votre Éminence, que
je ne subornerai point en ma faveur.

«\,Voilà, Monseigneur, les trois sujets de mon assassinat dans la propre
bouche de M. de Candale, qui hier, devant tout ce qu'il y a ici de gens
de qualité, fit venir dans une maison un des assassins, et lui ayant
fait conter l'assassinat, il dit\,: \emph{C'est moi qui l'ai ordonné\,;
je le dis afin que tout le monde le sache, et si Bartet s'en prend à
personne qu'à moi, je le ferai encore assassiner et tuer dans les rues,
et s'il fait encore aucune poursuite, je le ferai assassiner et tuer}.

«\,Votre Éminence, Monseigneur, qui sait si bien la science des rois,
sait bien qu'ils ne parlent ni ne font comme M. de Candale\,; et les
tyrans même, qui font un usage tyrannique de l'autorité qui est légitime
aux rois, n'en font point un de la qualité de M. de Candale. Je me mets
donc, Monseigneur, s'il vous plaît, sous la protection du roi, par celle
de Votre Éminence, et je la conjure, par tous les endroits qui lui
peuvent donner quelque sensible pour la disgrâce où je me trouve, de
laisser faire la justice au parlement de Paris, et que, pour avoir
l'honneur d'être au roi et au roi de Pologne, et au service de Votre
Éminence par l'action et le mouvement continuels de ma vie, je ne me
trouve pas dans une condition moins favorable que si j'étais un homme
d'une condition privée.

«\,Si, avec cela, Monseigneur, Votre Éminence avait la bonté de faire
considérer au roi comme le respect de sa personne est blessé en moi par
l'honneur que j'ai d'être son domestique, et le respect de son autorité
violé par l'assassinat commis en moi, et ensuite faire témoigner à M. de
Candale qu'il faut que le cours de la justice du royaume soit libre pour
moi, j'aurai l'obligation à Votre Éminence de me laisser un tribunal
qui, jugeant mon honneur suivant la loi, me tirera de l'opprobre du
monde, et me rétablira dans le même honneur dans lequel j'avais toujours
vécu jusqu'à cette heure.

«\,C'est là, Monseigneur, la très humble supplication que je fais à
Votre Éminence, avec une autre qui ne m'est guère moins nécessaire, qui
est de boucher son esprit à l'industrie et à la malice de mes ennemis,
qui, dans ce grand mouvement de ma mauvaise fortune, ne manqueront pas
de faire une autre sorte d'assassinat, moins déshonorant pour moi, mais
plus dangereux, pour varier les bonnes volontés de Votre Éminence en mon
endroit.

«\,Ce sont ces bonnes volontés-là, Monseigneur, par lesquelles je puis
parvenir à la protection de la justice que je suis sur le point de
demander au parlement de Paris contre mes assassins, je dis les gens qui
m'ont assassiné\,; et comme c'est l'endroit le plus capital de ma vie,
et un passage de fortune qui doit être presque regardé comme unique,
parce qu'il est presque toujours le dernier de celle d'un honnête homme,
je la supplie aussi de considérer ce que je devrai à sa protection, et
si, vous étant obligé du recouvrement de tout mon honneur, je ne dois
pas me préparer toute ma vie à l'employer pour le service de Votre
Éminence.

«\,Les personnes qui me compatissent sincèrement, et qui m'ont promis de
me donner les secours de leurs amitiés, attendent, Monseigneur, quelque
mouvement favorable de Votre Éminence en mon endroit, et par la bonté
qu'ils croient que vous avez naturellement pour moi, et parce que
l'action est si odieuse que l'autorité, dont vous avez la conduite, en
est blessée.

«\,M\textsuperscript{me} de Chevreuse en a parle ce matin à M. l'abbé
Ondedei, de qui j'ai reçu les dernières civilités. Je crois qu'elle lui
en écrira même encore\,; et M. le premier président, qui condamne
l'action par tous les endroits par lesquels elle est condamnable, m'a
promis ce que peut promettre un homme qui est à sa place\,; de sorte,
Monseigneur, que, si j'obtiens de Votre Éminence ce petit mouvement de
laisser faire, sans vous déplaire, le parlement de Paris, la plus grande
partie des juges, que j'ai déjà vus par précaution, voient en mon
affaire une fin fort honorable. Je trouverai la mienne bien glorieuse,
Monseigneur, si, après m'être rendu tout mon honneur qu'on m'a ôté, je
suis assez heureux pour l'employer pour votre service, qui est, comme
Dieu sait, la passion la plus forte que j'aie au monde.\,»

Mazarin parut compatir à l'affront qu'avait essuyé Bartet\,; il lui
écrivit une lettre dans laquelle il lui promit d'en tirer vengeance.
Mais soit qu'il ne voulût pas mécontenter la noblesse pour une cause de
si peu d'importance, soit qu'il fût lui-même blessé de la vanité de
Bartet, il laissa tomber l'affaire. Les contemporains ne firent que rire
de l'avanie infligée à un favori insolent. M\textsuperscript{me} de
Sévigné en parle en plaisantant à Bussy-Rabutin\footnote{Lettre du 19
  juillet 1655.}, et trouve le tour très bien imaginé. D'autres firent
sur l'aventure de Bartet une chanson dont voici un couplet\,:

Comme un autre homme

Vous étiez fait, monsieur Bartet\,;

Mais, quand vous iriez chez Prudhomme\footnote{Baigneur célèbre de cette
  époque, chez lequel on trouvait tous les raffinements du luxe.}.

De six mois vous ne seriez fait

Comme un autre homme.

Cependant Bartet n'en resta pas moins, après cette aventure
tragicomique, un des confidents de Mazarin. C'est à tort que Saint-Simon
dit (p.~121) que \emph{là commença son déclin, qui fut rapide et court}.
Quatre ans plus tard nous retrouvons encore Bartet à la cour rendant
compte de toutes choses au cardinal qui négociait la paix des Pyrénées
(1659). Les lettres fort nombreuses de Bartet forment une véritable
gazette de la cour de Louis XIV. Je n'en citerai qu'une, pour ne pas
allonger une note déjà trop étendue. Il écrivait à Mazarin, de Bordeaux,
le 23 septembre 1659\footnote{Archiv. des aff. étrang., France, t.
  CLXVIII, pièce 53 autographe.}\,:

«\,J'ai déjà rendu mille grâces très humbles à Votre Éminence de
l'honneur qu'elle m'a fait de nie choisir pour le voyage de Rome, et je
les lui rends encore une fois avec tout le ressentiment que je dois. Je
suis tout prêt, Monseigneur, pour le faire, et n'attends que les ordres
de Votre Éminence pour cela.

«\,J'ai su tout le particulier de l'accommodement de M. le Prince, et je
loue Dieu qu'il soit de manière que l'on puisse voir les confiances
rétablies. Il semblait que je l'eusse pressenti dès Fontainebleau, et si
Votre Éminence s'en souvient, je me donnai l'honneur de lui écrire, dès
ce temps-là, la plupart des choses là-dessus qui se sont faites
aujourd'hui. L'état de ces affaires-là n'est pourtant point encore su
ici de beaucoup de gens avec toutes circonstances, mais quelques-uns le
savent, avec la soumission qu'il a faite au roi en la personne de Votre
Éminence par M. Gaillet, de mettre à ses pieds toutes les grâces que les
Espagnols ont voulu lui faire ou lui procurer. Ce sera une grande
consolation à M\textsuperscript{me} de Longueville d'apprendre ces
nouvelles-là, elle, Monseigneur, qui a toujours conservé, depuis son
rétablissement, ce véritable esprit de rentrer dans son devoir par une
entière résignation aux volontés du roi, et par une confiance pareille à
l'amitié de Votre Éminence.

«\,J'espère qu'une si favorable et si naturelle constitution d'affaires
pourra engendrer d'autres choses aussi favorables qui l'affermiront, et
qu'ainsi la paix s'assurera de tous les côtés.

«\,Tout le monde craint ici le voyage de Toulouse\footnote{La cour alla
  en effet à Toulouse vers la fin de l'année 1659.}, et encore un plus
éloigné du même côté. Votre Éminence sait que, quand ces messieurs sont
à leur aise en un lieu, ils n'aiment guère à en sortir que pour aller à
Paris.

«\,Le roi témoigne assez d'impatience pour son mariage, et disait à la
reine, il y a trois jours, qu'il serait fort ennuyé s'il le croyait
différé encore longtemps. Il est certain que son esprit paraît fort
libre et assez dégagé, et il semble qu'il s'affectionne bien plus qu'il
ne faisait. Sa us doute que la cessation des commerces\footnote{Il
  s'agit des relations de Louis XIV avec Marie Mancini que le cardinal
  avait reléguée à Brouage.} à laquelle Votre Éminence a mis la main si
utilement, l'a mis en cet état et l'y maintient, qui est assurément pour
lui une situation d'un très grand repos\,; car sa santé était
visiblement altérée, et se sentait des impressions de son esprit, comme
je ne doute point que ceux qui en ont le soin ne vous en aient
particulièrement informé.

«\,La cour grossit à cette heure si extraordinairement qu'il ne se peut
rien voir de plus en un lieu si éloigné de Paris.

«\,M. le duc de Guise, MM. d'Harcourt, M. de Langres, MM. d'Albret et de
Roquelaure, comtes de Béthune, d'Estrées, de Brancas et cinquante autres
particuliers de qualité sont arrivés ici depuis peu, à trois ou quatre
jours les uns des autres, et de la façon qu'ils parlent je crois que M.
le commandeur de Jars se trouvera seul dans Paris de tous les gens qui
vont au Louvre, tous ceux qui y sont demeurés se disposant à venir ici.

«\,M. le duc de Guise s'en va voir M. le duc de Lorraine à la conférence
et ne demeurera ici que très peu de jours.

«\,Le roi va à cette heure à la comédie presque tous les soirs\,; il en
fit représenter une le jour de la naissance de l'infante\,; il prit un
habit magnifique, fit faire grand feu aux gardes françaises et suisses
et à ses mousquetaires\,; tout le canon de la ville fut tiré. Il y eut
grand bal où il dansa. L'on fit \emph{media hoche}, et il dit à la reine
n'y ayant que moi et deux personnes que c'était le moins qu'il pouvait
faire, puisqu'il était le principal acteur de la comédie, pour
s'expliquer dans les mêmes termes du roi d'Espagne.

«\,M. de Roquelaure perdit hier dix mille écus contre M. de Cauvisson au
piquet. Celui-ci n'en gagna que deux mille, Mais M. de Brancas, qui
pariait pour lui, en gagna six mille\footnote{Il faudrait \emph{huit
  mille} pour faire le chiffre indiqué par Bartet.}. M. de Roquelaure
n'a joué que deux fois contre M. de Cauvisson, et il a perdu quarante
mille francs qu'il a pariés. Je vous écris avec cette certitude, parce
que je les lui ai vu perdre. Sa chère n'en est pas moins grande, car il
la fait très bonne.

«\,M. de Gourville est passé ici qui a dit qu'il allait quérir M. le
surintendant\footnote{Nicolas Fouquet.}.

«\,M. de Langlade y est arrivé sans doute pour servir son quartier.

«\,M. de Vardes en est parti, il y a quatre jours, pour se rendre auprès
de Votre Éminence et s'y tenir. Rien n'est égal à la manière dont il a
parlé à tout le monde de ses intérêts, disant qu'il n'aurait jamais de
volonté que celle de Votre Éminence, et qu'il y était si résigné qu'il
prendrait le mal même pour bien, quand il lui viendrait de la main et du
choix de Votre Éminence. Il a édifié tout le monde par sa tristesse et
par sa modestie.

«\,M. de Bouillon est arrivé de la campagne, où il était allé pour
chasser quinze jours.

«\,Il arriva ici avant-hier des comédiens français qui étaient en
Hollande\,; ils ont passé à la Rochelle\,; on les appelle les comédiens
de M\textsuperscript{lle} Marianne\footnote{Marie-Anne Mancini, dernière
  nièce du cardinal Mazarin\,; elle épousa plus tard le duc de Bouillon.},
parce qu'elle les faisait jouer tous les jours. Ils vinrent hier chez la
reine, comme elle entrait au cercle. Elle leur fit diverses questions à
ce propos et les engagea à dire qu'il n'y avait jamais eu que
M\textsuperscript{lle} Marianne qui les eût vus jouer, et que les
demoiselles ses soeurs n'avaient jamais vu la comédie. Je regardai le
roi, qui fit assurément là-dessus les mêmes réflexions que Votre
Éminence fait dans ce moment.

«\,M. le duc de Noirmoutiers est ici préparé à donner l'estocade à Votre
Éminence pour la survivance du Mont-Olympe. Il a envoyé M. son fils à
Bayonne, pour faire le voyage de Madrid avec M. le maréchal de Grammont.
Il est fort alerte sur la nature de l'accommodement de M. le Prince, un
chacun étant appliqué à voir s'il est fait de manière qu'il puisse
établir entre vous de la confiance et de l'amitié, et Votre Éminence
sait que ces messieurs-là (j'entends ses amis) ont plus d'intérêt que
les autres gens à ces affaires-là par la manière dont ils sont restés
avec M. le Prince.

«\,Je l'ai étonné ce matin, au pied du lit du roi (car j'ai vu qu'il
n'en savait rien), quand je lui ai dit que j'étais assuré que Caillet,
par ordre de M. le Prince, avait été trouver Votre Éminence trois fois
pour vous dire qu'il mettait aux pieds du roi toutes les grâces que les
Espagnols lui voulaient faire, et qu'il n'en prétendait que de la bonté
de Sa Majesté.

«\,Voilà, Monseigneur, l'état de ce parti. Le marquis de Villeroy a
toujours la dysenterie avec un peu de fièvre\,; on n'en a point mauvaise
opinion\,; mais M. Félix\footnote{Premier chirurgien du roi.} m'a dit ce
matin que ce qui ne serait point dangereux en un autre l'était dans ce
corps-là.\,»

\hypertarget{note-iii.-jarzuxe9-son-aventure-avec-la-reine-anne-dautriche.}{%
\chapter{NOTE III. JARZÉ\,; SON AVENTURE AVEC LA REINE ANNE
D'AUTRICHE.}\label{note-iii.-jarzuxe9-son-aventure-avec-la-reine-anne-dautriche.}}

Saint-Simon renvoie (p.~208 de ce volume) pour les aventures de Jarzé
aux \emph{Mémoires de M\textsuperscript{me} de Motteville}, qui donne en
effet des détails très précis sur la folle passion qu'affecta ce
personnage pour Anne d'Autriche et sur les conséquences qu'elle eut\,;
mais ce que M\textsuperscript{me} de Motteville ne savait pas, et ce que
nous apprennent les \emph{carnets} encore inédits de Mazarin\footnote{M3.
  B. I, f. Baluze. Ces carnets sont autographes, et on y trouve, surtout
  pour la Fronde, les renseignements les plus complets et les plus
  authentiques.} c'est le rôle du cardinal dans cette affaire.

Condé, que ses victoires sur la maison d'Autriche et les services
récents rendus à la cour pendant la Fronde avaient enorgueilli jusqu'à
l'infatuation, traita Mazarin avec une hauteur blessante, et se rendit
coupable de l'insulte la plus sensible à l'égard d'une femme et d'une
reine, en prétendant imposer un amant à Anne d'Autriche (1649). Il
choisit pour ce rôle Jarzé, un de ces jeunes gens que leur fatuité et
leur présomption faisaient désigner sous le nom de
\emph{petits-maîtres}. Un pareil outrage porta le désespoir dans l'âme
d'Anne d'Autriche. «\,Je sais, dit Mazarin dans ses
\emph{carnets}\footnote{\emph{Carnets}, n° XIII, p.~79.}**, que la reine
ne dort plus, qu'elle soupire la nuit et pleure, et que tout procède du
mépris où elle croit être, et que tant s'en faut qu'elle attende
changement que, au contraire, elle est persuadée que cela empirera.\,»

Mazarin fut, dans cette situation délicate, le conseiller et le guide
d'Anne d'Autriche, et en rapprochant des \emph{carnets} le récit de
M\textsuperscript{me} de Motteville, on voit avec quelle docilité la
reine suivait les instructions du cardinal. Mazarin a consigné dans ses
\emph{carnets} les conseils qu'il donna à la reine \footnote{\emph{Ibidem},
  p.~95.}\,: «\,La reine pourrait dire devant beaucoup de princesses et
autres personnes\,: \emph{J'aurai grand tort à présent de me plaindre
plus de rien, ayant un galant si bien fait que Jarzé. Je crains
seulement de le perdre un de} ces jours, \emph{que je ne pourrai
empêcher qu'on ne le mène aux Petites-Maisons, et je n'aurai pas
l'avantage que l'on dise qu'il est devenu fou pour amour de moi, parce
qu'on sait qu'il y a longtemps qu'il est affligé de cette maladie}.
Après quoi, la première fois que Jarzé entrera dans le lieu que la reine
sera, s'il a l'effronterie après ce que dessus de s'y présenter, elle
lui pourrait dire en riant\,: \emph{Eh bien\,! monsieur de Jarzé\,; me
trouvez-vous à votre gré\,? Je ne pensai jamais avoir une si bonne
fortune. Il faut que cela vous vienne de race\,; car le bonhomme
Lavardin}\footnote{Il s'agit du maréchal de Lavardin, né en 1551, mort
  en 1614\,; il était aïeul maternel de Jarzé.}\emph{était} aussi
\emph{galant de la reine mère}\footnote{Marie de Médicis.}\emph{avec la
même joie de toute la cour qu'elle témoigne à présent de votre
amour}.\,»

M\textsuperscript{me} de Motteville assista à la scène qu'avait préparée
Mazarin, et son récit prouve que la mémoire d'Anne d'Autriche fut fidèle
et qu'elle prononça à peu de chose près les paroles que Mazarin lui
avait dictées\,: «\,Comme Jarzé, dit M\textsuperscript{me} de
Motteville\footnote{\emph{Mémoires}, collect. Petitot, 2° série, t.
  xxxviii, p.~405, 406.}, savait à peu près la disgrâce de son amie,
M\textsuperscript{me} de Beauvais\footnote{M\textsuperscript{me} de
  Beauvais était première femme de chambre d'Anne d'Autriche.
  M\textsuperscript{me} de Motteville en parle ainsi dans ses
  \emph{Mémoires} (collect. Petitot, \emph{ibidem}, p.~400, 401)\,:
  «\,M\textsuperscript{me} de Beauvais, première femme de chambre de la
  reine, était amie de Jarzé, qui n'étant ni belle ni jeune, et voulant
  avoir des amis, avait flatté Jarzé de cette pensée qu'elle le rendrait
  agréable à la reine, et lui ferait de bons offices.\,» L'époque de
  l'exil de M\textsuperscript{me} de Beauvais est marquée avec
  exactitude dans le Journal inédit de Dubuisson-Aubenay, gentilhomme
  attaché au secrétaire d'État Duplessis-Guénégaud\emph{\,: Le mercredi
  24 décembre (1649), les meubles de l'appartement de la dame de
  Beauvais, première femme de chambre de la reine, ont été enlevés du
  Palais-Royal et menés en la maison qu'elle a à Gentilly et où elle
  s'en alla dès le jour précédent avec toute sa famille, la reine lui
  ayant fait dire par Largentier, surnommé Legras, secrétaire de la
  reine, qu'elle eût à se retirer, sur le midi, comme Sa Majesté entrait
  en son carrosse pour aller ouïr messe aux Filles Sainte-Marie près la
  Bastille. Elle avait encore le matin été coiffée par ladite dame de
  Beauvais. «\,Le même journal fixe la date de la scène faite à Jarzé
  par la reine et la raconte ainsi\,: «\,Le vendredi (26 décembre 1649),
  la reine retournant de la galerie et chapelle du roi, où elle avait
  oui la messe, le marquis de Jarzé, peigné, poudré et vêtu à
  l'avantage, se trouve à son passage sur la terrasse, qui fait clôture
  à la cour intérieure et regarde sur le jardin du Palais-Royal, où il
  marche devant la reine, se tourne vers elle à certaines distances et
  pauses en l'attendant, et entré dans le grand cabinet se met en baie
  pour être vu de plus près d'elle à son passage, puis entre avec Sa
  Majesté dans la chambre du lit et plus outre dans la chambre du
  miroir, où la reine se coiffe ordinairement et se présente devant Sa
  Majesté qui lui fait signe de s'approcher d'elle et marche deux pas,
  puis s'arrêtant lui dit tout haut\,: }C'est une plaisante chose que
  l'on dise par la ville que vous, Jarzé, soyez pion galant. Vous en
  êtes bien aise, je m'assure, et vous ave\,; cette folie-là qui vous
  vient de votre grand-père. Mais vous ne prenez pas garde que cela vous
  fait passer pour impertinent et ridicule*.\,» L'auteur, qui n'avait
  pas assisté à la scène, altère un peu les paroles de la reine
  reproduites bien plus exactement par M\textsuperscript{me} de
  Motteville. Bibl. Maz., ms. in-fol., H, 1765 et non 1719, comme on a
  imprimé par erreur, t. V, p.~438 de cette édition des Mémoires de
  Saint-Simon.}, l'état \emph{où il} était à la cour, il crut faire voir
un tour d'habile politique de paraître ne penser à rien et ne rien
craindre\,; mais l'heure était venue qu'il devait être puni de son
impudence. La reine ayant dans l'esprit de le maltraiter, aussitôt
qu'elle l'aperçut ne manqua pas de l'attaquer et de lui dire avec un ton
méprisant ces mêmes paroles\,: \emph{Vraiment, monsieur de Jarzé, vous
êtes bien ridicule. On m'a dit que vous faites l'amoureux. Voyez un peu
le joli galant\,! Vous me faites pitié\,: il faudrait vous envoyer aux
Petites-Maisons. Mais il est vrai qu'il ne faut pas s'étonner de votre
folie, car vous tenez de race}. Voulant citer en cela le maréchal de
Lavardin, qui autrefois avait été passionnément amoureux de la reine
Marie de Médicis, et dont le roi son mari, Henri le Grand, se moquait
lui-même avec elle. Le pauvre Jarzé fut accablé de ce coup de foudre. Il
n'osa rien dire à sa justification. Il sortit du cabinet en bégayant,
mais plein de trouble, pâle et défait. Malgré sa douleur, peut-être se
flattait-il déjà de cette douce pensée que l'aventure était belle, que
ce crime était honorable et qu'il n'était pas honteux d'en être accusé.
Toute la cour fut aussitôt remplie de cet événement, et les ruelles des
dames retentissaient du bruit de ces royales paroles. On fut longtemps
que le nom de Jarzé s'entendait nommer dans Paris, et les provinces en
eurent bien vite leur part. Beaucoup de gens blâmèrent la reine d'avoir
voulu montrer ce ressentiment, et disaient qu'elle avait fait trop
d'honneur à Jarzé d'avoir daigné se rabaisser jusqu'à cette colère, et
que la dignité de la couronne en avait été blessée. Aussi peut-on dire
pour réparer cette petite faute, qu'elle ne l'aurait pas faite, si elle
n'y avait été forcée par les craintes du ministre, qui, voyant Jarzé
fidèle à M. le Prince, ingrat envers lui, ne pouvait pas manquer de
croire que, sous cette affectation de bouffonnerie, il y avait quelque
malignité frondeuse contre sa fortune.\,»

M\textsuperscript{me} de Motteville, comme on le voit, ne soupçonnait
pas à quel point Anne d'Autriche était dominée par son ministre, et que
la scène qu'elle venait de raconter avait été arrangée par le cardinal
jusque dans ses moindres détails. Cet exemple suffit pour montrer quel
intérêt présentent les \emph{carnets} de Mazarin comme document
historique. Déjà un écrivain célèbre en a signalé l'importance pour
l'année 1643\footnote{Voy. les articles de M. Cousin dans le
  \emph{Journal des Savants} (1854, 1855 et 1856).}\,; mais il est à
regretter qu'aucun des historiens de la Fronde n'ait tiré parti de ces
carnets. C'est en effet pour cette époque qu'ils fournissent le plus de
renseignements. Le cardinal y consigne jour par jour ses pensées, ses
projets, ses conversations. On ne trouve dans ces notes rapides aucune
des réticences qu'impose la correspondance officielle\,; c'est
l'épanchement du coeur, la révélation complète du génie et des
faiblesses de l'homme qui tenait dans ses mains les destinées de la
France.

\hypertarget{note-iv.-extraits-des-papiers-du-duc-de-noailles.}{%
\chapter{NOTE IV. EXTRAITS DES PAPIERS DU DUC DE
NOAILLES.}\label{note-iv.-extraits-des-papiers-du-duc-de-noailles.}}

J'ai déjà fait remarquer\footnote{Bibl. impér. du Louvre, ms., F. 325.}
que les papiers du duc de Noailles conservés à la bibliothèque impériale
du Louvre fournissent de curieux renseignements pour contrôler les
Mémoires de Saint-Simon. J'ajouterai ici quelques extraits relatifs aux
affaires d'Espagne, dont parle Saint-Simon.

\hypertarget{i.-extrait-dune-lettre-de-la-princesse-des-ursins-uxe0-torcybibl.-impuxe9r.-du}{%
\section{I. EXTRAIT D'UNE LETTRE DE LA PRINCESSE DES URSINS À
TORCY\^{}{[}Bibl. impér.
du}\label{i.-extrait-dune-lettre-de-la-princesse-des-ursins-uxe0-torcybibl.-impuxe9r.-du}}

Louvre, ms., F. 325, t. XXV, p.~18 et suiv.\,\,; copie du temps.{]}.

(4 mars 1708)

Sans contester l'anecdote racontée par Saint-Simon (p.~301, 302 de ce
volume) et par laquelle il explique les dispositions peu favorables de
la princesse des Ursins pour le duc d'Orléans, on peut remarquer
qu'avant l'arrivée de ce prince en Espagne, M\textsuperscript{me} des
Ursins se plaignait au ministre français du rappel de Berwick et lui
exprimait ses inquiétudes. Elle lui écrivait dès le 4 mars 1708\,\,:

«\,\,Nous sommes ici dans l'espérance d'y voir bientôt arriver M. le duc
d'Orléans. Si on veut en croire le public, nous perdons M. le maréchal
de Berwick, puisqu'on prétend qu'il retourne en France et même qu'il ira
commander en Dauphiné. Le roi et la reine ne sauraient s'imaginer,
monsieur, qu'on leur ôte un général qu'ils avaient demandé, qui leur est
très nécessaire, que les Espagnols aiment et qui a pris une parfaite
connaissance de tout ce qui regarde la guerre de ce pays-ci, sans que le
roi veuille bien les instruire du motif qui l'oblige à faire un pareil
changement, se fiant à la bonté du roi leur grand-père, qui ne voudrait
pas sans doute que les sujets du roi son petit-fils crussent qu'il en
fait peu de cas.

«\,\,On n'ajoutera donc pas de foi, monsieur, à une pareille
nouvelle\,\,; mais si, par malheur, elle se trouvait vraie, cela
produirait certainement un très mauvais effet. C'est vous dire mes
sentiments bien naïvement\,\,; mais je suis persuadée que je me fie à un
ami qui n'en fera pas moins bon usage, et qui connaît que ce n'est que
mon zèle pour les deux rois qui me fait sentir tout ce que je crains qui
pourrait les rendre moins contents l'un de l'autre qu'ils ne doivent
l'être.\,\,»

\hypertarget{ii.-arrivuxe9e-des-galions-en-espagne.}{%
\section{II. ARRIVÉE DES GALIONS EN
ESPAGNE.}\label{ii.-arrivuxe9e-des-galions-en-espagne.}}

Saint-Simon parle (p.~408 de ce volume) de l'arrivée des galions sous la
conduite de Ducasse. On voit par les lettres d'Amelot, ambassadeur de
France en Espagne, combien on y était préoccupé du sort des galions et
de la nouvelle répandue que les Anglais s'en étaient emparés.
L'ambassadeur écrivait à Louis XIV le 10 septembre 1708\footnote{\emph{Ibid}.,
  fol.~136 et suiv.\,\,; copie du temps.}\,\,: «\,\,Les avis du malheur
qu'on prétend qui est arrivé aux galions donnent ici beaucoup
d'inquiétude. La juste crainte qu'on a eue que ces avis ne se vérifient
est fortifiée par tout ce qu'écrit M. Ducasse du mauvais état des
galions. Ce qui rassure un peu est ce que dit le chevalier de Layet, qui
a été envoyé ici par M. Ducasse, en arrivant au Port-du-Passage. Il
prétend qu'avant de partir de la Havane, on a eu des lettres du général
des galions du 15 et du 20 juin, et que, suivant les nouvelles de
Londres et de Hollande, l'affaire doit s'être passée le 9 du même mois
(nouv. st.)\,\,; ce qui détruirait absolument la possibilité de cet
événement par les dates. La perte des galions dans les conjonctures
présentes serait une chose si terrible qu'on retardera tant qu'on pourra
d'y ajouter foi sans une pleine confirmation.\,\,»

Le 17 septembre, le même ambassadeur paraissait plus rassuré dans la
lettre qu'il adressait à Louis XIV \footnote{Bibl. imp. du Louvre,
  \emph{ibid}., fol.~141 et suiv.}\,\,: «\,\,L'inquiétude, Sire, qu'on
avait, il y a huit jours, pour les galions, a été diminuée par des avis
de Carthagène des Indes\footnote{Indes occidentales ou Amérique.}, du 28
juin, qui marquent qu'on y attend les galions, sans parler de combat ni
de rien d'approchant. Il est venu aussi des lettres écrites de la rade
de Saint-Domingue, du 7 et du 8 juillet, par des officiers embarqués sur
la flottille qui s'était arrêtée en cet endroit. Ces lettres disent
qu'il n'y avait aucune nouveauté en ces mers-là, et que jusqu'alors le
voyage de la flottille avait été très heureux. J'ai deux lettres de ces
deux dates, et dans ce sens, l'une d'un Espagnol et l'autre d'un
François. Cela donne lieu de croire que, si l'aventure des galions était
arrivée le 9 juin, comme les nouvelles de Hollande et d'Angleterre le
publient, on en aurait su quelque chose un mois après à Saint-Domingue
et à Puerto Rico, où la flottille avait mouillé dans les premiers jours
de juillet pour faire de l'eau.\,\,»

Enfin le roi d'Espagne, Philippe V, écrivit une lettre autographe au duc
d'Orléans qui l'avait félicité de l'arrivée des galions\footnote{Lettre
  du 19 septembre 1708, papiers du duc de Noailles, \emph{ibid}.,
  fol.~142\,\,; copie du temps.}\,\,: «\,\,Je vous remercie du
compliment que vous me faites sur l'arrivée de la flotte de la
Nouvelle-Espagne\,\,; c'est un secours qui nous est venu fort à propos,
et dont vous connaîtrez toute l'importance. J'ai écrit au roi mon
grand-père pour savoir son sentiment sur les projets que vous m'avez
communiqués pour la campagne prochaine, et je lui ai mandé que, s'il y
avait quelque apparence à pouvoir chasser entièrement l'archiduc de la
Catalogne, je ne balancerais pas à croire qu'il faudrait faire tous nos
efforts pour cela et laisser nos plus grandes forces de ce côté-là\,\,;
mais que, cela étant comme impossible par toutes les difficultés qui s'y
rencontrent, il me paraissait que le meilleur parti qu'on pourrait
prendre, était d'y laisser un nombre de troupes suffisant pour empêcher
les ennemis d'y pouvoir rien entreprendre, et d'agir vigoureusement
contre le Portugal avec le reste de nos forces. Vous savez que le projet
que vous avez formé pour ce côté-là a toujours été fort de mon goût, et
je vous assure qu'il me tient encore fort à coeur. Je suis fort inquiet
sur les affaires de Flandre, dont je ne sais point encore le dénouement.
Dieu veuille qu'il soit bon pour nous, car il est d'une grande
conséquence.\,\,»

Quant aux richesses rapportées par les galions et que Saint-Simon évalue
à soixante millions (argent et denrées), d'après les bruits répandus,
elles furent loin d'être aussi considérables. Amelot écrivait à Louis
XIV le 24 septembre 1708\footnote{Bibliothèque impér. du Louvre,
  \emph{ibid}., fol.~145.}\,\,: «\,\,On continue, au Port-du-Passage, à
décharger les effets de la flotte et à régler toutes les affaires qui en
dépendent par les soins et sous la direction de don Pedro Navarette. On
a voulu dire que cette flotte était riche de dix-sept, de vingt et
jusqu'à trente millions d'écus\,\,; niais ce sont des exagérations qu'on
fait toujours à l'arrivée des flottes et des galions, et les gens
instruits de l'état du commerce de la Nouvelle-Espagne savent bien que
cela n'est pas possible. Il est certain, Sire, que cela ne passe pas dix
à onze millions d'écus, y compris ce qui est venu pour le compte du roi
d'Espagne ou pour le commerce des Indes et les tribunaux qui en
dépendent.\,\,»

\end{document}
